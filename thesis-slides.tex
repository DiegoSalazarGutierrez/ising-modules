\documentclass{beamer}
\usetheme{Madrid}

\usepackage{enumitem}
\usepackage{tikz-cd}
\usepackage[nameinlink]{cleveref}

\setbeamertemplate{theorems}[numbered]

\setenumerate[0]{label = \normalfont(\roman*)}

\DeclareMathOperator{\Vir}{Vir}
\DeclareMathOperator{\Id}{Id}
\DeclareMathOperator{\End}{End}
\DeclareMathOperator{\ch}{ch}
\DeclareMathOperator{\vspan}{span}
\DeclareMathOperator{\Ind}{Ind}
\DeclareMathOperator{\Res}{Res}
\DeclareMathOperator{\vac}{|0\rangle}
\DeclareMathOperator{\vachalf}{|1/2\rangle}
\DeclareMathOperator{\vacsixteen}{|1/16\rangle}
\DeclareMathOperator{\zero}{\overline{0}}
\DeclareMathOperator{\Zhu}{Zhu}

\title{PBW bases of irreducible Ising modules}
\author{Diego Salazar}
\institute[IMPA]
{
  Instituto de Matemática Pura e Aplicada (IMPA) \\
  Rio de Janeiro -- Brasil \\ [1cm]
}
\date[26 September 2023]{26 September 2023}

\begin{document}

\maketitle

\begin{frame}
  \frametitle{Vertex superalgebras}

  A \emph{vertex superalgebra} is the data consisting of four elements $(V, \vac, T, Y)$ satisfying the following properties:
  \begin{enumerate}
  \item $V$ is a superspace called the \emph{state space};
  \item $\vac \in V_{\zero}$ is called the \emph{vacuum vector};
  \item $T \in \End(V)_{\zero}$ is called the \emph{translation operator};
  \item $Y: V \to \mathcal{F}(V)$ is a linear and parity preserving map called the \emph{state-field correspondence}, which is commonly written as $Y(a, z) = \sum_{n \in \mathbb{Z}}a_{(n)}z^{-n - 1}$ for $a \in V$.
  \end{enumerate}
  The data must satisfy the following axioms for $a \in V$:

\end{frame}

\begin{frame}

  \begin{enumerate}
  \item (Vacuum axiom)
    \begin{align*}
      Y(\vac,z) &= \Id_V, \\
      Y(a, z)\vac &\in V[[z]], \\
      Y(a, z)\vac|_{z = 0} &= a, \\
      T\vac &= 0;
    \end{align*}
  \item (Translation covariance) $[T, Y(a, z)] = \partial_zY(a, z)$;
  \item (Locality) $\{Y(b, z) \mid b \in V\}$ is a local family of fields.
  \end{enumerate}

  Axioms (i)--(iii) are equivalent to the following three axioms.
  For $a, b, c \in V$ and $m, n \in \mathbb{Z}$:
  \begin{align*}
    (\vac)_{(n)} &= \delta_{n, -1}\Id_V, \\
    [a_{(m)}, b_{(n)}] &= \sum_{j \in \mathbb{N}}(a_{(j)}b)_{(m + n - j)}, \\
    (a_{(n)}b)_{(m)}c &= \sum_{j \in \mathbb{N}}(-1)^j\binom{n}{j}(a_{(n - j)}b_{(m + j)}c - (-1)^np(a, b)b_{(n + m - j)}a_{(j)}c).
  \end{align*}

\end{frame}

\begin{frame}
  \frametitle{Modules over vertex superalgebras}

  Let $V$ be a vertex superalgebra.
  A $V$-module is a superspace $M$ together with a linear and parity preserving map $Y^M: V \to \mathcal{F}(M)$, written as $Y^M(a, z) = \sum_{n \in \mathbb{Z}}a^M_{(n)}z^{-n - 1}$, such that for $a, b \in V$, $u \in M$ and $m, n \in \mathbb{Z}$:
  \begin{align*}
    (\vac)^M_{(n)} &= \delta_{n, -1}\Id_M, \\
    [a^M_{(m)}, b^M_{(n)}] &= \sum_{j \in \mathbb{N}}(a_{(j)}b)^M_{(m + n - j)}, \\
    (a_{(n)}b)^M_{(m)}u &= \sum_{j \in \mathbb{N}}(-1)^j\binom{n}{j}(a^M_{(n - j)}b^M_{(m + j)}u - (-1)^np(a, b)b^M_{(n + m - j)}a^M_{(j)}u).
  \end{align*}

\end{frame}

\begin{frame}
  \frametitle{Examples of vertex algebras and their modules}

  \begin{example}[Commutative vertex algebras]
    \label{exa:1}
    Let $V$ be a differential commutative associative algebra with unit $1$ and derivation $T$.
    Then $V$ becomes a vertex algebra by setting  $\vac = 1$ and
    \begin{equation*}
      Y(a, z) = e^{Tz}a \quad \text{for $a \in V$}.
    \end{equation*}
  \end{example}

  \begin{example}[Virasoro Lie algebra]
    \label{exa:2}
    The Virasoro Lie algebra, denoted by $\Vir$, is the Lie algebra given by:
    \begin{align*}
      \Vir &= \bigoplus_{n \in \mathbb{Z}}\mathbb{C}L_n \oplus \mathbb{C}C, \\
      [L_m, L_n] &= (m - n)L_{m + n} + \delta_{m, -n}\frac{m^3 - m}{12}C \quad \text{for $m, n \in \mathbb{Z}$}, \\
      [\Vir, C] &= 0.
    \end{align*}
  \end{example}

\end{frame}

\begin{frame}

  \begin{example}[Virasoro vertex algebras]
    \label{exa:3}
    We make the subalgebra $\Vir_{\ge -1} \oplus \mathbb{C}C$ act on $\mathbb{C}$ as follows:
    \begin{equation*}
      \text{$L_n1 = 0$ for $n \ge -1$ and $C1 = c$}.
    \end{equation*}
    The \emph{universal Virasoro vertex algebra} of central charge $c$, denote by $\Vir^c$, is an induced $\Vir$-module given by:
    \begin{align*}
      \Vir^c &= \Ind^{\Vir}_{\Vir_{\ge -1} \oplus \mathbb{C}C}(\mathbb{C}) = U(\Vir) \otimes_{U(\Vir_{\ge -1} \oplus \mathbb{C}C)} \mathbb{C}, \\
      \vac &= 1\otimes 1, \\
      T & = L_{-1}, \\
      Y(L_{-2}\vac, z) &= \sum_{n \in \mathbb{Z}}L_{-n}z^{-n - 2}.
    \end{align*}

    The vertex algebra $\Vir^c$ has a unique maximal ideal, and the quotient $\Vir_c$ is the \emph{simple Virasoro vertex algebra} of central charge $c$.
  \end{example}

\end{frame}

\begin{frame}

  \begin{example}[Verma modules]
    \label{exa:4}
    We pick a pair $(c, h)$ of complex numbers.
    The subalgebra $\Vir_{\ge 0} \oplus \mathbb{C}C$ of $\Vir$ acts on $\mathbb{C}$ as follows:
    \begin{equation*}
      \text{$L_n1 = 0$ for $n \in \mathbb{Z}_+$, $L_01 = h$ and $C1 = c$}.
    \end{equation*}
    Then
    \begin{equation*}
      M(c, h) = \Ind^{\Vir}_{\Vir_{\ge 0} \oplus \mathbb{C}C}(\mathbb{C}) = U(\Vir) \otimes_{U(\Vir_{\ge 0} \oplus \mathbb{C}C)} \mathbb{C}
    \end{equation*}
    is a $\Vir$-module, where $\Vir$ acts by left multiplication.
    We set $|c, h\rangle = 1\otimes1$.
    We call this module the \emph{Verma module} with highest weight $(c, h)$.
    The representation $M(c, h)$ has a unique maximal subrepresentation $J(c, h)$, and the quotient
    \begin{equation*}
      L(c, h) = M(c, h)/J(c, h)
    \end{equation*}
    is the irreducible highest weight representation with highest weight $(c, h)$.
    Both $M(c, h)$ and $L(c, h)$ are modules over the vertex algebra $\Vir^c$.
  \end{example}

\end{frame}

\begin{frame}

  By the PBW theorem, a basis of $\Vir^c$, a.k.a.~PBW basis, is given by monomials $L_{-\lambda_1}\dots L_{-\lambda_m}\vac$ for partitions $\lambda = [\lambda_1, \dots, \lambda_m]$ with $\lambda_m \ge 2$.

  Let $p, q \ge 2$ be relatively prime integers, and we set $c_{p, q} = 1 - \frac{6(p - q)^2}{pq}$.

  \begin{theorem}[{\cite{gorelik_simplicity_2007}}]
    \label{thr:1}
    The following are equivalent:
    \begin{enumerate}
    \item $\Vir^c$ is not simple, i.e., $\Vir^c \neq \Vir_c$;
    \item $c$ is of the form $c_{p, q}$ for some $p, q \ge 2$ relatively prime integers.
    \end{enumerate}
    Moreover, the maximal ideal of $\Vir^{c_{p, q}}$ is generated by a singular vector of conformal weight $(p - 1)(q - 1)$, denoted by $a_{p, q}$.
    In the expression
    \begin{equation*}
      a_{p, q} = \sum_{\substack{i_1 \ge \dots \ge i_k \ge 2 \\ i_1 + \dots + i_k = (p - 1)(q - 1)}}c_{i_1\dots i_k}L_{-i_1}\dots L_{-i_k}\vac,
    \end{equation*}
    where $c_{i_1\dots i_k} \in \mathbb{Q}$, the coefficient of $L_{-2}^{(p - 1)(q - 1)/2}$ is nonzero.
  \end{theorem}

\end{frame}

\begin{frame}
  \frametitle{PBW basis of the Ising model}

  For the \emph{Ising model} $\Vir_{3, 4} = \Vir_{1/2} = L(1/2, 0)$, we have $a_{3, 4} = (L_{-2}^3 + \frac{93}{64}L_{-3}^2 - \frac{27}{16}L_{-6} - \frac{33}{8}L_{-4}L_{-2})\vac$.
  Let $R^0$ be the following set of partitions
  \footnotesize
  \begin{align*}
    &[r, r, r], [r + 1, r, r], [r + 1, r + 1, r], [r + 2, r + 1, r], [r + 2, r + 2, r], &(r \ge 2) \\
    &[r + 2, r, r], &(r \ge 3) \\
    &[r + 3, r + 3, r, r], [r + 4, r + 3, r, r],  [r + 4, r + 3, r + 1, r], [r + 4, r + 4, r + 1, r], &(r \ge 2) \\
    &[r + 6, r + 5, r + 3, r + 1, r], &(r \ge 2) \\
    &[5, 4, 2, 2], [7, 6, 4, 2, 2], [7, 7, 4, 2, 2], [9, 8, 6, 4, 2, 2].
  \end{align*}
  \normalsize
  Let $P^0$ be the set of partitions $\lambda = [\lambda_1, \dots, \lambda_m]$ with $\lambda_m \ge 2$ that do not contain any partition in $R^0$.

  \begin{theorem}[{\cite{andrews_singular_2022}}]
    \label{thr:2}
    The set $\{L_{-\lambda_1}L_{-\lambda_2}\dots L_{-\lambda_m}\vac \mid \lambda = [\lambda_1, \dots, \lambda_m] \in P^0\}$ is a vector space basis of the Ising model $\Vir_{3, 4}$.
  \end{theorem}

\end{frame}

\begin{frame}
  \frametitle{Graded and conformal vertex superalgebras}

  Let $V$ be a vertex superalgebra.
  A \emph{Hamiltonian operator} of $V$ is a diagonalizable operator $H \in \End(V)$ such that
  \begin{equation}
    \label{eq:1}
    [H, Y(a, z)] = z\partial_zY(a, z) + Y(Ha, z) \quad \text{for $a \in V$}.
  \end{equation}
  A vertex superalgebra with a Hamiltonian operator is called \emph{graded}.
  The \emph{grading} of $V$ is the eigenspace decomposition of $H$
  \begin{equation*}
    V = \bigoplus_{\Delta \in \mathbb{C}}V_{\Delta},
  \end{equation*}
  where
  \begin{equation*}
    V_{\Delta} = \ker(H - \Delta\Id_V) \quad \text{for $\Delta \in \mathbb{C}$}.
  \end{equation*}
  If $a$ is an eigenvector of $H$, it is called \emph{homogeneous}, its eigenvalue is called the \emph{conformal weight} of $a$, and it is denoted by $\Delta_a$.
  For $a \in V$ homogeneous, we write $Y(a, z) = \sum_{n \in \mathbb{Z} - \Delta_a}a_nz^{-n - \Delta_a}$.

\end{frame}

\begin{frame}

  Let $V$ be a vertex superalgebra.
  A \emph{conformal vector} of central charge $c \in \mathbb{C}$ of $V$ is a vector $\omega \in V$ such that $Y(\omega, z) = \sum_{n \in \mathbb{Z}}L_nz^{-n - 2}$ satisfies:
  \begin{enumerate}
  \item $Y(\omega, z)$ is a Virasoro formal distribution of central charge $C = c\Id_V$;
  \item $L_{-1} = T$;
  \item $L_0$ is diagonalizable.
  \end{enumerate}

  A \emph{conformal vertex superalgebra} (of central charge $c$) is a vertex superalgebra $V$ together with a conformal vector $\omega$ (of central charge $c$).
  Conformal vertex algebras are graded by $L_0$, and the conformal vector $\omega$ has conformal weight $\Delta_{\omega} = 2$.

  \begin{example}
    We pick $c \in \mathbb{C}$.
    Both $\Vir^c$ and $\Vir_c$ are conformal vertex algebras with conformal vector $L_{-2}\vac$.
  \end{example}

\end{frame}

\begin{frame}
  \frametitle{Characters of modules over vertex superalgebras}

  Let $M$ be a module over a graded vertex superalgebra $V$ with Hamiltonian operator $H$.
  A \emph{Hamiltonian operator} of $M$ is a diagonalizable operator $H^M \in \End(M)$ satisfying \eqref{eq:1}.
  If $V$ is conformal with conformal vector $\omega$ such that $L_0^M$ is diagonalizable, where $Y^M(\omega, z) = \sum_{n \in \mathbb{Z}}L_n^Mz^{-n - 2}$, then $M$ is graded by $L_0^M$.
  Furthermore, if $M$ is $h + \mathbb{N}$-graded for some $h \in \mathbb{R}$, we define the \emph{character} of $M$ by
  \begin{equation*}
    \ch_M(q) = \sum_{n \in \mathbb{N}}\dim(M_{h + n})q^{h + n} \in q^h\mathbb{C}[[q]].
  \end{equation*}

  \begin{example}
    The Verma module $M(c, h)$ for some highest weight $(c, h)$ satisfies
    \begin{equation*}
      \ch_{M(c, h)}(q) = \frac{q^h}{\prod_{j \in \mathbb{N}}(1 - q^j)}.
    \end{equation*}
  \end{example}

\end{frame}

\begin{frame}

  \begin{theorem}[{\cite{andrews_singular_2022}}]
    The characters of $\Vir_{3, 4} = L(1/2, 0)$, $L(1/2, 1/2)$ and $L(1/2, 1/16)$ are given by:
    \begin{align*}
      \ch_{L(1/2, 0)} &= \sum_{k_1, k_2 \in \mathbb{N}}\frac{q^{4k_1^2 + 3k_1k_2 + k_2^2}}{(q)_{k_1}(q)_{k_2}}(1 - q^{4k_1 + 2k_2 +1}), \\
      \ch_{L(1/2, 1/2)} &= q^{1/2}\left(\sum_{k_1, k_2 \in \mathbb{N}}\frac{q^{4k_1^2 +3k_1k_2 + k_2^2 + 2k_1}}{(q)_{k_1}(q)_{k_2}}(1 - q^{8k_1 + 4k_2 + 6})\right), \\
      \ch_{L(1/2, 1/16)} &= q^{1/16}\left(\sum_{k_1, k_2 \in \mathbb{N}}\frac{q^{4k_1^2 + 3k_1k_2 + k_2^2}}{(q)_{k_1}(q)_{k_2}}(q^{k_1 + k_2} + q^{4k_1 + k_2 + 1})\right).
    \end{align*}
  \end{theorem}

\end{frame}

\begin{frame}
  \frametitle{The Zhu algebra}

  Let $V$ be a $\mathbb{Z}$-graded vertex algebra.
  For $a, b \in V$ with $a$ homogeneous, we set:
  \begin{align*}
    a\circ b &= \Res_z\left(\frac{(1 + z)^{\Delta_a}}{z^2}Y(a, z)b\right) \in V, \\
    a*b &= \Res_z\left(\frac{(1 + z)^{\Delta_a}}{z}Y(a, z)b\right) \in V, \\
    O(V) &= \vspan\{a\circ b \mid a, b \in V\}.
  \end{align*}

  \begin{theorem}[{\cite{dong_twisted_1998}}]
    \label{thr:3}
    The subspace $O(V)$ is a two-sided ideal of $V$ under the operation $*$, and $V/O(V)$ becomes an associative algebra with unit $\vac + O(V)$.
  \end{theorem}

  The \emph{Zhu algebra} of $V$ is the quotient $\Zhu(V) = V/O(V)$.

\end{frame}

\begin{frame}

  For a $V$-module $M$, the subspace of \emph{lowest weight vectors of $M$} is
  \begin{equation*}
    \Omega(M) = \{u \in M \mid \text{for $a \in V$ homogeneous and $n \in \mathbb{Z}_+$, $a_nu = 0$}\}.
  \end{equation*}

  \begin{theorem}[{\cite{dong_twisted_1998}}]
    \label{thr:4}
    Let $V$ be a $\mathbb{Z}$-graded vertex algebra, and let $M$ be a $V$-module.
    Then the following map
    \begin{align*}
      \Zhu(V) &\to \End(\Omega(M)), \\
      a + O(V) &\mapsto a^M_0 = a^M_{(\Delta_a - 1)} \quad \text{for $a \in V$ homogeneous}
    \end{align*}
    defines a $\Zhu(V)$-module structure on $\Omega(M)$.
  \end{theorem}

  We have constructed a functor
  \begin{equation*}
    \Omega: \text{$V$-Mod} \to \text{$\Zhu(V)$-Mod}.
  \end{equation*}

\end{frame}

\begin{frame}[fragile]

  \begin{example}
    \label{exa:5}
    If $V = \Vir^c$ and $M$ is a $\Vir^c$-module, then $\Omega(M) = \{u \in M \mid \text{for $n \in \mathbb{Z}_+$, $L_nu = 0$}\}$ is the subspace of singular vectors of $M$.
  \end{example}

  We can construct a right inverse for $\Omega$, i.e., we have a pair of functors
  \begin{equation*}
    \begin{tikzcd}
      \text{$\Zhu(V)$-Mod} \arrow[r, "L", shift left] & \{\text{admissible $V$-modules}\} \arrow[l, "\Omega", shift left]
    \end{tikzcd}
  \end{equation*}
  such that $\Omega\circ L$ is equivalent to the identity.

  \begin{theorem}[{\cite{dong_twisted_1998}}]
    \label{thr:5}
    $L$ and $\Omega$ are equivalences when restricted to the full subcategories of completely reducible $\Zhu(V)$-modules and completely reducible admissible $V$-modules respectively.
  \end{theorem}

\end{frame}

\begin{frame}[fragile]
  \frametitle{Modules over the Virasoro vertex algebras}

  \begin{example}[{\cite{wang_rationality_1993}}]
    For $c \in \mathbb{C}$, we have $\Zhu(\Vir^c) \cong \mathbb{C}[x]$.
    For $c = c_{p, q}$ for some $p, q \ge 2$ relatively prime integers, we have $\Zhu(\Vir_c) \cong \mathbb{C}[x]/(G_{p, q}(x))$, where $G_{p ,q}(x)^2 = \prod_{m = 1}^{p - 1}\prod_{n = 1}^{q - 1}(x - h_{m, n})$ and $h_{m, n} = \frac{(np - mq)^2 - (p - q)^2}{4pq}$.
  \end{example}

  \begin{theorem}[{\cite{wang_rationality_1993}}]
    \label{thr:6}
    We set $c = c_{p, q}$ for some $p, q \ge 2$ relatively prime integers.
    Then the irreducible admissible modules over $\Vir_c$ are $L(c, h_{m, n})$ for integers $m, n$ such that $0 < m < p$ and $0 < n < q$, and the following diagram commutes
    \begin{equation*}
      \begin{tikzcd}
        \Vir^c \arrow[r, two heads] \arrow[rd, "{Y^{L(c, h_{m, n})}_{\Vir^c}}"'] & {\Vir_c} \arrow[d, "{Y^{L(c, h_{m, n})}_{\Vir_c}}"] \\
        & {\mathcal{F}(L(c, h_{m, n}))}
      \end{tikzcd}
    \end{equation*}
  \end{theorem}

\end{frame}

\begin{frame}
  \frametitle{PBW bases of irreducible Ising modules}

  It follows from \Cref{thr:6} that the irreducible admissible modules over the Ising model $\Vir_{3, 4}$, called \emph{Ising modules}, are $\Vir_{3, 4} = L(1/2, 0)$, $L(1/2, 1/2)$ and $L(1/2, 1/16)$.

  Let $R^{1/2}$ be the following set of partitions
  \footnotesize
  \begin{align*}
    &[r, r, r], [r + 1, r, r], [r + 1, r + 1, r], [r + 2, r + 1, r], [r + 2, r + 2, r], &(r \ge 3) \\
    &[r + 2, r, r], &(r \ge 3) \\
    &[r + 3, r + 3, r, r], [r + 4, r + 3, r, r],  [r + 4, r + 3, r + 1, r], [r + 4, r + 4, r + 1, r], &(r \ge 3)\\
    &[r + 6, r + 5, r + 3, r + 1, r], &(r \ge 3) \\
    &[2], [1, 1, 1], [3, 1, 1], [3, 3], [4, 3, 1], [4, 4, 1], [5, 4, 1, 1], [6, 5, 3, 1].
  \end{align*}
  \normalsize
  Let $P^{1/2}$ be the set of partitions that do not contain any partition in $R^{1/2}$.

  \begin{theorem}[\cite{salazar_pbw_2023}]
    \label{thr:7}
    The set $\{L_{-\lambda_1}L_{-\lambda_2}\dots L_{-\lambda_m}\vachalf \mid \lambda = [\lambda_1, \dots, \lambda_m] \in P^{1/2}\}$ is a vector space basis of $L(1/2, 1/2)$.
  \end{theorem}

\end{frame}

\begin{frame}

  Let $R^{1/16}$ be the following set of partitions
  \footnotesize
  \begin{align*}
    &[r, r, r], [r + 1, r, r], [r + 1, r + 1, r], [r + 2, r + 1, r], [r + 2, r + 2, r], &(r \ge 3) \\
    &[r + 2, r, r], &(r \ge 3) \\
    &[r + 3, r + 3, r, r], [r + 4, r + 3, r, r],  [r + 4, r + 3, r + 1, r], [r + 4, r + 4, r + 1, r], &(r \ge 3)\\
    &[r + 6, r + 5, r + 3, r + 1, r], &(r \ge 3) \\
    &[2], [1, 1, 1, 1], [3, 1, 1, 1], [3, 3, 1], [4, 3, 1], [4, 4, 1, 1], [5, 4, 1, 1, 1], [5, 5, 1, 1, 1], \\
    &[6, 5, 3, 1, 1], [6, 6, 3, 1, 1], [7, 6, 4, 1, 1, 1], [8, 7, 5, 3, 1, 1].
  \end{align*}
  \normalsize
  Let $P^{1/16}$ be the set of partitions that do not contain any partition in $R^{1/16}$.

  \begin{theorem}[\cite{salazar_pbw_2023}]
    The set $\{L_{-\lambda_1}L_{-\lambda_2}\dots L_{-\lambda_m}\vacsixteen \mid \lambda = [\lambda_1, \dots, \lambda_m] \in P^{1/16}\}$ is a vector space basis of $L(1/2, 1/16)$.
  \end{theorem}

\end{frame}

\begin{frame}
  \frametitle{References}

  \footnotesize
  \setbeamertemplate{bibliography item}{\insertbiblabel}
  \bibliographystyle{alpha}
  \bibliography{ising-modules.bib}

\end{frame}

\begin{frame}

  \begin{center}
    \Huge
    Thanks! \\
    Obrigado!
  \end{center}

\end{frame}

\end{document}