\documentclass[12pt, a4paper]{article}

\usepackage{amsmath}
\usepackage{amssymb}
\usepackage{amsthm}
\usepackage{enumerate}
\usepackage{hyperref}
\usepackage[margin = 0.8in]{geometry}
\usepackage{enumitem}
\usepackage{hyperref}
\usepackage{tikz-cd}
\usepackage[nameinlink]{cleveref}

\newtheorem{theorem}{Theorem}[section]
\newtheorem{lemma}[theorem]{Lemma}
\newtheorem{proposition}[theorem]{Proposition}
\newtheorem{corollary}[theorem]{Corollary}

\theoremstyle{remark}
\newtheorem{remark}[theorem]{Remark}

\DeclareMathOperator{\Vir}{Vir}
\DeclareMathOperator{\Id}{Id}
\DeclareMathOperator{\gr}{gr}
\DeclareMathOperator{\End}{End}
\DeclareMathOperator{\ch}{ch}
\DeclareMathOperator{\lm}{lm}
\DeclareMathOperator{\vspan}{span}
\DeclareMathOperator{\Ind}{Ind}
\DeclareMathOperator{\len}{len}
\DeclareMathOperator{\psn}{psn}

\newcommand{\vac}{|0\rangle}
\newcommand{\vachalf}{|1/2\rangle}

\begin{document}

\title{PBW bases of irreducible Ising modules}
\author{Diego Salazar Gutierrez \\ \href{mailto:diego.salazar@impa.br}{diego.salazar@impa.br}}
\date{\today}
\maketitle

\begin{abstract}
  ADD SOME ABSTRACT
\end{abstract}

\section{Introduction}
\label{sec:introduction}

In \cite{li_vertex_2004}, Li introduced an increasing filtration $(G^pV)_{p\in \mathbb{Z}}$ on an arbitrary $\mathbb{N}$-graded vertex algebra $V$.
The associated graded space $\gr^G(V)$ with respect to this increasing filtration then carries the structure of a $\mathbb{N}$-graded vertex Poisson algebra.

Then in \cite{li_abelianizing_2005}, Li introduces a decreasing filtration $(F_pV)_{p\in \mathbb{Z}}$ on an arbitrary vertex algebra $V$, not necessarily $\mathbb{N}$-graded.
The associated graded space $\gr_F(V)$ with respect to this decreasing filtration again carries the structure of a vertex Poisson algebra.
Li also introduces a decreasing filtration $(F_pM)_{p\in \mathbb{Z}}$ for modules $M$ over a vertex algebra $V$ and shows that the associated graded space $\gr_F(M)$ is a module over the vertex Poisson algebra $\gr_F(V)$.

In summary, Li constructed three functors
\begin{align*}
  \gr^G: \{\text{$\mathbb{N}$-graded vertex algebras}\} &\to \{\text{$\mathbb{N}$-graded vertex Poisson algebras}\}, \\
  \gr_F: \{\text{vertex algebras}\} &\to \{\text{vertex Poisson algebras}\}, \\
  \gr_F: \{\text{$V$-modules}\} &\to \{\text{$\gr_F(V)$-modules}\}.
\end{align*}

Then Arakawa shows in \cite[Proposition 2.6.1]{arakawa_remark_2012} that when $V$ is $\mathbb{N}$-graded, $\gr_F(V)$ and $\gr^G(V)$ are isomorphic as vertex Poisson algebras.

In this article, we define an increasing filtration $(G^pM)_{p\in \mathbb{Z}}$ for $h + \mathbb{N}$-graded modules $M$ over a $\mathbb{N}$-graded conformal vertex algebra $V$ which wasn't done in \cite{li_vertex_2004}.
So we construct a functor
\begin{equation*}
  \gr^G: \{\text{$h + \mathbb{N}$-graded $V$-modules}\} \to \{\text{$h+\mathbb{N}$-graded $\gr^G(V)$-modules}\}.
\end{equation*}
Parallel to what was done \cite{arakawa_remark_2012}, we show that $\gr_F(M)$ and $\gr^G(M)$ are isomorphic as modules.
However, for our purposes, the filtration $(G^pM)_{p\in \mathbb{Z}}$ is better suited.

In \cite{andrews_singular_2022}, two theorems about the Virasoro minimal model $\Vir_{3, 4} = L(1/2, 0)$, also known as the Ising model, are proved

\begin{theorem}
  \label{thr:1}
  The refined character of $\gr^G(\Vir_{3,4})$ is given by
  \begin{equation*}
    \ch_{\gr^G(\Vir_{3, 4})}(t, q) = \sum_{k_1, k_2\in \mathbb{N}}t^{4k_1 + 2k_2}\frac{q^{4k_1^2 + 3k_1k_2 + k_2^2}}{(q)_{k_1}(q)_{k_2}}(1 - q^{k_1} + q^{k_1 + k_2}).
  \end{equation*}
\end{theorem}

Let $Q_0$ be the following set of partitions
\begin{align*}
  &[q, q, q], [q + 1, q, q], [q + 1, q + 1, q], [q + 2, q + 1, q], [q + 2, q + 2, q], &(q \ge 2) \\
  &[q + 2, q, q], &(q \ge 3) \\
  &[q + 3, q + 3, q, q], [q + 4, q + 3, q, q],  [q + 4, q + 3, q + 1, q], [q + 4, q + 4, q + 1, q], &(q \ge 2) \\
  &[q + 6, q + 5, q + 3, q + 1, q], &(q \ge 2) \\
  &[5, 4, 2, 2], [7, 6, 4, 2, 2], [7, 7, 4, 2, 2], [9, 8, 6, 4, 2, 2].
\end{align*}
Let $P_0$ be the set of partitions $\lambda = [\lambda_1, \dots, \lambda_m]$ that don't contain any partition in $Q_0$ with $\lambda_m \ge 2$.

\begin{theorem}
  \label{thr:2}
  The set
  \begin{equation*}
    \{L_{-\lambda_1}L_{-\lambda_2}\dots L_{-\lambda_m}\vac \mid \lambda = [\lambda_1, \dots, \lambda_m] \in P_0\}
  \end{equation*}
  is a vector space basis of $\Vir_{3, 4}$.
\end{theorem}

In this article, we prove two similar results for $L(1/2,1/2)$.

\begin{theorem}
  \label{thr:3}
  The refined character of $\gr^G(L(1/2, 1/2))$ is given by
  \begin{equation*}
    \ch_{\gr^G(L(1/2, 1/2))}(t, q) = q^{1/2}\left( \sum_{k_1, k_2 \in \mathbb{N}}t^{4k_1 + 2k_2}\frac{q^{4k_1^2 + 3k_1k_2 + k_2^2}}{(q)_{k_1}(q)_{k_2}}(q^{3k_1 + 2k_2} + q^{5k_1 + 2k_2 + 1}t + q^{6k_1 + 3k_2 + 2}t^2) \right)
  \end{equation*}
\end{theorem}

Let $Q$ be the following set of partitions.
\begin{align}
  \nonumber
  &[q, q, q], [q + 1, q, q], [q + 1, q + 1, q], [q + 2, q + 1, q], [q + 2, q + 2, q], &(q \ge 3) \\
  \nonumber
  &[q + 2, q, q], &(q \ge 3) \\
  \label{eq:1}
  &[q + 3, q + 3, q, q], [q + 4, q + 3, q, q],  [q + 4, q + 3, q + 1, q], [q + 4, q + 4, q + 1, q], &(q \ge 3)\\
  \nonumber
  &[q + 6, q + 5, q + 3, q + 1, q], &(q \ge 3) \\
  \nonumber
  &[2], [1, 1, 1], [3, 1, 1], [3, 3], [4, 3, 1], [4, 4, 1], [5, 4, 1, 1], [6, 5, 3, 1].
\end{align}
Let $P$ be the set of partitions that don't contain any partition in $Q$.

\begin{theorem}
  \label{thr:4}
  The set
  \begin{equation*}
    \{L_{-\lambda_1}L_{-\lambda_2}\dots L_{-\lambda_m}\vachalf \mid \lambda=[\lambda_1, \dots, \lambda_m] \in P\}
  \end{equation*}
  is a vector space basis of $L(1/2, 1/2)$.
\end{theorem}

Specializing the character formula in \Cref{thr:3} to $t = 1$, we obtain that the character of $L(1/2, 1/2)$ is the sum of three Nahm sums for the same matrix $\left( \begin{smallmatrix} 8 & 3 \\ 3 & 2\end{smallmatrix} \right)$ (cf.\ \cite{Nahm2007}).

We can do a similar argument for $L(1/2, 1/16)$ and obtain the following theorem

\begin{theorem}
  \label{thr:5}
  SERIES AND BASIS OF $L(1/2,1/16)$
\end{theorem}

Finally, we show the SageMath \cite{sagemath} program used to compute the exceptional partitions appearing in \eqref{eq:1}.
The software systems Mathematica \cite{Mathematica} and Singular \cite{Singular} were also very useful to verify the series and compute Gr\"{o}bner bases.

I would like to thank my advisor Reimundo Heluani and Instituto de Matematica Pura e aplicada (IMPA) for their support.
Professor Jethro Van Ekeren made some valuable suggestions as well.

\section{Preliminaries and notation}
\label{sec:prel-notat}
All vector spaces are over $\mathbb{C}$, the field of complex numbers.
The set of natural numbers $\{0, 1, \dots\}$ is denoted by $\mathbb{N}$, the set of integers is denoted by $\mathbb{Z}$ and the set of positive integers $\{1, 2, \dots\}$ is denoted by $\mathbb{Z}_+$.

First, we review some parts of the theory of representations of the Virasoro Lie algebra following \cite{kac_bombay_2013}.
The \textbf{Virasoro Lie algebra} is a Lie algebra given by
\begin{equation*}
  \Vir = \bigoplus_{n \in \mathbb{Z}}\mathbb{C}L_n \oplus \mathbb{C}C.
\end{equation*}
These elements satisfy the following commutation relations
\begin{align*}
  [L_m, L_n] &= (m - n)L_{m + n} + \delta_{m, -n}\frac{m^3 -m}{12}C, \\
  [\Vir, C] &= 0, 
\end{align*}
for all $m$, $n \in \mathbb{Z}$.
Let $c, h \in \mathbb{C}$.
Set $\Vir_{\ge 0, C} = \bigoplus_{n \in \mathbb{N}}\mathbb{C}L_n \oplus \mathbb{C}C$.
Then $\Vir_{\ge 0, C}$ is a Lie subalgebra of $\Vir$ and we let it act on $\mathbb{C}$ as follows
\begin{align*}
  L_n1 &= 0, (n \in \mathbb{Z}_+) \\
  L_01 &= h, \\
  C1 &= c.
\end{align*}
The \textbf{Verma representation} of $\Vir$ with \textbf{highest weight} $(c, h)$ is defined as 
\begin{equation*}
  M(c, h) = \Ind^{\Vir}_{\Vir_{\ge 0, C}}(\mathbb{C}) = U(\Vir) \otimes_{U(\Vir_{\ge 0, C})} \mathbb{C},
\end{equation*}
where $\Vir$ acts by left multiplication.
Set $|c, h\rangle = 1 \otimes 1$.

A \textbf{partition} (of $n \in \mathbb{N}$) is a sequence $\lambda=[\lambda_1, \dots, \lambda_m]$ such that $\lambda_i \in \mathbb{Z}_+$ for $i=1, \dots, m$, $\lambda_1 \ge \dots \ge \lambda_m$ (and $\lambda_1 + \dots + \lambda_m = n$).
We also consider the \textbf{empty partition} $\emptyset$ which is the unique partition of $0$.
For a partition $\lambda = [\lambda_1, \dots, \lambda_m]$, define $L_{\lambda} = L_{-\lambda_1}\dots L_{-\lambda_m} \in U(\Vir)$.
By the PBW theorem, the set
\begin{equation*}
  \{L_\lambda|c,h\rangle \mid \lambda\text{ is a partition}\}
\end{equation*}
is a vector space basis of $M(c, h)$.
The representation $M(c, h)$ has a unique maximal subrepresentation $I(c, h)$ and the quotient
\begin{equation*}
  L(c, h)=M(c, h)/I(c, h)
\end{equation*}
is the \textbf{irreducible highest weight representation} of $\Vir$ with highest weight $(c, h)$.
We usually simplify $|c, h\rangle$ to just $|h\rangle$ when $c$ is understood.
In \cite{andrews_singular_2022} $\Vir_{3, 4}=L(1/2, 0)$ was studied.
In this article we study $L(1/2, 1/2)$ and $L(1/2, 1/16)$.

We do a short review of filtrations of vertex algebras and their modules.
The books \cite{kac_vertex_1998}, \cite{frenkel_vertex_2001} and \cite{lepowsky_introduction_2004} explain vertex algebras and their modules.
For details about vertex Poisson algebras and filtrations of vertex algebras see \cite{li_vertex_2004}, \cite{li_abelianizing_2005} and \cite{arakawa_remark_2012}.

Let $V$ be a vertex algebra and $(a^i)_{i \in I}$ a family of strong generators of $V$.
For $p \in \mathbb{Z}$, set
\begin{equation*}
  F_pV = \vspan\{a^{i_1}_{(-n_1 - 1)}\dots a^{i_s}_{(-n_s - 1)}\vac \mid s, n_1, \dots, n_s \in \mathbb{N}, i_1, \dots, i_s \in I, n_1 + \dots + n_s \ge p\}.
\end{equation*}

\begin{proposition}
  \label{prp:1}
  The filtration $(F_pV)_{p\in \mathbb{Z}}$ satisfies
  \begin{enumerate}[label={(\alph*)}]
  \item $F_pV = V$ for $p \le 0$,
  \item $F_0V \supseteq F_1V \supseteq \dots$,
  \item $TF_pV \subseteq F_{p+1}V$,
  \item $a_{(n)}F_qV \subseteq F_{p + q - n - 1}V$ for $a \in F_pV$, $n \in \mathbb{Z}$,
  \item $a_{(n)}F_qV \subseteq F_{p + q - n}V$ for $a \in F_pV$, $n \in \mathbb{N}$.
  \end{enumerate}
\end{proposition}

Let
\begin{equation*}
  \gr_F(V) = \bigoplus_{p\in \mathbb{N}}F_pV/F_{p + 1}V
\end{equation*}
be the associated graded vector space.
The vector space $\gr_F(V)$ is a vertex Poisson algebra with operations given as follows
\begin{align*}
  \sigma_p(a)\sigma_q(b) &= \sigma_{p + q}(a_{(-1)}b), \\
  T\sigma_p(a) &= \sigma_{p + 1}(Ta), \\
  Y_-(\sigma_p(a),z)\sigma_q(b) &= \sum_{n \in \mathbb{N}}\sigma_{p + q - n}(a_{(n)}b)z^{-n - 1},
\end{align*}
where $\sigma_p: F_pV \to \gr_F(V)$ is the \textbf{principal symbol map} which is the composition of the natural maps $F_pV \to F_pV/F_{p + 1}V$ and $F_pV/F_{p+1}V \to \gr_F(V)$.
The filtration $(F_pV)_{p\in \mathbb{Z}}$ is called the \textbf{Li filtration} of $V$.

Assume that $V$ is a $\mathbb{N}$-graded vertex algebra, meaning it is endowed with a Hamiltonian $H \in \End(V)$ whose eigenvalues are natural numbers.
We now define an increasing filtration $(G^pV)_{p \in \mathbb{Z}}$ called the \textbf{standard filtration}.
Let $(a^i)_{i \in I}$ be a family of homogeneous strong generators of $V$.
For $p \in \mathbb{Z}$ set
\begin{equation*}
  G^pV = \vspan\{a^{i_1}_{(-n_1 - 1)}\dots a^{i_s}_{(-n_s - 1)}\vac \mid s, n_1, \dots, n_s \in \mathbb{N}, i_1, \dots, i_s \in I, \Delta_{a^{i_1}} + \dots + \Delta_{a^{i_s}} \le p\},
\end{equation*}
where $\Delta$ denotes conformal weight.

\begin{proposition}
  \label{prp:2}
  The filtration $(G^pV)_{p\in \mathbb{Z}}$ satisfies
  \begin{enumerate}[label={(\alph*)}]
  \item $G^pV = 0$ for $p < 0$,
  \item $G^0V \subseteq G^1V \subseteq \dots$,
  \item $V = \bigcup_pG^pV$,
  \item $TG^pV \subseteq G^pV$,
  \item $a_{(n)}G^qV \subseteq G^{p + q}V$, for $a \in G^pV$, $n \in \mathbb{Z}$,
  \item $a_{(n)}G^qV \subseteq G^{p + q - 1}V$, for $a \in G^pV$, $n \in \mathbb{N}$.
  \end{enumerate}
\end{proposition}

Let
\begin{equation*}
  \gr^G(V)=\bigoplus_{p\in\mathbb{N}}G^pV/G^{p-1}V
\end{equation*}
be the associated graded vector space.
The vector space $\gr^G(V)$ a vertex Poisson algebra with operations given as follows
\begin{align*}
  \alpha^p(a)\alpha^q(b) &= \alpha^{p + q}(a_{(-1)}b), \\
  T\alpha^p(a) &= \alpha^p(Ta), \\
  Y_-(\alpha^p(a), z)\alpha^q(b) &= \sum_{n \in \mathbb{N}}\alpha^{p + q - 1}(a_{(n)}b)z^{-n - 1},
\end{align*}
where $\alpha^p: G^pV \to \gr^G(V)$ is the principal symbol map.

\begin{proposition}[{\cite[Proposition 2.6.1]{arakawa_remark_2012}}]
  \label{prp:3}
  We have
  \begin{equation*}
    F_pV_n = G^{n - p}V_n
  \end{equation*}
  for all $p, n \in \mathbb{N}$.
  Moreover, $\gr_F(V)$ and $\gr^G(V)$ are isomorphic as vertex Poisson algebras.
\end{proposition}

Let $V$ be a vertex algebra, $(a^i)_{i \in I}$ a family of strong generators of $V$ and $M$ a $V$-module.
For $p\in \mathbb{Z}$, we define the following family of subspaces of $M$
\begin{equation*}
  F_pM = \vspan \{a^{i_1M}_{(-n_1 - 1)}\dots a^{i_sM}_{(-n_s - 1)}m \mid s, n_1, \dots, n_s \in \mathbb{N}, i_1, \dots, i_s \in I, m \in M, n_1 + \dots + n_s \ge p\}.
\end{equation*}

\begin{proposition}
  \label{prp:4}
  The filtration $(F_pM)_{p \in \mathbb{Z}}$ satisfies
  \begin{enumerate}[label={(\alph*)}]
  \item $M = F_pM$ for $p \le 0$,
  \item $F_0M \supseteq F_1M \supseteq \dots$,
  \item $a_{(n)}F_qM \subseteq F_{p + q - n - 1}M$ for $a \in F_pV$, $n \in \mathbb{Z}$,
  \item $a_{(n)}F_qM \subseteq F_{p + q - n}M$ for $a \in F_pV$, $n \in \mathbb{N}$.
  \end{enumerate}
\end{proposition}

Let
\begin{equation*}
  \gr_F(M) = \bigoplus_{p\in \mathbb{N}}F_pM/F_{p + 1}M
\end{equation*}
be the associated graded vector space.
The vector space $\gr_F(M)$ is a module over the vertex Poisson algebra $\gr_F(V)$ by setting
\begin{align*}
  \sigma_p(a)\sigma^M_q(m) &= \sigma^M_{p + q}(a^M_{(-1)}b), \\
  Y_-(\sigma_p(a), z)\sigma^M_q(m) &= \sum_{n \in \mathbb{N}}\sigma^M_{p + q - n}(a^M_{(n)}m)z^{-n - 1},
\end{align*}
where $\sigma^M_p: F_pM \to \gr_F(M)$ is the principal symbol map.

Note that if $V$ were conformal with conformal vector $\omega$ then we don't always have the property $L_{-1}^MF_pM \subseteq F_{p + 1}M$ for $p \in \mathbb{Z}$ where $Y(\omega, z)=\sum_{n \in \mathbb{Z}}L^M_nz^{-n - 2}$.
For example for $V = \Vir^{1/2}$, $M = M(1/2, 1/2)$ and $\vachalf \in F_0M$ we have $L_{-1}\vachalf \notin F_1M$.

\begin{remark}
  \label{rmk:5}
  The observation above makes \cite[Lemma 3.1.2]{arakawa_remark_2012} incorrect because the expression $\sigma_{p - 1}(\omega_{(0)}m)$ that is written there (that should be $\sigma_{p + 1}(\omega_{(0)}m)$ but it still doesn't work) is not well defined.
  I couldn't fix this problem and I don't think it is possible to define a differential on $\gr_F(M)$ in a meaningful way.
  We won't need that differential in this article though.
\end{remark}

Now we introduce a definition not given by Li in \cite{li_vertex_2004}.
Let $V$ be a $\mathbb{N}$-graded conformal vertex algebra with conformal vector $\omega$, $(a^i)_{i\in I}$ a family of homogeneous strong generators of $V$ and let $M$ be a $h+\mathbb{N}$-graded $V$-module which means $M$ is a $V$-module with $L_0^M$ diagonalizable whose eigenvalues are in the set $h + \mathbb{N}$ for some $h \in \mathbb{C}$.
Set $M_\Delta = \ker(L^M_0 - \Delta\Id_M)$ for $\Delta \in \mathbb{C}$ so we have $M = \bigoplus_{n \in \mathbb{N}}M_{h + n}$.

The \textbf{standard filtration} $(G^pM)_{p \in \mathbb{Z}}$ of $M$ is defined as
\begin{equation*}
  \begin{split}
    G^pM = \vspan\{a^{i_1M}_{(-n_1 - 1)}\dots a^{i_sM}_{(-n_s - 1)}m \mid& s, n_1, \dots, n_s \in \mathbb{N}, i_1, \dots, i_s \in I, m \in M_{\Delta_m} \text{ and }\\
    &\Delta_{a^{i_1}} + \dots + \Delta_{a^{i_s}} + \Delta_m - h \le p\},
  \end{split}
\end{equation*}
where $\Delta$ denotes conformal weight.

\begin{proposition}
  \label{prp:5}
  The filtration $(G^pM)_{p\in \mathbb{Z}}$ satisfies
  \begin{enumerate}[label={(\alph*)}]
  \item $G^pM = 0$ for $p < 0$,
  \item $G^0M \subseteq G^1M \subseteq \dots$,
  \item $M = \bigcup_pG^pM$,
  \item $a^M_{(n)}G^qM \subseteq G^{p+q}M$ for $a \in G^pV$ and $n \in \mathbb{Z}$,
  \item $a^M_{(n)}G^qM \subseteq G^{p+q-1}M$ for $a \in G^pV$ and $n \in \mathbb{N}$,
  \item $L^M_0G^pM \subseteq G^pM$ and $L^M_{-1}G^pM \subseteq G^{p + 1}M$.
  \end{enumerate}
\end{proposition}

Note that we don't have the property $L_{-1}^MG^pM \subseteq G^pM$ as in the case of vertex algebras.
Let
\begin{equation*}
  \gr^G(M) = \bigoplus_{p \in \mathbb{N}}G^pM/G^{p - 1}M
\end{equation*}
be the associated graded vector space.
Then $\gr^G(M)$ is a module over the vertex Poisson algebra $\gr^G(V)$.
Call $\alpha^p_M: G^pM \to \gr^G(M)$ the principal symbol map for $p \in \mathbb{N}$.
The operations that make $\gr^G(M)$ into a module are defined by
\begin{align*}
  \alpha^p(a)\alpha^q_M(m) &= \alpha^{p+q}_M(a^M_{(-1)}m), \\
  Y^M_{-}(\alpha^p(a), z)\alpha^q_M(m) &= \sum_{n \in \mathbb{N}}\alpha^{p + q - 1}_M(a^M_{(n)}m)z^{-n - 1},
\end{align*}
where $p$, $q \in \mathbb{N}$, $a \in G^pV$ and $m \in G^qM$.
These are well defined by (d) and (e) of \Cref{prp:5}.

By (f) of \Cref{prp:5}, we can define a diagonalizable operator $H^M \in \End(\gr^G(M))$ by setting $H^M(\alpha^p_M(m)) = \alpha^p_M(L^M_0m)$ for $p \in \mathbb{Z}$, $m \in G^pM$.

For $p$, $n \in \mathbb{N}$, set $G^pM_{h + n} = G^pM \cap M_{h + n}$, then
\begin{equation}
  \label{eq:2}
  \gr^G(M)=\bigoplus_{n\in \mathbb{N}}\gr^G(M)_{h+n}
\end{equation}
where $\gr^G(M)_{h + n} = \bigoplus_{p \in \mathbb{N}}\alpha^p_M(G^pM_{h + n})$ and $\gr^G(M)_{h + n} = \ker(H^M - (h + n)\Id_{\gr^G(M)})$.
Besides gradation \eqref{eq:2}, we have the refined gradation
\begin{equation}
  \label{eq:3}
  \gr^G(M) = \bigoplus_{p, n \in \mathbb{N}}\alpha^p_M(G^pM_{h + n}).
\end{equation}
Assume each subspace $M_{h + n}$ is finite dimensional.
Under these hypotheses, $M$ is sometimes called an \textbf{ordinary} module of $V$ or  a \textbf{positive energy representation} of $V$ with \textbf{top degree component} $M_h$.
Define the \textbf{character} and the \textbf{refined character} of $\gr^G(M)$ (with respect to gradation \eqref{eq:2} and gradation \eqref{eq:3}) as
\begin{align*}
  \ch_{\gr^G(M)}(q) &= \sum_{n \in \mathbb{N}}\dim(\gr^G(M)_{h + n})q^{h + n} \in q^h\mathbb{C}[[q]],\\
  \ch_{\gr^G(M)}(t, q) &= \sum_{p, n \in \mathbb{N}}\dim(\alpha^p_M(G^pM_{h + n}))t^pq^{h + n} \in q^{h}\mathbb{C}[[t, q]].
\end{align*}
By property (c) of \Cref{prp:5}, we have
\begin{equation*}
  \ch_M(q) = \ch_{\gr^G(M)}(q) = \ch_{\gr^G(M)}(1, q).
\end{equation*}

If $f: M_1 \to M_2$ is a homomorphism of $V$-modules, then
\begin{align*}
  \gr^G(f): \gr^G(M_1) &\to \gr^G(M_2) \\
  \alpha^p_{M_1}(m_1) &\mapsto \alpha^p_{M_2}(f(m_1))
\end{align*}
for $p\in \mathbb{N}$ and $m_1\in G^pM_1$, defines a homomorphism of $\gr^G(V)$-modules.
Therefore we obtain a functor
\begin{equation*}
  \gr^G: \{\text{$h + \mathbb{N}$-graded $V$-modules}\} \to \{\text{$h + \mathbb{N}$-graded $\gr^G(V)$-modules}\}.
\end{equation*}

Similarly to \Cref{prp:3}, the following Proposition justifies the definition of the filtration $(G^pM)_{p\in \mathbb{Z}}$ given in this article.

\begin{proposition}
  \label{prp:6}
  We have
  \begin{equation*}
    F_pM_{h + n} = G^{n - p}M_{h + n}
  \end{equation*}
  for all $p$, $n \in \mathbb{N}$.
  An explicit isomorphism $\gr_F(M) \xrightarrow{\sim} \gr^G(M)$ of modules is defined by extending linearly the isomorphisms of vector spaces given by
  \begin{align*}
    \sigma^M_p(F_pM_{h + n}) &\xrightarrow{\sim} \alpha^{n - p}_M(G^{n - p}M_{h + n}) \\
    \sigma^M_p(m) &\mapsto \alpha^{n - p}_M(m),
  \end{align*}
  where $p$, $n\in \mathbb{N}$ and $m \in F_pM_{h + n}$.
\end{proposition}

\begin{proof}
  We recall three facts
  \begin{enumerate}[label={(\alph*)}]
  \item $G^nM \supseteq M_{h + n}$ for $n \in \mathbb{N}$,
  \item $\Delta_{a^M_{(n)}m} = \Delta_a + \Delta_m - n - 1$ for homogeneous $a \in V$, $m \in M$ and $n \in \mathbb{Z}$,
  \item $F_pM_{h + n} = \vspan\{a^{i_1M}_{(-n_1 - 1)}m \mid i_1 \in I, n_1 \in \mathbb{Z}_+\text{ and } m \in F_{p - n_1}M_{\Delta_m}\text{ with }\Delta_{a^{i_1}} + \Delta_m + n_1 = h + n\}$ for any $p \in \mathbb{Z}_+$ and $n \in \mathbb{N}$.
  \end{enumerate}
  First, we prove the inclusion $F_pM_{h + n} \subseteq G^{n - p}M_{h + n}$ for $p$, $n \in \mathbb{N}$.
  We do this by induction on $p \in \mathbb{N}$.
  The base case $p = 0$ is true by property (a) above.
  Now assume $p \ge 1$ and $F_qM_{h + n} \subseteq G^{n - q}M_{h + n}$ for $q < p$ and all $n \in \mathbb{N}$.
  Pick an element $a^{i_1M}_{(-n_1 - 1)}m$ from the spanning set of $F_pM_{h + n}$ in (c) above, with $i_1 \in I$, $n_1 \in \mathbb{Z}_+$ and $m \in F_{p - n_1}M_{\Delta_m}$.
  We know that $m \in G^{\Delta_m - h - p + n_1}M_{\Delta_m}$ by the induction hypothesis and also $a \in V_{\Delta_a^{i_1}} \subseteq G^{\Delta_{a^{i_1}}}V$.
  Therefore by property (d) of the filtration $(G^pM)_{p \in \mathbb{Z}}$, $a^{i_1M}_{(-n_1 - 1)}m \in G^{\Delta_{a^{i_1}} + \Delta_m - h - p + n_1}M = G^{n - p}M$.

  Now we prove the inclusion $G^{n - p}M_{h + n} \subseteq F_pM_{h + n}$.
  Pick an element $a^{i_1M}_{(-n_1 - 1)}\dots a^{i_sM}_{(-n_s - 1)}m$ from the spanning set of $G^{n - p}M_{h + n}$ where $m \in M_{\Delta_m}, i_j \in I, s, n_1, \dots, n_s\in \mathbb{N}\text{ and }\Delta_{a^{i_1}} + \dots + \Delta_{a^{i_s}} + \Delta_m - h \le n - p$ and we must have $\Delta_{a^{i_1}} + \dots + \Delta_{a^{i_s}} + \Delta_m + n_1 + \dots + n_s = h + n$ by property (b) above.
  Therefore $p \le n_1 + \dots + n_s$, so we get $a^{i_1M}_{(-n_1 - 1)}\dots a^{i_sM}_{(-n_s - 1)}m \in F_pM$ straight from the definition of $F_pM$.

  We verify that we obtain an isomorphism $\gr_F(M) \xrightarrow{\sim} \gr^G(M)$ directly from the definitions.
\end{proof}

In this article we always take $V = \Vir^{1/2}$ as the $\mathbb{N}$-graded vertex algebra with conformal vector $\omega = L_{-2}\vac$.
From now on, some indexes or subindexes of modules will not be written, so for example $\alpha^p_M$ simplifies to $\alpha^p$.
Set $M = M(1/1, 1/2)$ and $L = L(1/2, 1/2)$.
The representations $M$ and $L$ can be considered as modules over $V$ because they are smooth representations of $\Vir$ with central charge $1/2$.
Set $W = I(1/2, 1/2)$.
By definition we have $L = M/W$.
It can be shown (cf.\ \cite{feigin_verma_1984} and \cite{astashkevich_structure_1997}) that
\begin{equation}
  \label{eq:4}
  W = U(\Vir)\{m_2, m_3\},
\end{equation}
where
\begin{align*}
  m_2 &= \left(L_{-1}^2 - \tfrac{4}{3}L_{-2}\right)\vachalf, \\
  m_3 &= \left(L_{-1}^3 - 3L_{-2}L_{-1} + \tfrac{3}{4}L_{-3}\right)\vachalf.
\end{align*}
are the generating singular vectors.

We have a natural surjective homomorphism of representations of $\Vir$:
\begin{align*}
  \pi: M &\to L \\
  m &\mapsto m + W.
\end{align*}
We can consider $\pi$ as a homomorphism of modules over $\Vir^{1/2}$.
Applying the functor $\gr^G$ we obtain a surjective homomorphism of modules over $\gr^G(V)$:
\begin{equation*}
  \gr^G(\pi): \gr^G(M) \to \gr^G(L),
\end{equation*}
and this produces a natural isomorphism of modules over $\gr^G(V)$:
\begin{equation*}
  \gr^G(M)/K \xrightarrow{\sim} \gr^G(L)
\end{equation*}
where
\begin{equation}
  \label{eq:5}
  K = \ker(\gr^G(\pi)).
\end{equation}

Let us describe these objects and the isomorphisms explicitly.
For $V$, $\gr^G(V)$ is isomorphic to $\mathbb{C}[L_{-2}, L_{-3}, \dots]$, the polynomial algebra with infinitely many variables $L_{-2}, L_{-3}, \dots$ and coefficients in $\mathbb{C}$ because the standard filtration $(G^pV)$ is essentially the ``doubled'' PBW filtration $U_\bullet(\Vir_{\le -2})$ (cf.\ \cite{arakawa_remark_2012}).
A similar argument can be made for $\gr^G(M)$, obtaining that $\gr^G(M)$ is isomorphic to the free $\gr^G(V)$-module $\bigoplus_{k \in \mathbb{N}}\mathbb{C}[L_{-2}, L_{-3}, \dots]L_{-1}^k$.
The isomorphism is given by
\begin{align*}
  \bigoplus_{k \in \mathbb{N}}\mathbb{C}[L_{-2}, L_{-3}, \dots]L_{-1}^k &\xrightarrow{\sim} \gr^G(M) \\
  L_{-n_1 - 2}\dots L_{-n_s - 2}L_{-1}^k &\mapsto \alpha^{2s + k}(L_{-n_1 - 2}^M\dots L_{-n_s - 2}^M(L_{-1}^M)^k\vachalf),
\end{align*}
5where $s$, $k$, $n_1,\dots, n_s \in \mathbb{N}$.
\begin{remark}
  \label{rmk:6}
  That $+k$ in the isomorphism above is what makes this filtration different from the PBW filtration where all $L_n$ for $n \le -2$ have the same length equal to $2$.
  However, with the standard filtration $L_{-1}$ has length equal to $1$ while $L_{-2}$, $L_{-3}, \dots$ have length equal to $2$.
\end{remark}

These observations justify the following definitions.
For a partition $\lambda = [\lambda_1, \dots, \lambda_m, 1, \dots, 1]$ with exactly $n$ ones, define $m_\lambda \in \bigoplus_{k \in \mathbb{N}}\mathbb{C}[L_{-2}, L_{-3}, \dots]L_{-1}^k$ as $L_{-\lambda_1}\dots L_{-\lambda_m}L_{-1}^n$ where $L_{-1}^n$ is the $n$-th element of the canonical basis.
Define the \textbf{length} of $\lambda$ as
\begin{equation*}
  \len(\lambda) = 2m + n
\end{equation*}
and the \textbf{weight} of $\lambda$ as
\begin{equation*}
  \Delta(\lambda) = \lambda_1 + \dots + \lambda_m + n.
\end{equation*}
Therefore, for a partition $\lambda$ we have
\begin{equation*}
  L_{\lambda}\vachalf \in G^{\len(\lambda)}M_{1/2 + \Delta(\lambda)}.
\end{equation*}

From the surjective homomorphism of $V$-modules $\pi: M \to L$, we deduce the following four isomorphisms which are going to be used frequently implicitly.

\begin{proposition}
  \label{prp:8}
  We have four (conformal) weight-preserving vector space isomorphisms
  \begin{align*}
    M &\xrightarrow{\sim} \gr^G(M) \\
    L_\lambda\vachalf &\mapsto \alpha^{\len(\lambda)}(L_\lambda\vachalf) \\
    \gr^G(M) &\xrightarrow{\sim} \bigoplus_{k \in \mathbb{N}}\mathbb{C}[L_{-2}, L_{-3}, \dots]L_{-1}^k \\
    \alpha^{\len(\lambda)}(L_\lambda\vachalf) &\mapsto m_\lambda \\
    L &\xrightarrow{\sim} \gr^G(L) \\
    L_\lambda(\vachalf + W) &\mapsto \alpha^{\len(\lambda)}(L_\lambda(\vachalf + W)) \\
    \gr^G(L) &\xrightarrow{\sim} \gr^G(M)/K \\
    \alpha^{\len(\lambda)}(L_\lambda(\vachalf + W)) &\mapsto \alpha^{\len(\lambda)}(L_\lambda\vachalf) + K,
  \end{align*}
  where $\lambda$ is a partition.
\end{proposition}

In this article, we will deal with polynomial rings written as $\mathbb{C}[L_{-2}, L_{-3}, \dots, L_{-N}]$ and free modules of the form $\bigoplus_{n \le N}\mathbb{C}[L_{-2}, L_{-3}, \dots, L_{-N}]L_{-1}^n$ for some $N \in \mathbb{N}$.
We will always use the degree reverse lexicographical ordering with $L_{-2} > L_{-3} > \dots > L_{-N}$ and TOP (term over position) with $L_{-1}^0 < L_{-1}^1 < \dots < L_{-1}^N$.
We can extend the definition of $\lm$ as follows: for any element $a \in \bigoplus_{k \in \mathbb{N}}\mathbb{C}[L_{-2}, L_{-3}, \dots]L_{-1}^k$, pick $N$ large enough so that $a \in \bigoplus_{n \le N}\mathbb{C}[L_{-2}, L_{-3}, \dots, L_{-N}]L_{-1}^n$.
Then define $\lm(a)$ as the leading monomial in $\bigoplus_{n \le N}\mathbb{C}[L_{-2}, L_{-3}, \dots, L_{-N}]L_{-1}^n$ which is naturally a subset of $\bigoplus_{k \in \mathbb{N}}\mathbb{C}[L_{-2}, L_{-3}, \dots]L_{-1}^k$.
This doesn't depend on the choice of $N$.
For example, $\lm(L_{-2} - \frac{3}{4}L_{-1}^2) = L_{-2}$.

When working with $q$-series, the following notation is useful.
The $q$-Pochhammer symbol is $(q)_n = \prod_{j = 1}^n(1 - q^j) \in \mathbb{C}[[q]]$ for $n \in \mathbb{N}$.
The recursive proof of the series identities in \cite{andrews_singular_2022} suggest the following definition: for $a, b, c, d \in \mathbb{N}$, define
\begin{equation*}
  f_{a, b, c, d}(t, q) = \sum_{k_1, k_2 \in \mathbb{N}}t^{4k_1 + 2k_2 + d}\frac{q^{4k_1^2 + 3k_1k_2 + k_2^2 + ak_1 + bk_2 + c}}{(q)_{k_1}(q)_{k_2}} \in \mathbb{C}[[t, q]]
\end{equation*}
The following lemma will be very useful to verify series identities

\begin{lemma}
  \label{lmm:1}
  The series $f_{a, b, c, d}(t, q)$ satisfies
  \begin{enumerate}[label={(\alph*)}]
  \item $t^mq^nf_{a, b, c, d}(t, q) = f_{a, b, c + n, d + p}(t, q)$ for $m$, $n \in \mathbb{N}$,
  \item $f_{a, b, c, d}(tq^n, q) = f_{a + 4n, b + 2n, c + dn, d}(t, q)$ for $n \in \frac{1}{2}\mathbb{N}$, $d \in 2\mathbb{N}$.
  \item $f_{a, b, c, d}(t, q) - f_{a + n, b, c, d}(t, q) = \sum_{k = 0}^{n - 1}f_{a + 8 + k, b + 3, a + c + 4 + k, d + 4}(t, q)$ for $n \in \mathbb{Z}_+$,
  \item $f_{a, b, c, d}(t, q) - f_{a, b + n, c, d}(t, q) = \sum_{k = 0}^{n - 1}f_{a + 3, b + 2 + k, b + c + 1 + k, d + 2}(t, q)$ for $n \in \mathbb{Z}_+$.
  \end{enumerate}
\end{lemma}

\begin{proof}
  \begin{enumerate}[label=(\alph*)]
  \item Clear.
  \item Clear.
  \item The key step is replacing $k_1$ by $k_1 + 1$ in the following computation
    \begin{align*}
      f_{a, b, c, d}(t, q) - f_{a + n, b, c, d}(t, q) &= \sum_{k_1, k_2 \in \mathbb{N}}t^{4k_1 + 2k_2 + d}\frac{q^{4k_1^2 + 3k_1k_2 + k_2^2 + ak_1 + bk_2 + c}}{(q)_{k_1}(q)_{k_2}}(1 - q^{nk_1}) \\
      &= \sum_{k_1, k_2 \in \mathbb{N}}t^{4k_1 + 2k_2 + d}\frac{q^{4k_1^2 + 3k_1k_2 + k_2^2 + ak_1 + bk_2 + c}}{(q)_{k_1}(q)_{k_2}}(1 - q^{k_1})\sum_{k = 0}^{n - 1}q^{kk_1} \\
      &= \sum_{k = 0}^{n - 1}\sum_{k_1, k_2 \in \mathbb{N}}t^{4k_1 + 2k_2 + d}\frac{q^{4k_1^2 + 3k_1k_2 + k_2^2 + (a + k)k_1 + bk_2 + c}}{(q)_{k_1}(q)_{k_2}}(1 - q^{k_1}) \\
      &= \sum_{k = 0}^{n - 1}\sum_{k_1, k_2 \in \mathbb{N}}t^{4k_1 + 2k_2 + d + 4}\frac{q^{4k_1^2 + 3k_1k_2 + k_2^2 + (a + 8 + k)k_1 + (b + 3)k_2 + a + c + 4 + k}}{(q)_{k_1}(q)_{k_2}} \\
      &= \sum_{k = 0}^{n - 1}f_{a + 8 + k, b + 3, a + c + 4 + k, d + 4}(t, q).
    \end{align*}
  \item Same trick as (c) but now with $k_2$.
  \end{enumerate}
\end{proof}

\section{A combinatorial argument}
\label{sec:comb-argum}

A partition $\lambda = [\lambda_1, \dots, \lambda_m]$ \textbf{contains} a partition $\eta = [\eta_1, \dots, \eta_n]$, written as $\eta \subseteq \lambda$, if $m \ge n$ and there exist $i \in \mathbb{Z}_+$ such that $1 \le i \le m - n + 1$ and $[\lambda_i, \lambda_{i + 1}, \dots, \lambda_{i + n - 1}] = \eta$.

Define
\begin{equation*}
  p(t, q) = \sum_{\lambda \in P}t^{\len(\lambda)}q^{\Delta(\lambda)} \in \mathbb{C}[[t, q]],
\end{equation*}
where $P$ is the set of partitions that don't contain any partition in $Q$ as defined in \eqref{eq:1}, i.e.\
\begin{equation*}
  P = \{\lambda \mid \text{For all }\eta\in Q, \lambda \nsupseteq \eta\}.
\end{equation*}
We call the last eight partitions of \eqref{eq:1} \textbf{exceptional partitions} and the others involving $q$ are called \textbf{ordinary partitions}.
For $n, m \in \mathbb{N}$, set
\begin{align*}
  P(n) &= \{\lambda \in P \mid \Delta(\lambda) = n\}, \\
  p(q) &= \sum_{n \in \mathbb{N}}|P(n)|q^n \in \mathbb{C}[[q]], \\
  P(n, m) &= \{\lambda \in P \mid \len(\lambda) = m\text{ and }\Delta(\lambda) = n\}.
\end{align*}
Therefore, we have
\begin{align*}
  p(t, q) &= \sum_{n, m \in \mathbb{N}}|P(n, m)|t^mq^n, \\
  p(1, q) &= p(q).
\end{align*}
We wish to find an expression for $p(t, q)$ as a sum of series $f_{a, b, c, d}(t, q)$ for some tuples $(a, b, c, d)$.

We now define subsets of $P$ which will help in finding an expression for $p(t, q)$.
Set
\begin{align*}
  P_{>2} &= \{[\lambda_1, \dots, \lambda_m] \in P \mid \lambda_m > 2\text{ or }\lambda = \emptyset\}, \\
  P_2 &= \{[\lambda_1, \dots, \lambda_m] \in P \mid \lambda_m = 2\}
\end{align*}
and both $p_{>2}(t, q)$ and $P_{>2}(n, m)$ are defined like $p(t, q)$ and $P(n, m)$ were defined.
We similarly define objects like $P_{>6, 5, 3}$, $P_{6, 5, 3}$, $p_{>6, 5, 3}(t, q)$, $p_{6, 5, 3}(t, q)$, $P_{>6, 5, 3}(n, m)$ and $P_{6, 5, 3}(n, m)$.
It turns $P$ is made out of the these ``smaller'' objects and we can find recurrence relations between them to find our desired formula for $p(t, q)$.

\begin{lemma}
  \label{lmm:2}
  The formal power series $p_{>2}(t, q)$ is given explicitly by
  \begin{equation*}
    p_{>2}(t, q) = f_{3, 2, 0, 0}(t, q).
  \end{equation*}
\end{lemma}

\begin{proof}
  Consider the disjoint union
  \begin{equation*}
    P_{>2} = P_{4, 3} \cup P_{6, 5, 3} \cup P_{>6, 5, 3} \cup P_{>5, 3} \cup P_{4, 4} \cup P_{5, 4} \cup P_{>5, 4} \cup P_{>4}
  \end{equation*}
  from which we get the formula
  \begin{equation*}
    p_{>2}(t, q) = p_{4, 3}(t, q) + p_{6, 5, 3}(t, q) + p_{>6, 5, 3}(t, q) + p_{>5, 3}(t, q) + p_{4, 4}(t, q) + p_{5, 4}(t, q) + p_{>5, 4}(t, q) + p_{>4}(t, q).
  \end{equation*}
  These subseries must satisfy the following recurrences with initial conditions
  \begin{align*}
    p_{>4}(t, q) &= p_{4, 4}(tq^{1/2}, q) + p_{5, 4}(tq^{1/2}, q) + p_{>5, 4}(tq^{1/2}, q) + p_{>4}(tq^{1/2}, q), &p_{>4}(0, 0) &= 1, \\
    p_{>5, 4}(t, q) &= p_{6, 5, 3}(tq^{1/2}, q) + p_{>6, 5, 3}(tq^{1/2}, q) + p_{>5, 3}(tq^{1/2}, q), &p_{>5, 4}(0, 0) &= 0, \\
    p_{5, 4}(t, q) &= p_{4, 3}(tq^{1/2}, q), &p_{5, 4}(0, 0) &= 0, \\
    p_{4, 4}(t, q) &= t^2q^{3}p_{>6, 5, 3}(tq^{2/2}, q) + t^2q^3p_{>5, 3}(tq^{2/2}, q), &p_{4, 4}(0, 0) &= 0, \\
    p_{>5, 3}(t, q) &= t^2q^3p_{>4}(tq^{1/2}, q), &p_{>5, 3}(0, 0) &= 0, \\
    p_{>6, 5, 3}(t, q) &= t^2q^3p_{>5, 4}(tq^{1/2}, q), &p_{>6, 5, 3}(0, 0) &= 0, \\
    p_{6, 5, 3}(t, q) &= t^2q^3p_{5, 4}(tq^{1/2}, q), &p_{6, 5, 3}(0, 0) &= 0, \\
    p_{4, 3}(t, q) &= t^2q^3p_{4, 4}(tq^{1/2}, q) + t^2q^2p_{>5, 4}(tq^{1/2}, q), &p_{4, 3}(0, 0) &= 0.
  \end{align*}
  The solution to these equations is unique if it exists and we can verify using \Cref{lmm:1} that
  \begin{align*}
    p_{>4}(t, q) &= f_{6, 4, 0, 0}(t, q), \\
    p_{>5, 4}(t, q) &= f_{9, 5, 4, 2}(t, q), \\
    p_{5, 4}(t, q) &= f_{13, 6, 9, 4}(t, q), \\
    p_{4, 4}(t, q) &= f_{12, 6, 8, 4}(t, q), \\
    p_{>5, 3}(t, q) &= f_{8, 5, 3, 2}(t, q), \\
    p_{>6, 5, 3}(t, q) &= f_{11, 6, 8, 4}(t, q), \\
    p_{6, 5, 3}(t, q) &= f_{15, 7, 14, 6}(t, q), \\
    p_{4, 3}(t, q) &= f_{11, 5, 7, 4}(t, q), 
  \end{align*}
  is a solution to these equations.
  We derive the formula $p_{>2}(t, q) = f_{3, 2, 0, 0}(t, q)$ again from \Cref{lmm:1}.
  
  The recurrences follow from the following eight bijections for $n$, $m \in \mathbb{N}$
  \begin{align*}
    P_{>4}(n, 2m) &\xrightarrow{\sim} P_{4, 4}(n - m, 2m) \cup P_{5, 4}(n - m, 2m) \cup P_{>5, 4}(n - m, 2m) \cup P_{>4}(n - m, 2m) \\
    \lambda &\mapsto
              \begin{cases}
                [\lambda_1 - 1, \dots, \lambda_{m - 2} - 1, 4, 4] &\text{if }[\lambda_{m - 1}, \lambda_m] = [5, 5] \\
                [\lambda_1 - 1, \dots, \lambda_{m - 2} - 1, 5, 4] &\text{if }[\lambda_{m - 1}, \lambda_m] = [6, 5] \\
                [\lambda_1 - 1, \dots, \lambda_{m - 1} - 1, 4] &\text{if }[\lambda_m] = [5]\text{ and }\lambda_{m - 1} > 6 \\
                [\lambda_1 - 1, \dots, \lambda_m - 1] &\text{if }\lambda_m > 5,
              \end{cases}
  \end{align*}
  which can be verified directly from the definition of $P$ as we now show.
  Note that the operation on $\lambda$ is always ``substract 1 from each entry of $\lambda$''.
  For example, a partition $[\lambda_1, \dots, \lambda_m] \in P_{>4}(n, m)$ satisfies exactly one of the following four conditions:
  \begin{equation*}
    [\lambda_{m - 1}, \lambda_m] = [5, 5], [\lambda_{m - 1}, \lambda_m] = [6, 5], [\lambda_m] = [5]\text{ and }\lambda_{m - 1} > 6, \lambda_m > 5.
  \end{equation*}
  Say $\lambda$ satisfies $[\lambda_{m - 1}, \lambda_m] = [5, 5]$.
  Then $[\lambda_1 - 1, \dots, \lambda_{m - 2} - 1, 4, 4]$ belongs to $P$ because it can't contain some ordinary partition as defined in \eqref{eq:1} because: it can't contain some ordinary partition (otherwise $\lambda$ would contain some ordinary partition) and we can verify case by case that it can't contain some exceptional partition.
  Actually, we have $[\lambda_1 - 1, \dots, \lambda_{m - 2} - 1, 4, 4] \in P_{4, 4}(n - m, 2m)$ as we easily see, so we get the first part of our bijection.
  The other parts follow a similar reasoning and an explicit inverse of this function can be written if desired.
  We also note that $[5] \mapsto [4]$ and $\emptyset \mapsto \emptyset$ in this bijection.
  Now we verify that this bijection implies our first recurrence.
  One key step is replacing $n$ by $n - m$ in the following computation:
  \begin{align*}
    &p_{4, 4}(tq^{1/2}, q) + p_{5, 4}(tq^{1/2}, q) + p_{>5, 4}(tq^{1/2}, q) + p_{>4}(tq^{1/2}, q) \\
    &= \sum_{n, m\in \mathbb{N}}(|P_{4, 4}(n, m)| + |P_{5, 4}(n, m)| + |P_{>5, 4}(n, m)| + |P_{>4}(n, m)|)(tq^{1/2})^mq^n \\
    &= \sum_{n, m\in \mathbb{N}}(|P_{4, 4}(n, 2m)| + |P_{5, 4}(n, 2m)| + |P_{>5, 4}(n, 2m)| + |P_{>4}(n, 2m)|)(tq^{1/2})^{2m}q^n \\
    &= \sum_{n, m\in \mathbb{N}}(|P_{4, 4}(n - m, 2m)| + |P_{5, 4}(n - m, 2m)| + |P_{>5, 4}(n - m, 2m)| + |P_{>4}(n - m, 2m)|)t^{2m}q^n \\
    &= \sum_{n, m\in \mathbb{N}}|P_{>4}(n, 2m)|t^{2m}q^n \\
    &= p_{>4}(t, q).
  \end{align*}
  
  In the next bijections, similar observations are done, although in some cases the operation on $\lambda$ is ``substract 2 from each entry of $\lambda$'' and sometimes we need to change the tail of the result after doing this operation to get something in $P$.
  The next two lemmas also employ very similar arguments.
  \begin{align*}
    P_{>5, 4}(n, 2m) &\xrightarrow{\sim} P_{6, 5, 3}(n - m, 2m) \cup P_{>6, 5, 3}(n - m, 2m) \cup P_{>5, 3}(n - m, 2m) \\
    \lambda &\mapsto
              \begin{cases}
                [\lambda_1 - 1, \dots, \lambda_{m - 3} - 1, 6, 5, 3] &\text{if }[\lambda_{m - 2}, \lambda_{m - 1}, \lambda_m] = [7, 6, 4] \\
                [\lambda_1 - 1, \dots, \lambda_{m - 2} - 1, 5, 3] &\text{if }[\lambda_{m - 1}, \lambda_m] = [6, 4]\text{ and }\lambda_{m - 2} >7 \\
                [\lambda_1 - 1, \dots, \lambda_{m - 1} - 1, 3] &\text{if }[\lambda_m] = [4]\text{ and }\lambda_{m - 1} > 6
              \end{cases}
  \end{align*}
  \begin{align*}
    P_{5, 4}(n, 2m) &\xrightarrow{\sim} P_{4, 3}(n - m, 2m) \\
    \lambda &\mapsto [\lambda_1 - 1, \dots, \lambda_{m - 2} - 1, 4, 3]
  \end{align*}
  \begin{align*}
    P_{4, 4}(n, 2m) &\xrightarrow{\sim} P_{>6, 5, 3}(n - 2m - 1, 2(m - 1)) \cup P_{>5, 3}(n - 2m - 1, 2(m - 1)) \\
    \lambda &\mapsto
              \begin{cases}
                [\lambda_1 - 2, \dots, \lambda_{m - 3} - 2, 5, 3] &\text{if }[\lambda_{m - 2}, \lambda_{m - 1}, \lambda_m] = [7, 4, 4] \\
                [\lambda_1 - 2, \dots, \lambda_{m - 2} - 2, 3] &\text{if }[\lambda_{m - 1}, \lambda_m] = [4, 4]\text{ and }\lambda_{m - 2} > 7
              \end{cases}
  \end{align*}
  \begin{align*}
    P_{>5, 3}(n, 2m) &\xrightarrow{\sim} P_{>4}(n - m - 2, 2(m - 1)) \\
    \lambda &\mapsto [\lambda_1 - 1, \dots, \lambda_{m - 1} - 1]
  \end{align*}
  \begin{align*}
    P_{>6, 5, 3}(n, 2m) &\xrightarrow{\sim} P_{>5, 4}(n - m - 2, 2(m - 1)) \\
    \lambda &\mapsto [\lambda_1 - 1, \dots, \lambda_{m - 2} - 1, 4]
  \end{align*}
  \begin{align*}
    P_{6, 5, 3}(n, 2m) &\xrightarrow{\sim} P_{5, 4}(n - m - 2, 2(m - 1)) \\
    \lambda &\mapsto [\lambda_1 - 1, \dots, \lambda_{m - 3} - 1, 5, 4]
  \end{align*}
  \begin{align*}
    P_{4, 3}(n, 2m) &\xrightarrow{\sim} P_{4, 4}(n - m - 2, 2(m - 1)) \cup P_{>5, 4}(n - m - 1, 2(m - 1)) \\
    \lambda &\mapsto
              \begin{cases}
                [\lambda_1 - 1, \dots, \lambda_{m - 3} - 1, 4, 4] &\text{if }[\lambda_{m - 2}, \lambda_{m - 1}, \lambda_m] = [6, 4, 3] \\
                [\lambda_1 - 1, \dots, \lambda_{m - 2} - 1, 4] &\text{if }[\lambda_{m - 1}, \lambda_m] = [4, 3]\text{ and }\lambda_{m - 2} > 6
              \end{cases}
  \end{align*}
\end{proof}

\begin{lemma}
  \label{lmm:3}
  The formal power series $p_1(t, q)$ is given by
  \begin{equation*}
    p_1(t, q)=f_{5, 2, 1, 1}(t, q).
  \end{equation*}
\end{lemma}

\begin{proof}
  Consider the disjoint union
  \begin{equation*}
    P_1 = P_{5, 3, 1} \cup P_{>5, 3, 1} \cup P_{5, 4, 1} \cup P_{>5, 4, 1} \cup P_{>4, 1}
  \end{equation*}
  from which we get the formula
  \begin{equation*}
    p_1(t, q) = p_{5, 3, 1}(t, q) + p_{>5, 3, 1}(t, q) + p_{5, 4, 1}(t, q) + p_{>5, 4, 1}(t, q) + p_{>4, 1}(t, q).
  \end{equation*}
  We have bijections
  \begin{align*}
    P_{>4, 1}(n, 2m + 1) &\xrightarrow{\sim} P_{>4}(n - 1, 2m) \\
    \lambda &\mapsto [\lambda_1, \dots, \lambda_m]
  \end{align*}
  \begin{align*}
    P_{>5, 4, 1}(n, 2m + 1) &\xrightarrow{\sim} P_{>5, 4}(n - 1, 2m) \\
    \lambda & \mapsto [\lambda_1, \dots, \lambda_m]
  \end{align*}
  \begin{align*}
    P_{5, 4, 1}(n, 2m + 1) &\xrightarrow{\sim} P_{5, 4}(n - 1, 2m) \\
    \lambda &\mapsto [\lambda_1, \dots, \lambda_m]
  \end{align*}
  \begin{align*}
    P_{>5, 3, 1}(n, 2m + 1) &\xrightarrow{\sim} P_{>5, 3}(n - 1, 2m) \\
    \lambda &\mapsto [\lambda_1, \dots, \lambda_m]
  \end{align*}
  \begin{align*}
    P_{5, 3, 1}(n, 2m + 1) &\xrightarrow{\sim} P_{>6, 5, 3}(n - 1, 2m) \\
    \lambda &\mapsto [\lambda_1, \dots, \lambda_m]
  \end{align*}
  from which together with the proof of \Cref{lmm:2} we get
  \begin{align*}
    p_{>4, 1}(t, q) &= tqp_{>4}(t, q) = f_{6, 4, 1, 1}(t, q), \\
    p_{>5, 4, 1}(t, q) &= tqp_{>5, 4}(t, q) = f_{9, 5, 5, 3}(t, q), \\
    p_{5, 4, 1}(t, q) &= tqp_{5, 4}(t, q) = f_{13, 6, 10, 5}(t, q), \\
    p_{>5, 3, 1}(t, q) &= tqp_{>5, 3}(t, q) = f_{8, 5, 4, 3}(t, q), \\
    p_{5, 3, 1}(t, q) &= tqp_{>6, 5, 3}(t, q) = f_{11, 6, 9, 5}(t, q).
  \end{align*}
  The formula for $p_1(t, q)$ then follows from \Cref{lmm:1}.
\end{proof}

\begin{lemma}
  \label{lmm:4}
  The formal power series $p_{1, 1}(t, q)$ is given by
  \begin{equation*}
    p_{1, 1}(t, q) = f_{6, 3, 2, 2}(t, q).
  \end{equation*}
\end{lemma}

\begin{proof}
  Consider the disjoint union
  \begin{equation*}
    P_{1, 1} = P_{4, 1, 1} \cup P_{>4, 1, 1}
  \end{equation*}
  from which we get the formula
  \begin{equation*}
    p_{1, 1}(t, q) = p_{4, 1, 1}(t, q) + p_{>4, 1, 1}(t, q).
  \end{equation*}
  We have bijections
  \begin{align*}
    P_{>4, 1, 1}(n, 2m + 2) &\xrightarrow{\sim} P_{>4, 1}(n - 1, 2m + 1) \\
    \lambda &\mapsto [\lambda_1, \dots, \lambda_m, 1]
  \end{align*}
  \begin{align*}
    P_{4, 1, 1}(n, 2m + 2) &\xrightarrow{\sim} P_{>5, 4, 1}(n - 1, 2m + 1) \\
    \lambda &\mapsto [\lambda_1, \dots, \lambda_m, 1]
  \end{align*}
  from which together with the proof of \Cref{lmm:3} we get
  \begin{align*}
    p_{>4, 1, 1}(t, q) &= tqp_{>4, 1}(t, q) = f_{6, 4, 2, 2}(t, q), \\
    p_{4, 1, 1}(t, q) &= tqp_{>5, 4, 1}(t, q) = f_{9, 5, 6, 4}(t, q).
  \end{align*}
  The formula for $p_{1, 1}(t, q)$ then follows from \Cref{lmm:1}.
\end{proof}

\begin{lemma}
  \label{lmm:5}
  The formal power series $p(t, q)$ is given by
  \begin{equation*}
    p(t, q) = f_{3, 2, 0, 0}(t, q) + f_{5, 2, 1, 1}(t, q) + f_{6, 3, 2, 2}(t, q).
  \end{equation*}
\end{lemma}

\begin{proof}
  This follows from the disjoint union
  \begin{equation*}
    P = P_{>2} \cup P_{1} \cup P_{1, 1}
  \end{equation*}
  together with \Cref{lmm:2}, \Cref{lmm:3} and \Cref{lmm:4}.
\end{proof}

\begin{lemma}
  \label{lmm:6}
  The formal power series $p(q)$ satifies
  \begin{equation*}
    \ch_{L}(q) = q^{1/2}p(q).
  \end{equation*}
\end{lemma}

\begin{proof}
  This follows from \Cref{lmm:5} by setting $t = 1$ and \cite[Theorem 4]{andrews_singular_2022} together with \Cref{lmm:1}.
\end{proof}

\section{Computing leading monomials}
\label{sec:comp-lead-monom}
We recall that we have defined $V = \Vir^{1/2}$ and $M = M(1/2, 1/2)$.
\begin{lemma}
  \label{lmm:8}
  Let $\lambda=[\lambda_1,\dots, \lambda_s]$ be a partition with $\lambda=\emptyset$ or $\lambda_s\ge 2$ and consider $M$ as a module over $V$.
  Then
  \begin{equation*}
    (L_\lambda\vac)_{(-1)}\vachalf=L_\lambda\vachalf+m
  \end{equation*}
  for some $m\in G^{2s-1}M$.
\end{lemma}
\begin{proof}
  This follows from the isomorphism between $\gr^G(M)$ and $\bigoplus_{k \in \mathbb{N}}\mathbb{C}[L_{-2}, L_{-3}, \dots]L_{-1}^k$ given in \Cref{prp:8}.
\end{proof}

We know $\gr^G(M)$ is a free module over $\gr^G(V)$:
\begin{equation*}
  \gr^G(M) = \bigoplus_{n\in \mathbb{N}}\gr^G(V)L_{-1}^n.
\end{equation*}
For $n\in \mathbb{N}$ call
\begin{equation*}
  \iota_n:\gr^G(V)\to\gr^G(M)
\end{equation*}
the insertion of $\gr^G(V)$ into the $n$-th component.

\begin{lemma}
  \label{lmm:9}
  Let $v \in G^pV$ for some $p \in \mathbb{N}$.
  Then $v_{(-1)}\vachalf \in G^pM$ and
  \begin{equation*}
    \iota_0(\alpha^p(v)) = \alpha^p(v_{(-1)}\vachalf).
  \end{equation*}
\end{lemma}

\begin{proof}
  Follows immediately from \Cref{lmm:8}.
\end{proof}

We have a natural quotient map
\begin{align*}
  \pi_0: V &\to \Vir_{3, 4} \\
 v &\mapsto v + U(\Vir)\{v_{3, 4}\},
\end{align*}
where
\begin{equation*}
  v_{3, 4} = (L_{-2}^3 + \tfrac{93}{64}L_{-3}^2 - \tfrac{27}{16}L_{-6} - \tfrac{33}{8}L_{-4}L_{-2})\vac
\end{equation*}
is the singular vector of $V$ that generates its maximal subrepresentation as in \cite{andrews_singular_2022}.
Applying the functor $\gr^G$ we obtain a surjective homomorphism of $\mathbb{N}$-graded vertex Poisson algebras:
\begin{equation*}
  \gr^G(\pi_0): \gr^G(V) \to \gr^G(\Vir_{3, 4}).
\end{equation*}
Set $I = \ker(\gr^G(\pi_0))$ following the notation of \cite{andrews_singular_2022} and recall the definition of $K$ given in \eqref{eq:5}.

\begin{lemma}
  \label{lmm:10}
  We have the inclusion
  \begin{equation*}
    \iota_0(I) \subseteq K.
  \end{equation*}
\end{lemma}

\begin{proof}
  We can consider $L$ as a module over $V$ with state-field correspondence map $Y^L_{V}: V \to \mathcal{F}(L)$.
  By \cite[Theorem 4.2]{wang_rationality_1993}, $L$ is a module over $\Vir_{3,4}$ with state-field correspondence map $Y^L_{\Vir_{3, 4}}: \Vir_{3, 4} \to \mathcal{F}(L)$ map such that the following diagram commutes
  \begin{equation*}
    \begin{tikzcd}
      V \arrow[r, "\pi_0"] \arrow[rd, "{Y^{L}_{V}}"'] & {\Vir_{3, 4}} \arrow[d, "{Y^{L}_{\Vir_{3, 4}}}"] \\
      & {\mathcal{F}(L)}
    \end{tikzcd}
  \end{equation*}
  The commutativity of this diagram implies the following statement:
  \begin{equation*}
    \text{For all }v \in U(\Vir)\{v_{3, 4}\}, \text{all } m \in M\text{ and  all }n \in \mathbb{Z}, v_{(n)}m \in W,
  \end{equation*}
  where $W$ is defined in \eqref{eq:4}.
  We use this statement in the following simplified form
  \begin{equation}
    \label{eq:6}
    \text{For all }v \in U(\Vir)\{v_{3, 4}\}, v_{(-1)}\vachalf \in W.
  \end{equation}
  
  Note that
  \begin{equation*}
    I = \sum_{p \in \mathbb{N}}\alpha^p(U(\Vir)\{v_{3, 4}\} \cap G^pV).
  \end{equation*}
  Similarly, we have a formula for $K$
  \begin{equation*}
    K = \sum_{p \in \mathbb{N}}\alpha^p(W \cap G^pM).
  \end{equation*}
  Let $\alpha^p(v) \in I$ with $v \in U(\Vir)\{v_{3, 4}\} \cap G^pV$.
  By \Cref{lmm:9} and \eqref{eq:6}, we have $v_{(-1)}\vachalf \in W \cap G^pM$ and also $\iota_0(\alpha^p(v)) = \alpha^p(v_{(-1)}\vachalf) \in K$, finishing the proof.  
\end{proof}

We need to compute all leading monomials of elements of $K$.
To do this, we need to order the PBW basis of $U(\Vir_{\le -1})=\vspan\{L_\lambda\mid \lambda\text{ is a partition}\}$ first by length, then by degree reverse lexicographic order and finally by position.
Formally, for any partitions $\lambda$ and $\eta$ define
\begin{equation*}
  L_\lambda \le L_\eta \iff
  \begin{cases}
    \len(\lambda) < \len(\eta)\text{ or } \\
    \len(\lambda) = \len(\eta)\text{ and }m_\lambda \le m_\eta.
  \end{cases}
\end{equation*}
For $a \in U(\Vir_{\le -1})$ with $a \neq 0$, we may write
\begin{equation*}
  a = c_1L_{\lambda_1} + c_2L_{\lambda_2} + \dots + c_rL_{\lambda_r},
\end{equation*}
where for $1 \le i\le r$, $0 \neq c_i \in \mathbb{C}$ and $L_{\lambda_1} > L_{\lambda_2} > \dots > L_{\lambda_r}$.
We define the \textbf{leading monomial} of $a$ as $\lm(a) = L_\lambda$. 
We set $\lm(0) = 0$.
We extend the definition of $\lm$ from $U(\Vir_{\le 1})$ to $V$ and $M$ by considering the isomorphisms of vector spaces: $U(\Vir_{\le -2}) \xrightarrow{\sim} V, L_\lambda \mapsto L_\lambda\vac$ and $U(\Vir_{\le -1}) \xrightarrow{\sim} M, L_\lambda \mapsto L_\lambda\vachalf$ where $\lambda$ is a partition and $U(\Vir_{\le 2}) = \{L_\lambda \mid \lambda = [\lambda_1, \dots, \lambda_m]\text{ satisfies } \lambda_m \ge 2\}$.
For example, we have $(L_{-1}^4 - 3L_{-3}L_{-1} - 6L_{-4})\vachalf \in W$ and $\lm((L_{-1}^4 - 3L_{-3}L_{-1} - 6L_{-4})\vachalf) = L_{-1}^4\vachalf$.
If we were using only the degree reverse lexicographical order with TOP, then we would get $L_{-3}L_{-1}\vachalf$ as leading monomial and we don't want that.

\begin{remark}
  \label{rmk:1}
  The definition of the order in the PBW basis of $U(\Vir_{\le 1})$ was made to make the following statements true:
  \begin{enumerate}[label={(\alph*)}]
  \item For all partitions $\lambda$ and all $v \in V$, if $\lm(v) = L_\lambda\vac$ then $\lm(\alpha^{\len(\lambda)}(v)) = p_\lambda$.
  \item For all partitions $\lambda$ and all $m \in M$, if $\lm(m) = L_\lambda\vachalf$ then $\lm(\alpha^{\len(\lambda)}(m)) = m_\lambda$.
  \end{enumerate}
\end{remark}

For $n \in \mathbb{N}$, let $p(n)$ denote the number of partitions of $n$.
A basis of $\gr^G(M)_{1/2 + n}$ is given by $\{L_\lambda\vachalf \mid \Delta(\lambda) = n\}$ and has $p(n)$ elements. 
For $n\in \mathbb{N}$, compute the matrix $A_n$ with $p(n - 2) + p(n - 3)$ rows and $p(n)$ columns which is given by stacking the matrix $A^{m_3}_n$ below the matrix $A^{m_2}_n$ where $A^{m_k}_n$ is the matrix given by
\begin{equation*}
  A^{m_k}_n(i, j) = \text{coefficient of }L_{\lambda_j}\vachalf\text{ in }L_{\lambda_i}m_k,
\end{equation*}
for $1 \le i \le p(n - k)$, $1 \le j \le p(n)$, $n \in \mathbb{N}$ and $k = 2, 3$ where $\lambda_1, \dots, \lambda_{p(n)}$ are the partitions of $n$ ordered in such a way that $L_{\lambda_1} > L_{\lambda_2} > \dots > L_{\lambda_{p(n)}}$.
Now transform $A_n$ into row reduced echelon form obtaining a matrix $A^W_n$ which has an unknown number of rows and $p(n)$ columns.
For each pivot $\lambda$ of $A^W_n$, let $m^W_\lambda$ be the element of $\gr^G(M)$ corresponding to the row which has $L_\lambda\vachalf$ as pivot.
In other words, $m^W_\lambda = L_\lambda\vachalf + \text{(lower order terms)}$.
Set $m^K_\lambda = \alpha^{\len(\lambda)}(m^W_\lambda)$ for each pivot of $A^W_n$.

\begin{remark}
  \label{rmk:2}
  By construction and \Cref{rmk:1}, for each pivot $\lambda$ of $A^W_n$, we have $m^W_\lambda\in W$, $m^K_\lambda \in K$ and $\lm(m^K_\lambda) = m_\lambda$.
\end{remark}

For example, when $n = 4$, the partitions are ordered in the following way:
  \begin{equation*}
    [[2, 2], [2, 1, 1], [1, 1, 1, 1], [3, 1], [4]],
  \end{equation*}
  the matrix $A^W_4$ is given by:
  \begin{equation*}
    A^W_4=
    \begin{pmatrix}
      1 & 0 & 0 & -\frac{3}{16} & -\frac{15}{8} \\
      0 & 1 & 0 & -\frac{1}{4} & -\frac{5}{2} \\
      0 & 0 & 1 & -3 & -6
    \end{pmatrix},
  \end{equation*}
  and the pivots are $[2, 2]$, $[2, 1, 1]$ and $[1, 1, 1, 1]$.
  Therefore,
  \begin{align*}
    m^W_{[2, 2]} &= (L_{[2, 2]} - \tfrac{3}{16}L_{[3, 1]} - \tfrac{15}{8}L_{[4]})\vachalf, &m^K_{[2, 2]} &= L_{-2}L_{-2}, \\
    m^W_{[2, 1, 1]} &= (L_{[2, 1, 1]} - \tfrac{1}{4}L_{[3, 1]} - \tfrac{5}{2}L_{[4]})\vachalf, &m^K_{[2, 1, 1]} &= L_{-2}L_{-1}^2, \\
    m^W_{[1, 1, 1, 1]} &= (L_{[1, 1, 1, 1]} - 3L_{[3, 1]} - 6L_{[4]})\vachalf, &m^K_{[1, 1, 1, 1]} &= L_{-1}^4.
  \end{align*}
  
The partitions $[2]$, $[1, 1, 1]$, $[3, 1, 1]$, $[3, 3]$, $[4, 3, 1]$, $[4, 4, 1]$, $[5, 4, 1, 1]$, $[6, 5, 3, 1]$ are pivots of the matrices $A^W_2, A^W_3, A^W_5, A^W_6, A^W_8, A^W_9, A^W_{11}$ and $A^W_{15}$ respectively, see \Cref{sec:sagem-progr-comp}.
Set
\begin{equation*}
  K' = (m^K_{[2]}, m^K_{[1, 1, 1]}, m^K_{[3, 1, 1]}, m^K_{[3, 3]}, m^K_{[4, 3, 1]}, m^K_{[4, 4, 1]}, m^K_{[5, 4, 1, 1]}, m^K_{[6, 5, 3, 1]}, \iota_0(I))_{\psn},
\end{equation*}
where $\psn$ denotes the Poisson submodule generated by the given subset.
By \Cref{lmm:10}, \Cref{rmk:2} and the fact that $K$ is a Poisson submodule (not just a submodule), we have
\begin{equation*}
K' \subseteq K.
\end{equation*}

Let us consider the Poisson structure of $\gr^G(M)$ as a module over $\gr^G(V)$.
We have
\begin{align*}
  (L_{-2})_{(0)}(m_\lambda) &= \alpha^2(\omega)_{(0)}\alpha^{\len(\lambda)}(L_\lambda\vachalf) \\
  &= \alpha^{\len(\lambda) + 1}(L_{-1}L_\lambda\vachalf) \\
  &= m_{[\lambda, 1]},
\end{align*}
where $\lambda$ is any partition and $[\lambda, 1]$ denotes the partition $\lambda$ with a one appended at the end.
\begin{remark}
  \label{rmk:3}
  While the Poisson structure of $\gr^G(V)$ is trivial (i.e.\ is zero), the Poisson structure of $\gr^G(M)$ is not.
  In fact, if $x \in \gr^G(M)$ and $\lambda$ is a partition then
  \begin{equation*}
    \lm(x) = m_\lambda \text{ implies } \lm((L_{-2})_{(0)}x) = m_{[\lambda,1]}.
  \end{equation*}
\end{remark}

Let $\overline{Q}$ be the set of partitions containing some element of $Q$:
\begin{equation*}
  \overline{Q} = \{\lambda \mid \text{ There is some partition }\eta \in Q \text{ such that } \lambda \supseteq \eta\}.
\end{equation*}

\begin{lemma}
  \label{lmm:11}
  For all $\lambda \in \overline{Q}$ there exists $x \in K'$ such that $\lm(x) = m_\lambda$.
\end{lemma}

\begin{proof}
  By the definition of $K'$, we know that for any $\lambda \in Q$ there exists $x \in K'$ such that $\lm(x) = m_\lambda$.
  Now assume $\lambda \in \overline{Q}$, which means there is $\eta \in Q$ such that $\lambda \supseteq \eta$.
  Therefore, $\lambda$ is obtained from $\eta$ by adding some integers greater than or equal to two and adding $k$ ones.
  Pick $y \in K'$ such that $\lm(y) = m_\eta$ and some power product $p_\tau \in \mathbb{C}[L_{-2}, L_{-3}, \dots]$ for some partition $\tau$ such that $\lm((L_{-2})^k_{(0)}(p_\tau y)) = m_\lambda$ (this can be done because of \Cref{rmk:3} and the fact that $\lm$ is multiplicative) so we set $x = (L_{-2})^k_{(0)}p_\tau y \in K'$ to get $\lm(x) = m_\lambda$.
  For example, take $\lambda = [3, 2, 1, 1] \in \overline{Q}$, $\eta = [2] \in Q$.
  In this case $y = L_{-2}- 3/4L_{-1}^2\in K'$ is such that $\lm(y) = m_{[2]}$ and we take $x = L_{-3}L_{-2}L_{-1}^2 - 3/4L_{-3}L_{-1}^4 \in K'$.
\end{proof}

\begin{remark}
  \label{rmk:4}
  For any partitions $\lambda$ and $\eta$ with the same number of ones, if $\lambda \supseteq \eta$ then $m_\eta \mid m_\lambda$.
  The converse is not true.
  For example, $m_{[4, 2]} \mid m_{[4, 3, 2]}$ but $[4, 3, 2] \nsupseteq [4, 2]$.
  However, if $\eta = [\eta_1, \dots, \eta_m, 1, \dots, 1]$ with $\eta_m \ge 2$ and $\eta_1 - \eta_m \le 1$ then $\lambda \supseteq \eta$ if and only if $m_\eta \mid m_\lambda$ provided $\lambda$ and $\eta$ have the same number of ones.
\end{remark}

The following lemma says our set of partitions $Q$ is special.
If $Q$ were $\{[4, 2]\}$, for example, the following lemma wouldn't be true.

\begin{lemma}
  \label{lmm:12}
  There is an alternative description for $P$:
  \begin{equation*}
    P = \{\lambda \mid \text{For all }\eta \in \overline{Q}, m_\eta \nmid m_\lambda\}.
  \end{equation*}
\end{lemma}

\begin{proof}
  We have to prove the following equality
  \begin{equation*}
    \{\lambda \mid \text{For all }\eta\in Q, \lambda \nsupseteq \eta\} = \{\lambda \mid \text{For all } \eta\in \overline{Q}, m_\eta \nmid m_\lambda\}.
  \end{equation*}
  
  First we prove the inclusion $(\supseteq)$.
  Assume $\lambda$ belongs to the right set and $\lambda \supseteq \eta$ for some $\eta \in Q$.
  Then $\lambda$ has $k$ more ones than $\eta$ for some $k\in \mathbb{N}$.
  By \Cref{rmk:4}, $m_{[\eta, 1, \dots, 1]} \mid m_\lambda$ where we added $k$ ones to $\eta$, a contradiction.
  Therefore, $\lambda \nsupseteq \eta$ for all $\eta \in Q$ which means $\lambda$ belongs to the left set.

  Now we prove inclusion $(\subseteq)$.
  Assume $\lambda$ belongs the left set.
  Then $\lambda$ has $0$, $1$ or $2$ ones.
  Assume $\lambda$ has $0$ ones.
  It is enough to prove that for all $\eta \in Q$ with $0$ ones, $m_\eta \nmid m_\lambda$.
  By \Cref{rmk:4}, if $\eta$ is equal to $[2]$, $[3, 3]$, $[q, q, q]$, $[q + 1, q, q]$ or $[q + 1, q + 1, q]$ for some $q \ge 3$ then $m_\eta \nmid m_\lambda$.
  Assume $\eta = [q + 2, q + 1, q]$ for some $q \ge 3$ and $m_\eta \mid m_\lambda$.
  Then $\lambda$ contains $[q + 2, q + 1, \dots, q + 1, q]$ where $q + 1$ appears $k \ge 1$ times.
  If $k \ge 2$ then $\lambda$ contains $[q + 1, q + 1, q]$ which is not possible.
  If $k = 1$ then $\lambda$ contains $[q + 2, q + 1, q]$ which is not possible.
  Therefore, we can't have $m_\eta \mid m_\lambda$.
  Continuing this way we obtain that $m_\eta \nmid m_\lambda$ for all $\eta \in Q$ with $0$ ones.
  The same argument can be made when $\eta$ contains $1$ or $2$ ones.
  Therefore, $\lambda$ belongs to the right set.
\end{proof}

\section{Proofs of the main theorems}
\label{sec:proofs-main-theorems}

\begin{proof}[Proof of \Cref{thr:4}]
  Since we can't apply Gr\"{o}bner basis theory directly on the free module $\bigoplus_{k\in \mathbb{N}}\mathbb{C}[L_{-2},L_{-3},\dots]L_{-1}^k$, we need to truncate somehow.
  Keep in mind the isomorphisms given in \Cref{prp:8}.
  For $N \in \mathbb{N}$, define
  \begin{align*}
    \gr^G(M)_{\le 1/2 + N} &= \sum_{n \le N}\gr^G(M)_{1/2 + n}, \\
    \gr^G(L)_{\le 1/2 + N} &= \sum_{n \le N}\gr^G(L)_{1/2 + n}.
  \end{align*}
  Note that $\gr^G(M)_{\le 1/2 + N}$ is a vector subspace of the free module
  \begin{equation*}
    F_N = \bigoplus_{n \le N}\mathbb{C}[L_{-2}, L_{-3}, \dots, L_{-N}]L_{-1}^n
  \end{equation*}
  with base ring $\mathbb{C}[L_{-2}, L_{-3}, \dots, L_{-N}]$ because a basis of $\gr^G(M)_{\le 1/2 + N}$ is given by elements of the form $m_\lambda$ with all the elements of $\lambda$ being less than $N$ and with $\lambda$ having at most $N$ ones.
  We note that $K \cap F_N$ is a submodule of $F_N$ for $N \in \mathbb{N}$.
  
  For each $N \in \mathbb{N}$, we have natural vector space isomorphisms
  \begin{equation}
    \label{eq:7}
    \frac{F_N}{K \cap F_N} \supseteq \frac{\gr^G(M)_{\le 1/2 + N}}{K \cap F_N} \xrightarrow{\sim} \frac{\gr^G(M)_{\le 1/2 + N}}{K \cap \gr^G(M)_{\le 1/2 + N}} \xrightarrow{\sim} \gr^G(L)_{1/2 + N}.
  \end{equation}
  Therefore, if we find a basis for each vector space $\gr^G(M)_{\le 1/2 + N}/K \cap F_N \subseteq F_N/K \cap F_N$ such that each basis is contained in the next one when considering the isomorphism \eqref{eq:7}, we get a basis of $\gr^G(L)$ by taking the union of these bases because $\bigcup_{N \in \mathbb{N}}\gr^G(L)_{\le 1/2 + N} = \gr^G(L)$.
  
  Let $G_N$ be a Gr\"{o}bner basis of $K \cap F_N$.
  Define
  \begin{equation*}
    B_N = \{m_\lambda \mid \Delta(\lambda) \le N\text{ and for all }x \in G_N, \lm(x) \nmid m_\lambda\}.
  \end{equation*}
  By \cite[Proposition 3.6.4]{adams_introduction_1994},
  \begin{equation}
    \label{eq:8}
    \{m_\lambda + K \cap F_N \mid m_\lambda \in B_N\}
  \end{equation}
  is a vector space basis of $\gr^G(M)_{\le 1/2 + N}/K \cap F_N$.
  Therefore, by \Cref{lmm:6}, isomorphism \eqref{eq:7} and the isomorphism between $\gr^G(L)_{1/2 + n}$ and $L_{1/2 + n}$ of \Cref{prp:8}, we have
  \begin{equation*}
    |B_N| = \sum_{n \le N}\dim(\gr^G(L)_{1/2 + n}) = \sum_{n \le N}\dim(L_{1/2 + n}) = \sum_{n \le N}|P(n)|.
  \end{equation*}
  
  Define
  \begin{equation*}
    \overline{B_N} = \{m_\lambda \mid \Delta(\lambda) \le N\text{ and for all }\eta \in \overline{Q}, m_\eta \nmid m_\lambda\}.
  \end{equation*}
  By \Cref{lmm:12}, we also have
  \begin{equation*}
    \overline{B_N} = \{m_\lambda \mid \Delta(\lambda) \le N\text{ and }\lambda \in P\}.
  \end{equation*}
  From the definition of $P(n)$ and $\overline{B_N}$, we see that
  \begin{equation*}
    |\overline{B_N}| = \sum_{n \le N}|P(n)|.
  \end{equation*}
  
  On the other hand, we have $B_N \subseteq \overline{B_N}$ because if $m_\lambda \in B_N$ and $\eta \in \overline{Q}$ then we have $m_\eta \nmid m_\lambda$ because if we had $m_\eta \mid m_\lambda$ then by \Cref{lmm:11}, there would exist $x \in K' \cap F_N \subseteq K \cap F_N$ such that $\lm(x) = m_\eta$, then by our definition of Gr\"{o}bner basis, there would exist $y \in G_N$ such that $\lm(y) \mid \lm(x)$, implying $\lm(y) \mid m_\lambda$, a contradiction.
  As both $B_N$ and $\overline{B_N}$ are finite sets and $|B_N| = |\overline{B_N}|$, we get $B_N = \overline{B_N}$.
  We see that the family of bases given by \eqref{eq:8} satisfies the property that one basis is contained in the next one and this concludes the proof of \Cref{thr:4}.
\end{proof}

\begin{proof}[Proof of \Cref{thr:3}]
  Note that the basis of \Cref{thr:4} is also a basis of $\gr^G(L)$ under the isomorphism between $\gr^G(L)$ and $L$ of \Cref{prp:8}.
  For each $\lambda \in P$, we have $\alpha^{\len(\lambda)}(L_\lambda(\vachalf + W)) \in \alpha^{\len(\lambda)}(G^{\len(\lambda)}L_{1/2 + \Delta(\lambda)})$ so $\Delta(\lambda)$ is summed with $1/2$.
  Therefore, $\ch_{\gr^G(L)}(t, q) = q^{1/2}p(t,q)$ and together with \Cref{lmm:5}, we conclude the proof of \Cref{thr:3}.
\end{proof}

\begin{corollary}
  \label{crl:1}
  We have the equality
  \begin{equation*}
    K' = K.
  \end{equation*}
\end{corollary}

\begin{proof}
  This proof is a copy of the proof of \Cref{thr:3}.
  We know $K' \subseteq K$.
  
  Let $N \in \mathbb{N}$ and let $G'_N$ be a Gr\"{o}bner basis of $K' \cap F_N$ considered as a submodule of $F_N$.
  Like in the proof of \Cref{thr:3}, set
  \begin{equation*}
    B'_N = \{m_\lambda \mid \Delta(\lambda) \le N\text{ and for all }x \in G'_N, \lm(x) \nmid m_\lambda\}.
  \end{equation*}
  By \cite[Proposition 3.6.4]{adams_introduction_1994},
  \begin{equation*}
    |B'_N| = \dim\left(\frac{\gr^G(M)_{\le 1/2 + N}}{K' \cap F_N}\right) = \dim\left( \frac{\gr^G(M)_{\le 1/2 + N}}{K' \cap \gr^G(M)_{\le 1/2 + N}} \right).
  \end{equation*}

  We can do the same argument as in the start of the last paragraph of the proof of \Cref{thr:3} and conclude that $B'_N \subseteq \overline{B_N}$, where $\overline{B_N}$ is the defined in the same way.
  But $B_N = \overline{B_N}$ and
  \begin{equation*}
    |B_N| = \dim\left( \frac{\gr^G(M)_{\le 1/2 + N}}{K \cap \gr^G(M)_{\le 1/2 + N}} \right).
  \end{equation*}
  Therefore, $|B'_N| \le |B_N|$ which implies
  \begin{equation*}
    \dim(K' \cap \gr^G(M)_{\le 1/2 + N}) \ge \dim(K \cap \gr^G(M)_{\le 1/2 + N}) \ge \dim(K' \cap \gr^G(M)_{\le 1/2 + N}).
  \end{equation*}
  Then $K' \cap \gr^G(M)_{\le 1/2 + N} = K \cap \gr^G(M)_{\le 1/2 + N}$ for all $N \in \mathbb{N}$.
  Taking the union of these subspaces, we get $K' = K$.
\end{proof}

\appendix
\section{An explicit description of $K$}
\label{sec:sagem-progr-comp}

In \cite[ising-modules.ipynb]{sagemath2}, I wrote a SageMath program to compute $m^W_\lambda$ for all partitions $\lambda$ (if $\lambda$ is not a pivot, it returns $0$).
From the output, we obtain
\begin{align*}
  m^K_{[2]} &= L_{-2} - \tfrac{3}{4}L_{-1}^2, \\
  m^K_{[1, 1, 1]} &= L_{-1}^3, \\
  m^K_{[3, 1, 1]} &= L_{-3}L_{-1}^2, \\
  m^K_{[3, 3]} &= L_{-3}L_{-3} + \tfrac{1}{3}L_{-4}L_{-1}^2, \\
  m^K_{[4, 3, 1]} &= L_{-4}L_{-3}L_{-1}, \\
  m^K_{[4, 4, 1]} &= L_{-4}L_{-4}L_{-1} + \tfrac{9}{8}L_{-5}L_{-3}L_{-1}, \\
  m^K_{[5, 4, 1, 1]} &= L_{-5}L_{-4}L_{-1}^2, \\
  m^K_{[6, 5, 3, 1]} &= L_{-6}L_{-5}L_{-3}L_{-1}.
\end{align*}

However, it turns out $m^K_{[5, 4, 1, 1]}$ and $m^K_{[6, 5, 3, 1]}$ are redundant, as can be seen from \cite[m11-m15.ipynb]{sagemath2}.
From \cite[Theorem 2]{andrews_singular_2022}, we obtain the following explicit expression for $K$:
\begin{equation*}
  K = (m^K_{[2]}, m^K_{[1, 1, 1]}, m^K_{[3, 1, 1]}, m^K_{[3, 3]}, m^K_{[4, 3, 1]}, m^K_{[4, 4, 1]}, \iota_0((a, b)_\partial))_{\psn},
\end{equation*}
where
\begin{align*}
  a = L_{-2}^3, b = L_{-2}L_{-3}L_{-4} + \tfrac{1}{6}L_{-5}L_{-2}^2.
\end{align*}

\bibliographystyle{alpha}
\bibliography{ising-modules}
\end{document}
