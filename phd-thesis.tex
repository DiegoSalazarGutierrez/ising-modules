\documentclass[a4paper, 12pt, reqno]{amsart}

\usepackage{amssymb}
\usepackage{enumerate}
\usepackage{hyperref}
\usepackage[margin = 0.8in]{geometry}
\usepackage{enumitem}
\usepackage{tikz-cd}
\usepackage[nameinlink]{cleveref}
\usepackage{stmaryrd}

\newtheorem{theorem}{Theorem}[subsection]
\newtheorem{lemma}[theorem]{Lemma}
\newtheorem{proposition}[theorem]{Proposition}
\newtheorem{corollary}[theorem]{Corollary}

\theoremstyle{remark}
\newtheorem{remark}[theorem]{Remark}
\newtheorem{example}[theorem]{Example}

\crefformat{section}{\S#2#1#3}
\crefformat{subsection}{\S#2#1#3}

\numberwithin{equation}{subsection}

\setenumerate[0]{label = \normalfont(\roman*)}

\DeclareMathOperator{\Vir}{Vir}
\DeclareMathOperator{\Id}{Id}
\DeclareMathOperator{\gr}{gr}
\DeclareMathOperator{\End}{End}
\DeclareMathOperator{\ch}{ch}
\DeclareMathOperator{\lm}{lm}
\DeclareMathOperator{\vspan}{span}
\DeclareMathOperator{\Ind}{Ind}
\DeclareMathOperator{\len}{len}
\DeclareMathOperator{\psn}{psn}
\DeclareMathOperator{\Frac}{Frac}
\DeclareMathOperator{\Res}{Res}
\DeclareMathOperator{\vac}{|0\rangle}
\DeclareMathOperator{\vachalf}{|1/2\rangle}
\DeclareMathOperator{\vacsixteen}{|1/16\rangle}
\DeclareMathOperator{\zero}{\overline{0}}
\DeclareMathOperator{\one}{\overline{1}}
\DeclareMathOperator{\Der}{Der}
\DeclareMathOperator{\Lie}{Lie}
\DeclareMathOperator{\ad}{ad}
\DeclareMathOperator{\Cur}{Cur}
\DeclareMathOperator{\Hom}{Hom}
\DeclareMathOperator{\fs}{fs}
\DeclareMathOperator{\tc}{tc}
\DeclareMathOperator{\sdim}{sdim}
\DeclareMathOperator{\Sing}{Sing}
\DeclareMathOperator{\Zhu}{Zhu}

\makeindex

\begin{document}

\begin{abstract}
  To every $h + \mathbb{N}$-graded module $M$ over a $\mathbb{N}$-graded conformal vertex algebra $V$ we associate an increasing filtration $(G^pM)_{p \in \mathbb{Z}}$ which is compatible with the filtrations introduced by Haisheng Li.
  The associated graded vector space $\gr^G(M)$ is naturally a module over the Poisson vertex algebra $\gr^G(V)$.
  We study $\gr^G(M)$ for the three irreducible modules of the Ising model $\Vir_{3, 4}$, namely $\Vir_{3,4} = L(1/2, 0)$, $L(1/2, 1/2)$ and $L(1/2, 1/16)$.
  We obtain an explicit monomial basis for each of these modules and a formula for their refined characters which are related to Nahm sums for the matrix $\left(\begin{smallmatrix} 8 & 3 \\ 3 & 2 \end{smallmatrix}\right)$.
\end{abstract}

\title{PBW bases of irreducible Ising modules}
\author{Diego Salazar Gutierrez}
\address{Instituto de Matemática Pura e Aplicada, Rio de Janeiro, RJ, Brazil}
\email{diego.salazar@impa.br}
\date{\today}
\maketitle

\tableofcontents

\section{Introduction}
\label{sec:introduction}

\section{Vertex algebras and their modules}
\label{sec:vert-algebr-their}

\subsection{Formal calculus}
\label{sec:formal-calculus}

All vector spaces and all algebras are over $\mathbb{C}$, the field of complex numbers.
The set of natural numbers $\{0, 1, \dots\}$ is denoted by $\mathbb{N}$, the set of integers is denoted by $\mathbb{Z}$, the set of positive integers $\{1, 2, \dots\}$ is denoted by $\mathbb{Z}_+$ and the set of negative integers $\{-1, -2, \dots\}$ is denoted by $\mathbb{Z}_-$.

The vector space of \index{formal distribution|emph}\emph{formal distributions} in $n \in \mathbb{N}$ variables, denoted by $\mathbb{C}[[X_1^{\pm 1}, \dots, X_n^{\pm 1}]]$, is the set of functions $f: \mathbb{Z}^n \to \mathbb{C}$, written as $f(X_1, \dots, X_n) = \sum_{m_1, \dots, m_n \in \mathbb{Z}}f_{m_1, \dots, m_n}X_1^{m_1}\dots X_n^{m_n} $ with the natural operations of addition and multiplication by a scalar.
The field of \index{rational function|emph}\emph{rational functions} in $n$ variables, denoted by $\mathbb{C}(X_1, \dots, X_n)$, is the field of fractions $\Frac(\mathbb{C}[X_1, \dots, X_n])$.
The field of \index{formal Laurent series|emph}\emph{formal Laurent series} is the subalgebra of elements $f(X) \in \mathbb{C}[[X^{\pm 1}]]$ such that there is $N \in \mathbb{Z}$ with $f_n = 0$ for $n \le N$.
We also have $\mathbb{C}((X)) = \Frac(\mathbb{C}[[X]])$.
The field of \index{joint Laurent series|emph}\emph{joint Laurent series} in $n$ variables, denoted by $\mathbb{C}((X_1, \dots, X_n))$, is $\Frac(\mathbb{C}[[X_1, \dots, X_n]])$.

If $V$ is a vector space, we similarly define $V[[X_1^{\pm 1}, \dots, X_n^{\pm 1}]]$ and $V((X))$ but in this case, $V((X))$ is only a vector space.

Let $V$ be a vector space.
The \index{formal distribution!Fourier expansion of|emph}\emph{Fourier expansion} of a formal distribution $a(z) \in V[[z^{\pm 1}]]$, written as $a = \sum_{n \in \mathbb{Z}} a_nz^n$, is conventionally written in the theory of vertex algebras as
\begin{equation*}
  a(z) = \sum_{n \in \mathbb{Z}}a_{(n)}z^{-n - 1},
\end{equation*}
where
\begin{equation*}
  a_{(n)} = a_{-n-1}.
\end{equation*}
The \index{formal distribution!residue of|emph}\emph{residue} of a formal distribution $a(z) \in V[[z^{\pm 1}]]$ is defined as
\begin{equation*}
  \Res_z(a(z)) = a_{(0)} = a_{-1}.
\end{equation*}

If $P \in R[[z_1^{\pm 1}, \dots, z_n^{\pm 1}]]$ and $Q \in R[[w_1^{\pm 1}, \dots, w_m^{\pm 1}]]$, then $PQ \in R[[z_1^{\pm 1}, \dots z_n^{\pm 1}, w_1^{\pm 1}, \dots, w_m^{\pm 1}]]$ is defined in the natural way.
However, if both $P$ and $Q$ belong to $R[[z_1^{\pm 1}, \dots, z_n^{\pm 1}]]$, we may run into trouble because infinite sums may appear.

An important formal distribution in two variables $z, w$ is the \index{formal distribution!delta|emph}\emph{formal delta distribution} which is defined by
\begin{equation*}
  \delta(z, w) = \sum_{n \in \mathbb{Z}}z^nw^{-n - 1} \in \mathbb{C}[[z^{\pm 1}, w^{\pm 1}]].
\end{equation*}

The \emph{expansion in the domain} $|z| > |w|$ is the field homomorphism $i_{z, w}: \mathbb{C}((z, w)) \to \mathbb{C}((z))((w))$ such that the following diagram commutes
\begin{equation*}
  \begin{tikzcd}
    {\mathbb{C}[[z, w]]} \arrow[rd] \arrow[r] & {\mathbb{C}((z, w))} \arrow[d, "{i_{z,w}}"] \\
    & \mathbb{C}((z))((w))                                          
  \end{tikzcd}
\end{equation*}
where the unlabeled homomorphisms are the natural ones.
Similarly, \index{expansion in the domain|emph}expansion in the domain $|w| > |z|$ is the field homomorphism $i_{w, z}: \mathbb{C}((z, w)) \to \mathbb{C}((w))((z))$ such that the following diagram commutes
\begin{equation*}
  \begin{tikzcd}
    {\mathbb{C}[[z, w]]} \arrow[rd] \arrow[r] & {\mathbb{C}((z, w))}\arrow[d, "{i_{w, z}}"] \\
    & \mathbb{C}((w))((z))                                          
  \end{tikzcd}
\end{equation*}

We have natural inclusions $\mathbb{C}((z))((w)) \to \mathbb{C}[[z^{\pm 1}, w^{\pm 1}]]$ and $\mathbb{C}((w))((z)) \to \mathbb{C}[[z^{\pm 1}, w^{\pm 1}]]$.
The diagram
\begin{equation*}
  \begin{tikzcd}
    & {\mathbb{C}((z, w))} \arrow[ld, "{i_{z,w}}"'] \arrow[rd, "{i_{w,z}}"] &                                 \\
    \mathbb{C}((z))((w)) \arrow[rd] &                                                                      & \mathbb{C}((w))((z)) \arrow[ld] \\
    & {\mathbb{C}[[z^{\pm 1}, w^{\pm 1}]]}                                  &                                
  \end{tikzcd}
\end{equation*}
doesn't commute. In fact, the formal delta distribution can be expressed as
\begin{equation*} 
  \delta(z, w) = i_{z, w}\left(\frac{1}{z - w}\right) - i_{w, z}\left(\frac{1}{z - w}\right),
\end{equation*}
where we consider $i_{z, w}(\frac{1}{z - w})$ and $i_{w, z}(\frac{1}{z - w})$ as elements of $\mathbb{C}[[z^{\pm 1}, w^{\pm 1}]]$.
From now on, we will consider $i_{z, w}$ and $i_{w, z}$ as mapped into $\mathbb{C}[[z^{\pm 1}, w^{\pm 1}]]$.

Let $V$ be a vector space.
A formal distribution $a(z, w) \in V[[z^{\pm 1}, w^{\pm 1}]]$ is \index{formal distribution!local|emph}\emph{local} if there is $N \in \mathbb{N}$ such that
\begin{equation*}
  (z - w)^Na(z, w)=0.
\end{equation*}
For example, the \index{formal distribution!delta}formal delta distribution $\delta(z, w)$ is local.

\begin{theorem}[{\cite[Proposition 2.2]{kac_vertex_1998}}]
  \label{thr:1}
  Let $a(z, w) \in V[[z^{\pm 1}, w^{\pm 1}]]$ be a local formal distribution.
  Then $a(z, w)$ can be written uniquely as a sum
  \begin{equation}
    \label{eq:1}
    a(z, w) = \sum_{j \in \mathbb{N}}c^j(w)\frac{\partial_w^j\delta(z, w)}{j!}
  \end{equation}
  where $c^j(w) \in V[[w^{\pm 1}]]$ are formal distributions given by
  \begin{equation}
    \label{eq:2}
    c^j(w) = \Res_z(z - w)^ja(z, w).
  \end{equation}
  In addition, the converse is true.
\end{theorem}

Let $V$ be a vector space and  $a(z) \in V[[z^{\pm 1}]]$ be a formal distribution.
Define
\begin{equation*}
  i_{z, w}a(z + w) = \sum_{n \in \mathbb{Z}} a_ni_{z, w}(z + w)^n.
\end{equation*}
We have a formal version of the usual Taylor series expansion.

\begin{theorem}
  \label{thr:2}
  We have
  \begin{equation*}
    i_{z, w}a(z + w) = \sum_{j \in \mathbb{N}}\frac{\partial^ja(z)}{j!}w^j.
  \end{equation*}
\end{theorem}

We now define the notion of Fourier transform in two cases: in one and two variables. Let $V$ be a vector space and let $a(z) \in V[[z^{\pm 1}]]$.
We define the \index{Fourier transform!in one variable|emph}\emph{Fourier transform in one variable} of $a(z)$ by
\begin{equation*}
  F^\lambda_za(z) = \Res_z(e^{\lambda z}a(z)) \in V[[\lambda]].
\end{equation*}

\begin{proposition}[{\cite[Proposition 1.5.2]{nozaradan_introduction_2008}}]
  \label{prp:1}
  $F^\lambda_z$ satisfies the following properties:
  \begin{enumerate}
  \item $F^\lambda_z\partial_za(z) = -\lambda F^\lambda_za(z)$;
  \item $F^\lambda_z(e^{zT}a(z)) = F^{z + T}_z(a(z))$ where $T \in \End(V)$ and $a(z) \in V((z))$;
  \item $F^\lambda_z(a(-z)) = -F^{-\lambda}_za(z)$;
  \item $F^\lambda_z\partial^n_w\delta(z, w) = e^{\lambda w}\lambda^n$.
  \end{enumerate}
\end{proposition}

Now let $a(z, w) \in V[[z^{\pm 1}, w^{\pm 1}]]$.
We define the \index{Fourier transform!in two variables|emph}\emph{Fourier transform in two variables} of $a(z, w)$ by
\begin{equation*}
  F^\lambda_{z, w}a(z, w) = \Res_ze^{\lambda(z - w)}a(z, w) \in V[[w^{\pm 1}]][[\lambda]].
\end{equation*}

Expanding the definition of $F^\lambda_{z, w}$, we obtain another expression
\begin{equation*}
  F^\lambda_{z, w}a(z, w) = \sum_{j \in \mathbb{N}}\frac{\lambda^j}{j!}c^j(w),
\end{equation*}
where
\begin{equation*}
  c^j(w) = \Res_z(z - w)^ja(z, w).
\end{equation*}
\begin{proposition}[{\cite[Proposition 1.5.4]{nozaradan_introduction_2008}}]
  \label{prp:2}
  $F^\lambda_{z, w}$ satifies the following properties
  \begin{enumerate}
  \item If $a(z, w)$ is local then $F^\lambda_{z, w}a(z, w) \in V[[w^{\pm 1}]][\lambda]$;
  \item $F^\lambda_{z, w}\partial_za(z, w) = -\lambda F^\lambda_{z, w}a(z, w) = [\partial_w, F^{\lambda}_{z, w}]a(z, w)$;
  \item If $a(z, w)$ is local, then $F^\lambda_{z, w}a(w, z) = F^{-\lambda - \partial_w}_{z, w}a(z, w)$, where $F^{-\lambda - \partial_w}_{z, w}a(z, w) = F^\mu_{z, w}a(z, w)|_{u = -\lambda - \partial_w}$.
  \end{enumerate}
\end{proposition}

\subsection{Lie conformal superalgebras}
\label{sec:lie-conf-super}

A \index{vector superspace|emph}\emph{vector superspace} is a $\mathbb{Z}_2$-graded vector space $V = V_{\zero} \oplus V_{\one}$ where $\mathbb{Z}_2 = \mathbb{Z}/2\mathbb{Z} = \{\zero, \one\}$, $\zero = 0 + 2\mathbb{Z}$ and $\one = 1 + 2\mathbb{Z}$.
% Elements of $V_{\zero}$ are called \index{vector superspace!even element of |emph}\emph{even}, elements of $V_{\one}$ are called \index{vector superspace!odd element of|emph}\emph{odd}.
We call $V_{\zero}$ the \index{vector superspace!even subspace of|emph}\emph{even subspace} of $V$ and $V_{\one}$ the \index{vector superspace!odd subspace of|emph}\emph{odd subspace} of $V$.
Nonzero elements of $V_{\zero}\cup V_{\one}$ are called \index{vector superspace!homogeneous element of|emph}\emph{homogeneous}.
If $V$ is finite dimensional, its we define its superdimension by setting $\sdim(V) = \dim(V_{\zero}) - \dim(V_{\one})$.
A \index{superalgebra|emph}\emph{superalgebra} is a $\mathbb{Z}_2$-graded algebra $A = A_{\zero} \oplus A_{\one}$.
This means $A_{\alpha}A_{\beta} \subseteq A_{\alpha + \beta}$ for all $\alpha, \beta \in \mathbb{Z}_2$.

We set $(-1)^{\zero} = 1$, $(-1)^{\one} = -1$.
If $a\in V_\alpha$, $v\neq 0$ is homogeneous, we set $p(a) = \alpha$ and call it the \index{parity!of homogeneous element|emph}\emph{parity} of $a$.
If $a$ and $b$ are homogeneous, we set $p(a, b) = (-1)^{p(a)p(b)}$.
A \index{Lie superalgebra|emph}\emph{Lie superalgebra} is a superalgebra $\mathfrak{g} = \mathfrak{g}_{\zero} \oplus \mathfrak{g}_{\one}$ with the product written $[a, b]$ for $a, b \in \mathfrak{g}$ and called \index{Lie superalgebra!Lie superbracket of|emph}\emph{Lie superbracket} satisfying the following properties:
\begin{enumerate}
\item $[\bullet, \bullet]$ is \index{graded antisymmetric|emph}\emph{graded antisymmetric}
  \begin{equation*}
    [a, b] = -p(a, b)[b, a];
  \end{equation*}
\item $[\bullet, \bullet]$ satisfies the \index{graded Jacobi identity|emph}\emph{graded Jacobi identity}
  \begin{equation*}
    p(a, c)[a, [b, c]] + p(b, a)[b, [c, a]] + p(c, b)[c, [a, b]] = 0.
  \end{equation*}
\end{enumerate}

In an \index{associative superalgebra}associative superalgebra $A$, we can define the superbracket of homogeneous elements by
\begin{equation*}
  [a, b] = ab - p(a, b)ba.
\end{equation*}
It can then be extended by linearity to nonhomogeneous elements.
With this superbracket, $A$ becomes a Lie superalgebra called the \index{associative superalgebra!underlying Lie superalgebra of|emph}\emph{underlying Lie superalgebra} of $A$ and is denoted by $[A]$.

\begin{remark}
  \label{rmk:1}
  The even part $\mathfrak{g}_{\zero}$ of a Lie superalgebra is just a standard Lie algebra.
  However, unlike superspaces and superalgebras, a Lie superalgebra is not always a Lie algebra.
  This is why we prefer the term Lie superalgebra instead of $\mathbb{Z}_2$-graded Lie algebra.
\end{remark}

The most important example of an associative superalgebra is the \index{endomorphism superalgebra}endomorphism superalgebra $\End(V)$ of a superspace $V$ with the $\mathbb{Z}_2$-grading given by:
\index{parity!of endomorphism|emph}
\begin{equation*}
  \End(V)_{\alpha} = \{a \in \End(V) \mid \text{For }\beta \in \mathbb{Z}_2, aV_\beta \subseteq V_{\alpha + \beta}\},
\end{equation*}
for $\alpha \in \mathbb{Z}_2$.
We denote $\mathfrak{gl}(V) = [\End(V)]$.

Let $A$ be a not necessarily associative superalgebra.
A \index{superalgebra!superderivation of|emph}\emph{superderivation} of $A$ is an homogeneous endomorphism $\partial \in \End(A)$ such that
\begin{equation*}
  \partial(ab) = \partial(a)b + (-1)^{p(\partial)p(a)}b\partial(c),
\end{equation*}
for all $a, b \in A$.
The \index{superalgebra!subspace of superderivations of|emph}\emph{subspace of superderivations} of $A$ is denoted by $\Der(A)$.
A differential superalgebra is a superalgebra $A$ together with a superderivation $\partial$ of $A$.
Assume $\mathfrak{g}$ is a Lie superalgebra and $\partial$ is a superderivation of $\mathfrak{g}$.
We can form the universal enveloping algebra $U(\mathfrak{g})$ which is now an associative superalgebra.
The derivation $\partial: \mathfrak{g} \to \mathfrak{g}$ can be extended uniquely to a derivation $DU(\partial): U(\mathfrak{g}) \to U(\mathfrak{g})$.
We have, in fact, constructed a functor
\begin{align*}
  DU: \{\text{differential Lie superalgebras}\} &\to \{\text{associative differential superalgebras}\} \\
  (\mathfrak{g}, \partial) &\mapsto (U(\mathfrak{g}), DU(\partial))
\end{align*}

Let $\mathfrak{g}$ be a Lie superalgebra.
We first extend the Lie superbracket on $\mathfrak{g}$ to the \index{formal distribution!Lie superbracket of|emph}\emph{Lie superbracket} between two $\mathfrak{g}$-valued formal distributions in one variable. Starting from $a(z) = \sum_{m \in \mathbb{Z}}a_{(m)}z^{-m - 1} \in \mathfrak{g}[[z^{\pm 1}]]$ and $b(w) = \sum_{n \in \mathbb{Z}}b_{(n)}z^{-n - 1} \in \mathfrak{g}[[w^{\pm 1}]]$, we define a new formal distribution in two variables by defining the superbracket
\begin{equation*}
  [a(z), b(w)] = \sum_{m, n \in \mathbb{Z}}[a_{(m)}, b_{(n)}]z^{-m - 1}w^{-n - 1} \in \mathfrak{g}[[z^{\pm 1}, w^{\pm 1}]]
\end{equation*}

Let $\mathfrak{g}$ be a Lie superalgebra.
A pair $(a(z), b(z))$ of $\mathfrak{g}$-valued formal distributions is said \index{formal distribution!local pair of|emph}\emph{local} if $[a(z), b(w)]$ is local.
By \Cref{thr:1}, this means that
\begin{equation*}
  [a(z), b(w)] = \sum_{j \in \mathbb{N}}c^j(w)\frac{\partial^j_w\delta(z, w)}{j!}.
\end{equation*}
with $c^j(w) = \Res_z(z - w)^j[a(z), b(w)] \in \mathfrak{g}[[w^{\pm 1}]]$.
Equivalently, we can write this equation as
\begin{equation}
  \label{eq:3}
  [a_{(m)}, b_{(n)}] = \sum_{j \in \mathbb{N}}c^j(w)_{(m + n - j)}.
\end{equation}
\begin{remark}
  \label{rmk:2}
  If $(a(z), b(z))$ is a local pair then $(\partial_za(z), b(z))$ is also a local pair.
\end{remark}

Let $\mathfrak{g}$ be a Lie superalgebra.
A subset $\mathfrak{F} \subseteq \mathfrak{g}[[z^{\pm 1}]]$ of formal distributions is called a \index{formal distribution!local family of|emph}\emph{local family} if all pairs of its elements are local.
Let $a(w)$ and $b(w)$ be two $\mathfrak{g}$-valued formal distributions.
For $j \in \mathbb{N}$, their \index{formal distribution!$j$-product of|emph}\emph{$j$-product} is the $\mathbb{C}$-bilinear map defined by
\begin{align}
  \nonumber
  \bullet_{(j)}\bullet: \mathfrak{g}[[w^{\pm 1}]] \times \mathfrak{g}[[w^{\pm 1}]] &\to \mathfrak{g}[[w^{\pm 1}]]\\
  \label{eq:4}
  a(w)_{(j)}b(w) &= \Res_z(z - w)^j[a(z), b(w)].
\end{align}
Expanding the right hand side we get
\begin{equation}
  \label{eq:5}
  (a(w)_{(j)}b(w))_{(m)} = \sum_{k = 0}^j\binom{j}{k}(-1)^k[a_{(j - k)},b_{(m + k)}].
\end{equation}
We define $a(w)_{(j)} \in \End(\mathfrak{g}[[z^{\pm 1}]])$ in the natural way.
If $(a(z), b(z))$ is a local pair, then \eqref{eq:3} becomes
\begin{equation}
  \label{eq:6}
  [a_{(m)}, b_{(n)}] = \sum_{j \in \mathbb{N}}\binom{m}{j}(a(w)_{(j)}b(w))_{(m + n - j)}.
\end{equation}
By \Cref{thr:1}, we also have
\begin{equation*}
  [a(z),b(w)]=\sum_{j\in \mathbb{N}}(a(w)_{(j)}b(w))\frac{\partial_w^j\delta(z,w)}{j!}.
\end{equation*}
All these identities led us to the define the following new algebraic structure that encodes the relevant information compactly.

Let $\mathfrak{g}$ be a Lie superalgebra.
The \index{formal distribution!$\lambda$-bracket of|emph}\emph{$\lambda$-bracket} of two $\mathfrak{g}$-valued formal distributions is defined by the $\mathbb{C}$-bilinear map
\begin{align*}
  [\bullet_{\lambda}\bullet]: \mathfrak{g}[[w^{\pm 1}]] \times \mathfrak{g}[[w^{\pm 1}]] &\to \mathfrak{g}[[w^{\pm 1}]][[\lambda]] \\
  [a(w)_{\lambda}b(w)] &= F^{\lambda}_{z, w}[a(z), b(w)]
\end{align*}
It can easily be shown that the $\lambda$-bracket is related to the $j$-products by
\begin{equation*}
  [a(w)_{\lambda}b(w)] = \sum_{j \in \mathbb{N}}a(w)_{(j)}b(w)\frac{\lambda^j}{j!}.
\end{equation*}
This suggest to see the $\lambda$-bracket as the generating function of the $j$-products.
It allows us to gather all the $j$-products in one product alone, the price to pay being the additional formal variable $\lambda$.
Note that for a local pair, the sum in the expansion of $[a(w)_{\lambda} b(w)]$ in terms of the $j$-products is finite, i.e.\ $[a(w)_{\lambda}b(w)] \in \mathfrak{g}[[w^{\pm 1}]][\lambda]$.

\begin{theorem}[{\cite[\S2.3]{nozaradan_introduction_2008}}]
  \label{thr:3}
  The $j$-products and the $\lambda$-bracket satisfy the following properties:
  \begin{enumerate}
  \item $(\partial a(w))_{(j)}b = -ja(w)_{(j - 1)}b(w)$;
  \item $a(w)_{(j)}\partial b(w) = \partial(a(w)_{(j)}b(w)) + ja(w)_{(j - 1)}b(w)$;
  \item $\partial(a(w)_{(j)}b(w))=(\partial a(w))_{(j)}b(w)+a(w)_{(j)}\partial b(w)$;
  \item $[\partial a(w)_{\lambda}b(w)] = -\lambda [a(w)_{\lambda}b(w)]$;
  \item $[a(w)_{\lambda}\partial b(w)] = (\partial + \lambda)[a(w)_{\lambda}b(w)]$;
  \item $\partial[a(w)_\lambda b(w)]=[\partial a(w)_\lambda b(w)]+[a(w)_\lambda \partial b(w)]$.
  \end{enumerate}
\end{theorem}
\begin{remark}
  \label{rmk:3}
  Properties (c) and (f) tell us that $\partial: \mathfrak{g}[[z^{\pm 1}]] \to \mathfrak{g}[[z^{\pm 1}]]$ acts as a derivation on the $j-$products and the $\lambda$-bracket.
\end{remark}

Let $V$ be vector superspace.
From now on, all coefficients of a formal distribution are assumed to have the same parity.
Therefore, we can define the \index{parity!of formal distribution|emph}\emph{parity} of a formal distribution $a(z) \in V[[z^{\pm 1}]]$ as $p(a(z)) = p(a_{(n)})$ for any $n \in \mathbb{Z}$.

\begin{theorem}[{\cite[\S2.3]{nozaradan_introduction_2008}}]
  \label{thr:4}
  Let $\mathfrak{g}$ be a Lie superalgebra.
  The $j$-products and the $\lambda$-brackets between $\mathfrak{g}$-valued formal distributions satisfy the following properties
  \begin{enumerate}
  \item $b(w)_{(j)}a(w) = -p(a(w), b(w))\sum_{l = 0}^{\infty}(-1)^{j + l}\frac{\partial^l(a(w)_{(j + l)}b(w))}{l!}$ if $(a(w), b(w))$ is a local pair;
  \item $[a(w)_{(p)}, b(w)_{(m)}] = \sum_{k = 0}^p\binom{p}{k}(a(w)_{(k)}b(w))_{(p + m - k)}$;
  \item $[b(w)_{\lambda}a(w)] = -p(a(w), b(w))[a(w)_{-\lambda - \partial}b(w)]$ if $(a(w), b(w))$ is a local pair;
  \item $[a(w)_{\lambda}[b(w)_{\mu}c(w)]] = [[a(w)_{\lambda}b(w)]_{\lambda + \mu}c(w)] + p(a(w), b(w))[b(w)_{\mu}[a(w)_{\lambda}c(w)]]$;
  \item $F^{\lambda + \mu}_z[a(z)_{\lambda}b(z)] = [F^{\lambda}_za(z), F^{\mu}_zb(z)]$.
  \end{enumerate}
\end{theorem}

Let $\mathfrak{g}$ be a Lie superalgebra.
A \index{formal distribution Lie superalgebra|emph}\emph{formal distribution Lie superalgebra} is a pair $(\mathfrak{g}, \mathfrak{F})$ such that $\mathfrak{F}$ is a local family of $\mathfrak{g}$-valued formal distributions, denoted by $\{a^j(z) = \sum_{n \in \mathbb{Z}}a^j_{(n)}z^{-n - 1}\}_{j \in J}$, such that the coefficients $\{a^j_{(n)} \mid j \in J, n \in \mathbb{Z}\}$ span the whole $\mathfrak{g}$.
A \index{formal distribution Lie superalgebra!regular|emph}\emph{regular} formal distribution Lie superalgebra is a triple $(\mathfrak{g}, \mathfrak{F}, T)$ such that
\begin{enumerate}
\item $(\mathfrak{g}, \mathfrak{F})$ is a formal distribution Lie superalgebra;
\item $\mathbb{C}[\partial_z]\mathfrak{F}$ is closed under all $n$-th products for $n \in \mathbb{N}$;
\item $T \in \Der(\mathfrak{g})$ satisfies
  \begin{equation*}
    T(a^j(z)) = \partial_za^j(z)
  \end{equation*}
  which is equivalent to
  \begin{equation}
    \label{eq:7}
    T(a^j_{(n)}) = -na^j_{(n - 1)},
  \end{equation}
  for all $j \in J$ and $n \in \mathbb{Z}$. 
\end{enumerate}
\begin{remark}
  \label{rmk:4}
  Note that \eqref{eq:7} and the fact that $\{a^j_{(n)} \mid j \in J, n \in \mathbb{Z}\}$ span $\mathfrak{g}$ imply that if such $T$ exists, it is even and unique so we could remove $T$ from the notation, but we won't.
\end{remark}

Let $(\mathfrak{g}, \mathfrak{F})$ be a formal distribution Lie superalgebra.
The \index{formal distribution Lie superalgebra!annihilation subalgebra of|emph}\emph{annihilation subalgebra} of $(\mathfrak{g}, \mathfrak{F})$ is
\begin{equation*}
  \mathfrak{g}_- = \vspan\{a^j_{(n)} \mid j \in J, n \in \mathbb{N}\},
\end{equation*}
the \index{formal distribution Lie superalgebra!creation subalgebra of|emph}\emph{creation subalgebra} of $(\mathfrak{g}, \mathfrak{F})$ is
\begin{equation*}
  \mathfrak{g}_+ = \vspan\{a^j_{(-n - 1)} \mid j \in J, n \in \mathbb{N}\},
\end{equation*}
and the \index{formal distribution Lie superalgebra!polar decomposition|emph}\emph{polar decomposition} of $(\mathfrak{g}, \mathfrak{F})$ is
\begin{equation*}
  \mathfrak{g} = \mathfrak{g}_- \oplus \mathfrak{g}_+.
\end{equation*}

By \Cref{rmk:2}, if $(\mathfrak{g}, \mathfrak{F})$ is a formal distribution Lie superalgebra then $\mathbb{C}[\partial_z]\mathfrak{F}$ is a local family.

The notions of $j$-products and $\lambda$-bracket were previously defined from $\mathfrak{g}$-valued formal distributions with $\mathfrak{g}$ being a given Lie superalgebra.
Those products were shown to satisfy several properties, coming either from their definition or from the fact that $\mathfrak{g}$ is a Lie superalgebra.
Now we take those properties as axioms of a new algebraic structure, defined intrinsically, without any reference either to $\mathfrak{g}$, nor to formal distributions.
For this reason, we write $\partial$ instead of $\partial_z$ in the following definition.

A $\mathbb{C}[\partial]$-module $\mathcal{R}$ is called a Lie conformal superalgebra if it is endowed with a $\mathbb{C}$-bilinear map called $\lambda$-bracket
\begin{equation*}
  [\bullet_{\lambda}\bullet]: \mathcal{R}\times\mathcal{R} \to \mathcal{R}[\lambda]
\end{equation*}
and satisfying the following properties for all $a, b, c \in \mathcal{R}$:
\begin{enumerate}
\item (sesquilinearity) $[\partial a_{\lambda}b] = -\lambda[a_{\lambda}b]$;
\item (skewsymmetry) $[b_{\lambda}a] = -p(a, b)[a_{-\lambda - \partial}b]$;
\item (Jacobi identity) $[a_{\lambda}[b_{\mu}c]] = [[a_{\lambda}b]_{\lambda +\mu}c] + p(a, b)[b_{\mu}[a_{\lambda}c]]$.
\end{enumerate}

If we write
\begin{equation*}
  [a_{\lambda}b] = \sum_{j \in \mathbb{N}}(a_{(j)}b)\frac{\lambda^j}{j!},
\end{equation*}
where $a_{(j)} \in \End(\mathcal{R})$, these properties translate in terms of $j$-products as follows
\begin{enumerate}[label = (\alph*)]
\item $(\partial a)_{(j)} = -j a_{(j - 1)}$; 
\item $b_{(j)}a = -p(a,b)\sum_{l = 0}^\infty(-1)^{j + l}\frac{\partial^l(a_{(j + l)}b)}{l!}$;
\item $[a_{(p)},b_{(m)}] = \sum_{k = 0}^p\binom{p}{k}(a_{(k)}b)_{(p + m - k)}$.
\end{enumerate}

\begin{proposition}[{\cite[Remark 2.5.3]{nozaradan_introduction_2008}}]
  \label{prp:3}
  Let $\mathcal{R}$ be a Lie conformal superalgebra and $a, b \in \mathcal{R}$.
  Then
  \begin{equation*}
    [a_{\lambda}\partial b] = (\partial + \lambda)[a_{\lambda}b],
  \end{equation*}
  or equivalently
  \begin{equation*}
    a_{(j)}\partial b = \partial(a_{(j)}b) + ja_{(j - 1)}b,
  \end{equation*}
  for all $j \in \mathbb{N}$.
  In particular, $\partial$ is a derivation of the $\lambda$-bracket.
\end{proposition}

We have previously shown that any regular formal distribution Lie superalgebra $(\mathfrak{g}, \mathfrak{F}, T)$ was given the structure of Lie Conformal superalgebra $\mathcal{R}$ with $\mathcal{R} = \mathbb{C}[\partial_z]\mathfrak{F}$, $\partial = \partial_z$ and $[a_{\lambda}b] = F^{\lambda}_{z, w}([a(z), b(w)])$.
It turns out that the process can be reverted: to any conformal superalgebra we can associate a regular formal distribution Lie superalgebra.
According to the definition of a formal distribution Lie superalgebra, we first have to define a Lie superalgebra, denoted by $\Lie(\mathcal{R})$ and then associate to it a conformal family $\mathcal{R}$ of $\Lie(\mathcal{R})$-valued formal distributions, whose coefficients span $\Lie(\mathcal{R})$ so that $(\Lie(\mathcal{R}), \mathcal{R})$ is then the expected formal distribution Lie algebra.
We proceed in two steps.

We first consider the space $\widetilde{\mathcal{R}} = \mathcal{R}[t, t^{-1}] = \mathcal{R}\otimes\mathbb{C}[t, t^{-1}]$ with $\widetilde{\partial} = \partial \otimes I + I \otimes \partial_t$ where $I$ appearing on the left (resp.\ right) of $\otimes$ is the identity operator acting on $\mathcal{R}$ (resp.\ $\mathbb{C}[t, t^{-1}]$).
This space is called the affinization of $\mathcal{R}$.
Its generating elements can be written $a\otimes t^m$, where $a \in \mathcal{R}$ and $m \in \mathbb{Z}$.
For clarity, we will use the notation $at^m$ for its elements and $\widetilde{\partial} = \partial + \partial_t$.
Define the commutation relation on $\widetilde{\mathcal{R}}$ as follows:
\begin{equation*}
  [at^m, bt^n] = \sum_{j \in \mathbb{N}}\binom{m}{j}(a_{(j)}b)t^{m + n - j},
\end{equation*}
which gives $\widetilde{\mathcal{R}}$ the structure of algebra, denoted by $(\widetilde{\mathcal{R}},[\bullet, \bullet])$.

Now the second step. We have to check that the commutator verifies the antisymmetry and Jacobi identities, considering that the terms $a_{(j)}b$ of the definition of $[\bullet, \bullet]$ satisfy the axioms.
The latter ones are not sufficient.
Indeed, as we will see, another constraint has to be imposed on elements of $\widetilde{\mathcal{R}}$, namely $\widetilde{\partial}(at^m) = 0$.
The algebraic formulation of the latter condition is as follows: the space $\widetilde{\mathcal{R}}$ has to be quotiented by the subspace $I$ spanned by the elements of the form $\{(\partial a)t^n + nat^{n - 1} \mid n \in \mathbb{Z}\}$.
Using $\widetilde{\partial}$, we can write $I = \widetilde{\partial}\widetilde{\mathcal{R}}$.
This process has two goals: first transferring on $\widetilde{\mathcal{R}}$ the structure of algebra of $(\widetilde{\mathcal{R}}, [\bullet, \bullet])$ and then endowing $(\widetilde{\mathcal{R}}/\widetilde{\partial}\widetilde{\mathcal{R}}, [\bullet, \bullet])$ with the structure of Lie superalgebra.
The first goal is not direct.
Indeed, $\widetilde{\partial}\widetilde{\mathcal{R}}$ has to be a two-sided ideal of the algebra $(\widetilde{\mathcal{R}}, [\bullet, \bullet])$, which is the case.

\begin{lemma}[{\cite[Proposition 2.6.1]{nozaradan_introduction_2008}}]
  \label{lmm:1}
  $\widetilde{\partial}\widetilde{\mathcal{R}}$ is a two-sided ideal of the algebra $(\widetilde{\mathcal{R}}, [\bullet, \bullet])$.
\end{lemma}

Define the homomorphism $\phi: \widetilde{\mathcal{R}} \to \widetilde{\mathcal{R}}/\widetilde{\partial}\widetilde{\mathcal{R}}$ as the natural quotient map.
The commutator between two elements of $\widetilde{\partial}\widetilde{\mathcal{R}}$ is defined by
\begin{equation}
  [\phi(at^m), \phi(bt^n)] = \sum_{j \in \mathbb{N}}\binom{m}{j}\phi((a_{(j)}b)t^{m + n - j}),
\end{equation}
where $a, b \in \mathcal{R}$.

\begin{proposition}[{\cite[Proposition 2.6.3]{nozaradan_introduction_2008}}]
  \label{prp:4}
  $(\widetilde{\mathcal{R}}/\widetilde{\partial}\widetilde{\mathcal{R}}, [\bullet, \bullet])$ is a Lie superalgebra.
\end{proposition}

We set
\begin{equation*}
  \Lie(\mathcal{R}) = \widetilde{\mathcal{R}}/\widetilde{\partial}\widetilde{\mathcal{R}}.
\end{equation*}
Abusing notation, we define the family $\mathcal{R}$ of $\Lie(\mathcal{R})$-valued formal distributions, whose coefficients span $\Lie(\mathcal{R})$, by
\begin{equation*}
  \mathcal{R} = \left\{\sum_{n \in \mathbb{Z}}\phi(at^n)z^{-n - 1} \mid a \in \mathcal{R}\right\}.
\end{equation*}

\begin{theorem}[{\cite[Proposition 2.6.4]{nozaradan_introduction_2008}}]
  \label{thr:5}
  Let $\mathcal{R}$ be a Lie conformal superalgebra.
  Then $(\Lie(\mathcal{R}), \mathcal{R},-\partial_t)$ is a regular formal distribution Lie superalgebra. 
\end{theorem}

\begin{remark}
  \label{rmk:5}
  We haven't defined the category of regular formal distribution Lie superalgebras nor the category of Lie conformal superalgebras.
  But it's clear how they should be and they are equivalent categories:
  \begin{align*}
    \{\text{regular formal distribution Lie superalgebra}\} &\leftrightarrow \{\text{Lie conformal superalgebra}\} \\
    (\mathfrak{g}, \mathfrak{F}, T) &\mapsto (\mathbb{C}[\partial_z]\mathfrak{F}, F^{\lambda}_{z, w}([\bullet, \bullet])) \\
    (\Lie(\mathcal{R}), \mathcal{R},-\partial_t) &\mapsfrom \mathcal{R}
  \end{align*}
\end{remark}

\begin{theorem}
  \label{thr:6}
  Let $\mathcal{R}$ be a Lie conformal superalgebra, $a, b \in R$ and $j, m \in \mathbb{N}$.
  Then
  \begin{equation*}
    (a_{(j)}b)_{(m)} = \sum_{k = 0}^j\binom{j}{k}(-1)^k[a_{(j - k)},b_{(m + k)}].
  \end{equation*}
\end{theorem}

\begin{proof}
  This is just \eqref{eq:5} in the language of Lie conformal suprealgebras.
\end{proof}

We now show three examples of regular formal distribution Lie superalgebras and their respective Lie conformal superalgebras.

\begin{example}[\index{Virasoro!Lie conformal algebra|emph}\emph{Virasoro Lie conformal algebra}]
  \label{exa:1}
  The Virasoro Lie algebra, denoted $\Vir$, is the complex Lie algebra given by
  \begin{equation*}
    \Vir = \bigoplus_{n \in \mathbb{Z}}\mathbb{C}L_{n} \oplus \mathbb{C}C.
  \end{equation*}
  These elements satisfy the following commutation relations:
  \begin{equation}
    \label{eq:8}
    \begin{aligned}
      [L_m, L_n] &= (m - n)L_{m + n} + \delta_{m, -n}\frac{m^3 - m}{12}C, \\
      [\Vir, C] &= 0,
    \end{aligned}
  \end{equation}
  for all $m, n \in \mathbb{Z}$.
  We construct a $\Vir$-valued formal distribution by setting
  
  \begin{equation*}
    L(z) = \sum_{n \in \mathbb{Z}}L_{(n)}z^{-n - 1}\text{ with }L_{(n)} = L_{n - 1}.
  \end{equation*}
  Hence $L(z) = \sum_{n \in \mathbb{Z}}L_nz^{-n - 2}$.
  In terms of formal distributions, the commutation relations become:
  \begin{equation}
    \label{eq:9}
    \begin{aligned}
      [L(z), L(w)] &= \partial_wL(w)\delta(z, w)+2L(w)\partial_w\delta(z,w)+\frac{C}{12}\partial^3_w\delta(z,w) \\
      [L(z), C] &= 0,
    \end{aligned}
  \end{equation}
  where $C$ denotes the constant formal distribution equal to $C \in \Vir$.
  In terms of $j$-products, the commutation relations become:
  \begin{equation}
    \label{eq:10}
    \begin{aligned}
      L(z)_{(0)}L(z) &= \partial_zL(z), \\
      L(z)_{(1)}L(z) &= 2L(z), \\
      L(z)_{(3)}L(z) &= \frac{C}{2}, \\
      L(z)_{(j)}L(z) &= 0 \quad (j \neq 0, 1, 3), \\
      L(z)_{(j)}C &= 0 \quad (j \in \mathbb{N}).
    \end{aligned}
\end{equation}
  In terms of the $\lambda$-bracket, the commutation relations become:
  \begin{equation}
    \label{eq:11}
    \begin{aligned}
      [L(z)_{\lambda}L(z)] &= (\partial + 2\lambda)L(z) + \frac{\lambda^3}{12}C \\
      [L(z)_{\lambda}C] &= 0.
    \end{aligned}
\end{equation}
  By \Cref{thr:1}, $\{L(z), C\}$ is a local family.
  Therefore, $(\Vir, \{L(z), C\})$ is a formal distribution Lie algebra.
  Moreover, we can verify directly that $(\Vir, \{L(z), C\}, \ad(L_{-1}))$ is regular.
  We obtain a Lie conformal algebra $\mathcal{R} = \mathbb{C}[\partial]L + \mathbb{C}C$, with $L = L(z)$, $\partial C = 0$ and $\partial = \partial_z$.
  This is actually a direct sum and  we get the Virasoro Lie conformal algebra
  \begin{equation*}
    \Vir = \mathbb{C}[\partial]L \oplus \mathbb{C}C.
  \end{equation*}
\end{example}

\begin{remark}
  \label{rmk:6}
  The notation $L(z) = \sum_{n \in \mathbb{Z}}L_nz^{-n - 2}$ is contradictory with the notation we wrote in \cref{sec:formal-calculus}.
  However, this notation will acquire a meaning when we treat the notion of weight of an eigendistribution.
  In fact, this notation usually simplifies calculations, as we will see later.
\end{remark}

Let $\mathfrak{g}$ be a Lie superalgebra and $C \in \mathfrak{g}$.
A Virasoro formal distribution $L(z) \in \mathfrak{g}[[z^{\pm 1}]]$ with central charge $C$ is a $\mathfrak{g}$-valued formal distribution satisfying \eqref{eq:8}, or equivalently, \eqref{eq:9}, \eqref{eq:10} or \eqref{eq:11}.

\begin{example}[Current Lie conformal superalgebra]
  \label{exa:2}
  Let $\mathfrak{g} = \mathfrak{g}_{\zero} \oplus \mathfrak{g}_{\one}$ be a Lie superalgebra.
  A supersymmetric bilinear form is a bilinear map $(\bullet| \bullet): \mathfrak{g} \times \mathfrak{g} \to \mathbb{C}$ such that
  \begin{equation*}
    (a| b) = (-1)^{p(a)}(b| a),
  \end{equation*}
  where $a$ and $b$ are homogeneous elements of $\mathfrak{g}$.
  Alternatively, we can define a supersymmetric bilinear form as a bilinear form that vanishes on $\mathfrak{g}_{\zero} \oplus \mathfrak{g}_{\one}$ and $\mathfrak{g}_{\one} \oplus \mathfrak{g}_{\zero}$, symmetric on $\mathfrak{g}_{\zero} \oplus \mathfrak{g}_{\zero}$ and antisymmetric on $\mathfrak{g}_{\one} \oplus \mathfrak{g}_{\one}$.
  A bilinear form $(\bullet| \bullet): \mathfrak{g} \times \mathfrak{g} \to \mathbb{C}$ is said invariant if
  \begin{equation*}
    ([a, b]| c)=(a| [b, c])\quad a, b, c \in \mathfrak{g}.
  \end{equation*}

  Let $\mathfrak{g}$ be a Lie superalgebra endowed with a supersymmetric, invariant bilinear form $(\bullet| \bullet)$.
  The associated loop algebra, denoted by $\tilde{\mathfrak{g}}$, is the algebra $\tilde{\mathfrak{g}} = \mathfrak{g} \otimes \mathbb{C}[t, t^{-1}] = \mathfrak{g}[t, t^{-1}]$, endowed with the superbracket defined by
  \begin{equation*}
    [a\otimes f(t), b\otimes g(t)] = [a, b] \otimes f(t)g(t),
  \end{equation*}
  where $a, b \in \mathfrak{g}$, $f(t), g(t) \in \mathbb{C}[t, t^{-1}]$ and with parity given by $p(a\otimes f(t)) = p(a)$.
  This makes $\tilde{\mathfrak{g}}$ into a Lie superalgebra.
  Replacing $a\otimes t^n$ by $at^n$ for brevity, the commutation relations become
  \begin{equation*}
    [at^m, bt^n]=[a, b]t^{m + n}.
  \end{equation*}
  The central extension of the loop algebra is the algebra $\hat{\mathfrak{g}} = \tilde{\mathfrak{g}} \oplus \mathbb{C}K$ with the superbracket
  \begin{equation*}
    \begin{aligned}
    [at^m, bt^n] &= [a, b]t^{m + n} + m\delta_{m, -n}(a| b)K, \\
    [\hat{\mathfrak{g}}, K] &= 0
    \end{aligned}
  \end{equation*}
  where $a, b \in \mathfrak{g}$, $n, m \in \mathbb{Z}$ and with parity given by $p(K) = \zero$.
  This makes $\hat{\mathfrak{g}}$ into a Lie superalgebra called the affinization of $\mathfrak{g}$.
  If $\mathfrak{g}$ is a finite-dimensional simple Lie superalgebra, then the affinization of $\mathfrak{g}$ leads to a Kac-Moody affinization.
  We now construct $\hat{\mathfrak{g}}$-valued formal distributions by setting
  \begin{equation*}
    a(z) = \sum_{n \in \mathbb{Z}}at^nz^{-n - 1},
  \end{equation*}
  for $a \in \mathfrak{g}$.
  These formal distributions are called currents.
  In terms of currents, the commutation relations become:
  \begin{equation*}
    \begin{aligned}
    [a(z), b(w)] &= [a, b](w)\delta(z, w) + K(a| b)\partial_w\delta(z, w), \\
    [a(z), K] &= 0
    \end{aligned}
  \end{equation*}
  In terms of the $\lambda$-bracket, the commutation relations become:
  \begin{equation*}
    \begin{aligned}
    [a(z)_{\lambda}b(z)] &= [a(z), b(z)] + (a| b)K\lambda, \\
    [a(z)_{\lambda}K] &= 0.
    \end{aligned}
  \end{equation*}
  By \Cref{thr:1}, $\{a(z) \mid a \in \mathfrak{g}\} \cup \{K\}$ is a local family.
  Therefore, $(\hat{\mathfrak{g}}, \{a(z) \mid a \in \mathfrak{g}\} \cup \{K\})$ is a formal distribution Lie superalgebra.
  Moreover, we can verify directly that $(\hat{\mathfrak{g}}, \{a(z) \mid a \in \mathfrak{g}\} \cup \{K\}, -\partial_t)$ is regular.
  In a similar way to the Virasoro Lie conformal algebra, we obtain the current Lie conformal superalgebra
  \begin{equation*}
    \Cur(\mathfrak{g}) = \mathbb{C}[\partial]\mathfrak{g} \oplus \mathbb{C}K.
  \end{equation*}
  Let $\mathfrak{g}$ be an abelian Lie superalgebra.
  In that case, $\Cur(\mathfrak{g})$ is known as the conformal algebra of free bosons associated with the free bosons algebra $\hat{\mathfrak{g}}$, the latter being endowed with the relations $[at^m, bt^n] = m(a| b)\delta_{m, -n}K$. 
\end{example}

\begin{example}[Fermionic Lie conformal superalgebra]
  \label{exa:3}
  Let $V = V_{\zero} \oplus V_{\one}$ be a superspace. A bilinear form $\langle\bullet, \bullet\rangle: V\times V \to \mathbb{C}$ is antisupersymmetric if it satisfies the relation
  \begin{equation*}
    \langle a, b\rangle = -(-1)^{p(a)}\langle b, a\rangle
  \end{equation*}
  for homogeneous elements $a, b\in V$.
  Alternatively, we can define an antisymmetric bilinear form  as a bilinear form that vanishes on $V_{\zero} \oplus V_{\one}$ and $V_{\one}\oplus V_{\zero}$, is antisymmetric on the even part $V_{\zero}\oplus V_{\zero}$ and symmetric on the odd part $V_{\one}\oplus V_{\one}$.

  The Clifford affinization of $V$ is defined by
  \begin{equation*}
    \widehat{V} = V[t, t^{-1}] \oplus \mathbb{C}K
  \end{equation*}
  with the superbracket
  \begin{equation}
    \label{eq:12}
    \begin{aligned}
      [at^m, bt^n] &= \delta_{m, -n - 1}\langle a, b\rangle K, \\
      [at^m, K] &= 0,
    \end{aligned}
  \end{equation}
  where $a, b\in V$ and with parity given by $p(at^n) = p(a)$ and $p(K) = \zero$.
  This makes $\widehat{V}$ into a Lie superalgebra.
  We now construct $\widehat{V}$-valued formal distributions by setting
  \begin{equation*}
    a(z) = \sum_{m \in \mathbb{Z}}at^mz^{-m - 1},
  \end{equation*}
  for $a \in V$.
  In terms of formal distributions, the commutation relations become:
  \begin{equation*}
    \begin{aligned}
      [a(z), b(w)] &= \langle a, b\rangle K\delta(z, w), \\
      [a(z),K] &= 0.
    \end{aligned}
  \end{equation*}
  In terms of the $\lambda$-bracket, the commutation relations become:
  \begin{equation*}
    \begin{aligned}
      [a(z)_{\lambda}b(z)] &= \langle a, b\rangle K, \\
      [a(z)_{\lambda}K] &= 0.
    \end{aligned}
  \end{equation*}
  As before, we obtain a regular formal distribution Lie superalgebra $(\widehat{V}, \{a(z) \mid a \in V\} \cup \{K\}, -\partial_t)$ from which we obtain the Fermionic Lie conformal superalgebra
  \begin{equation*}
    F(V) = \mathbb{C}[\partial]V \oplus \mathbb{C}K.
  \end{equation*}
\end{example}

\subsection{Fields over vector spaces}
\label{sec:fields-over-vector}

In this section, we fix a vector superspace $V = V_{\zero} \oplus V_{\one}$ and all formal distributions are $\End(V)$-valued unless otherwise stated.
Let $a(z)$ be a formal distribution.
Set
\begin{align*}
  a(z)_+&=\sum_{n\le -1}a_{(n)}z^{-n-1} \\
  a(z)_-&=\sum_{n\ge 0}a_{(n)}z^{-n-1}.
\end{align*}

Let $a(z)$ and $b(z)$ be two formal distributions.
We define the normal product between $a(z)$ and $b(z)$ as the following formal distribution in two variables
\begin{equation*}
  :a(z)b(w):=a(z)_+b(w)+p(a,b)b(w)a(z)_-.
\end{equation*}

\begin{theorem}[{\cite[Proposition 3.2.3]{nozaradan_introduction_2008}}]
  \label{thr:7}
  Let $(a(z), b(z))$ be a pair of local formal distributions.
  The following identities are known as the operator product expansion of $a(z)$ and $b(w)$:
  \begin{align*}
    a(z)b(w) &= \sum_{j \in \mathbb{N}}a(w)_{(j)}b(w)i_{z, w}\frac{1}{(z - w)^{j + 1}} + :a(z)b(w):, \\
    p(a, b)b(w)a(z) &= \sum_{j \in \mathbb{N}}a(w)_{(j)}b(w)i_{w, z}\frac{1}{(z - w)^{j + 1}} + :a(z)b(w):.
  \end{align*}
\end{theorem}

A formal distribution $a(z)$ is a field if
\begin{equation*}
  a(z)b = \sum_{n \in \mathbb{Z}}a_{(n)}bz^{-n - 1} \in V((z))
\end{equation*}
for all $b \in V$.
The set of fields on $V$ is denoted by $\mathcal{F}(V)$.
Note that
\begin{equation*}
  \mathcal{F}(V) = \Hom(V, V((z))).
\end{equation*}
Therefore, we can define a field $a(z)$ by defining $a(z)b \in V((z))$ for $b \in V$.

\begin{proposition}[{\cite[Proposition 3.3.2]{nozaradan_introduction_2008}}]
  \label{prp:5}
  Let $a(z), b(z)\in \mathcal{F}(V)$ be two fields.
  Then $:a(z)b(z):\in \End(V)[[z,z^{-1}]]$ is again a field where $:a(z)b(z):$ is defined by
  \begin{equation*}
    :a(z)b(z):c = a(z)_+b(z)c + p(a, b)b(z)a(z)_-c,
  \end{equation*}
  for $c \in V$.
\end{proposition}

We thus defined the notion of Normal Ordered Product $:a(z)b(z):$ between fields $a(z), b(z) \in \mathcal{F}(V)$.
In general, the operation of normal ordered product is neither commutative nor associative. We follow the convention that the normal ordered ordering is read from right to left, so that by definition
\begin{equation*}
  :a(z)b(z)c(z): = :a(z)(:b(z)c(z):):.
\end{equation*}
The normal ordered product between $a(z)$ and $b(z)$ is explicitly given in the following proposition.

\begin{lemma}[{\cite[Proposition 3.3.3]{nozaradan_introduction_2008}}]
  \label{lmm:2}
  Let $a(z)$ and $b(z)$ be two fields. Their normal ordered product is written explicitly as
  \begin{equation*}
    :a(z)b(z): = \sum_{j \in \mathbb{Z}}:ab:_{(j)}z^{-j - 1},
  \end{equation*}
  with
  \begin{equation*}
    :ab:_{(j)}c = \sum_{n \le -1}a_{(n)}b_{(j - n - 1)}c + p(a, b)\sum_{n \ge 0}b_{(j - n - 1)}a_{(n)}c,
  \end{equation*}
  for $c \in V$.
\end{lemma}

\begin{lemma}
  \label{lmm:3}
  Consider $s$ fields $a^1(z), \dots, a^s(z) \in \mathcal{F}(V)$ with $s \ge 2$ and $v \in V$.
  For $l \in \mathbb{Z}$,
  \begin{equation*}
    :a^1(z)a^2(z)\dots a^s(z):_{(l)}v = \sum_{n_1, \dots, n_{s - 1} \in \mathbb{N}}\sum_{k = 0}^{s - 1}R^{l, k}_{n_1, \dots, n_{s - 1}}(a^1(z), \dots, a^s(z))v
  \end{equation*}
  where
  \begin{align*}
    &R^{l, k}_{n_1, \dots, n_{s - 1}}(a^1(z), \dots, a^s(z)) =\\
    &\sum_{\substack{1 \le i_1 < \dots < i_k \le s - 1 \\ 1 \le j_1 < \dots < j_{s - 1 - k} \le s - 1 \\ \{i_1, \dots, i_k\} \cup \{j_1, \dots, j_{s - 1 - k}\} = \{1, \dots, s - 1\}}}a^{j_1}_{(-n_{j_1} - 1)}\dots a^{j_{s - 1 - k}}_{(-n_{j_{s - 1 - k}} - 1)}a^s_{(l - k - \sum_{r = 1}^k n_{i_r} + \sum_{r = 1}^{s - 1 - k}n_{j_r})}a^{i_k}_{(n_{i_k})}\dots a^{i_1}_{(n_{i_1})}.
  \end{align*}
\end{lemma}

\begin{proof}
  This follows from \Cref{lmm:2} and induction on $s$.
\end{proof}

Now let's extend the $j$-products.
For $j \in \mathbb{N}$, set
\begin{equation}
  \label{eq:13}
  a(w)_{(-1 - j)}b(w) = \frac{:(\partial^j_wa(w))b(w):}{j!}.
\end{equation}

\begin{theorem}[{\cite[Proposition 3.4.3]{nozaradan_introduction_2008}}]
  \label{thr:8}
  The $j$-products \eqref{eq:4} and \eqref{eq:1} are special cases of the following generalized $j$-product defined by
  \begin{equation*}
    a(w)_{(j)}b(w)c = \Res_z(i_{z, w}(z - w)^ja(z)b(w)c - p(a,b)i_{w, z}(z - w)^jb(w)a(z)c),
  \end{equation*}
  for $c \in V$.
\end{theorem}

The usual properties of $j$-products for $j \in \mathbb{N}$ carry over to the generalized $j$-products.

\begin{proposition}[{\cite[Proposition 3.4.4]{nozaradan_introduction_2008}}]
  \label{prp:6}
  Let $a(z)$ and $b(z)$ be formal distributions.
  For all $j \in \mathbb{Z}$, we have:
  \begin{enumerate}
  \item $(\partial a(z))_{(j)}b(z) = -ja(z)_{(j - 1)}b(z)$;
  \item $\partial(a(z)_{(j)}b(z)) = (\partial_za(z))_{(j)}b(z) + a(z)_{(j)}\partial_zb(z)$.
  \end{enumerate}
\end{proposition}

\begin{lemma}[Dong's lemma {\cite[Lemma 3.2]{kac_vertex_1998}}]
  \label{lmm:4}
  If $a(z)$, $b(z)$ and $c(z)$ are pairwise mutually local fields, then $(a(z),b(z)_{(n)}c(z))$ is a local pair of fields as well for $n\in \mathbb{Z}$.
\end{lemma}

Let $L(z)$ be a formal distribution.
An eigendistribution of weight $\Delta_a$ with respect to $L(z)$ is a formal distribution $a(z)$ satisfying
\begin{equation}
  \label{eq:14}
  [L_{\lambda}a] = (\partial + \Delta_a\lambda)a + O(\lambda^2),
\end{equation}
where $O(\lambda^2)$ denotes sums of terms of monomials $s\lambda^j$ with $j \ge 2$.
If $L$ is a Virasoro formal distribution, $\Delta_a$ is called the conformal weight.
Clearly, $a(z)$ is an eigendistribution in the sense that it is an eigenvector of the endomorphism $L_{(1)}$ whose action is defined by $L_{(1)}(b) = L_{(1)}b$.
Indeed, \eqref{eq:14} implies $L_{(1)}a = \Delta_aa$.
Note that, by definition, a Virasoro formal distribution $L(z)$ is an eigendistribution of conformal weight $2$ with respect to itself.

\begin{theorem}[{\cite[Proposition 3.7.4]{nozaradan_introduction_2008}}]
  \label{thr:9}
  If $a(z)$ and $b(z)$ have weights $\Delta_a$ and $\Delta_b$ with an even formal distribution $L$, then $\Delta_{a_{(n)}b} = \Delta_a + \Delta_b - n - 1$ with respect to $L$.
  In particular, $\Delta_{:ab:} = \Delta_a + \Delta_b$ and $\Delta_{\partial a} = \Delta_a + 1$.
\end{theorem}

The expansion of an eigendistribution $a(z)$ of weight $\Delta_a$ is often adapted as follows:
\begin{equation*}
  a(z) = \sum_{n \in \mathbb{Z} - \Delta_a}a_nz^{-n - \Delta_a}.
\end{equation*}
This justifies the way we wrote the Virasoro formal distribution when we defined the Virasoro Lie conformal algebra.
By comparison with the usual way of writing $a(z) = \sum_{m \in \mathbb{Z}}a_{(m)}z^{-m - 1}$, we must have
\begin{equation*}
  a_{(m)} = a_{m - \Delta_a + 1},
\end{equation*}
or the other way
\begin{equation*}
  a_n = a_{(n + \Delta_a - 1)}.
\end{equation*}
One of the interesting features of this change of notation is that it reveals the gradation of the superbracket.

\begin{theorem}[{\cite[Proposition 3.7.6]{nozaradan_introduction_2008}}]
  \label{thr:10}
  In the new notation, we can write
  \begin{equation*}
    [a_m, b_n] = \sum_{j \in \mathbb{N}}\binom{m + \Delta_a - 1}{j}(a_{(j)}b)_{m + n}.
  \end{equation*}
\end{theorem}

An eigendistribution $a(z)$ is called primary of conformal weight $\Delta_a$ if
\begin{equation*}
  [L_\lambda a]=(\partial +\Delta_a\lambda)a,
\end{equation*}
where $L(z)$ is a Virasoro formal distribution.

Let's fix an operator $T\in \End(V)$.
A formal distribution $a(z)$ is covariant with respect to $T$ if
\begin{equation*}
  [T, a(z)] = \partial_za(z).
\end{equation*}

\begin{theorem}[{\cite[Lemma 1]{callegaro_introduction_2017}}]
  \label{thr:11}
  Assume that $\vac \in V$ is such that $T\vac = 0$.
  Then
  \begin{enumerate}
  \item For any translation covariant field $a(z)$ we have $a(z)\vac \in V[[z]]$.
  \item Set $a = a_{(-1)}\vac$.
    Then
    \begin{equation*}
      a(z)\vac = e^{Tz}a = \sum_{n = 0}^\infty\frac{T^n(a)}{n!}z^n.
    \end{equation*}
  \end{enumerate}
\end{theorem}

Let
\begin{equation*}
  \mathcal{F}_{\tc} = \{a(z) \in \mathcal{F}(V) \mid [T, a(z)] = \partial_za(z)\}
\end{equation*}
be the subspace of translation covariant fields.

\begin{lemma}[{\cite[Lemma 3]{callegaro_introduction_2017}}]
  \label{lmm:5}
  $\mathcal{F}_{\tc}$ contains $\Id_V$, it is $\partial_z$-invariant and is closed under all $n$-products, i.e., $\partial_za(z), a(z)_{(n)}b(z) \in \mathcal{F}_{\tc}$ for $a(z), b(z) \in \mathcal{F}_{\tc}$ and $n \in \mathbb{Z}$.
\end{lemma}

By \Cref{thr:11}, we can define a linear map
\begin{align*}
  \fs: \mathcal{F}_{\tc} &\to V \\
  \fs(a(z)) &= a(z)\vac|_{z = 0},
\end{align*}
called the field-state correspondence.

\begin{lemma}[{\cite[Proposition 4.3.2]{nozaradan_introduction_2008}}]
  \label{lmm:6}
  Let $A$ be a linear operator on a linear space $V$.
  The formal differential equation
  \begin{equation*}
    \frac{df(z)}{dz} = Af(z) \quad (f(z) \in V[[z]])
  \end{equation*}
  admits a unique solution, given an initial condition $f(0)=f_0$.
\end{lemma}

\begin{theorem}
  \label{thr:12}
  Let $a(z), b(z) \in \mathcal{F}_{\tc}$, $a = \fs(a(z))$, $b = \fs(b(z))$ and $n \in \mathbb{Z}$.
  Write $a(z) = \sum_{j \in \mathbb{Z}}a_{(j)}z^{-j - 1}$ and $b(z) = \sum_{j \in \mathbb{Z}}b_{(j)}z^{-j - 1}$.
  Then
  \begin{enumerate}
  \item $\fs(\Id_V) = \vac$;
  \item $\fs(\partial_za(z)) = Ta$;
  \item ($n$-product identity) $\fs(a(z)_{(n)}b(z)) = a_{(n)}b$;
  \item $T(a_{(n)}b) = -na_{(n - 1)}b + a_{(n)}Tb$;
  \item $e^{Tw}a(z)e^{-Tw} = i_{z, w}a(z + w)$;
  \item (Borcherds identity). If $a(z)$ and $b(z)$ are local then
    \begin{equation*}
      i_{z,w}(z-w)^na(z)b(w)c-p(a,b)i_{w,z}(z-w)^nb(w)a(z)c=\sum_{j\in \mathbb{N}}a(w)_{(n+j)}b(w)\frac{\partial^j_w\delta(z,w)}{j!}c,
    \end{equation*}
    for $c\in V$;
  \item (Skewsymmetry) If $a(z)$ and $b(z)$ are local then
    \begin{equation*}
      a(z)b=p(a,b)e^{Tz}b(-z)a.
    \end{equation*}
  \end{enumerate}
\end{theorem}

\begin{proof}
  \begin{enumerate}
  \item Clear.
  \item
    \begin{align*}
      \fs(\partial_za(z)) &= [T, a(z)]\vac|_{z = 0} \\
      &= (Ta(z) - p(a(z), T)a(z)T)\vac|_{z = 0} \\
      &= Ta.
    \end{align*}
  \item By definition, we have
    \begin{equation*}
      \fs(a(z)_{(n)}b(z)) = a(z)_{(n)}b(z)\vac|_{z = 0},
    \end{equation*}
    and the right hand side, by \Cref{thr:8}, is equal to
    \begin{equation*}
      \Res_w(a(w)b(z)i_{w, z}(w - z)^n\vac - p(a, b)b(z)a(w)i_{z, w}(w - z)^n\vac)|_{z = 0}.
    \end{equation*}
    Now, since $a(w)\vac \in V[[w]]$ and $i_{z, w}(w-z)^n$ has only nonnegative powers of $w$, we have
    \begin{equation*}
      \Res_w(b(z)a(w)i_{z, w}(w - z)^n\vac) = 0.
    \end{equation*}
    For the first term, since $b(z)\vac \in V[[z]]$, we can let $z = 0$ before we calculate the residue, which gives
    \begin{equation*}
      \Res_w(a(w)b(z)i_{w, z}(w - z)^n\vac)|_{z = 0} = \Res_w(a(w)bw^n) = a_{(n)}b.
    \end{equation*}
  \item This follows from $[T, a_{(n)}] = -na_{(n - 1)}$ which is equivalent to translation covariance of the field $a(z)$.
  \item Set $b_1(z, w) = i_{z, w}a(z + w)$ and $b_2(z, w) = e^{Tw}a(z)e^{-Tw}$.
    By \Cref{thr:2} and translation covariance,
    \begin{align*}
      \frac{\partial b_1(z, w)}{\partial w} &= \sum_{j \in \mathbb{N}}\frac{\partial^{j + 1}_za(z)}{j!}w^j = \sum_{j \in \mathbb{N}}\frac{\partial^j[T, a(z)]}{j!} = [T, b_1(z, w)], \\
      b_1(z, 0) &= a(z), \\
      \frac{\partial b_1(z, w)}{\partial w} &= Te^{Tw}a(z)e^{-Tw} + e^{Tw}a(z)(-T)e^{-Tw} = [T, b_2(z, w)], \\
      b_2(z, 0) &= a(z).
    \end{align*}
    By \Cref{lmm:6}, $b_1(z, w) = b_2(z, w)$.
  \item The left hand side of the Borcherds identity is a local formal distribution in $z$ and $w$ applied to $c$.
    Apply \Cref{thr:1} to it to get that it is equal to
    \begin{equation*}
      \sum_{j \in \mathbb{N}} c^j(w)\frac{\partial^j_w\delta(z,w)}{j!}c,
    \end{equation*}
    where
    \begin{align*}
      c^j(w)c &= (\Res_z(z - w)^j(i_{z, w}(z - w)^na(z)b(w) - p(a, b)i_{w, z}(z - w)^nb(w)a(z)))c \\
      &= \Res_z(i_{z, w}(z - w)^{n + j}a(z)b(w)c - p(a, b)i_{w, z}(z - w)^{n + j}b(w)a(z)c) \\
      &= a(w)_{(n + j)}b(w)c.
    \end{align*}
  \item By locality, there is $N \in \mathbb{Z}$ such that
    \begin{equation*}
      (z - w)^Na(z)b(w) = p(a, b)(z - w)^Nb(w)a(z).
    \end{equation*}
    Apply $\vac$ to both sides; by \Cref{thr:11}, we get
    \begin{equation*}
      (z - w)^Na(z)e^{Tw}b = p(a, b)(z - w)^Nb(w)e^{Tz}a.
    \end{equation*}
    Now use (f) and \Cref{thr:2}:
    \begin{equation*}
      RHS = p(a, b)(z - w)^Ne^{Tz}e^{-Tz}b(w)e^{Tz}a = p(a, b)(z - w)^Ne^{Tz}i_{w, z}b(w - z)a.
    \end{equation*}
    For $N$ big enough, this is a formal power series in $(z - w)$, so we can set $w = 0$ and get
    \begin{equation*}
      LHS = z^Na(z)b = p(a, b)e^{Tz}z^Nb(-z)a = RHS,
    \end{equation*}
    which proves the desired formula. \qedhere
  \end{enumerate}
\end{proof}

\begin{lemma}[{\cite[Lemma 3]{callegaro_introduction_2017}}]
  \label{lmm:7}
  Let $\mathcal{F}' \subseteq \mathcal{F}_{\tc}$ and $a(z) \in \mathcal{F}_{\tc}$.
  Assume that
  \begin{enumerate}
  \item $\fs(a(z)) = 0$;
  \item $a(z)$ is local with any element in $\mathcal{F}'$;
  \item $\fs(\mathcal{F}') = V$.
  \end{enumerate}
  Then $a(z) = 0$.
\end{lemma}

\begin{proof}
  Let $b(z) \in \mathcal{F}'$.
  By the locality of $a(z)$ and $b(z)$, we have $(z - w)^N[a(z), b(w)] = 0$, for some $N \in \mathbb{N}$.
  Apply $\vac$ to both sides to get
  \begin{equation*}
    (z - w)^Na(z)b(w)\vac = \pm(z - w)^Nb(w)a(z)\vac.
  \end{equation*}
  By the property (i), we have $a_{(-1)}\vac = 0$ and $a(z)$ is translation covariant, hence by \Cref{thr:11}(i), $b(w)\vac \in V[[w]]$, so we can let $w = 0$ and get $z^Na(z)b = 0$, which means $a_{(n)}b = 0$ for any $n \in \mathbb{Z}$.
  This is true for any $b \in V$ by the property (c).
  So in fact, we have $a(z) = 0$.
\end{proof}

\subsection{Vertex algebras}
\label{sec:vertex-algebras}

A vertex algebra is the data consisting of four elements $(V, \vac, T, Y)$ satisfying the following properties:
\begin{enumerate}
\item $V$ is a superspace called the state space;
\item $\vac \in V_{\zero}$ is called the vacuum vector;
\item $T \in \End(V)_{\zero}$ is called the translation operator;
\item $Y: V \to \mathcal{F}(V)$ is a linear and parity preserving map called the state-field correspondence.
\end{enumerate}
By parity preserving map we mean that for $a \in V$ homogeneous, $p(a_{(n)}) = p(a)$ for all $n \in \mathbb{Z}$.
The operator $(a) = Y(a, z) \in \End(V)[[z^{\pm 1}]]$ for $a \in V$ is sometimes called a vertex operator.
The data must satisfy the following axioms for $a \in V$:
\begin{enumerate}
\item (Vacuum axiom)
  \begin{align*}
    Y(\vac,z) &= \Id_V, \\
    Y(a, z)\vac &\in V[[z]], \\
    Y(a, z)\vac|_{z = 0} &= a, \\
    T\vac &= 0;
  \end{align*}
\item (Translation covariance) $[T, Y(a, z)] = \partial_zY(a, z)$;
\item (Locality) $\{Y(a, z) \mid a \in V\}$ is a local family of fields.
\end{enumerate}

\begin{remark}
  \label{rmk:7}
  Writing $Y(a, z) = \sum_{n \in \mathbb{Z}}a_{(n)}z^{-n - 1}$ for $a \in V$, the first two vertex algebra axioms imply that for $n \in \mathbb{Z}$:
  \begin{align*}
    \vac_{(n)}a &= a_{(n)}\vac = \delta_{n, -1}a, \\  
    [T, a_{(n)}] &= -na_{(n - 1)}.
  \end{align*}
\end{remark}

\begin{remark}
  \label{rmk:8}
  The translation covariance axiom together with \Cref{thr:11} permit us to express $T$ by
  \begin{equation}
    \label{eq:15}
    Ta = a_{(-2)}\vac.
  \end{equation}
  As a consequence, the data of the translation operator $T$ is redundant.
  The original definition with $T$ appears to be more natural, though.
\end{remark}

\begin{remark}
  \label{rmk:9}
  Even though a vertex algebra might be a superspace, we prefer the term vertex algebra rather than vertex superalgebra.
\end{remark}

A vertex algebra homomorphism $f: (V, \vac, T, Y) \to (V', \vac', T', Y')$ is a linear map $f: V \to V'$ such that $f(\vac) = \vac'$ and for $a, b \in V$,
\begin{equation*}
  f(Y(a, z)b) = \sum_{n \in \mathbb{Z}}f(a_{(n)}b)z^{-n - 1} = \sum_{n \in \mathbb{Z}}f(a)_{(n)}f(b)z^{-n - 1} = Y'(f(a), z)f(b).
\end{equation*}
We obtain the category of vertex algebras.

\begin{remark}
  \label{rmk:10}
  It is easy to verify that category of vertex algebras is abelian in a natural way.
\end{remark}

\begin{example}[Commutative vertex algebras]
  \label{exa:4}
  A vertex algebra $V$ is called commutative if all vertex operators $Y(a, z)$, $a \in V$ commute with each other.
  
  Suppose we are given a commutative vertex algebra $V$.
  Then for $a, b \in A$,
  \begin{equation*}
    Y(a, z)b = Y(a, z)Y(b, w)\vac|_{w = 0} = Y(b, w)Y(a, z)\vac|_{w = 0}.
  \end{equation*}
  But by the vacuum axiom, the last expression has no negative power of $z$.
  Therefore $Y(a, z)b \in V[[z]]$ for $a, b \in V$ so $Y(a, z) \in \End(V)[[z]]$ for $a \in V$.
  Conversely, suppose that we are given a vertex algebra $V$ in which $Y(a, z) \in \End(V)[[z]]$ for all $a \in V$.
  Observe that if the equality $(z - w)^Nf_1(z, w) = (z - w)^Nf_2(z, w)$ holds for $f_1, f_2 \in \mathbb{C}[[z, w]]$ and $N \in \mathbb{N}$, then necessarily $f_1(z, w) = f_w(z, w)$.
  Therefore, we obtain that $[Y(a, z), Y(b, w)] = 0$ for all $a, b \in V$, so $V$ is commutative.
  
  Thus, a commutative vertex algebra may be defined as one in which all $Y(a, z)$ belong to $\End(V)[[z]]$.

  Denote by $Y_a$ the endomorphism of a commutative vertex algebra $V$ which is the constant term of $Y(a, z)$ for $a \in V$, and define a bilinear operation $\circ$ on $V$ by setting $a\circ b=Y_ab$.
  By construction, $Y_aY_b = Y_bY_a$.
  This implies both commutativity and associativity of $\circ$.
  Furthermore, the vacuum vector $\vac$ is a unit, and the operator $T$ is a derivation with respect to this product.
  Thus, we have the structure of a commutative algebra with a derivation on $V$.

  Conversely, let $V$ be an associative and commutative algebra with a unit $1$ and a derivation $T$.
  We call this object a differential algebra $(V, T)$.
  Then $V$ can become a vertex algebra.
  Take $\vac = 1$, and set
  \begin{equation*}
    Y(a, z) = e^{Tz}a.
  \end{equation*}
  It is straightforward to check that all the axioms of a commutative vertex algebra are satisfied.

  Therefore, we obtain an isomorphism between the category of differential algebras and the category of commutative vertex algebras.
\end{example}

The theory done in the previous sections translates into the language of vertex algebras as the following theorem shows.

\begin{theorem}[Properties of vertex algebras]
  \label{thr:13}
  Let $V$ be a vertex algebra.
  For $a, b, c \in V$ and $m, n \in \mathbb{Z}$:
  \begin{enumerate}
  \item $Y: V \to \mathcal{F}(V)$ is injective;
  \item $Y(a, z)\vac \in V[[z]]$ and $Y(a, z)\vac = e^{Tz}a$ so $T^ja = j!a_{(-j - 1)}\vac$ for $j \in \mathbb{N}$;
  \item $Y(Ta, z) = \partial_zY(a, z)$ or, equivalently, $(Ta)_{(n)} = -na_{(n - 1)}$;
  \item ($n$-product identity) $Y(a, z)_{(n)}Y(b, z) = Y(a_{(n)}b, z)$;
  \item$ [a_{(m)}, b_{(n)}] = \sum_{j \in \mathbb{N}}\binom{m}{j}(a_{(j)}b)_{(m + n - j)}$;
  \item ($T$ as an even derivation) $T(Y(a, z)b) = Y(Ta, z)b + Y(a, z)Tb$;
  \item $e^{Tw}Y(a,z)e^{-Tw}=i_{z,w}Y(a,z+w)$;
  \item (Borcherds identity)
    \begin{equation*}
      i_{z, w}(z - w)^nY(a, z)Y(b, w)c - p(a, b)i_{w, z}(z - w)^nY(b, w)Y(a, z)c = \sum_{j \in \mathbb{N}}Y(a_{(n + j)}b, w)\frac{\partial^j_w\delta(z, w)}{j!}c;
    \end{equation*}
  \item (Skew-symmetry) $Y(a, z)b = p(a, b)e^{Tz}Y(b, -z)a$;
  \item $(a_{(n)}b)_{(m)}c = \sum_{j \in \mathbb{N}}(-1)^j\binom{n}{j}(a_{(n - j)}b_{(m + j)}c - (-1)^np(a, b)b_{(n + m - j)}a_{(j)}c)$.
  \end{enumerate}
\end{theorem}

\begin{theorem}[Original Borcherds identity]
  \label{thr:14}
  Let $V$ be a vertex algebra.
  For $a, b, c \in V$,
  \begin{equation*}
    \sum_{j \in \mathbb{N}}(-1)^j\binom{n}{j}\left(a_{(m + n - j)}(b_{(k + j)}c) - (-1)^np(a, b)b_{(n + k - j)}(a_{(m + j)}c)\right) = \sum_{j \in \mathbb{N}}\binom{m}{j}(a_{(n + j)}b)_{(m + k - j)}c,
  \end{equation*}
  or, equivalently, for all $F(z, w) = z^mw(z - w)^k$ where $m, n, k \in \mathbb{Z}$,
  \begin{equation*}
    \begin{split}
      &\Res_{z - w}(Y(Y(a, z - w)b, w)ci_{w, z - w}F(z, w)) = \\
      &\Res_z(Y(a, z)Y(b, w)ci_{z, w}F(z, w)) - \Res_z(Y(b, w)Y(a, z)ci_{w, z}F(z, w)).
    \end{split}
  \end{equation*}
\end{theorem}

Let $V$ be a vertex algebra.
A vertex subalgebra of $V$ is a subspace of $W$ of $V$ which contains $\vac$ and such that $Y(a, z)b \in W((z))$ for $a, b\in W$.
Because $Ta = a_{(-2)}\vac$ and $\vac\in W$, this implies that $TW \subseteq W$.
Thus, $(W, \vac, T|_{W}: W \to W, Y|_{W}: W \to \mathcal{F}(W))$ is a vertex algebra in its own right.
Let $S \subseteq V$ be a subset.
The vertex subalgebra generated by $S$ is the smallest vertex subalgebra containing $S$ which is the intersection of all subalgebras containing $S$.
It is denoted $\langle S \rangle$ and we can prove that
\begin{equation*}
  \langle S \rangle = \vspan \{a^1_{(n_1)}\dots a^s_{(n_s)}\vac \mid s \in \mathbb{N}, a^i \in S, n_1, \dots, n_s \in \mathbb{Z}\}.
\end{equation*}
The vertex algebra $V$ is strongly generated by $S \subseteq V$ if
\begin{equation*}
  V = \vspan \{a^1_{(-n_1 - 1)}\dots a^s_{(-n_s - 1)}\vac \mid s \in \mathbb{N}, a^i \in S, n_1, \dots, n_s \in \mathbb{N}\}.
\end{equation*}

An ideal of a vertex algebra $V$ is a subspace $I$ of $V$ such that $Y(a, z)b \in I((z))$ and $Y(b, z)a \in I((z))$ for all $a \in V$ and $b \in I$.
For example, the kernel of vertex algebra homomorphism is an ideal.

\begin{remark}
  \label{rmk:11}
  Note that a right ideal $I$ is automatically $T$-invariant ($TI \subseteq I$) because of \eqref{eq:15}.
  Also, right ideals and $T$-invariant left ideals are automatically two-sided ideals because of skewsymmetry.
  However, to prove that a subspace is an ideal, it is usually easier to check that it is $T$-invariant and a left ideal.
\end{remark}

It follows that for any ideal $I$, $V/I$ inherits a natural quotient vertex algebra structure $(V/I, \vac + I,T_{V/I}: V/I \to V/I, Y_{V/I}: V/I \to \mathcal{F}(V/I))$.
Let $S \subseteq V$ be a subset.
The ideal generated by $S$ is the smallest ideal containing $S$ which is the intersection of all ideals containing $S$.
It is denoted by $(S)$ and we can prove that
\begin{equation*}
  (S) = \vspan\{b_{(n)}T^ma \mid b \in V, n \in \mathbb{Z}, m \in \mathbb{N}, a \in S\}.
\end{equation*}

We don't examples of vertex algebras other than the ones coming from differential algebras.
It turns out is not an easy task to construct nontrivial vertex algebras.
We need a preliminary concept to do that task. 
A pre-vertex algebra is a quadruple $(V, \vac, T, \mathcal{F})$ where $V = V_{\zero} \oplus V_{\one}$ is a vector superspace, $\vac \in V_{\zero}$, $T \in \End(V)_{\zero}$ and $\mathcal{F} = \{a^j(z) = \sum_{n \in \mathbb{Z}}a^j_{(n)}z^{-n - 1}\}_{j \in J}$ is a collection of $\End(V)$-valued fields such that for $j \in J$, all $a^j_{(n)}$ for $n \in \mathbb{Z}$ have the same parity.
The above data satisfies the following axioms
\begin{enumerate}
\item (Vacuum axiom) $T\vac = 0$;
\item (Translation covariance) $[T, a^j(z)] = \partial_za^j(z)$ for all $j \in J$;
\item (Locality) $a^i(z)$ and $a^j(z)$ are mutually local for all $i, j \in J$;
\item (Completeness) $\vspan\{a^{j_1}_{(n_1)}\dots a^{j_s}_{(n_s)}\vac \mid s \in \mathbb{N}, j_i \in J, n_i \in \mathbb{Z}\} = V$.
\end{enumerate}

Let $(V, \vac, T, \mathcal{F})$ be a pre-vertex algebra.
Define the following subspaces of $\mathcal{F}(V)$:
\begin{align*}
  \mathcal{F}_{\min} &= \vspan\{a^{j_1}(z)_{(n_1)}(a^{j_2}(z)_{(n_2)}\dots(a^{j_s}(z)_{(n_s)}\Id_V)\dots) \mid s \in \mathbb{N}, n_i \in \mathbb{Z}, j_i \in J\}, \\
  \mathcal{F}_{\max} &= \{a(z) \in \mathcal{F}(V) \mid [T, a(z)] = \partial_za(z)\text{ and for }j \in J, (a(z),a^j(z))\text{ is a local pair}\}.
\end{align*}
We have inclusions
\begin{equation*}
  \mathcal{F} \subseteq \mathcal{F}_{\min} \subseteq \mathcal{F}_{\max} \subseteq \mathcal{F}_{\tc}.
\end{equation*}
The first inclusion is because for $a(z) \in \mathcal{F}$, $a(z)_{(-1)}\Id_V = a(z) \in \mathcal{F}_{\min}$.
The second inclusion is by \Cref{lmm:7} and Dong's Lemma.
The last inclusion is by definition.

Now we come to a very fundamental theorem, which allows us to construct noncommutative vertex algebras and is the backbone of several of our most important examples of vertex algebras.
\begin{theorem}[Extension theorem]
  \label{thr:15}
  Let $(V, \vac, T, \mathcal{F})$ be a pre-vertex algebra and $\mathcal{F}_{\min}, \mathcal{F}_{\max}$ defined as above.
  Then:
  \begin{enumerate}
  \item $\mathcal{F}_{\min} = \mathcal{F}_{\max}$;
  \item The linear map
    \begin{align*}
      \fs: \mathcal{F}_{\max} &\to V \\
      \fs(a(z)) &= a(z)\vac|_{z = 0}
    \end{align*}
    is well-defined and bijective.
    Denote by $Y: V \to \mathcal{F}(V)$ the inverse map;
  \item $(V, \vac, T, Y)$ is a vertex algebra with $Y: V \to \mathcal{F}(V)$ given explicitly by
    \begin{equation}
      \label{eq:16}
      Y(a^{j_1}_{(n_1)}a^{j_2}_{(n_2)}\dots a^{j_s}_{(n_s)}\vac) = a^{j_1}(z)_{(n_1)}(a^{j_2}(z)_{(n_2)}\dots (a^{j_s}(z)_{(n_s)}\Id_V)\dots),
    \end{equation}
    for $s \in \mathbb{N}$, $j_1, \dots, j_s \in J$ and $n_1, \dots, n_s \in \mathbb{Z}$;
  \item The vertex algebra $V$ is generated by $\{a^j_{(-1)}\vac \mid j \in J\}$;
  \item The only vertex algebra structure on $V$ with $Y(a^j_{(-1)}\vac, z) = a^j(z)$ for $j \in J$ is the one given by \eqref{eq:16}.
  \end{enumerate}
\end{theorem}

\begin{proof}
  By \Cref{thr:11}, the map $\fs$ is well defined as was noted already in \Cref{sec:fields-over-vector}.
  By \Cref{thr:12}(i) and \Cref{thr:12}(iii), $\fs|_{\mathcal{F}_{\min}}: \mathcal{F}_{\mathcal{F}_{\min}} \to V$ is given by
  \begin{equation}
    \label{eq:17}
    \fs|_{\mathcal{F}_{\min}}(a^{j_1}(z)_{(n_1)}(a^{j_2}(z)_{(n_2)}\dots (a^{j_s}(z)_{(n_s)}\Id_V)\dots)) = a^{j_1}_{(n_1)}a^{j_2}_{(n_2)}\dots a^{j_s}_{(n_s)}\vac.
  \end{equation}
  By the completeness axiom of pre-vertex algebras, $\fs|_{\mathcal{F}_{\min}}$ is surjective.

  The map $\fs: \mathcal{F}_{\max} \to V$ is injective using \Cref{lmm:7} with $\mathcal{F}' = \mathcal{F}_{\min}$.
  Recall the inclusion $\mathcal{F}_{\min} \subseteq \mathcal{F}_{\max}$.
  We know that $\fs|_{\mathcal{F}_{\min}}$ is surjective and $\fs$ is injective, so we can conclude that is in fact bijective and $\mathcal{F}_{\min} = \mathcal{F}_{\max}$.
  This proves (i) and (ii).

  For (iii), we need to show that $Y(a, z)$ is translation covariant for $a \in V$ and that each pair $(Y(a, z), Y(b, w))$ is local for $a, b \in V$.
  Translation covariance comes from \Cref{lmm:5} and locality comes from Dong's lemma.
  
  Note that we have $Y(a^j_{(-1)}\vac, z) = a^j(z)$ for $j \in J$.
  Therefore, $(a^j_{(-1)}\vac)_{(n)} = a^j_{(n)}$ for $j \in J$ and $n \in \mathbb{Z}$.
  By the completeness axiom, we get (iv).

  Uniqueness of the vertex algebra structure follows from the completeness axiom of pre-vertex algebras, the $n$-product identity and the fact that $\vac \mapsto \Id_V$ in any vertex algebra.
  This finishes (v) and the proof of the theorem.
  
\end{proof}

\begin{corollary}
  \label{crl:1}
  Let $V$ be a vertex algebra, $s \in \mathbb{N}$, $a^1, \dots, a^s \in V$ and $n_1, \dots, n_s \in \mathbb{Z}$.
  Then 
  \begin{equation*}
    Y(a^1_{(n_1)}a^2_{(n_2)}\dots a^s_{(n_s)}\vac, z) = Y(a^j, z)_{(n_1)}(Y(a^2, z)_{(n_2)}\dots (Y(a^s, z)_{(n_s)}\Id_V)\dots).
  \end{equation*}
  In particular, for $s, n_1, \dots, n_s \in \mathbb{N}$ and $a^1, \dots, a^s \in V$,
  \begin{equation*}
    Y(a^1_{(-n_1 - 1)}\dots a^s_{(-n_s - 1)}\vac, z) = \frac{:\partial^{n_1}_zY(a^1,z)\dots \partial^{n_s}_zY(a^s,z):}{n_1!\dots n_s!}.
  \end{equation*}
  If $V$ is given by a pre-vertex algebra $(V, \vac, T, \mathcal{F})$ where $\mathcal{F} = \{a^j(z) = \sum_{n \in \mathbb{Z}}a^j_{(n)}z^{-n - 1}\}_{j \in J}$, then for $s, n_1, \dots, n_s \in \mathbb{N}$ and $j_1, \dots, j_s \in J$,
  \begin{equation*}
    Y(a^{j_1}_{(-n_1 - 1)}\dots a^{j_s}_{(-n_s - 1)}\vac, z) = \frac{:\partial^{n_1}_za^{j_1}(z)\dots \partial^{n_s}_za^{j_s}(z):}{n_1!\dots n_s!}.
  \end{equation*}
\end{corollary}

Let $(\mathfrak{g}, \mathfrak{F}, T_0)$ be a regular formal distribution Lie superalgebra with $\mathfrak{F} = \{a^j(z) = \sum_{n \in \mathbb{Z}}a^j_{(n)}z^{-n - 1}\}_{j \in J}$ and let $\mathfrak{g}_-$ be the annihilation subalgebra.
Since $T_0(\mathfrak{g}_-) \subseteq \mathfrak{g}_-$, $DU(T_0): U(\mathfrak{g}) \to U(\mathfrak{g})$ is a $(U(\mathfrak{g}), U(\mathfrak{g}_-))$-bimodule homomorphism.
Consider the trivial representation $0: \mathfrak{g}_- \to \mathfrak{gl}(\mathbb{C})$.
We define:
\begin{align*}
  V &= \Ind^{\mathfrak{g}}_{\mathfrak{g}_-}(\mathbb{C}) = U(\mathfrak{g}) \otimes_{U(\mathfrak{g}_-)} \mathbb{C}, \\
  \pi &= \Ind^{\mathfrak{g}}_{\mathfrak{g}_-}(0): \mathfrak{g} \to \mathfrak{gl}(V), \\
  \vac &= 1\otimes1 \in V, \\
  T &= DU(T_0) \otimes \Id_{\mathbb{C}} \in \End(V), \\
  \mathcal{F} &= \{\pi(a^j(z)) = \sum_{n \in \mathbb{Z}}\pi(a^j_{(n)})z^{-n - 1} \mid j \in J\}.
\end{align*}

\begin{theorem}
  \label{thr:16}
  With the notation above, $\mathcal{F}$ consists of fields and $(V, \vac, T, \mathcal{F})$ is a pre-vertex algebra.
\end{theorem}

\begin{proof}
  First, we prove that $\pi(a^j(z))$ is a field for all $j \in J$.
  We do this by induction on $s$ in $v = a^{j_1}_{(n_1)}\dots a^{j_s}_{(n_s)}\vac$ where $s \in \mathbb{N}$, $n_1, \dots, n_s \in \mathbb{Z}_-$ and $j_1, \dots, j_s \in J$.
  By \eqref{eq:20}, these elements form a spanning set for $V$.
  For $s = 0$, we have $v = \vac$ and
  \begin{equation*}
    \pi(a^j(z))\vac = \sum_{n \in \mathbb{Z}} a^j_{(n)}\vac z^{-n - 1} = \sum_{n \in \mathbb{Z}_-} a^j_{(n)}\vac z^{-n - 1} \in V((z)).
  \end{equation*}
  The last equality is true because $a_{(n)}\vac = 0$ for $n \in \mathbb{N}$.
  Now we proceed by proving the induction step:
  \begin{align}
    \nonumber
    \pi(a^j(z))v &= \sum_{n \in \mathbb{Z}}a^j_{(n)}a^{j_1}_{(n_1)}\dots a^{j_s}_{(n_s)}\vac z^{-n - 1} \\
    \label{eq:18}
    &= \sum_{n \in \mathbb{Z}} [a^j_{(n)}, a^{j_1}_{(n_1)}]a^{j_2}_{(n_2)}\dots a^{j_s}_{(n_s)}\vac z^{-n - 1} \pm \sum_{n \in \mathbb{Z}}a^{j_1}_{(n_1)}a^j_{(n)}a^{j_2}_{(n_2)}\dots a^{j_s}_{(n_s)}\vac z^{-n - 1}.
  \end{align}
  By the induction hypothesis, the second sum in \eqref{eq:18} is in $V((z))$, so we only need to show that the first sum is also in $V((z))$.
  By \eqref{eq:6},
  \begin{equation}
    \label{eq:19}
    [a^j_{(n)}, a^{j_1}_{(n_1)}] = \sum_{k \in \mathbb{N}}\binom{n}{k}(a^j(z)_{(k)}a^{j_1}(z))_{(n + n_1 - k)}.
  \end{equation}
  By the regularity property, we know that $a^j(z)_{(k)}a^{j_1}(z) \in \mathbb{C}[\partial_z]\mathfrak{F}$, thus we can assume that
  \begin{equation*}
    a^j(z)_{(k)}a^{j_1}(z) = \sum_{l \in J}f^k_l(\partial_z)a^l(z),
  \end{equation*}
  for some polynomials $f^k_l$.
  Since $(a^j(z), a^{j_1}(z))$ is a local pair, there exists $N \in \mathbb{N}$ such that $a^j(z)_{(k)}a^{j_1}(z) = 0$ for $k \ge N$.
  This allows us to rewrite \eqref{eq:19} as
  \begin{equation*}
    [a^j_{(n)}, a^{j_1}_{(n_1)}] = \sum_{0 \le k \le N}\binom{n}{k}\left(\sum_{l \in J}f^k_l(\partial_z)a^l(z)\right)_{(n + n_1 - k)}.
  \end{equation*}
  Therefore, we can rewrite the first sum in \eqref{eq:18} as
  \begin{equation*}
    \sum_{0 \le k \le N}\sum_{n \in \mathbb{Z}}\binom{n}{k}\left(\sum_{l \in J}f^k_l(\partial_z)a^l(z)\right)_{(n + n_1 - k)}a^{j_2}_{(n_2)}\dots a^{j_s}_{(n_s)}\vac z^{-n - 1}.
  \end{equation*}
  By the induction hypothesis, for each $k$,
  \begin{equation*}
    \sum_{n \in \mathbb{Z}}\binom{n}{k}\left(\sum_{l \in J}f^k_l(\partial_z)a^l(z)\right)_{(n + n_1 - k)}a^{j_2}_{(n_2)}\dots a^{j_s}_{(n_s)}\vac z^{-n - 1} \in V((z)).
  \end{equation*}
  Finally, the first sum in \eqref{eq:18} is also in $V((z))$.

  We now verify the four axioms of a pre-vertex algebra.
  \begin{enumerate}
  \item $T\vac = DU(T_0)\otimes \Id_{\mathbb{C}}(1\otimes1) = DU(T_0)(1)\otimes1 = 0\otimes1 = 0$.
  \item Recall that $T$ is an even endomorphism.
    For $j \in J$,
    \begin{equation*}
      [T, \pi(a^j(z))] = \left[T, \sum_{n \in \mathbb{Z}}\pi(a^j_{(n)})z^{-n - 1}\right] = \sum_{n \in \mathbb{Z}}[T, \pi(a^j_{(n)})]z^{-n - 1} = \sum_{n \in \mathbb{Z}}T\pi(a^j_{(n)}) - \pi(a^j_{(n)})T.
    \end{equation*}
    For $x \in U(\mathfrak{g})$,
    \begin{align*}
      T\pi(a^j_{(n)})(x\otimes1) &= T(a^j_{(n)}x\otimes1) \\
      &= DU(T_0)(a^j_{(n)}x)\otimes1 \\
      &= (T_0(a^j_{(n)}) + a^j_{(n)}T_0(x))\otimes1 \\
      &= (-na^j_{(n - 1)}x + a^j_{(n)}T_0(x))\otimes1 \\
      &= -na^j_{(n - 1)}(x\otimes1) + \pi(a^j_{(n)})T(x\otimes 1).
    \end{align*}
    The last two equalities imply that for all $j \in J$, $[T, \pi(a^j(z))] = \partial_z\pi(a^j(z))$.
  \item Note that for $i, j \in J$ and $N \in \mathbb{N}$,
    \begin{equation*}
      (z - w)^N[\pi(a^i(z)), \pi(a^j(z))] = (z - w)^N\pi([a^i(z), a^j(z)]) = \pi((z - w)^N[a^i(z), a^j(z)]).
    \end{equation*}
  \item This was already done. \qedhere
  \end{enumerate}
\end{proof}

By \Cref{thr:16} and the Extension theorem, $V$ is a vertex algebra, denoted by $V(\mathfrak{g}, \mathfrak{F}, T_0)$, and by the PBW theorem, it is explicitly given by
\begin{equation}
  \label{eq:20}
  V = \vspan\{a^{j_1}_{(-n_1 - 1)}\dots a^{j_s}_{(-n_s - 1)}\vac \mid s, n_1, \dots, n_s \in \mathbb{N}, j_1, \dots, j_s \in J\}
\end{equation}
which means $V$ is strongly generated by $\{a^j_{(-1)}\vac\}_{j\in J}$.

\begin{remark}
  \label{rmk:12}
  The PBW theorem for Lie superalgebras is actually more precise than \eqref{eq:20}.
  Let $\le$ be a total order on $J$.
  Then
  \begin{equation*}
    \begin{split}
      V = \vspan\{a^{j_1}_{(-n_1 - 1)}\dots a^{j_s}_{(-n_s - 1)}\vac &\mid s, n_1, \dots, n_s \in \mathbb{N}, j_1, \dots, j_s \in J\text{ and for } k = 1, \dots, s - 1,\\
      &\quad \text{if }a^{j_k}_{(n_k)},a^{j_{k + 1}}_{(n_{k + 1})}\in V_{\one},\text{ then }j_k < j_{k + 1}\}.
    \end{split}
  \end{equation*}
\end{remark}

\begin{remark}
  \label{rmk:13}
  We have constructed a functor
  \begin{align*}
    VA: \{\text{regular formal distribution Lie superalgebra}\} &\to \{\text{vertex algebras}\} \\
    (\mathfrak{g}, \mathfrak{F}, T_0) &\mapsto V(\mathfrak{g}, \mathfrak{F}, T_0).
  \end{align*}
  By \Cref{rmk:5}, we could've constructed a functor (see~\cite[Theorem 2.15]{li_vertex_2004})
  \begin{align*}
    VA: \{\text{Lie conformal superalgebras}\} &\to \{\text{vertex algebras}\} \\
    \mathcal{R} &\mapsto VA(\mathcal{R}).
  \end{align*}
\end{remark}

Usually we need to quotient the vertex algebras obtained this way.
Let $(\mathfrak{g}, \mathfrak{F}, T_0)$ be a regular formal distribution Lie superalgebra and $\lambda: \mathfrak{h} \to \mathbb{C}$ a linear functional where $\mathfrak{h}$ is a subalgebra of $\mathfrak{g}_+$ with $\mathfrak{h} \subseteq \ker(T_0)$.
Denote by $I^{\lambda}$ the $\mathfrak{g}$-submodule of $V(\mathfrak{g}, \mathfrak{F}, T_0)$ generated by the vectors $(a - \lambda(a))\vac$ for $a \in \mathfrak{h}$.
For $a \in \mathfrak{h}$ and $x \in U(\mathfrak{g})$,
\begin{align*}
  T(x(a - \lambda(a))\vac) &= DU(T_0)(x(a - \lambda(a)))\otimes1 \\
  &= (DU(T_0)(x)(a - \lambda(a)) + xDU(T_0)(a - \lambda(a)))\otimes1 \\
  &= DU(T_0)(x)(a - \lambda(a))\otimes1 \\
  &= DU(T_0)(x)(a - \lambda(a))\vac.
\end{align*}
Thus, $I^{\lambda}$ is $T$-invariant.
By \Cref{thr:16}, $\{a^j_{(-1)}\vac \}_{j \in J}$ strongly generates $V$ and from this we see that $I^{\lambda}$ is a left ideal.
By \Cref{rmk:11}, $I^{\lambda}$ is an ideal of $V(\mathfrak{g}, \mathfrak{F}, T_0)$.
We get a vertex algebra denoted by
\begin{equation*}
  V^{\lambda}(\mathfrak{g}, \mathfrak{F}, T_0) = V(\mathfrak{g}, \mathfrak{F}, T_0)/I^{\lambda}.
\end{equation*}

Let $\mathbb{C}_{\lambda}$ be the representation of $\mathfrak{g}_- \oplus \mathfrak{h}$ on which $\mathfrak{g}_-$ acts as $0$ and $a$ acts as $\lambda(a)$ for $a \in \mathfrak{h}$.
Using the universal property of the induced representation, we find a $\mathfrak{g}$-module homomorphism $\Ind^{\mathfrak{g}}_{\mathfrak{g}_-}(\mathbb{C}) \to \Ind^{\mathfrak{g}}_{\mathfrak{g}_- \oplus \mathfrak{h}}(\mathbb{C}_{\lambda})$.
By the universal property of the quotient, we find a homomorphism of $\mathfrak{g}$-modules $f: V^{\lambda}(\mathfrak{g}, \mathfrak{F}, T_0) \to \Ind^{\mathfrak{g}}_{\mathfrak{g}_- \oplus \mathfrak{h}}(\mathbb{C}_{\lambda})$ such that the following diagram commutes
\begin{equation*}
  \begin{tikzcd}
    {\Ind^{\mathfrak{g}}_{\mathfrak{g}_-}(\mathbb{C})} \arrow[rd] \arrow[r] & {V^\lambda(\mathfrak{g}, \mathfrak{F}, T_0)} \arrow[d, "f"] \\
    & \Ind^{\mathfrak{g}}_{\mathfrak{g}_- \oplus \mathfrak{h}}(\mathbb{C}_{\lambda})
  \end{tikzcd}
\end{equation*}
Using again the universal property of the induced representation we find a homomorphism $g: \Ind^{\mathfrak{g}}_{\mathfrak{g}_-}(\mathbb{C}) \to V^{\lambda}(\mathfrak{g}, \mathfrak{F}, T_0)$.
We can verify that $f$ and $g$ are inverses of each other.
Therefore, it is better to think of $V^{\lambda}(\mathfrak{g}, \mathfrak{F}, T_0)$ as being $\Ind^{\mathfrak{g}}_{\mathfrak{g}_- \oplus \mathfrak{h}}(\mathbb{C}_{\lambda}) = U(\mathfrak{g})\otimes_{U(\mathfrak{g}_- \oplus \mathfrak{h})}\mathbb{C}$.
We now use this construction to obtain several important examples of vertex algebras.

\begin{remark}
  \label{rmk:14}
  The vertex algebra $V^{\lambda}(\mathfrak{g}, \mathfrak{F}, T_0)$ comes equipped with a vertex algebra homomorphism $\pi: V(\mathfrak{g}, \mathfrak{F}, T_0) \to V^{\lambda}(\mathfrak{g}, \mathfrak{F}, T_0)$ that satisfies $\pi(a\vac) = \lambda(a)\vac$ for all $a \in \mathfrak{h}$ and is universal with this property, i.e. if $f: V(\mathfrak{g}, \mathfrak{F}, T_0)\to W$ is a vertex algebra homomorphism such that $f(a)\vac = \lambda(a)\vac$ for all $a \in \mathfrak{h}$, then there exist a unique vertex algebra homomorphism $\bar{f}: V^{\lambda}(\mathfrak{g}, \mathfrak{F}, T_0)\to W$ such that the following diagram commutes
  \begin{equation*}
    \begin{tikzcd}
      {V(\mathfrak{g}, \mathfrak{F}, T_0)} \arrow[rd, "f"] \arrow[r, "\pi"] & {V^\lambda(\mathfrak{g}, \mathfrak{F}, T_0)} \arrow[d, "\bar{f}"] \\
      & W                                                    
    \end{tikzcd}
  \end{equation*}
\end{remark}

\begin{example}[Universal Virasoro vertex algebra with central charge $c$]
  \label{exa:5}
  Pick $c \in \mathbb{C}$.
  Take $(\Vir, \{L(z), C\}, \ad(L_{-1}))$ as the regular formal distribution Lie superalgebra as constructed in \Cref{exa:1} and $\lambda: \mathbb{C}C \to \mathbb{C}, \lambda(C) = c$ as the linear functional.
  The resulting vertex algebra is the Universal Virasoro vertex algebra with central charge $c$, denoted by $\Vir^c$.

  A partition (of $n \in \mathbb{N}$) is a sequence $\lambda = [\lambda_1, \dots, \lambda_m]$ such that $\lambda_i \in \mathbb{Z}_+$ for $i = 1, \dots, m$, $\lambda_1 \ge \dots \ge \lambda_m$ (and $\lambda_1 + \dots + \lambda_m = n$).
  We also consider the empty partition $\emptyset$ which is the unique partition of $0$.
  By \Cref{rmk:12}, for any $c \in \mathbb{C}$, a basis for $\Vir^c$ is given by
  \begin{equation*}
    \{L_{-\lambda_1}\dots L_{-\lambda_m}\vac \mid [\lambda_1, \dots, \lambda_m]\text{ is a partition with } \lambda_m \ge 2\}.
  \end{equation*}
  When we deal with conformal vertex algebras, we will explain why is this vertex algebra called universal.
\end{example}

\begin{example}[Universal affine vertex algebra of level $k$]
  \label{exa:6}
  Pick $k \in \mathbb{C}$.
  Take $(\hat{\mathfrak{g}}, \{a(z) \mid a \in \mathfrak{g}\} \cup \{K\}, -\partial_t)$ as the regular formal distribution Lie superalgebra as constructed in \Cref{exa:2} and $\lambda: \mathbb{C}K \to \mathbb{C}, \lambda(K) = k$ as the linear functional.
  The resulting vertex algebra is the Universal affine vertex algebra of level $k$, denoted by $V^k(\mathfrak{g})$.
\end{example}

\begin{example}[Fermionic vertex algebra]
  \label{exa:7}
  Take $(\widehat{V}, \{a(z) \mid a \in V\} \cup \{K\}, -\partial_t)$ as the regular formal distribution Lie superalgebra as constructed in \Cref{exa:3} and $\lambda: \mathbb{C}K \to \mathbb{C}, \lambda(K) = 1$ as the linear functional.
  The resulting vertex algebra is the fermionic vertex algebra, denoted by $F(V)$.
\end{example}

\subsection{Graded and conformal vertex algebras}
\label{sec:grad-conf-vert}

Let $V$ be a vertex algebra.
A Hamiltonian operator of $V$ is a diagonalizable operator $H \in \End(V)$ such that
\begin{equation}
  \label{eq:21}
  [H, Y(a, z)] = z\partial_zY(a, z) + Y(Ha, z) \quad \text{for }a \in V.
\end{equation}
A vertex algebra with a Hamiltonian operator is called graded.
The grading of $V$ is the eigenspace decomposition for $H$:
\begin{equation*}
  V = \bigoplus_{\Delta \in \mathbb{C}}V_{\Delta},
\end{equation*}
where
\begin{equation*}
  V_{\Delta} = \ker(H - \Delta\Id_V) \quad \text{for }\Delta \in \mathbb{C}.
\end{equation*}
If $a$ is an eigenvector of $H$, it is called homogeneous, its eigenvalue is called the conformal weight of $a$ and it is denoted by $\Delta_a$.
Condition \eqref{eq:21} is equivalent to
\begin{equation}
  \label{eq:22}
  [H,a_{(n)}] = -(n + 1)a_{(n)} + (Ha)_{(n)} \quad \text{for }a \in V\text{ and }n \in \mathbb{Z}
\end{equation}
and to
\begin{equation}
  \label{eq:23}
  [H, a_{(n)}] = (\Delta_a - n - 1)a_{(n)} \quad \text{for }a \in V\text{ homogeneous and }n \in \mathbb{Z} - \Delta_a.
\end{equation}

For $a \in V$ homogeneous with conformal weight $\Delta_a$, we write, as was done with eigendistributions, $Y(a,z) = \sum_{n \in \mathbb{Z} - \Delta_a}a_nz^{-n - \Delta_a}$ which is equivalent to
\begin{equation*}
  a_{(n)} = a_{n - \Delta_a + 1} \quad \text{for }n \in \mathbb{Z}.
\end{equation*}
With this notation, \eqref{eq:23} is equivalent to
\begin{equation*}
  [H, a_n] = -na_n \quad \text{for } a \in V\text{ homogeneous and }n \in \mathbb{Z} - \Delta_a.
\end{equation*}
When a formula involves $\Delta_a$ it is assumed that $a$ is an eigenvector of $H$ with eigenvalue $\Delta_a$, and the formula is extended to arbitrary $a$ by linearity.

\begin{theorem}[{\cite[\S4.9]{kac_vertex_1998}}]
  \label{thr:17}
  Let $V$ be a vertex algebra with a Hamiltonian $H$ and grading $V = \bigoplus_{\Delta \in \mathbb{C}}V_{\Delta}$.
  Then:
  \begin{enumerate}
  \item $H\vac = 0$ which means that $\Delta_{\vac} = 0$;
  \item $[H, T] = T$;
  \item $[T, a_n] = (-n - \Delta_a + 1)a_{n - 1}$ for $a \in V$ homogeneous and $n \in \mathbb{Z} - \Delta_a$;
  \item $a_nV_{\Delta} \subseteq V_{\Delta - n}$ for $a \in V$ homogeneous, $\Delta \in \mathbb{C}$ and $n \in \mathbb{Z} - \Delta_a$;
  \item $TV_\Delta \subseteq V_{\Delta + 1}$ or equivalently $\Delta_{Ta} = \Delta_a + 1$ for homogeneous $a \in V$;
  \item $\Delta_{a_{(n)}b} = \Delta_a + \Delta_b - n - 1$ for $a, b \in V$ homogeneous and $n \in \mathbb{Z}$.
  \end{enumerate}
\end{theorem}

A $\mathbb{Z}$, $\mathbb{N}$ or $\frac{1}{m}\mathbb{N}$-graded vertex algebra $V$ ($m \in \mathbb{Z}_+$) is a graded vertex algebra $V$ such that $V_{\Delta} = 0$ for $\Delta \notin \mathbb{Z}$, $\mathbb{N}$ or $\frac{1}{m}\mathbb{N}$, respectively.

Homomorphisms of graded vertex algebras are assumed to respect the gradings, i.e. if $f: V \to W$ is a homomorphism of graded vertex algebra then $f\circ H^V = H^w\circ f$ where $H^V$ is the Hamiltonian of $V$ and $H^W$ is the Hamiltonian of $W$.

Let $V$ be a vertex algebra.
A conformal vector of central charge $c \in \mathbb{C}$ of $V$ is a vector $\omega \in V$ such that $Y(\omega, z) = \sum_{n \in \mathbb{Z}}L_nz^{-n - 2}$ satisfies:
\begin{enumerate}
\item $Y(\omega, z)$ is a Virasoro formal distribution with central charge $C = c\Id_V$;
\item $L_{-1} = T$;
\item $L_0$ is diagonalizable. 
\end{enumerate}

A conformal vertex algebra (of central charge $c$) is a vertex algebra $V$ together with a conformal vector $\omega$ (of central charge $c$).

\begin{remark}
  \label{rmk:15}
  Because of property (ii), a conformal vector $\omega$ is necessarily even.
  Note that $\omega_{(n)} = L_{n - 1}$ for $n \in \mathbb{Z}$.
\end{remark}

\begin{theorem}
  \label{thr:18}
  If $\omega$ is a conformal vector for $V$, then $L_0$ is a Hamiltonian for $V$ and for $a \in V$, $Y(a, z)$ is an eigendistribution of conformal weight $\Delta_a$ with respect to $Y(\omega, z)$ if and only if $a$ is homogeneous of conformal weight $\Delta_a$.
  Moreover, $\omega$ has conformal weight $2$.
\end{theorem}

\begin{proof}
  By \Cref{thr:13}(iii) and \Cref{thr:13}(v), for $a \in V$ homogeneous and $n \in \mathbb{Z}$,
  \begin{align*}
    [L_0, a_{(n)}] &= [\omega_{(1)}, a_{(n)}] \\
    &= \sum_{j \in \mathbb{N}}\binom{1}{j}(\omega_{(1)}a)_{1 + n - j} \\
    &= (\omega_{(0)}a)_{( n + 1)} + (\omega_{(1)}a)_{(n)} \\
    &= (Ta)_{(n + 1)} + (L_0a)_{(n)} \\
    &= -(n + 1)a_{(n)} + (L_0a)_{(n)}.
  \end{align*}
  By \eqref{eq:22}, this is equivalent to $L_0$ being a Hamiltonian for $V$.

  By the $n$-product identity for vertex algebras and \Cref{thr:13}(iii), for $a \in V$,
  \begin{align*}
    [Y(\omega, z)_{\lambda}Y(a, z)] &= \sum_{j \in \mathbb{N}}\frac{Y(\omega, z)_{(j)}Y(a, z)}{j!}\lambda^j \\
    &= \sum_{j \in \mathbb{N}}\frac{Y(\omega_{(j)}a, z)}{j!}\lambda^j \\
    &= Y(Ta, z) + Y(L_0a, z)\lambda + O(\lambda^2) \\
    &= \partial_zY(a, z) + Y(L_0a, z)\lambda + O(\lambda^2).
  \end{align*}
  Because $Y: V \to \mathcal{F}(V)$ is injective, this implies that $[Y(\omega, z)_{\lambda}Y(a, z)] = \partial_zY(a, z) + \Delta_aY(a, z)\lambda + O(\lambda^2)$ if and only if $L_0a = \Delta_aa$.
  As $Y(\omega, z)$ has conformal weight $2$ with respect to itself, $\omega$ has conformal weight $2$.
\end{proof}

\begin{remark}
  \label{rmk:16}
  A conformal vertex algebra can have many different conformal vectors, see \cite[2.5.9. Examples.]{frenkel_vertex_2001}
\end{remark}

We can naturally endow $V$ with the structure of a Lie conformal superalgebra by taking $\partial = T$ and defining the following $\lambda$-bracket on $V$:
\begin{align*}
  [\bullet_{\lambda}\bullet]: V \times V &\to V[\lambda] \\
  [a_{\lambda}b] &= F^{\lambda}(a(z)b) = \sum_{j \in \mathbb{N}}(a_{(j)}b)\frac{\lambda^j}{j!}.
\end{align*}
To verify the Lie conformal superalgebra axioms we do the following:
Set $\mathcal{G}=\{Y(a,z)\mid a\in V\}$.
Then $\mathcal{G}\subseteq \mathcal{F}(V)$ is a Lie conformal superalgebra and the following diagram commutes:
\begin{equation*}
  \begin{tikzcd}
    V\times V \arrow[d, "{[\bullet_{\lambda}\bullet]_V}"] \arrow[r, "Y\times Y"] & \mathcal{G}\times\mathcal{G} \arrow[d, "{[\bullet_{\lambda}\bullet]_{\mathcal{G}}}"] \\
    {V[\lambda]} \arrow[r, "{Y[\lambda]}"]                       & {\mathcal{G}[\lambda]}                                                 
  \end{tikzcd}
\end{equation*}
As $Y: V \to \mathcal{F}(V)$ is injective, we obtain that $V$ itself is a Lie conformal superalgebra.
We have constructed a functor
\begin{align*}
  LCA: \{\text{Vertex algebras}\} &\to \{\text{Lie conformal superalgebras}\} \\
  V &\mapsto (V, T, [\bullet_{\lambda}\bullet])
\end{align*}

\begin{remark}
  \label{rmk:17}
  Recall the functor $VA: \{\text{Lie conformal superalgebras}\} \to \{\text{Vertex algebras}\}$ constructed in \Cref{rmk:13}.
  We could prove that $(VA, LCA)$ is an adjoint pair of functors:
  \begin{equation*}
    \Hom(VA(\mathcal{R}), V) \cong \Hom(\mathcal{R}, LCA(V)).
  \end{equation*}
\end{remark}

\begin{example}
  \label{exa:8}
  It is straightforward to verify that for any $c \in \mathbb{C}$, $\Vir^c$ is a conformal vertex algebra with conformal vector $\omega = L_{-2}\vac$.
\end{example}

\begin{example}[Sugawara construction {\cite[Theorem 5.7]{kac_vertex_1998}}]
  \label{exa:9}
  Recall the universal affine vertex algebra of level $k$ of \Cref{exa:6} and assume that $\mathfrak{g}$ is simple.
  Let $(a_i), (b^i)$ be dual bases of $(\bullet| \bullet)$, which means $(a_i| b^j) = \delta_{ij}$.
  Denote by $c_{\mathfrak{g}} = \sum_ia_ib^i \in U(\mathfrak{g})$ the universal Casimir element of $\mathfrak{g}$.
  Let $2h^{\wedge}$ be the eigenvalue of $\ad(c_{\mathfrak{g}}) \in \End(\mathfrak{g})$ in the adjoint representation, i.e.\ $\ad(c_{\mathfrak{g}}) = \sum_i\ad(a_i)\ad(b^i) = 2h^{\wedge}\Id_V$. 
  Assume $k \neq -h^\wedge$.
  Set
  \begin{equation*}
    \omega = \frac{1}{2(k + h^\wedge)}\sum_ia_it^{-1}b^it^{-1}\vac \in V^k(\mathfrak{g}).
  \end{equation*}
  Then $\omega$ is a conformal vector with central charge $\frac{k \sdim\mathfrak{g}}{2(k + h^\wedge)}$ and for any $a\in \mathfrak{g}$, $a(z)$ is a primary eigendistribution of weight $1$.
\end{example}

\begin{example}[{\cite[Proposition 4.10]{kac_vertex_1998}}]
  \label{exa:10}
  Recall the fermionic vertex algebra of \Cref{exa:7}.
  Let $(a_i)_{i \in I}$ and $(b^i)_{i \in I}$ be a pair of dual bases of $V$, i.e.\ $<a_i, b^i> = \delta_{ij}$
  Set
  \begin{equation*}
    \omega = \frac{1}{2}\sum_{i \in I}:(T(b^it^{-1}\vac))a_it^{-1}\vac:\in F(V).
  \end{equation*}
  Then $\omega$ is a conformal vector with central charge $-\frac{1}{2}\sdim(V)$ and all fields $a(z)$ where $a \in V$ are primary of comformal weight $1/2$.
\end{example}

A homomorphism $(V, \omega) \to (V', \omega')$ of conformal vertex algebras is a homomorphism of vertex algebras such that $\omega \mapsto \omega'$.

\begin{theorem}[Universal property of $\Vir^c$]
  \label{thr:19}
  Let $V$ be a conformal vertex algebra with conformal vector $\omega$.
  There exists a unique homomorphism of conformal vertex algebras $\Vir^c \to V$.
\end{theorem}

\begin{proof}
  Let $\Vir$ also denote the Virasoro Lie conformal algebra.
  Because $V$ is conformal, the map
  \begin{align*}
    g: \Vir &\to LCA(V) \\
    g(L) &= \omega \\
    g(C) &= c\vac
  \end{align*}
  gives a homomorphism of Lie conformal superalgebras.
  By \Cref{rmk:17}, we obtain a homomorphism
  \begin{align*}
    f: VA(\Vir) &\to V \\
    f(L_{-2}\vac) &= \omega \\
    f(C) &= c\vac.
  \end{align*}
  Using the universal property of $\Vir^c$ (cf.\ \Cref{rmk:14}), we obtain our desired homomorphism of vertex algebras $\bar{f}: \Vir^c \to V$ such that $\bar{f}(L_{-2}\vac) = \omega$.
  As $\{L_{-2}\vac\}$ strongly generates $\Vir^c$ and a homomorphism of conformal vertex algebras is required to send $L_{-2}\vac$ to $\omega$, $\bar{f}$ is the only homomorphism of conformal vertex algebras $\Vir^c \to V$.
  
  Alternatively, we could use \Cref{thr:21} ahead to obtain a state-field correspondence $f: \Vir^c \to \mathcal{F}(V)$.
  Actually, the image of $f$ is contained in the image of the state-field correspondence $Y^V: V \to \mathcal{F}(V)$.
  Because $Y^V$ is injective, we can simply define $\bar{f} = (Y^V)^{-1}\circ f$.
\end{proof}

A vertex operator algebra (of central charge $c$) is a $\mathbb{Z}$-graded vertex algebra $V$ (of central charge $c$) such that $\dim(V)_{\Delta} < \infty$ for all $\Delta \in \mathbb{Z}$ and $V_{\Delta} = 0$ for $\Delta$ sufficiently small.

\subsection{Modules over vertex algebras}
\label{sec:modules-over-vertex}

Let $V$ be a vertex algebra.
A $V$-module or representation of $V$ is a vector space $M$ together with a linear parity preserving map
\begin{align*}
  Y^M(\bullet, z): V &\to \mathcal{F}(M) \\
  a &\mapsto Y^M(a,z) = \sum_{n \in \mathbb{Z}}a^M_{(n)}z^{-n - 1}
\end{align*}
satisfying:
\begin{enumerate}
\item $Y(\vac,z) = \Id_M$.
\item (Borcherds identity) For $a, b \in V$, $c \in M$ and $m, n, k\in \mathbb{Z}$,
  \begin{equation*}
    \sum_{j \in \mathbb{N}}(-1)^j\binom{n}{j}\left(a^M_{(m + n - j)}(b^M_{(k + j)}c) - (-1)^np(a, b)b^M_{(n + k - j)}(a^M_{(m + j)}c)\right) = \sum_{j \in \mathbb{N}}\binom{m}{j}(a_{(n + j)}b)^M_{(m + k - j)}c.
  \end{equation*}
\end{enumerate}
The vertex algebra $V$ is clearly a $V$-module and is sometimes called the adjoint representation of $V$.
A submodule of $M$ is a subspace $U$ of $M$ such that $Y^M(a, z)m \in U((x))$ for $a \in V$ and $m \in U$, i.e.\ $a^M_{(n)} m \in U$ for all $n\in \mathbb{Z}$.
The quotient module $M/U$ is defined in the usual way.
A module whose only submodules are $0$ and itself is called simple or irreducible.
Most of the theorems about vertex algebras and their proofs carry over to modules over vertex algebras (cf. \Cref{thr:13}) as the following theorem shows.

\begin{theorem}
  \label{thr:20}
  Let $V$ be a vertex algebra, $Y^M:V\to \mathcal{F}(M)$ a $V$-module, $a, b\in V$, $c \in M$ and $m, n \in \mathbb{Z}$.
  Then:
  \begin{enumerate}
  \item $(\vac)^M_{(n)} = \delta_{n, -1}\Id_M$;
  \item $Y^M(Ta,z)=\partial_zY^M(a,z)$ or, equivalently, $(Ta)^M_{(n)} = -na^M_{(n - 1)}$;
  \item All fields $Y^M(a,z)$ are mutually local;
  \item ($n$-product identity) $Y^M(a,z)_{(n)}Y^M(b,z)=Y^M(a_{(n)}b,z)$;
  \item $[a^M_{(m)},b^M_{(n)}]=\sum_{j\in \mathbb{N}}(a_{(j)}b)^M_{(m+n-j)}$;
  \item $e^{Tw}Y^M(a,z)e^{-Tw}=i_{z,w}Y^M(a,z+w)$;
  \item (Borcherds identity)
    \begin{equation}
      \label{eq:26}
      i_{z,w}(z-w)^nY^M(a,z)Y^M(b,w)c-p(a,b)i_{w,z}(z-w)^nY^M(b,w)Y^M(a,z)c=\sum_{j\in \mathbb{N}}Y^M(a_{(n+j)}b,w)\frac{\partial^j_w\delta(z,w)}{j!}c;
    \end{equation}
  \item $(a_{(n)}b)^M_{(m)}c = \sum_{j \in \mathbb{N}}(-1)^j\binom{n}{j}(a^M_{(n - j)}b^M_{(m + j)}c - (-1)^np(a, b)b^M_{(n + m - j)}a^M_{(j)}c)$.
  \end{enumerate}
\end{theorem}

\begin{remark}
  \label{rmk:18}
  A $V$-module is equivalently a vector space $M$ such that (i), (v) and (viii) in \Cref{thr:20} hold.
\end{remark}

\begin{remark}
  \label{rmk:19}
  In contrast to vertex algebras, the map $Y^M: V \to \mathcal{F}(M)$ is in general not injective for a module $M$ over a vertex algebra $V$.
\end{remark}

Let $V$ be a vertex algebra and let $M$ a $V$-module.
We say  $T^M \in \End(M)$ is a differential of $M$ if
\begin{equation*}
  [T^M, a^M_{(n)}] = -na^M_{(n - 1)} \quad \text{for } a \in V\text{ and }n \in \mathbb{Z}
\end{equation*}
or, equivalently,
\begin{equation*}
  [T^M, Y^M(a, z)] = Y^M(Ta, z).
\end{equation*}
A differential $V$-module is a $V$-module equipped with a differential.

Let $(\mathfrak{g}, \mathfrak{F}, T_0)$ be a regular formal distribution Lie superalgebra with $\mathfrak{F} = \{a^j(z) = \sum_{n \in \mathbb{Z}}a^j_{(n)}z^{-n - 1}\}_{j \in J}$.
A smooth $(\mathfrak{g}, \mathfrak{F}, T_0)$-module is a $\mathfrak{g}$-module $M$ such for $j \in J$ and $m \in M$, $a^j(z)m \in M((z))$.
Let $\lambda: \mathfrak{h} \to \mathbb{C}$ be a linear functional where $\mathfrak{h}$ is a subalgebra of $\mathfrak{g}_+$ with $\mathfrak{h} \subseteq \ker(T_0)$.
We say $\mathfrak{h}$ acts as $\lambda$ if for $h \in \mathfrak{h}$ and $m \in M$, $hm = \lambda(m)m$.


\begin{theorem}[{\cite[Theorem 2.15]{li_vertex_2004}}]
  \label{thr:21}
  Let $(\mathfrak{g}, \mathfrak{F}, T_0)$ be a regular formal distribution Lie superalgebra with $\mathfrak{F} = \{a^j(z) = \sum_{n \in \mathbb{Z}}a^j_{(n)}z^{-n - 1}\}_{j \in J}$.
  Given a smooth $(\mathfrak{g}, \mathfrak{F}, T_0)$-module $M$ there is a unique module structure $Y^M: V(\mathfrak{g}, \mathfrak{F}, T_0) \to \mathcal{F}(M)$ such that for $j \in J$ and $m \in M$, $Y^M(a^j_{(-1)}\vac)m = a^j(z)m$.
  Let $\lambda: \mathfrak{h} \to \mathbb{C}$ be a linear functional where $\mathfrak{h}$ is a subalgebra of $\mathfrak{g}_+$ with $\mathfrak{h} \subseteq \ker(T_0)$.
  Given a smooth $(\mathfrak{g}, \mathfrak{F}, T_0)$-module $M$ where $h$ acts as $\lambda$ there is a unique module structure $Y^M: V^{\lambda}(\mathfrak{g}, \mathfrak{F}, T_0) \to \mathcal{F}(M)$ such that for $j \in J$ and $m \in M$, $Y^M(a^j_{(-1)}\vac)m = a^j(z)m$.
\end{theorem}

\begin{remark}
  \label{rmk:20}
  We define in a natural way the categories $\{\text{smooth }(\mathfrak{g}, \mathfrak{F}, T_0)\text{-modules}\}$ and $\{\text{smooth }(\mathfrak{g}, \mathfrak{F}, T_0)\text{-modules where }\mathfrak{h}\text{ acts as }\lambda\}$.
  \Cref{thr:21} constructs a pair of functors
  \begin{align*}
    \{\text{smooth }(\mathfrak{g}, \mathfrak{F}, T_0)\text{-modules}\} &\to V(\mathfrak{g}, \mathfrak{F}, T_0)\text{-mod} \\
    \{\text{smooth }(\mathfrak{g}, \mathfrak{F}, T_0)\text{-modules where }\mathfrak{h}\text{ acts as }\lambda\} &\to V^{\lambda}(\mathfrak{g}, \mathfrak{F}, T_0)\text{-mod}
  \end{align*}
  which are actually isomorphisms.
\end{remark}

Let $M$ be a module of a graded vertex algebra $V$ with Hamiltonian operator $H$. 
A Hamiltonian operator of $M$ is a diagonalizable operator $H^M \in \End(M)$ such that
\begin{equation}
  \label{eq:27}
  [H^M, Y^M(a, z)] = z\partial_zY^M(a, z) + Y^M(Ha, z) \quad \text{for }a \in V.
\end{equation}
A $V$-module together with a Hamiltonian operator is called graded.
The grading of $M$ is the eigenspace decomposition for $H^M$:
\begin{equation*}
  M = \bigoplus_{\Delta \in \mathbb{C}}M_{\Delta},
\end{equation*}
where
\begin{equation*}
  M_{\Delta} = \{m \in M \mid H^Mm = \Delta m\}.
\end{equation*}

\begin{theorem}
  \label{thr:22}
  Let $V$ be a graded vertex algebra with Hamiltonian $H$ and $M$ a graded $V$-module with Hamiltonian $H^M$.
  Then:
  \begin{enumerate}
  \item $a^M_nM_\Delta\subseteq M_{\Delta - n}$ for $a$ homogeneous and $n \in \mathbb{Z}-\Delta_a$;
  \item $\Delta_{a^M_{(n)}m}=\Delta_a+\Delta_m-n-1$ for homogeneous $a\in V$, $m\in M$ and $n\in \mathbb{Z}$;
  \item (Graded Borcherds identity).
    Write $Y^M(a, z) = \sum_{n \in \mathbb{Z} - \Delta_a}a^M_nz^{-n - \Delta_a}$ for $a \in V$ homogeneous with eigenvalue $\Delta_a$.
    For $m \in \mathbb{Z} - \Delta_a$, $k \in \mathbb{Z}- \Delta_b$ and $n \in \mathbb{Z}$,
    \begin{equation}
      \label{eq:28}
      \sum_{j \in \mathbb{N}}(-1)^j\binom{n}{j}(a^M_{m + n - j}(b^M_{k + j - n}c)-(-1)^np(a, b)b^M_{k - j}(a^M_{m + j}c)) = \sum_{j \in \mathbb{N}}\binom{m + \Delta_a - 1}{j}(a_{(n + j)}b)^M_{m + k}c
    \end{equation}
  \end{enumerate}
\end{theorem}

\begin{proof}
  The proof for \Cref{thr:17} also works here.
\end{proof}

\begin{theorem}[{\cite[Proposition 4.1.5 and (4.1.18)]{lepowsky_introduction_2004}}]
  \label{thr:24}
  Let $V$ be a conformal vertex algebra of central charge $c$ with conformal vector $\omega$ and $M$ a $V$-module.
  Write $Y^M(\omega, z) = \sum_{n \in \mathbb{Z}}L^M_nz^{-n - 2}$.
  For $a \in V$ and $m, n \in \mathbb{Z}$,
  \begin{align*}
    [L^M_{-1}, Y^M(a, z)] &= Y^M(L_{-1}a, z) = \partial_zY^M(a, z), \\
    [L^M_m, L^M_n] &=(m - n)L^M_{m + n} + \delta_{m, -n}\frac{m^3 - m}{12}c\Id_M, \\
    [L^M_0, Y^M(a, z)] &= z\partial_zY^M(a, z) + Y^M(L_0a, z).
  \end{align*}
  In particular, $M$ is a differential $V$-module with $L_{-1}^M $ and $M$ is a smooth $\Vir$-module with central charge $c$.
  If $L^M_0$ is diagonalizable, then $M$ is graded by $L_0^M$ and
  \begin{align*}
    [L_0^M,L_{-1}^M]&=L_{-1}^M, \\
    L^M_{-1}M_\Delta&\subseteq M_{\Delta+1} \quad \text{for }\Delta \in \mathbb{C}.
  \end{align*}
\end{theorem}

\begin{remark}
  \label{rmk:21}
  \Cref{thr:24} is analogous to \Cref{thr:17} with $L_{-1}^M$ in place of $T$ and $L_0^M$ in place of $H$.
\end{remark}

Let $V$ be a conformal vertex algebra with conformal vector $\omega$.
By \Cref{thr:24}, the following definition makes sense.
A $V$-module as a conformal vertex algebra is a $V$-module $M$ as a vertex algebra such that $L^M_0$ is diagonalizable where we write $Y^M(\omega, z) = \sum_{n \in \mathbb{Z}}L^M_nz^{-n - 2}$.
In particular, $M$ is differential by $L^M_{-1}$ and graded by $L^M_0$.
The $V$-module $M$ is called a positive energy representation of $V$ if $M = \bigoplus_{n \in \mathbb{N}}M_{h + n}$ for some $h \in \mathbb{C}$.
The subspace $M_h$ is called the top degree component of $M$.
A positive energy representation $M$ of $V$ is called ordinary if for $n \in \mathbb{N}$, $\dim(M_{h + n}) < \infty$.
For an ordinary representation $M$, the character of $M$ is defined as
\begin{equation*}
  \ch_M(q) = \sum_{n \in \mathbb{N}}\dim(M_{h + n})q^{h + n} \in q^h\mathbb{C}[[q]].
\end{equation*}

Let $V$ be a graded vertex algebra.
An admissible $V$-module is a $V$-module $M$ together with a gradation $M = \bigoplus_{n \in \mathbb{N}}M(n)$ such that
\begin{equation*}
  a_{(n)}M(m) \subseteq M(m + \Delta_a - n - 1) \quad \text{for }a \in V, n \in \mathbb{N}\text{ and }m \in \mathbb{Z}.
\end{equation*}
The subspace $M(0)$ is called the top degree component of $M$.

\begin{example}
  \label{exa:13}
  WRITE ABOUT HIGHEST WEIGHT INSTEAD HERE, IT IS MORE GENERAL.
  The Verma module $M(c, h)$ is a smooth $\Vir$-module with central charge $c$.
  Note that $L_0$ is diagonalizable.
  By \Cref{thr:21}, $M(c, h)$ is a graded $\Vir^c$-module.
  Set $M(c, h)(n) = M_{h + n}$ for $n \in \mathbb{N}$.
  By \Cref{thr:20}, $M(c, h) = \bigoplus_{n \in \mathbb{N}}M(c, h)(n)$ is an admissible $\Vir^c$-module with top degree component $\mathbb{C}|c, h\rangle$.
\end{example}

\subsection{Lie algebras associated to vertex algebras}
\label{sec:lie-algebr-assoc}

\begin{lemma}
  \label{lmm:8}
  For a vertex algebra $V$, $V/TV$ is a Lie algebra with bracket
  \begin{equation*}
    [a + TV, b + TV] = a_{(0)}b + TV \quad \text{for }a, b \in V.
  \end{equation*}
\end{lemma}

\begin{proof}
  The skewsymmetry of the bracket follows from skewsymmetry for vertex algebras.
  The Jacobi identity follows from \Cref{thr:13}(x) with $m = n = 0$.
\end{proof}

\begin{lemma}
  \label{lmm:9}
  Let $V$ be a vertex algebra and $(R, \partial)$ a differential algebra.
  Then
  \begin{equation*}
    \Lie(V, R) = (V \otimes R)/(T\otimes\Id_R + \Id_V\otimes\partial)(V \otimes R)
  \end{equation*}
  is a Lie algebra with bracket
  \begin{equation*}
    [a\otimes r, b\otimes s] = \sum_{j \in \mathbb{N}}a_{(j)}b\otimes\left(\frac{\partial^jr}{j!}\right)s \quad \text{for }a, b \in V\text{ and }r, s \in R.
  \end{equation*}
\end{lemma}

\begin{proof}
  Since $R$ is a commutative vertex algebra, $V \otimes R$ is a vertex algebra with translation operator $T\otimes\Id_R + \Id_V\otimes\partial$.
  The assertion follows by applying \Cref{lmm:9} to the vertex algebra $V \otimes R$.
\end{proof}

The Borcherds Lie algebra associated with a vertex algebra $V$ is the Lie algebra
\begin{equation*}
  \Lie(V) = \Lie(V, \mathbb{C}[t, t^{-1}]),
\end{equation*}
where $\mathbb{C}[t, t^{-1}]$ is viewed as a differential algebra with the derivation $\partial_t$.
For $a \in V$ and $n \in \mathbb{Z}$, let $a_{\{n\}}$ be the class of $a\otimes t^n \in V \otimes \mathbb{C}[t, t^{-1}]$.
By definition, we have
\begin{equation}
  \label{eq:24}
  [a_{\{m\}}, b_{\{n\}}] = \sum_{j \in \mathbb{N}}\binom{m}{j}(a_{(j)}b)_{\{m + n - j\}}\quad \text{for }a, b \in V\text{ and }m, n \in \mathbb{Z}.
\end{equation}
Therefore, we have constructed a functor
\begin{equation*}
  \Lie: \{\text{Vertex algebras}\} \to \{\text{Lie algebras}\}.
\end{equation*}

\begin{lemma}
  \label{lmm:10}
  Any $V$-module $M$ is a $\Lie(V)$-module by setting $a_{\{n\}} \mapsto a^M_{(n)}$ for $a \in V$ and $n \in \mathbb{Z}$.
\end{lemma}

\begin{proof}
  First, this map is well defined because of \Cref{thr:20}(ii).
  It is a Lie algebra homomorphism because of \Cref{thr:20}(viii).
\end{proof}

Therefore, we have constructed a functor
\begin{equation*}
  \{\text{$V$-modules}\} \to \{\text{$\Lie(V)$-modules}\}.
\end{equation*}

Assume now that $V$ is $\mathbb{Z}$-graded.
Then $\Lie(V)$ is a graded Lie algebra by defining $H \in \End(\Lie(V))$ as
\begin{equation*}
  H(a_{\{n\}}) = -(n + 1)a_{\{n\}} + (Ha)_{\{n\}} \quad \text{for }a \in V\text{ and }n \in \mathbb{Z}.
\end{equation*}
This operator is diagonalizable because for $a$ homogeneous and $n \in \mathbb{Z}$, $H(a_{\{n\}}) = (\Delta_a - n - 1)a_{\{n\}}$.
This means that
\begin{equation*}
  \Delta_{a_{\{n\}}} = \Delta_a - n - 1\quad \text{for }a \in V\text{ homogeneous and }n \in \mathbb{Z}.
\end{equation*}
We have a grading
\begin{equation*}
  \Lie(V) = \bigoplus_{n \in \mathbb{Z}}\Lie(V)_n,
\end{equation*}
and a triangular decomposition
\begin{equation*}
  \Lie(V) = \Lie(V)_+ \oplus \Lie(V)_0 \oplus \Lie(V)_-,
\end{equation*}
where
\begin{align*}
  \Lie(V)_+ &= \bigoplus_{n > 0}\Lie(V)_n, \\
  \Lie(V)_- &= \bigoplus_{n < 0}\Lie(V)_n.
\end{align*}
Note that $\Lie(V)_0$ is spanned by elements of the form $a_{\{\Delta_a - 1\}}$ for $a \in V$ homogeneous and it forms a Lie subalgebra of $\Lie(V)$.
By \eqref{eq:24} and \Cref{thr:17}(vi), the bracket in $\Lie(V)_0$ is given by
\begin{equation}
  \label{eq:25}
  [a_{\{\Delta_a - 1\}}, b_{\{\Delta_b - 1\}}] = \sum_{j \in \mathbb{N}}\binom{\Delta_a - 1}{j}(a_{(j)}b)_{\{\Delta_{a_{(j)}b} - 1\}}\quad \text{for }a, b \in V.
\end{equation}

Consider the surjective linear map
\begin{align*}
  V &\to \Lie(V)_0 \\
  a &\mapsto a_{\{\Delta_a - 1\}},
\end{align*}
whose kernel is $(T + H)V$.
By \eqref{eq:25}, we are led to define
\begin{equation*}
  \Lie_0(V) = V/(T + H)V,
\end{equation*}
whose bracket is given by
\begin{equation*}
  [a + (T + H)V, b + (T + H)V] = \sum_{j \in \mathbb{N}}\binom{\Delta_a - 1}{j}a_{(j)}b \quad \text{for }a, b \in V.
\end{equation*}
From what we have done, there is a natural Lie algebra isomorphism
\begin{align*}
  \Lie_0(V) &\xrightarrow{\sim} \Lie(V)_0 \\
  a + (T + H)V &\mapsto a_{\{\Delta_a - 1\}}
\end{align*}
and we have another functor
\begin{equation*}
  \Lie_0: \{\text{$\mathbb{Z}$-graded Vertex algebras}\} \to \{\text{Lie algebras}\}.
\end{equation*}

\section{Highest weight representations of the Virasoro Lie algebra}
\label{sec:high-weight-repr}

\subsection{Unitary and contravariant representations of Lie algebras}
\label{sec:unit-contr-repr}

We start by reviewing some basic facts about linear algebra and forms on vector spaces.

\begin{lemma}[{\cite[\S6.2]{hoffman_linear_1971}}]
  \label{lmm:11}
  Let $V$ be a vector space, $H \in \End(V)$ and for $\Delta \in \mathbb{C}$, set $V_{\Delta} = \ker(H - \Delta\Id_V)$.
  Then the family of vector spaces $(V_{\Delta})_{\Delta \in \mathbb{C}}$ is linearly independent.
\end{lemma}

\begin{lemma}[{\cite[Corollary 1.1]{kac_bombay_2013}}]
  \label{lmm:12}
  Let $V$ be a vector space (not necessarily a vertex algebra) and $H \in \End(V)$ a diagonalizable operator with $H$-grading $V = \bigoplus_{\Delta \in \mathbb{C}}V_{\Delta}$.
  Let $U$ be a $H$-invariant subspace of $V$, this means $H(U) \subseteq U$.
  Then $U$ is also $H$-graded:
  \begin{equation*}
    U = \sum_{\Delta \in \mathbb{C}}U \cap V_{\Delta}.
  \end{equation*}
\end{lemma}

Let $V$ be a vector space.
A (sesquilinear) form on $V$ is a function
\begin{align*}
  \langle \bullet| \bullet\rangle: V \times V &\to \mathbb{C} \\
  (a, b) &\mapsto \langle a| b\rangle
\end{align*}
such that for $u, v, w \in V$ and $a \in \mathbb{C}$:
\begin{enumerate}
\item $\langle au + v| w\rangle = \overline{a}\langle u| w\rangle + \langle v| w\rangle$;
\item $\langle u| av + w\rangle = a\langle u| v\rangle a + \langle u| w\rangle$.
\end{enumerate}
All forms are assumed to be sesquilinear.

An Hermitian form on $V$ is a form $\langle \bullet| \bullet\rangle$ satisfying
\begin{equation*}
  \langle u| v\rangle = \overline{\langle v| u\rangle} \quad \text{for }u, v \in V.
\end{equation*}

Let $S \subseteq V$ be a subset of a vector space $V$ equipped with a form $\langle \bullet| \bullet\rangle$.
Define the orthogonal complement of $S$ as the subspace
\begin{equation*}
  S^{\perp} = \{v \in V \mid \text{for }u\in S, \langle u| v\rangle = 0\}.
\end{equation*}

A form $\langle \bullet| \bullet\rangle$ on $V$ is called nondegenerate if $V^{\perp} = 0$ and it is called positive-definite if it is Hermitian and
\begin{equation*}
  \langle v| v\rangle > 0 \quad \text{for }v \in V \text{ with } v \neq 0.
\end{equation*}
An inner product space is a vector space $V$ together with a positive-definite form.

\begin{lemma}[{\cite[\S8.2 Theorem 5]{hoffman_linear_1971}}]
  \label{lmm:13}
  Let $V$ be an inner product space and $W$ a finite dimensional subspace of $V$.
  Then $W \oplus W^{\perp}=V$.
\end{lemma}

Let $V$ be a  vector space.
We say a function $\omega: V \to V$ is antilinear if for all $u, v \in V$ and $a \in \mathbb{C}$,
\begin{align*}
  \omega(u + v) &= \omega(u) + \omega(v), \\
  \omega(au) &= \overline{a}\omega(u).
\end{align*}
Let $\mathfrak{g}$ be a lie algebra over $\mathbb{C}$.
We say a function $\omega: \mathfrak{g} \to \mathfrak{g}$ is an antilinear anti-involution if $\omega$ is antilinear and for all $x, y \in \mathfrak{g}$,
\begin{align*}
  \omega([x, y]) &= [\omega(y), \omega(x)], \\
  \omega(\omega(x)) &= x.
\end{align*}
Note that this means we can extend $\omega$ to the universal enveloping algebra $U(\mathfrak{g})$ of $\mathfrak{g}$ obtaining a function $\omega: U(\mathfrak{g}) \to U(\mathfrak{g})$ that is still an antilinear anti-involution, i.e.\ for $x, y \in U(\mathfrak{g})$,
\begin{align*}
  \omega(xy) &= \omega(y)\omega(x), \\
  \omega(\omega(x)) &= x.
\end{align*}

We shall be mostly interested in the Virasoro Lie algebra $\Vir$ which has the following antilinear anti-involution:
\begin{align*}
  \omega: \Vir &\to \Vir \\
  \omega(L_n) &= L_{-n} \quad \text{for }n \in \mathbb{Z}, \\
  \omega(C) &= C,
\end{align*}
which is extended by antilinearity.

Let $\mathfrak{g}$ be a Lie algebra with an antilinear anti-involution $\omega: \mathfrak{g} \to \mathfrak{g}$ and $V$ a $\mathfrak{g}$-module with an Hermitian form $\langle \bullet, \bullet \rangle$.
We say $\langle \bullet, \bullet \rangle$ is contravariant if,
\begin{equation*}
  \langle xu| v \rangle = \langle u| \omega(x)v\rangle \quad \text{for }x \in \mathfrak{g}\text{ and }u, v \in V.
\end{equation*}
We further say that this representation is unitary if in addition it is positive-definite.

\subsection{Verma modules}
\label{sec:verma-modules}

Let $V$ be a $\Vir$-module.
We say $V$ is smooth if $L(z) = \sum_{n \in \mathbb{Z}}L_{n}z^{-n - 2} \in \End(V)[[z^{\pm 1}]]$ is a field.
We say $V$ is of central charge $c\in \mathbb{C}$ if the central element $C$ acts as multiplication by $c$.

A highest weight representation of $\Vir$ is a $\Vir$-module $V$ which has a nonzero vector $v$ such that there exists complex numbers $c, h \in \mathbb{C}$ such that:
\begin{enumerate}
  \item $Cv = cv$; 
  \item $L_0v = hv$;
  \item $V = \vspan\{L_{-i_k}\dots L_{-i_1}v \mid i_k \ge \dots \ge i_1 > 0\}$.
\end{enumerate}
The numbers $c$ and $h$ are uniquely determined and the pair $(c, h)$ is called the highest weight of $V$.
The vector $v$ is not uinquely determined and can be replaced by any nonzero scalar multiple of it and is called a highest weight vector of $V$.

For $\Delta \in \mathbb{C}$, set $V_{\Delta} = \ker(L_0 - \Delta\Id_V)$.
We observe that all vectors of the form $L_{-i_k}\dots L_{-i_1}v$ with $i_k \ge \dots \ge i_1 > 0$ and a fixed value of $j = i_1 + \dots + i_k$ belong to $V_{h + j}$.
By axiom (iii), $V = \sum_{j \in \mathbb{N}}V_{h + j}$ and by \Cref{lmm:11}, this is in fact a direct sum:
\begin{equation}
  \label{eq:29}
  V = \bigoplus_{j \in \mathbb{N}}V_{h + j}.
\end{equation}
By axiom (ii),
\begin{equation}
  \label{eq:30}
  V_h = \mathbb{C}v
\end{equation}
Note that
\begin{equation}
  \label{eq:31}
  L_nV_{h + j}\subseteq V_{h + j - n} \quad \text{for }n \in \mathbb{Z}\text{ and }j \in \mathbb{N}.
\end{equation}
In particular we have
\begin{equation}
  \label{eq:32}
  L_nv = 0 \quad \text{for } n \in \mathbb{Z}_+.
\end{equation}

For $j \in \mathbb{N}$, let $p(j)$ denote the number of partitions of $j$ into a sum of positive integers $j = i_1 + \dots + i_k$ with $i_k \ge \dots \ge i_1 > 0$.
It is clear that
\begin{equation}
  \label{eq:33}
  \dim(V_{h + j}) \le p(j)
\end{equation}
with equality if and only if all vectors of the form $L_{-i_k}\dots L_{-i_1}v$ with $i_k \ge \dots \ge i_1 > 0$ and $j = i_1 + \dots + i_k$ are linearly independent.
By axiom (i), $C$ acts on $V$ as multiplication by $c$ because it commutes with every $L_n$ for $n \in \mathbb{Z}$.
Thus, a highest weight representation of $\Vir$ with highest weight $(c, h)$ is necessarily smooth and of central charge $c$.

\begin{lemma}
  \label{lmm:14}
  Let $V$ be a highest weight representation of $\Vir$ with highest weight $(c, h)$, highest weight vector $v$ and decomposition $V = \bigoplus_{j \in \mathbb{N}}V_{h + j}$.
  Then:
  \begin{enumerate}
  \item Any subrepresentation $U$ of $V$ is graded:
    \begin{equation*}
      U = \bigoplus_{j \in \mathbb{N}}U \cap V_{h + j}.
    \end{equation*}
  \item $V$ is indecomposable, i.e.\ we can't find nontrivial subrepresentations $U, W$ such that
    \begin{equation*}
      V = U \oplus W.
    \end{equation*}
  \item $V$ has a unique maximal proper subrepresentation $J_{\max}$ and $V/J_{\max}$ is the unique irreducible quotient of $V$ and it is also a highest weight representation with highest weight $(c, h)$.
  \end{enumerate}
\end{lemma}

\begin{proof}\leavevmode
  \begin{enumerate}
  \item This is \Cref{lmm:12} with $H = L_0$.
  \item Assume we find such a decomposition.
    Both $U$ and $W$ are graded subrepresentations and therefore we must have either $v \in U$ or $v \in W$ which implies either $U = V$ or $W = V$.
  \item Just take $J_{\max}$ as the sum of all proper subrepresentations of $V$.\qedhere
  \end{enumerate}
\end{proof}

Let $V$ be an highest weight representation of $\Vir$ with highest weight $(c, h)$.
The character of $V$, denoted by $\ch_V(q)$, is formal power series given by
\begin{equation*}
  \ch_V(q) = \sum_{j \in \mathbb{N}}\dim(V_{h + j})q^{h + j} \in q^h\mathbb{C}[[q]].
\end{equation*}
The character of $V$ satisfies the inequality
\begin{equation*}
  \ch_V(q) \le \frac{q^h}{\prod_{j \in \mathbb{N}}(1 - q^j)}.
\end{equation*}

A Verma representation is a highest weight representation of $\Vir$ in which all vectors of the form $L_{-i_k}\dots L_{-i_1}v$ with $i_k \ge \dots \ge i_1 > 0$ are linearly independent.

Since all vectors of the form $L_{-i_k}\dots L_{-i_1}v$ with $i_k \ge \dots \ge i_1 > 0$ in a Verma representation $V$ with highest weight $(c, h)$ are linearly independent, it follows that they form a basis of $V$ and that there is a homomorphism from $V$ to any other highest weight representation of $\Vir$ with the same highest weight $(c, h)$ mapping a highest weight vector to a highest weight vector.
In particular, for any pair $(c, h)$ of complex numbers, there is at most one Verma representation having $(c, h)$ as its highest weight.
We now show we do have a Verma representation for each pair $(c, h)$ of complex numbers using standard Lie algebra techniques.
Consider the subalgebra $\Vir_{\ge 0, C} = \bigoplus_{n \in \mathbb{N}}\mathbb{C}L_n\oplus \mathbb{C}C$ of $\Vir$ acting on $\mathbb{C}$ as follows
\begin{align*}
  L_n1 &= 0 \quad \text{for }n \in \mathbb{Z}_+, \\
  L_01 &= h, \\
  C1 &= c.
\end{align*}
Then
\begin{equation*}
  M(c, h) = \Ind^{\Vir}_{\Vir_{\ge 0, C}}(\mathbb{C}) = U(\Vir) \otimes_{U(\Vir_{\ge 0, C})} \mathbb{C},
\end{equation*}
is a $\Vir$-module where $\Vir$ acts by left multiplication.

For a partition $\lambda = [\lambda_1, \dots, \lambda_m]$, define $L_{\lambda} = L_{-\lambda_1}\dots L_{-\lambda_m} \in U(\Vir)$.
By the PBW theorem, the set
\begin{equation*}
  \{L_\lambda|c,h\rangle \mid \lambda\text{ is a partition}\}
\end{equation*}
is a vector space basis of $M(c, h)$.
Therefore, $M(c, h)$ is a Verma representation of $\Vir$ with highest weight $(c, h)$ and highest weight vector $|c, h\rangle = 1\otimes1$.
We usually simplify $|c, h\rangle$ to just $|h\rangle$ when $c$ is understood.
By \eqref{eq:33}, the character of a Verma representation is given by
\begin{equation*}
  \ch_{M(c, h)}(q) = \frac{q^h}{\prod_{j \in \mathbb{N}}(1 - q^j)}.
\end{equation*}

Any other highest weight representation $V$ with highest weight $(c, h)$ is a quotient of $M(c, h)$, just map $v$ to a highest weight vector of $V$, the resulting homomorphism is surjective and $V$ is isomorphic to $M(c, h)$ quotiented by the kernel of this homomorphism.

By \Cref{lmm:14}(iii), $M(c, h)$ has a unique maximal subrepresentation $J(c, h)$.
The quotient
\begin{equation*}
  L(c, h) = M(c, h)/J(c, h)
\end{equation*}
is an irreducible highest weight representation with highest weight $(c, h)$.
Actually, this is the unique irreducible highest weight representation with highest weight $(c, h)$ because if $V$ is such a representation, then $V$ is isomorphic to $M(c, h)/U$ for some subrepresentation $U$ of $M(c, h)$ which is maximal, then $U = J(c, h)$.
We wish to determine when is $L(c, h)$ equal to $M(c, h)$.

A vector $u$ in $\Vir$-module is called singular if it is nonzero and
\begin{equation}
  \label{eq:34}
  L_nu = 0 \quad \text{for }n \in \mathbb{Z}_+.
\end{equation}

\begin{lemma}
  \label{lmm:15}
  Let $V$ be a $\Vir$-module.
  If $u \in V$ is nonzero and $L_1u = L_2u = 0$ then $u$ is a singular vector. 
\end{lemma}

\begin{proof}
  The condition $L_1u = L_2u = 0$ implies, using induction, that $L_nu = 0$ for $n \ge 3$.
\end{proof}

\begin{remark}
  \label{rmk:22}
  By \eqref{eq:32}, a highest weight vector is singular. 
  By \eqref{eq:29}, we can  write a singular vector $u$ as $u = \sum_{j \in \mathbb{N}}u_{h + j}$ where $u_{h + j} \in V_{h + j}$.
  Then all the nonzero vectors $u_{h + j}$ are also singular vectors.
  Therefore, we can focus on homogeneous singular vectors.
\end{remark}

\begin{lemma}
  \label{lmm:16}
  A highest weight representation is irreducible if and only if it has no singular vectors other than nonzero scalar multiples of a highest weight vector.
\end{lemma}

\begin{theorem}
  \label{thr:25}
  Let $V$ be a highest weight representation of $\Vir$ with a contravariant form $\langle \bullet, \bullet\rangle$.
  Then all eigenspaces of $L_0$ are pairwise orthogonal.
  Let $U$ be a subrepresentation of $V$.
  Then $U^\perp$ is a subrepresentation of $V$.
  Moreover, if $\langle \bullet, \bullet\rangle$ is unitary, then $U\oplus U^\perp=V$.
\end{theorem}

\begin{proof}
  Let $(c, h)$ be the highest weight of $V$ and let $v$ be a highest weight vector in $V$.
  We have the decomposition $V = \bigoplus_{j \in \mathbb{N}}V_{h + j}$.
  First, we note that if $j_1 \neq j_2$, then $\langle V_{h + j_1}| V_{h + j_2}\rangle = 0$.
  This is because if $u \in V_{h + j_1}$ and $w \in V_{h + j_2}$, then
  \begin{equation*}
    (\overline{h} + j_1)\langle u| w\rangle = \langle L_0u| w\rangle = \langle u| L_0w\rangle = (h + j_2)\langle u| w\rangle
  \end{equation*}
  which implies $(h - \overline{h} + j_2 - j_1)\langle u| w\rangle = 0$ and $\langle u| w\rangle = 0$.
  We know $\langle U| U^{\perp}\rangle = 0$ and $L_jU \subseteq U$ for $j \in \mathbb{Z}$.
  Then $0 = \langle L_jU| U^{\perp}\rangle = \langle U| L_{-j}U^{\perp}\rangle$ which implies $L_jU^{\perp} \subseteq U^{\perp}$ for $j \in \mathbb{Z}$, so $U^{\perp}$ is a subrepresentation of $V$.
  By \Cref{lmm:14},
  \begin{equation*}
     U^{\perp} = \bigoplus_{j \in \mathbb{N}}U^{\perp} \cap V_{h + j} = \bigoplus_{j \in \mathbb{N}}(U \cap V_{h + j})^{\perp|V_{h + j}},
  \end{equation*}
  where $(U \cap V_{h + j})^{\perp|V_{h + j}}$ denotes the orthogonal subspace of $U \cap V_{h + j}$ in $V_{h + j}$.
  If $\langle \bullet| \bullet\rangle$ is unitary, then by \Cref{lmm:13}, $U \cap V_{h + j} \oplus (U \cap V_{h + j})^{\perp|V_{h + j}} = V_{h + j}$ for $j \in \mathbb{N}$ because all the vector spaces $V_{h + j}$ are finite dimensional.
  It is now clear that $U \oplus U^{\perp} = V$.
\end{proof}

\begin{corollary}
  \label{crl:2}
  A highest weight representation $V$ with a unitary form is irreducible.
\end{corollary}

\begin{proof}
  Let $U$ be a subrepresentation of $V$.
  By \Cref{thr:25}, $U^{\perp}$ is a subrepresentation of $V$ and $U\oplus U^{\perp} = V$.
  By \Cref{lmm:14}(ii), either $U = V$ or $U = 0$.
  We conclude that $V$ is irreducible.
\end{proof}

\subsection{Kac determinant formula and singular vectors}
\label{sec:kac-dete-form}

Let $V$ be a highest weight representation with highest weight $(c, h)$ and pick a highest weight vector $v$.
We wish to define a contravariant form $\langle \bullet| \bullet\rangle$ on $V$ such that $\langle v| v\rangle = 1$.
We now show we don't have much choice.
Since $V_h = \mathbb{C}v$, it makes sense to define for $u \in V$, its expectation value $\langle u\rangle$ as the coefficient of $v$ with respect to the direct sum $V = \bigoplus_{j \in \mathbb{N}}V_{h + j}$, i.e.\ as the unique $t \in \mathbb{C}$ such that $u - tv \in \bigoplus_{j \in \mathbb{Z}_+}V_{h + j}$.
Let $P = L_{-i_t}\dots L_{-i_1} \in U(\Vir)$ with $i_t \ge \dots \ge i_1 > 0$ and $u \in V$.
By contravariance, we must have $\langle Pv| u\rangle = \langle v| \omega(P)u\rangle$.
But $\langle v| \omega(P)u\rangle = \langle \omega(P)u\rangle$ because all eigenspaces of $L_0$ are pairwise orthogonal by \Cref{thr:25}.
Therefore, if $\langle \bullet| \bullet\rangle$ is a contravariant Hermitian nonzero form on $V$ such that $\langle v| v\rangle = 1$, we are forced to have
\begin{equation}
  \label{eq:35}
  \langle Pv| u\rangle = \langle \omega(P)u\rangle,
\end{equation}
for $P = L_{-i_t} \dots L_{-i_1} \in U(\Vir)$ with $i_t \ge \dots \ge i_1 > 0$ and $u \in V$.

Contravariance of $\langle \bullet| \bullet\rangle$ and \eqref{eq:35} imposes conditions on the highest weight $(c, h)$.
Let $R\in U(\Vir)$.
Then
\begin{equation*}
  \overline{\langle Rv\rangle} = \overline{\langle \omega(\omega(R))v\rangle} = \overline{\langle \omega(R)v| v\rangle} = \langle v| \omega(R)v\rangle = \langle Rv| v\rangle = \langle \omega(R)v\rangle.
\end{equation*}
Taking $R = C$ and $R = L_0$, we obtain $c, h \in \mathbb{R}$.

Note, however, that initially we cannot define the form using \eqref{eq:35} for a general highest weight representation because there may be linear dependences between terms of the form $L_{-i_t}\dots L_{-i_1}v$ with $i_t \ge \dots \ge i_1 > 0$ so we may ask if it is well defined in the first place.
That is only possible for $M(c, h)$ and later we will see that is also possible for any highest weight representation.

\begin{theorem}[{\cite[Proposition 3.4]{kac_bombay_2013}}]
  \label{thr:26}
  Let $M(c, h)$ be the Verma representation of $\Vir$ with highest weight $(c, h)$ where $c, h \in \mathbb{R}$.
  Pick a highest weight vector $v$ and define the form $\langle \bullet| \bullet\rangle$ on $M(c, h)$ using equation \eqref{eq:35}.
  Then:
  \begin{enumerate}
  \item $\langle \bullet| \bullet\rangle$ is a contravariant Hermitian form on $M(c, h)$ such that $\langle v| v\rangle = 1$;
  \item The eigenspaces of $L_0$ are pairwise orthogonal;
  \item $M(c, h)^{\perp} = J(c, h)$;
  \item Every highest weight representation $V$ of $\Vir$ with highest weight $(c, h)$ and highest weight vector $v'$ carries a contravariant form such that $\langle v'| v'\rangle = 1$.
    This form on $V$ is defined by \eqref{eq:35} with $v'$ in place of $v$ and satisfies all of the properties of this Theorem.
    In particular, for $L(c, h)$, this form is nondegenerate;
  \item Let $V$ be a highest weight representation with highest weight vector $v$.
    There is a unique contravariant form $\langle \bullet| \bullet\rangle$ on $V$ such that $\langle v| v\rangle = 1$ and is defined by equation \eqref{eq:35}.
    If we pick other highest weight vector, the resulting contravariant form is the previously defined form times a positive constant.
  \end{enumerate}
\end{theorem}

Let $V$ be a highest weight representation with highest weight $(c, h)$ and pick a highest weight vector $v$.
Let $\langle \bullet| \bullet\rangle$ be the contravariant form on $V$ defined by \eqref{eq:35}.
We say that $V$ is unitary if this form is unitary.
By \Cref{thr:26}, this is independent of the choice of the highest weight vector $v$ and there is essentially one form.
By \Cref{crl:2}, a unitary highest weight representation is necessarily irreducible, therefore it is of the form $L(c, h)$ for some real numbers $c, h \in \mathbb{R}$.

We wish to study when is $L(c,h)$ unitary.
A simple necessary condition is given by the next theorem.

\begin{theorem}
  \label{thr:27}
  If $L(c, h)$ is unitary then $c \ge 0$ and $h \ge 0$.
\end{theorem}

\begin{proof}
  Pick a highest weight vector $v$ and assume $\langle v| v\rangle=1$.
  A necessary condition for unitarity is that
  \begin{equation*}
    c_n = \langle L_{-n}v| L_{-n}v\rangle > 0 \quad \text{for } n \in \mathbb{N}.
  \end{equation*}
  But contravariance and the commutation rules of Virasoro show that
  \begin{equation}
    \label{eq:36}
    c_n = 2nh + c(n^3 - n)/12.
  \end{equation}
  Putting $n = 1$ we get $c_1 = 2h$ so that we must have $h \ge 0$.
  Moreover, \eqref{eq:36} shows that $c_n$ is dominated by $cn^3$ for large $n$, so that $c \ge 0$ is also necessary.
\end{proof}

Let $M(c, h)$ be the Verma representation with highest weight $(c, h)$.
For $n \in \mathbb{N}$, the subspace $M(c, h)_{h + n}$ is finite dimensional.
Therefore, we can consider the determinant $\det_n(c, h)$ of the contravariant Hermitian form $\langle \bullet| \bullet\rangle$ restricted to $M(c, h)_{h + n}$.
This is well defined up to a nonzero constant.

\begin{theorem}
  \label{thr:28}
  The Verma representation $M(c, h)$ is irreducible if and only if for $n \in \mathbb{Z}_+$, $\det_n(c, h) \neq 0$.
\end{theorem}

\begin{proof}
  By Theorem \ref{thr:26}, $M(c, h)$ is irreducible if and only if $M(c, h)^{\perp} = 0$.
  By Theorem \ref{thr:25}, $M(c, h)^{\perp} = \bigoplus_{j \in \mathbb{N}}M(c, h)^{\perp} \cap M(c, h)_{h + j} = \bigoplus_{j \in \mathbb{N}}M(c, h)_{h + j}^{\perp}$.
  Therefore, $M(c, h)$ is irreducible if and only if for $j \in \mathbb{N}$, $M(c, h)_{h + j}^{\perp} = 0$ which is equivalent to for $n \in \mathbb{Z}_+$, $\det_n(c, h) \neq 0$.
\end{proof}

Thus, to determine when $=M(c, h)$ is irreducible, it is worthwhile to study the number $\det_n(c, h)$ for $n \in \mathbb{N}$.
Fortunately, there is a formula for this.

\begin{theorem}[Kac determinant formula {\cite[Theorem 4.2]{iohara_representation_2011}}]
  \label{thr:29}
  For $n \in \mathbb{N}$,
  \begin{equation*}
    \textstyle\det_n(c, h) = \displaystyle\text{constant}\cdot \prod_{\substack{k, l \in \mathbb{Z}_+\\k \ge l\\ 1 \le kl \le n}} \phi_{k, l}(c, h)^{p(n - kl)}
  \end{equation*}
  where
  \begin{equation*}
    \phi_{k, l}(c, h)=
    \begin{cases}
      (h + \frac{(k^2 - 1)(c - 13)}{24} + \frac{kl - 1}{2})(h + \frac{(l^2 - 1)(c - 13)}{24} + \frac{kl - 1}{2}) + \frac{(k^2 - l^2)^2}{16} & \text{if }k \neq l\\
      h + \frac{(k^2 - 1)(c - 13)}{24} + \frac{k^2 - 1}{2} & \text{if }k = l
    \end{cases}
  \end{equation*}
\end{theorem}

\begin{remark}
  \label{rmk:23}
  Using Kac determinant formula, it is possible to prove that $L(1/2, h)$ unitary implies $h = 0$, $1/2$ or $1/16$ (see \cite[\S3]{kac_bombay_2013}).
  This is one of the reasons we study $L(1/2, 1/2)$ and $L(1/2, 1/16)$ in this work.
\end{remark}

We will need to find the maximal submodules of $M(1/2, 1/2)$ and $M(1/2, 1/16)$ explicitly.
Kac determinant formula also helps with this, allowing us to compute $J(c, h)$ for rational numbers $c$ and $h$.

It turns out $J(c, h)$ is generated by at most two singular vectors which can be computed explicitly for isolated cases.
It is always possible to write (nonuniquely)
\begin{equation*}
  c = \frac{(3p + 2q)(3q + 2p)}{pq}, h = \frac{(p + q)^2 - m^2}{4pq}
\end{equation*}
for some $p,q\in \mathbb{C}\setminus\{0\}$ and $m\in \mathbb{C}$.
Then

\begin{equation*}
  \phi_{k, l}(c, h) =
  \begin{cases}
    \frac{(pk + ql + m)(qk + pl + m)(pk + ql - m)(qk + pl - m)}{pq} & k \neq l\\
    \frac{(pk + qk + m)(pk + qk - m)}{4pq} & k = l
  \end{cases}
\end{equation*}

Therefore, to find singular vectors in $M(c, h)$ we have to study integral solutions to the linear equation $pk + ql + m = 0$.
Let $l_{c, h}$ denote the real solutions to this linear equation.

\begin{theorem}[{\cite{astashkevich_structure_1997}}]
  \label{thr:30}
  The integral points (the points in $\mathbb{Z}^2$) of $l_{c, h}$ determine the maximal subrepresentation of $M(c, h)$ completely according to the following three cases:
  \begin{description}[leftmargin = !]
  \item[Case I] The line $l_{c, h}$ contains no integral points.
    In this case $J(c, h) = 0$
  \item[Case II] The line $l_{c, h}$ contains exactly one integral point $(k, l)$.
    We have three subcases:
    \begin{description}[leftmargin = !]
    \item [Subcase II$_+$] The product $kl > 0$.
      Let $u$ be a singular vector in $M_{c, h + kl}$.
      Then $J(c, h) = U(\Vir)\{u\}$.
    \item[Subcase II$_0$] The product $kl = 0$.
      In this subcase, $J(c, h) = 0$.
    \item[Subcase II$_-$] The product $kl < 0$.
      In this subcase, $J(c, h) = 0$.
    \end{description}
  \item[Case III] The line $l_{c, h}$ contains infinitely many integral points.
    Let $(k_1, l_1), (k_2, l_2), \dots$ be all integral points on the line $l_{c, h}$ up to equivalence relation $(k, l)\sim(k', l')$ if and only if  $kl = k'l'$ and such that $kl > 0$.
    We ordered them in such a way that $k_il_i < k_{i + 1}l_{i + 1}$ for $i \in \mathbb{Z}_+$.
    We have two subcases:
    \begin{description}[leftmargin = !]
    \item[Subcase $c\le 1$] We have three subsubcases:
      \begin{description}[leftmargin = !]
      \item [Subsubcase III$^{00}_-$] Line $l_{c, h}$ intersects both axes at integral points.
        Let $u$ be a singular vector in $M_{c, h + k_1l_1}$.
        Then $J(c, h) = U(\Vir)\{u\}$.
      \item[Subsubcase III$^0_-$] Line $l_{c, h}$ intersects only one axis at integral point.
        Let $u$ be a singular vector in $M_{c, h + k_1l_1}$.
        Then $J(c, h) = U(\Vir)\{u\}$.
      \item[Subsubcase III$_-$] Line $l_{c, h}$ intersects both axes at nonintegral points.
        Let $u$ and $w$ be singular vectors in $M_{c, h + k_1l_1}$ and $M_{c, h + k_2l_2}$, respectively.
        Then $J(c, h) = U(\Vir)\{u, w\}$.
      \end{description}
    \item[Subcase $c\ge 25$] We have three subsubcases:
      \begin{description}[leftmargin = !]
      \item[Subsubcase III$^{00}_+$] Line $l_{c, h}$ intersects both axes at integral points.
        Let $u$ be a singular vector in $M_{c, h + k_1l_1}$.
        Then $J(c, h) = U(\Vir)\{u\}$.
      \item[Subsubcase III$^0_+$] Line $l_{c, h}$ intersects only one axis at integral point.
        Let $u$ be a singular vector in $M_{c, h + k_1l_1}$.
        Then $J(c, h) = U(\Vir)\{u\}$.
      \item[Subsubcase III$_+$] Line $l_{c, h}$ intersects both axes at nonintegral points.
        Let $u$ and $w$ be singular vectors in $M_{c, h + k_1l_1}$ and $M_{c, h + k_2l_2}$, respectively.
        Then $J(c, h) = U(\Vir)\{u, w\}$.
      \end{description}
    \end{description}
  \end{description}
\end{theorem}

\begin{remark}
  \label{rmk:24}
  \Cref{thr:29} gives us an algorithm to find $J(c, h)$ for a given highest weight $(c, h)$.
  We merely need to find the levels at which the singular vectors that generate $J(c, h)$ lie and then by \Cref{lmm:15}, we have to solve the equation $L_1u = L_2 u = 0$ assuming $u$ lies in the right level to obtain our desired singular vector.
\end{remark}

One of the objectives of this work is to find a monomial basis for $L(1/2, 1/2)$ and $L(1/2, 1/16)$ similar to what was done for the Ising model $\Vir_{3, 4} = L(1/2, 0)$ in \cite{andrews_singular_2022}.
The Ising model is also a simple vertex algebra because $\Vir^{1/2} \cong M(1/2, 0)/U(\Vir)\{L_{-1}\vac\}$ as a $\Vir$-module and by \Cref{rmk:20}, the ideals of $\Vir^{1/2}$ are in bijection with the submodules of $M(1/2, 0)/U(\Vir)\{L_{-1}\vac\}$.
Let $I^{1/2}$ be the maximal ideal of $\Vir^{1/2}$ and by what we just said, to $I^{1/2}$ corresponds the maximal submodule of $M(1/2, 0)/U(\Vir)\{L_{-1}\vac\}$.
Therefore, $\Vir_{1/2} = \Vir^{1/2}/I^{1/2} \cong L(1/2, 0) = \Vir_{3, 4}$.

In fact, the singular vector $v_{3, 4}$ = ... can be obtained using the algorithm described in remark... See (code) and (section) ahead.

The modules $L(1/2, 1/2)$ and $L(1/2, 1/1/16)$ are not vertex algebras but they are modules (vertex algebra modules) over both $\Vir^{1/2}$ and $\Vir_{3, 4}$, as we will see later.

\begin{remark}
  \label{rmk:25}
  The irreducible highest weight representations $L(1/2, 1/2)$ and $L(1/2, 1/16)$ can be constructed explicitly as the even or odd part of some induced representations without using Verma representations and passing to the quotient as was done here (cf.\ \cite[\S3]{kac_bombay_2013}).
  Moreover, $L(1/2, 1/2)$ is isomorphic to $F_{\one}$ where $F = F(\mathbb{C}\varphi)$ is the fermionic vertex algebra associated to a purely odd one dimensional superspace $\mathbb{C}\varphi$ with an antisupersymmetric form defined by $\langle \varphi| \varphi\rangle = 1$ (cf.\ \Cref{exa:10}).
\end{remark}

\section{Vertex Poisson algebras and filtrations of vertex algebras}
\label{sec:vert-poiss-algebr}

\subsection{Vertex Lie algebras and their modules}
\label{sec:vert-lie-algebr}

Let $V$ be a vector space.
Given a formal distribution $f(X_1, \dots, X_n) \in V[[X_1^{\pm 1}, \dots, X_n^{\pm 1}]]$, we can write it as
\begin{equation*}
  f(X_1, \dots, X_n) = \sum_{m_1, \dots, m_n \in \mathbb{Z}}f_{(m_1, \dots, m_n)}X_1^{-m_1 - 1}\dots X_n^{-m_n - 1}.
\end{equation*}
We set
\begin{equation*}
  \Sing f(X_1, \dots, X_n) = \sum_{m_1, \dots, m_n \in \mathbb{N}}f_{(m_1, \dots, m_n)}X_1^{-m_1 - 1}\dots X_n^{-m_n - 1}.
\end{equation*}

A vertex Lie algebra is the data $(V, T, Y_-)$ where $V$ is a vector superspace, $Y_-: V \to \mathcal{F}(V)$ is a linear parity preserving map such that $Y_-(a,z) = \Sing(Y_-(a, z))$ (i.e.\ $Y_-: V \to \Hom(V, z^{-1}V[z^{-1}])$) for $a\in V$ and $T \in \End(V)$.
The data must satisfy the following axioms for $a, b \in V$:
\begin{enumerate}
\item $Y_-(Ta, z) = \partial_zY_-(a, z)$;
\item $Y_-(a, z)b = \Sing(p(a, b)(e^{Tz}Y_-(b, -z)a))$;
\item $[Y_-(a, z),Y_-(b, w)] = \Sing(\sum_{j \in \mathbb{N}}Y_-(a_{(j)}b, w)\frac{\partial^j_w\delta(z, w)}{j!})$ where $Y_-(a, z) = \sum_{n \in \mathbb{N}}a_{(n)}z^{-n - 1}$, $a_{(n)} \in \End(V)$.
\end{enumerate}

\begin{remark}
  \label{rmk:26}
  Axiom (i) of vertex Lie algebras implies that $T \in \End(V)_{\zero}$.
\end{remark}

Concepts like homomorphism, vertex Lie subalgebras and ideals are defined in the usual way.
We obtain the category of vertex Lie algebras.

Let $V$ be vertex Lie algebra.
We can make $V$ into a $C[\partial]$-module by declaring $\partial = T$.
Then axioms (i) and (ii) are the respective axioms of Lie conformal superalgebras given in \Cref{sec:lie-conf-super}.
Since
\begin{equation*}
  \sum_{j \in \mathbb{N}}Y_-(a_{(j)}b, w)\frac{\partial^j_w\delta(z, w)}{j!} = \sum_{m, n \in \mathbb{Z}}\left(\sum_{j, n \in \mathbb{N}, j + k = m + n}\binom{m}{j}(a_{(j)}b)_{(k)}\right)z^{-m - 1}w^{-n - 1},
\end{equation*}
we have
\begin{equation*}
  \Sing\left(\sum_{j \in \mathbb{N}}Y_-(a_{(j)}b, w)\frac{\partial^j_w\delta(z, w)}{j!}\right) = \sum_{m, n \in \mathbb{N}}\sum_{j = 0}^m\binom{m}{j}(a_{(j)}b)_{(m + n - j)}z^{-m - 1}w^{-n - 1}.
\end{equation*}
On the other hand,
\begin{equation*}
  [Y_-(a, z), Y_-(b, w)] = \sum_{m, n \in \mathbb{N}}[a_{(m)}, b_{(n)}]z^{-m - 1}w^{-n - 1}.
\end{equation*}
Therefore, axiom (iii) is equivalent to
\begin{equation*}
  [a_{(m)}, b_{(n)}] = \sum_{j = 0}^m\binom{m}{j}(a_{(j)}b)_{(m + n - j)} \quad \text{for }a, b \in V\text{ and }m, n \in \mathbb{Z}.
\end{equation*}

This is just axiom (iii) of Lie conformal superalgebras.
Thus, the category of vertex Lie algebras and the category of Lie conformal superalgebras are isomorphic.
So we have three equivalent concepts: vertex Lie algebras, Lie conformal superalgebras y regular formal distribution Lie superalgebras.

\begin{theorem}
  \label{thr:31}
  Let $V$ be a vertex Lie algebra.
  For $a, b \in V$ and $m, j \in \mathbb{N}$:
  \begin{enumerate}
  \item $[T, Y_-(a, z)] = Y_-(Ta, z) = \partial_zY_-(a, z)$;
  \item $(a_{(j)}b)_{(m)} = \sum_{k = 0}^j\binom{j}{k}(-1)^k[a_{(j - k)},b_{(m + k)}]$.
  \end{enumerate}
\end{theorem}

\begin{proof}\leavevmode
  \begin{enumerate}
  \item This is \Cref{prp:3}.
  \item This is \Cref{thr:6}. \qedhere
  \end{enumerate}
\end{proof}

Let $V$ be a vertex Lie algebra.
A Hamiltonian operator of $V$ is a diagonalizable operator $H \in \End(V)$ such that
\begin{equation}
  \label{eq:37}
  [H, Y_-(a, z)] = z\partial_zY_-(a, z) + Y_-(Ha, z) \quad \text{for }a \in V.
\end{equation}
A vertex Lie algebra with a Hamiltonian operator is called graded.
The grading of $V$ is the eigenspace decomposition for $H$:
\begin{equation*}
  V = \bigoplus_{\Delta \in \mathbb{C}}V_{\Delta}.
\end{equation*}

A module over a vertex Lie algebra $V$ is a vector superspace $M$ together with a linear parity preserving map $Y^M_-: V \to \Hom(M, z^{-1}M[z^{-1}])$, written as $Y^M_-(a, z) = \sum_{n \in \mathbb{N}}a^M_{(n)}z^{-n - 1}$, $a^M_{(n)} \in \End(M)$, such that for $a, b \in V$ and $m, n \in \mathbb{N}$:
\begin{enumerate}
\item $(Ta)^M_{(n)} = -na^M_{(n - 1)}$;
\item $[a^M_{(m)}, b^M_{(n)}] = \sum_{j = 0}^m\binom{m}{j}(a_{(j)}b)^M_{(m + n - j)}$.
\end{enumerate}
Concepts like homomorphisms and vertex Lie submodules are defined in the usual way.
Given a vertex Lie algebra $V$, we obtain the abelian category $V$-mod of modules over $V$.

Let $M$ be a module over a graded vertex Lie algebra $V$ with Hamiltonian operator $H$. 
A Hamiltonian operator of $M$ is a diagonalizable operator $H^M \in \End(M)$ such that
\begin{equation}
  \label{eq:38}
  [H^M, Y^M_-(a, z)] = z\partial_zY^M_-(a, z) + Y^M_-(Ha, z) \quad \text{for }a \in V.
\end{equation}
A $V$-module together with a Hamiltonian operator is called graded.
The grading of $M$ is the eigenspace decomposition for $H^M$:
\begin{equation*}
  M = \bigoplus_{\Delta \in \mathbb{C}}M_{\Delta}.
\end{equation*}

Let $V$ be a vertex Lie algebra and $M$ a $V$-module.
We say $T^M \in \End(M)$ is a differential of $M$ if
\begin{equation*}
  [T^M, Y^M_-(a, z)] = Y^M_-(Ta, z) \quad \text{for }a \in V.
\end{equation*}
A differential $V$-module is a module equipped with a differential.

\subsection{Vertex Poisson algebras and their modules}
\label{sec:vert-poiss-algebr}

A Vertex Poisson algebra is a triple $(V, T, Y_-)$ such that:
\begin{enumerate}
\item $(V, T)$ is a differential commutative associative algebra;
\item $(V, T, Y_-)$ is a vertex Lie algebra;
\item The left Leibniz rule holds:
  \begin{equation*}
    Y_-(a, z)(bc) = (Y_-(a, z)b)c + b(Y_-(a, z)c) \quad \text{for }a, b, c \in V.
  \end{equation*}
\end{enumerate}
The left Leibniz rule implies that $a_{(n)} \in \Der(A)$ for $a \in A$ and $n \in \mathbb{N}$.
Therefore, we have
\begin{equation*}
  Y_-(a, z) \in z^{-1}(\Der(A)) \text{for }a \in A.
\end{equation*}
In particular, $a_{(n)}1 = 0$ for $a \in A$ and $n\in \mathbb{N}$.
Therefore,
\begin{equation*}
  Y_-(a, z)1 = 0 \quad \text{for } a \in A.
\end{equation*}
By skewsymmetry, 
\begin{equation*}
  Y_-(1,z)=0.
\end{equation*}

Let $V$ and $W$ be vertex Poisson algebras.
A vertex Poisson algebra homomorphism $f: V \to W$ is simultaneously a differential algebra and a vertex Lie algebra homomorphism.
We obtain the category of vertex Poisson algebras.

A module over a vertex Poisson algebra $(V, T, Y_-)$ is a module $(M, Y^M_-)$ for $V$ as a vertex Lie algebra and a module over $V$ as a commutative associative algebra such that
\begin{equation*}
  a^M_{(n)}(bm) = (a_{(n)}b)m+b(a^M_{(n)}m) \quad \text{for }a, b \in V, m \in M\text{ and }n \in \mathbb{N}.
\end{equation*}
Concepts like homomorphisms and vertex Poisson submodules are defined in the usual way.
Given a vertex Poisson algebra $V$, we obtain the abelian category $V$-mod of modules over $V$.

\subsection{Filtrations of vertex algebras}
\label{sec:filtr-vert-algebr}

Let $V$ be a vertex algebra and $(a^i)_{i \in I}$ a family of strong generators of $V$.
For $p \in \mathbb{Z}$, set
\begin{equation*}
  F_pV = \vspan\{a^{i_1}_{(-n_1 - 1)}\dots a^{i_s}_{(-n_s - 1)}\vac \mid s, n_1, \dots, n_s \in \mathbb{N}, i_1, \dots, i_s \in I, n_1 + \dots + n_s \ge p\}.
\end{equation*}

\begin{proposition}[{\cite{li_abelianizing_2005}}]
  \label{prp:7}
  The filtration $(F_pV)_{p\in \mathbb{Z}}$ satisfies:
  \begin{enumerate}
  \item $F_pV = V$ for $p \le 0$;
  \item $F_0V \supseteq F_1V \supseteq \dots$;
  \item $TF_pV \subseteq F_{p + 1}V$;
  \item $a_{(n)}F_qV \subseteq F_{p + q - n - 1}V$ for $a \in F_pV$ and $n \in \mathbb{Z}$;
  \item $a_{(n)}F_qV \subseteq F_{p + q - n}V$ for $a \in F_pV$ and $n \in \mathbb{N}$.
  \end{enumerate}
\end{proposition}

Let
\begin{equation*}
  \gr_F(V) = \bigoplus_{p \in \mathbb{N}}F_pV/F_{p + 1}V
\end{equation*}
be the associated graded vector space.
By \cite{li_abelianizing_2005}, the vector space $\gr_F(V)$ is a vertex Poisson algebra with operations given as follows: for $p, q \in \mathbb{N}$, $a \in F_pV$ and $b \in F_qV$, set
\begin{align*}
  \sigma_p(a)\sigma_q(b) &= \sigma_{p + q}(a_{(-1)}b), \\
  T\sigma_p(a) &= \sigma_{p + 1}(Ta), \\
  Y_-(\sigma_p(a),z)\sigma_q(b) &= \sum_{n \in \mathbb{N}}\sigma_{p + q - n}(a_{(n)}b)z^{-n - 1},
\end{align*}
where $\sigma_p: F_pV \to \gr_F(V)$ is the principal symbol map which is the composition of the natural maps $F_pV \to F_pV/F_{p + 1}V$ and $F_pV/F_{p+1}V \to \gr_F(V)$.
The filtration $(F_pV)_{p \in \mathbb{Z}}$ is called the Li filtration of $V$.

\begin{lemma}[{\cite[Lemma 2.9]{li_abelianizing_2005}}]
  \label{lmm:17}
  Let $V$ be a vertex algebra.
  Then
  \begin{equation*}
    F_pV = \vspan_{\mathbb{C}}\{a_{(-i - 1)}b \mid a \in V, i \ge 1, b \in F_{p - i}V\} \quad \text{for }p \in \mathbb{Z}_+.
  \end{equation*}
\end{lemma}

By \Cref{lmm:17}, the Li filtration depends only on $V$ and not on the choice of the strong generators.
If $V$ is graded by a Hamiltonian $H$ with grading $V = \bigoplus_{\Delta \in \mathbb{C}}V_{\Delta}$, then $H(F_pV) \subseteq F_pV$ because in that case, for $p \in \mathbb{Z}$,
\begin{equation*}
F_pV =\vspan \{a^1_{(-n_1 - 1)}\dots a^s_{(-n_s - 1)}\vac \mid s, n_1, \dots, n_s \in \mathbb{N}, a^i \in V\text{ homogeneous}, n_1 + \dots + n_s \ge p\}.
\end{equation*}

Therefore, we can define an operator $H \in \End(\gr_F(V))$ as $H(\sigma_p(a)) = \sigma_p(Ha)$ for $p \in \mathbb{N}$ and $a \in F_pV$.
For $\Delta \in \mathbb{C}$, define $F_pV_{\Delta} = F_pV \cap V_{\Delta}$.
Since $H(F_pV) \subseteq F_pV$ for $p \in \mathbb{Z}$, \Cref{lmm:12} implies that
\begin{equation}
  \label{eq:39}
  F_pV = \bigoplus_{\Delta \in \mathbb{C}}F_pV_{\Delta} \quad \text{for }p \in \mathbb{Z}.
\end{equation}
For $\Delta\in \mathbb{C}$, define $\gr_F(V)_{\Delta} = \bigoplus_{p \in \mathbb{N}}\sigma_p(F_pV_{\Delta})$.
Then $Ha = \Delta a$ for $a \in \gr_F(V)_{\Delta}$.
The family of subspaces $(\gr_F(V)_{\Delta})_{\Delta \in \mathbb{C}}$ satisfy $\gr_F(V) = \bigoplus_{\Delta \in \mathbb{C}} \gr_F(V)_{\Delta}$.
Therefore, the operator $H \in \End(\gr_F(V))$ is diagonalizable with $\gr_F(V)_{\Delta} = \ker(H - \Delta\Id_{\gr_F(V)})$.
In fact, more is true.
\begin{theorem}
  \label{thr:32}
  This diagonalizable operator $H$ is a Hamiltonian for $\gr_F(V)$.
\end{theorem}

\begin{proof}
  For $p, q \in \mathbb{N}$, $a \in F_pV$ and $b \in F_qV$,
  \begin{align*}
    [H, Y_-(\sigma_p(a), z)]\sigma_q(b) &= \sum_{n \in \mathbb{N}}\sigma_{p + q -n}([H, a_{(n)}]b)z^{-n -1} \\
    &= \sum_{n \in \mathbb{N}}\sigma_{p + q -n}((-(n + 1)a_{(n)} + (Ha)_{(n)})b)z^{-n - 1} \\
    &= (z\partial_zY_-(\sigma_p(a), z) + Y_-(\sigma_p(Ha), z))\sigma_q(b). \qedhere
  \end{align*}
\end{proof}

We have the natural vector space isomorphisms
\begin{equation*}
  \sigma_p(F_pV_{\Delta}) \cong F_pV_{\Delta}/F_{p + 1}V_{\Delta} \quad \text{for }p \in \mathbb{Z}\text{ and }\Delta \in \mathbb{C}
\end{equation*}
and the double gradation
\begin{equation}
  \label{eq:40}
  \gr_F(V) =\bigoplus_{\substack{p \in \mathbb{N} \\ \Delta \in \mathbb{C}}}\sigma_p(F_pV_{\Delta}).
\end{equation}
By \eqref{eq:40}, it is natural to define refined character of $V$ with respect to the Li filtration as
\begin{equation*}
  \ch_{\gr_F(V)}(t, q) = \sum_{\substack{p \in \mathbb{N} \\ \Delta \in \mathbb{C}}}\dim(\sigma_p(F_pV_{\Delta}))t^pq^{\Delta}.
\end{equation*}

If $f: V \to W$ is a homomorphism of vertex algebras, then
\begin{align*}
  \gr_F(f): \gr_F(V) &\to \gr_F(W) \\
  \sigma^V_p(v) &\mapsto \sigma^W_p(f(v)) \quad \text{for }p \in \mathbb{N}\text{ and } v \in F_pV,
\end{align*}
defines a homomorphism of vertex Poisson algebras.
If $V$ and $W$ are graded, then we require that $f$ respects the gradings of $V$ and $W$ and this implies that $\gr_F(f)$ also respects the gradings of $\gr_F(V)$ and $\gr_F(W)$.
Therefore, we obtain a functor
\begin{equation*}
  \gr_F: \{\text{(graded) vertex algebras}\} \to \{\text{(graded) vertex Poisson algebras}\}.
\end{equation*}

Assume that $V$ is a $\mathbb{N}$-graded vertex algebra with Hamiltonian $H$.
Let $(a^i)_{i \in I}$ be a family of homogeneous strong generators of $V$.
For $p \in \mathbb{Z}$, set
\begin{equation*}
  G^pV = \vspan\{a^{i_1}_{(-n_1 - 1)}\dots a^{i_s}_{(-n_s - 1)}\vac \mid s, n_1, \dots, n_s \in \mathbb{N}, i_1, \dots, i_s \in I, \Delta_{a^{i_1}} + \dots + \Delta_{a^{i_s}} \le p\}.
\end{equation*}

\begin{proposition}[{\cite{li_vertex_2004}}]
  \label{prp:8}
  The filtration $(G^pV)_{p \in \mathbb{Z}}$ satisfies:
  \begin{enumerate}
  \item $G^pV = 0$ for $p < 0$;
  \item $G^0V \subseteq G^1V \subseteq \dots$;
  \item $V = \bigcup_pG^pV$;
  \item $V_n \subseteq G^nV$ for $n \in \mathbb{Z}$;
  \item $a_{(n)}G^qV \subseteq G^{p + q}V$ for $a \in G^pV$ and $n \in \mathbb{Z}$;
  \item $a_{(n)}G^qV \subseteq G^{p + q - 1}V$ for $a \in G^pV$ and $n \in \mathbb{N}$;
  \item $H(G^pV) \subseteq G^pV$ and $TG^pV \subseteq G^pV$ for $p \in \mathbb{Z}$.
  \end{enumerate}
\end{proposition}

Let
\begin{equation*}
  \gr^G(V) = \bigoplus_{p \in \mathbb{N}}G^pV/G^{p - 1}V
\end{equation*}
be the associated graded vector space.
By \cite{li_vertex_2004}, the vector space $\gr^G(V)$ a vertex Poisson algebra with operations given as follows: for $p, q \in \mathbb{N}$, $a \in G^pV$ and $b \in G^qV$, set
\begin{align*}
  \alpha^p(a)\alpha^q(b) &= \alpha^{p + q}(a_{(-1)}b), \\
  T\alpha^p(a) &= \alpha^p(Ta), \\
  Y_-(\alpha^p(a), z)\alpha^q(b) &= \sum_{n \in \mathbb{N}}\alpha^{p + q - 1}(a_{(n)}b)z^{-n - 1},
\end{align*}
where $\alpha^p: G^pV \to \gr^G(V)$ is the principal symbol map.
The filtration $(G^pV)_{p \in \mathbb{Z}}$ is called the standard filtration of $V$.
By \Cref{prp:9} ahead, the standard filtration does not depend on the choice of the strong generators of $V$.

By \Cref{prp:8}(vii), we can define an operator $H \in \End(\gr^G(V))$ as $H(\alpha^p(a)) = \alpha^p(Ha)$ for $p \in \mathbb{Z}$ and $a \in G^pV$.
For $n \in \mathbb{N}$, define $G^pV_n = G^pV \cap V_n$.

Since $H(G^pV) \subseteq G^pV$ for $p \in \mathbb{Z}$, \Cref{lmm:12} implies that
\begin{equation}
  \label{eq:41}
  G^pV = \bigoplus_{n \in \mathbb{N}}G^pV_n \quad \text{for }p \in \mathbb{Z}.
\end{equation}
For $n \in \mathbb{N}$, define $\gr^G(V)_n = \bigoplus_{p \in \mathbb{N}}\alpha^p(G^pV_n)$.
Then $Ha = na$ for $a \in \gr^G(V)_n$.
The family of subspaces $(\gr^G(V)_n)_{n \in \mathbb{N}}$ satisfy $\gr^G(V) = \bigoplus_{n \in \mathbb{N}} \gr^G(V)_n$.
Therefore, the operator $H \in \End(\gr^G(V))$ is diagonalizable with $\gr^G(V)_n = \ker(H - n\Id_{\gr^G(V)})$.
In fact, more is true.

\begin{theorem}
  \label{thr:33}
  This diagonalizable operator $H$ is a hamiltonian for $\gr^G(V)$.
\end{theorem}

\begin{proof}
  The proof for \Cref{thr:32} also works here.
\end{proof}

We have the natural vector space isomorphisms
\begin{equation*}
  \alpha^p(G^pV_n) \cong G^pV_n/G^{p - 1}V_n \quad \text{for }p \in \mathbb{Z}\text{ and }n \in \mathbb{N}
\end{equation*}
and the double gradation
\begin{equation}
  \label{eq:42}
  \gr^G(V) =\bigoplus_{p, n \in \mathbb{N}}\alpha^p(G^pV_n).
\end{equation}
By \eqref{eq:42}, it is natural to define refined character of $V$ with respect to the standard filtration as
\begin{equation*}
  \ch_{\gr^G(V)}(t, q) = \sum_{p, n \in \mathbb{N}}\dim(\alpha^p(G^pV_n))t^pq^n.
\end{equation*}

If $f: V \to W$ is a homomorphism of $\mathbb{N}$-graded vertex algebras, then
\begin{align*}
  \gr^GF(f): \gr^G(V) &\to \gr^G(W) \\
  \alpha_V^p(v) &\mapsto \alpha_W^p(f(v)) \quad \text{for }p \in \mathbb{N}\text{ and } v \in G^pV,
\end{align*}
defines a homomorphism of vertex Poisson algebras.
Therefore, we obtain a functor
\begin{equation*}
  \gr^GF: \{\text{$\mathbb{N}$-graded vertex algebras}\} \to \{\text{$\mathbb{N}$-graded vertex Poisson algebras}\}.
\end{equation*}

\begin{proposition}[{\cite[Proposition 2.6.1]{arakawa_remark_2012}}]
  \label{prp:9}
  Let $V$ be a $\mathbb{N}$-graded vertex algebra.
  Then the Li filtration and standard filtration satisfy
  \begin{equation*}
    F_pV_n = G^{n - p}V_n \quad \text{for }p, n \in \mathbb{N}.
  \end{equation*}
  An explicit isomorphism $\gr_F(V) \xrightarrow{\sim} \gr^G(V)$ of vertex Poisson algebras is defined by extending linearly the isomorphisms of vector spaces given by
  \begin{align*}
    \sigma_p(F_pV_n) &\xrightarrow{\sim} \alpha^{n - p}(G^{n - p}V_n) \\
    \sigma_p(v) &\mapsto \alpha^{n - p}(v),
  \end{align*}
  where $p, n \in \mathbb{N}$ and $v \in F_pV_n$.
\end{proposition}

\begin{remark}
  \label{rmk:27}
  In \cite{arakawa_remark_2012}, it is suggested in a footnote that it is possible to consider more general $\tfrac{1}{r_0}\mathbb{N}$-graded vertex algebras instead of $\mathbb{N}$-graded vertex algebras.
  However, I don't think that is possible.
  For example, if $V = \Vir^{1/2}$ (which is the case we are interested here), then $V$ is $\tfrac{1}{2}\mathbb{N}$-graded.
  According to the definition of $\gr^G(V)$ given in \cite{arakawa_remark_2012}, we should have $\gr^G(V) = \bigoplus_{p \in \tfrac{1}{r_0}\mathbb{N}}G_pV/G_{p - 1}V = G_0V/G_{-1}V \oplus G_{1/2}V/G_{-1/2}V \oplus \dots = \mathbb{C}\vac \oplus \mathbb{C}\vac \oplus \dots$, which means ``the vacuum is doubled''.
  I don't think that is intended and also the proof of \cite[Proposition 2.6.1]{arakawa_remark_2012} does not work for $\tfrac{1}{r_0}\mathbb{N}$-graded vertex algebras.
  But I do think the hypothesis $V_0 = \mathbb{C}\vac$ can be removed, just as Arakawa wrote.
\end{remark}

\begin{theorem}
  \label{thr:34}
  Let $V$ be a vertex algebra.
  The refined characters of $V$ are related as follows:
  \begin{enumerate}
  \item $\ch_V(q) = \ch_{\gr_F(V)}(q) = \ch_{\gr^G(V)}(q) = \ch_{\gr_F(V)}(1, q) = \ch_{\gr^G(V)}(1, q)$;
  \item $\ch_{\gr^G(V)}(t^{-1}, tq) = \ch_{\gr_F(V)}(t, q)$.
  \end{enumerate}
\end{theorem}

\begin{proof}
  See the proof of \Cref{prp:13} ahead.
\end{proof}

\begin{example}[$\gr^G(\Vir^c)$]
  \label{exa:11}
  Pick any $c \in \mathbb{C}$.
  Define the subalgebra $\Vir_{\le -2} = \bigoplus_{n \le -2} L_{n}$.
  By the PBW theorem, a basis of $U(\Vir_{\le -2})_s$ (see \Cref{sec:almost-comm-algebr}) is given by
  \begin{equation*}
    \{L_{-\lambda_1}\dots L_{-\lambda_t} \mid t \le s\text{ and }[\lambda_1, \dots, \lambda_t]\text{ is a partition with } \lambda_t \ge 2\}.
  \end{equation*}
  Recall that $\Delta_{L_{-2}\vac} = 2$.
  From the definition of the standard filtration and \Cref{exa:5}, we see that for $s \in \mathbb{Z}$, $G^{2s}\Vir^c = G^{2s + 1}\Vir^c \cong U(\Vir_{\le -2})_s$.
  This implies that the quotients $G^{2s}\Vir^c/G^{2s + 1}\Vir^c$ are $0$ for $s \in \mathbb{N}$.
  Therefore, we have a vector space isomorphism
  \begin{align*}
    \gr^G(\Vir^c) &\xrightarrow{\sim} \gr(U(\Vir_{\le -2})) \\
    \alpha^{2s}(L_{\lambda}\vac) &\mapsto \gamma^s(L_{\lambda}) \quad \text{for }s \in \mathbb{N}\text{ and }\lambda = [\lambda_1, \dots, \lambda_s] \text{ a partition}
  \end{align*}
  We now show this is an algebra homomorphism.
  We need to show that for $s, t \in \mathbb{N}$ and partitions $[\lambda_1, \dots, \lambda_s], [\eta_1, \dots, \eta_t]$ with $\lambda_s, \eta_t \ge 2$,
  \begin{equation*}
    \alpha^{2s}(L_{\lambda}\vac)\alpha^{2t}(L_{\eta}\vac) \mapsto \gamma^{s + t}(L_{\lambda}L_{\eta}),
  \end{equation*}
  which is equivalent to
  \begin{equation*}
    \alpha^{2s + 2t}((L_{\lambda}\vac)_{(-1)}(L_{\eta}\vac)) \mapsto \gamma^{s + t}(L_{\lambda}L_{\eta}).
  \end{equation*}
  Therefore, we have to show that
  \begin{equation}
    \label{eq:43}
    (L_{\lambda}\vac)_{(-1)}(L_{\eta}\vac) = (L_{\lambda}L_{\eta})\vac + v \quad \text{for some }v \in G^{2s - 1}\Vir^c.
  \end{equation}

  If $s = 0$ or $s = 1$, \eqref{eq:43} is clear.
  Assume $s \ge 2$.
  By \Cref{crl:1},
  \begin{align*}
    (L_\lambda\vac)_{(-1)}(L_{\eta}\vac) &= \left(\frac{:\partial_z^{\lambda_1 - 2}L(z)\dots\partial_z^{\lambda_s - 2}L(z):}{(\lambda_1 - 2)!\dots(\lambda_s - 2)!} \right)_{(-1)}(L_{\eta}\vac) \\
    &= \frac{(:\partial_z^{\lambda_1 - 2}L(z)\dots\partial_z^{\lambda_s - 2}L(z):)_{(-1)}(L_{\eta}\vac)}{(\lambda_1 - 2)!\dots(\lambda_s - 2)!},
  \end{align*}
  where $L(z) = \sum_{n \in \mathbb{Z}}L_nz^{-n - 2} \in \mathcal{F}(V)$.
  Now we use \Cref{lmm:3} with $V = \Vir^c$, $a^k(z) = \partial^{\lambda_k - 2}L(z)$ for $k = 1, \dots, s$ and $u = (L_{\eta}\vac)$.
  First, note that for $k = 1, \dots, s$,
  \begin{align*}
    a^k(z) &= \partial^{\lambda_k - 2}L(z) \\
    &= \sum_{n \in \mathbb{Z}}(-n - 2)\dots(-n - 2 - (\lambda_k - 3))L_nz^{-n - 2 - (\lambda_k - 2)} \\
    &= \sum_{n \in \mathbb{Z}}(-n + \lambda_k - 3)\dots(-n)L_{n + 1 - \lambda_k}z^{-n - 1},
  \end{align*}
  which says that
  \begin{equation*}
    a^k_{(n)} = (-n + \lambda_k - 3)\dots(-n)L_{n + 1 - \lambda_k} \quad \text{for }k = 1, \dots, s\text{ and }n \in \mathbb{Z}.
  \end{equation*}
  
  Note that each expression $R^{-1, k}_{n_1, \dots, n_{s - 1}}L_{\eta}\vac$ in \Cref{lmm:3} (where we omitted the fields $a^k(z)$ from the notation) is the sum of elements of the form
  \begin{equation*}
    \text{scalar}\cdot L_{\eta_1}\dots L_{\eta_s}(L_{\eta}\vac),
  \end{equation*}
  where $\eta_k \in \mathbb{Z}$ for $k = 1, \dots, s$.
  Note that if $\eta_k \ge -1$ for some $k = 1, \dots, s$ then $L_{\eta_1}\dots L_{\eta_s}(L_{\eta}\vac) \in G^{2s - 1}V$.
  Now we study these expressions $R^{-1, k}_{n_1, \dots, n_{s - 1}}L_{\eta}\vac$ by considering the elements that appear in the sum defining it.
  We consider several disjoint cases
  \begin{enumerate}
  \item $k > 0$. If $n_{i_1} \ge \lambda_{i_1} - 2$ then $n_{i_1} + 1 - \lambda_{i_1} \ge -1$, so we get
    \begin{equation*}
      a^{j_1}_{(-n_{j_1} - 1)}\dots a^{j_{s - 1 - k}}_{(-n_{j_{s - 1 - k}} - 1)}a^s_{(l - k - \sum_{r = 1}^kn_{i_r} + \sum_{r = 1}^{s - 1 - k}n_{j_r})}a^{i_k}_{(n_{i_k})}\dots a^{i_1}_{(n_{i_1})}L_{\eta}\vac \in G^{2s - 1}\Vir^c.
    \end{equation*}
    If $n_{i_1} \le \lambda_{i_1} - 3$, then $a^{i_1}(z)_{(n_{i_1})}L_{\eta}\vac = 0$ because $(-n + \lambda_k - 3)\dots(-n) = 0$, so we get
    \begin{equation*}
      a^{j_1}_{(-n_{j_1} - 1)}\dots a^{j_{s - 1 - k}}_{(-n_{j_{s - 1 - k}} - 1)}a^s_{(l - k - \sum_{r = 1}^kn_{i_r} + \sum_{r = 1}^{s - 1 - k}n_{j_r})}a^{i_k}_{(n_{i_k})}\dots a^{i_1}_{(n_{i_1})}L_{\eta}\vac = 0.
    \end{equation*}
    Therefore, we obtain
    \begin{equation*}
      R^{-1, k}_{n_1, \dots, n_{s - 1}}L_{\eta}\vac \in G^{2s - 1}\Vir^c.
    \end{equation*}
  \item $k = 0$ and there exists some $j = 1, \dots, s - 1$ such that $n_j > 0$.
    In this case, we get
    \begin{equation*}
      R^{-1, 0}_{n_1, \dots, n_{s - 1}}L_{\eta}\vac = a^1_{(-n_1 - 1)}\dots a^{s - 1}_{(-n_{s - 1} - 1)}a^s_{(-1 + \sum_{r = 1}^{s - 1}n_r)}L_{\eta}\vac \in G^{2s - 1}\Vir^c
    \end{equation*}
    because we can repeat the reasoning of (i) with $-1 + \sum_{r = 1}^{s - 1}n_r$ instead of $n_{i_1}$.
  \item $k = 0$ and $n_1 = \dots = n_{s - 1} = 0$. In this case, we get
    \begin{equation*}
      R^{-1, 0}_{0, \dots, 0} = a^1_{(-1)}\dots a^{s - 1}_{(-1)}a^s_{(-1)} L_{\eta}\vac = (\lambda_1 - 2)!\dots(\lambda_s - 2)!L_{\lambda} L_{\eta}\vac.
    \end{equation*}
  \end{enumerate}
  From these three cases we obtain \eqref{eq:43}.

  We conclude that we have an algebra isomorphism $\gr^G(\Vir^c) \xrightarrow{\sim} \gr(U(\Vir_{\le -2}))$.
  Composing this with the inverse of the isomorphism $S(\mathfrak{g}) \to \gr(U(\Vir_{\le -2}))$ in \Cref{sec:almost-comm-algebr} and taking $(L_n)_{n \le -2}$ as the basis of $\Vir_{\le -2}$, we obtain the following isomorphism of commutative, associative algebras with unit:
  \begin{align*}
  \gr^G(\Vir^c) &\xrightarrow{\sim} \mathbb{C}[L_{-2}, L_{-3}, \dots] \\
  \alpha^{2s}(L_{-n_1 - 2}\dots L_{-n_s - 2}\vac) &\mapsto L_{-n_1 - 2}\dots L_{-n_s - 2} \quad \text{for }s, n_1, \dots, n_s \in \mathbb{N}.
  \end{align*}
  In particular, the isomorphism does not depend on $c$.

  The Poisson structure on $\gr^G(V)$ (i.e.\ the map $Y_-$ is zero) is trivial because for $s \in \mathbb{Z}$, $G^{2s}\Vir^c = G^{2s + 1}\Vir^c$.
\end{example}

\subsection{Filtrations of modules over vertex algebras}
\label{sec:filtr-modul-over}

Let $V$ be a vertex algebra, $(a^i)_{i \in I}$ a family of strong generators of $V$ and $M$ a $V$-module.
For $p \in \mathbb{Z}$, set
\begin{equation*}
  F_pM = \vspan \{a^{i_1M}_{(-n_1 - 1)}\dots a^{i_sM}_{(-n_s - 1)}m \mid s, n_1, \dots, n_s \in \mathbb{N}, i_1, \dots, i_s \in I, m \in M, n_1 + \dots + n_s \ge p\}.
\end{equation*}

\begin{proposition}\cite{li_abelianizing_2005}
  \label{prp:10}
  The filtration $(F_pM)_{p \in \mathbb{Z}}$ satisfies:
  \begin{enumerate}
  \item $M = F_pM$ for $p \le 0$;
  \item $F_0M \supseteq F_1M \supseteq \dots$;
  \item $a_{(n)}F_qM \subseteq F_{p + q - n - 1}M$ for $a \in F_pV$ and $n \in \mathbb{Z}$;
  \item $a_{(n)}F_qM \subseteq F_{p + q - n}M$ for $a \in F_pV$ and $n \in \mathbb{N}$.
  \end{enumerate}
\end{proposition}

Let
\begin{equation*}
  \gr_F(M) = \bigoplus_{p \in \mathbb{N}}F_pM/F_{p + 1}M
\end{equation*}
be the associated graded vector space.
The vector space $\gr_F(M)$ is a module over the vertex Poisson algebra $\gr_F(V)$ by setting
\begin{align*}
  \sigma_p(a)\sigma^M_q(m) &= \sigma^M_{p + q}(a^M_{(-1)}b), \\
  Y^M_-(\sigma_p(a), z)\sigma^M_q(m) &= \sum_{n \in \mathbb{N}}\sigma^M_{p + q - n}(a^M_{(n)}m)z^{-n - 1},
\end{align*}
where $\sigma^M_p: F_pM \to \gr_F(M)$ is the principal symbol map.
The filtration $(F_pM)_{p \in \mathbb{Z}}$ is called the Li filtration of $M$.

Note that if $V$ were conformal with conformal vector $\omega$, then we don't always have the property $L_{-1}^MF_pM \subseteq F_{p + 1}M$ for $p \in \mathbb{Z}$ where $Y(\omega, z) = \sum_{n \in \mathbb{Z}}L^M_nz^{-n - 2}$.
For example, for $V = \Vir^{1/2}$, $M = M(1/2, 1/2)$ and $\vachalf \in F_0M$, we have $L_{-1}\vachalf \notin F_1M$.

\begin{remark}
  \label{rmk:28}
  The observation above makes \cite[Lemma 3.1.2]{arakawa_remark_2012} incorrect because the expression $\sigma_{p - 1}(\omega_{(0)}m)$ that is written there (that should be $\sigma_{p + 1}(\omega_{(0)}m)$ but it still doesn't work) is not well defined.
  I couldn't fix this problem and I don't think it is possible to define a differential on $\gr_F(M)$ in a meaningful way.
  We won't need that differential in this article, though.
\end{remark}

\begin{lemma}[{\cite[Lemma 2.9]{li_abelianizing_2005}}]
  \label{lmm:18}
  Let $V$ be a vertex algebra.
  Then
  \begin{equation*}
    F_pM = \vspan_{\mathbb{C}}\{a^M_{(-i - 1)}b \mid a \in V, i \ge 1, b \in F_{p - i}M\} \quad \text{for }p \in \mathbb{Z}_+.
  \end{equation*}
\end{lemma}

By \Cref{lmm:18}, the Li filtration depends only on $M$ and not on the choice of the strong generators of $V$.
If $V$ is graded by a Hamiltonian $H$ and $M$ is graded by a Hamiltonian $H^M$, then $H^M(F_pM) \subseteq F_pM$ because in that case, for $p \in \mathbb{Z}$,
\begin{equation*}
  \begin{split}
    F_pM =\vspan \{a^1_{(-n_1 - 1)}\dots a^s_{(-n_s - 1)}m &\mid s, n_1, \dots, n_s \in \mathbb{N}, a^i \in V\text{ homogeneous},\\
    &\quad  m\in M\text{ homogeneous}, n_1 + \dots + n_s \ge p\}.
  \end{split}
\end{equation*}

Therefore, we can define an operator $H^M \in \End(\gr_F(M))$ as $H^M(\sigma^M_p(m)) = \sigma^M_p(Hm)$ for $p \in \mathbb{N}$ and $m \in F_pM$.
For $\Delta \in \mathbb{C}$, define $F_pM_{\Delta} = F_pM \cap M_{\Delta}$.
Since $H^M(F_pM) \subseteq F_pM$ for $p \in \mathbb{Z}$, \Cref{lmm:12} implies that
\begin{equation}
  \label{eq:44}
  F_pM = \bigoplus_{\Delta \in \mathbb{C}}F_pM_{\Delta} \quad \text{for }p \in \mathbb{Z}.
\end{equation}
For $\Delta\in \mathbb{C}$, define $\gr_F(M)_{\Delta} = \bigoplus_{p \in \mathbb{N}}\sigma^M_p(F_pM_{\Delta})$.
Then $H^Mm = \Delta m$ for $m \in \gr_F(M)_{\Delta}$.
The family of subspaces $(\gr_F(M)_{\Delta})_{\Delta \in \mathbb{C}}$ satisfy $\gr_F(M) = \bigoplus_{\Delta \in \mathbb{C}} \gr_F(M)_{\Delta}$.
Therefore, the operator $H^M \in \End(\gr_F(M))$ is diagonalizable with grading $\gr_F(M)_{\Delta} = \ker(H^M - \Delta\Id_{\gr_F(M)})$.
In fact, more is true.
\begin{theorem}
  \label{thr:35}
  This diagonalizable operator $H^M$ is a Hamiltonian for $\gr_F(M)$.
\end{theorem}

\begin{proof}
  The proof for \Cref{thr:32} also works here.
\end{proof}

We have the natural vector space isomorphisms
\begin{equation*}
  \sigma^M_p(F_pM_{\Delta}) \cong F_pM_{\Delta}/F_{p + 1}M_{\Delta} \quad \text{for }p \in \mathbb{Z}\text{ and }\Delta \in \mathbb{C}
\end{equation*}
and the double gradation
\begin{equation}
  \label{eq:45}
  \gr_F(M) =\bigoplus_{\substack{p \in \mathbb{N} \\ \Delta \in \mathbb{C}}}\sigma_p(F_pM_{\Delta}).
\end{equation}
By \eqref{eq:45}, it is natural to define refined series of $M$ with respect to the Li filtration as
\begin{equation*}
  \ch_{\gr_F(M)}(t, q) = \sum_{\substack{p \in \mathbb{N} \\ \Delta \in \mathbb{C}}}\dim(\sigma_p(F_pM_{\Delta}))t^pq^{\Delta}.
\end{equation*}

If $f: M_1 \to M_2$ is a homomorphism of vertex algebras, then
\begin{align*}
  \gr_F(f): \gr_F(M_1) &\to \gr_F(M_2) \\
  \sigma^{M_1}_p(m_1) &\mapsto \sigma^{M_2}_p(f(m_1)) \quad \text{for }p \in \mathbb{N}\text{ and } m_1 \in F_pM_1,
\end{align*}
defines a homomorphism of vertex Poisson algebras.
If $M_1$ and $M_2$ are graded, then we require that $f$ respects the gradings of $M_1$ and $M_2$ and this implies that $\gr_F(f)$ also respects the gradings of $\gr_F(M_1)$ and $\gr_F(M_2)$.
Therefore, we obtain a functor
\begin{equation*}
  \gr_F: \{\text{(graded) $V$-modules}\} \to \{\text{(graded) $\gr_F(V)$-modules}\}.
\end{equation*}

Now we introduce a definition not given by Li in \cite{li_vertex_2004}.
Let $V$ be a $\mathbb{N}$-graded conformal vertex algebra with conformal vector $\omega$, $(a^i)_{i \in I}$ a family of homogeneous strong generators of $V$ and let $M$ be a $h + \mathbb{N}$-graded $V$-module which means $M$ is a $V$-module with $L_0^M$ diagonalizable whose eigenvalues are in the set $h + \mathbb{N}$ for some $h \in \mathbb{C}$.
Set $M_\Delta = \ker(L^M_0 - \Delta\Id_M)$ for $\Delta \in \mathbb{C}$ so we have $M = \bigoplus_{n \in \mathbb{N}}M_{h + n}$.
For $p \in \mathbb{Z}$, set

\begin{equation*}
  \begin{split}
    G^pM = \vspan\{a^{i_1M}_{(-n_1 - 1)}\dots a^{i_sM}_{(-n_s - 1)}m &\mid s, n_1, \dots, n_s \in \mathbb{N}, i_1, \dots, i_s \in I, m \in M_{\Delta_m} \text{ and }\\
    &\quad \Delta_{a^{i_1}} + \dots + \Delta_{a^{i_s}} + \Delta_m - h \le p\}.
  \end{split}
\end{equation*}

\begin{proposition}
  \label{prp:11}
  The filtration $(G^pM)_{p\in \mathbb{Z}}$ satisfies:
  \begin{enumerate}
  \item $G^pM = 0$ for $p < 0$;
  \item $G^0M \subseteq G^1M \subseteq \dots$;
  \item $M = \bigcup_pG^pM$;
  \item $a^M_{(n)}G^qM \subseteq G^{p + q}M$ for $a \in G^pV$ and $n \in \mathbb{Z}$;
  \item $a^M_{(n)}G^qM \subseteq G^{p + q - 1}M$ for $a \in G^pV$ and $n \in \mathbb{N}$;
  \item $L^M_0G^pM \subseteq G^pM$ and $L^M_{-1}G^pM \subseteq G^{p + 1}M$.
  \end{enumerate}
\end{proposition}

\begin{proof}
  The proofs in \cite{li_vertex_2004} also work here but there is an additional $\pm h$ hanging around.
\end{proof}

Let
\begin{equation*}
  \gr^G(M) = \bigoplus_{p \in \mathbb{N}}G^pM/G^{p - 1}M
\end{equation*}
be the associated graded vector space.
The vector space $\gr^G(M)$ a module over $\gr^G(V)$ with operations given as follows: for $p, q \in \mathbb{N}$, $a \in G^pV$ and $m \in G^qM$, set
\begin{align*}
  \alpha^p(a)\alpha_M^q(m) &= \alpha_M^{p + q}(a^M_{(-1)}m), \\
  Y^M_-(\alpha^p(a), z)\alpha_M^q(m) &= \sum_{n \in \mathbb{N}}\alpha_M^{p + q - 1}(a_{(n)}m)z^{-n - 1},
\end{align*}
where $\alpha_M^p: G^pM \to \gr^G(M)$ is the principal symbol map.
The filtration $(G^pM)_{p \in \mathbb{Z}}$ is called the standard filtration of $M$.
By ... ahead, the standard filtration does not depend on the choice of the strong generators of $V$.

By \Cref{prp:8}(vii), we can define an operator $H^M \in \End(\gr^G(M))$ as $H^M(\alpha_M^p(m)) = \alpha_M^p(L^M_0m)$ for $p \in \mathbb{Z}$ and $m \in G^pM$.
For $n \in \mathbb{N}$, define $G^pM_{h + n} = G^pV \cap M_{h + n}$.

Since $L_0^M(G^pM) \subseteq G^pM$ for $p \in \mathbb{Z}$, \Cref{lmm:12} implies that
\begin{equation}
  \label{eq:46}
  G^pM = \bigoplus_{n \in \mathbb{N}}G^pM_{h + n} \quad \text{for }p \in \mathbb{Z}.
\end{equation}
For $n \in \mathbb{N}$, define $\gr^G(M)_{h + n} = \bigoplus_{p \in \mathbb{N}}\alpha_M^p(G^pM_{h + n})$.
Then $H^Mm = (h + n)m$ for $m \in \gr^G(M)_{h + n}$.
The family of subspaces $(\gr^G(M)_{h + n})_{n \in \mathbb{N}}$ satisfy $\gr^G(M) = \bigoplus_{n \in \mathbb{N}} \gr^G(M)_{h + n}$.
Therefore, the operator $H^M \in \End(\gr^G(M))$ is diagonalizable with $\gr^G(M)_{h + n} = \ker(H - (h + n)\Id_{\gr^G(M)})$.
In fact, more is true.

\begin{theorem}
  \label{thr:36}
  This diagonalizable operator $H^M$ is a Hamiltonian for $\gr^G(M)$.
\end{theorem}

\begin{proof}
  The proof for \Cref{thr:32} also works here.
\end{proof}

We have the natural vector space isomorphisms
\begin{equation*}
  \alpha_M^p(G^pM_{h + n}) \cong G^pM_{h + n}/G^{p - 1}M_{h + n} \quad \text{for }p \in \mathbb{Z}\text{ and }n \in \mathbb{N}
\end{equation*}
and the double gradation
\begin{equation}
  \label{eq:47}
  \gr^G(M) =\bigoplus_{p, n \in \mathbb{N}}\alpha_M^p(G^pM_{h + n}).
\end{equation}
By \eqref{eq:47}, it is natural to define refined series of $M$ with respect to the standard filtration as
\begin{equation*}
  \ch_{\gr^G(M)}(t, q) = \sum_{p, n \in \mathbb{N}}\dim(\alpha_m^p(G^pM_{h + n}))t^pq^{h + n}.
\end{equation*}

If $f: M_1 \to M_2$ is a homomorphism of $V$-modules, then
\begin{align*}
  \gr^G(f): \gr^G(M_1) &\to \gr^G(M_2) \\
  \alpha^p_{M_1}(m_1) &\mapsto \alpha^p_{M_2}(f(m_1)) \quad \text{for }p \in \mathbb{N}\text{ and } m_1 \in G^pM_1,
\end{align*}
defines a homomorphism of $\gr^G(V)$-modules.
Therefore, we obtain a functor
\begin{equation*}
  \gr^G: \{\text{$h + \mathbb{N}$-graded $V$-modules}\} \to \{\text{$h + \mathbb{N}$-graded $\gr^G(V)$-modules}\}.
\end{equation*}

\begin{proposition}
  \label{prp:12}
  Let $V$ be a $\mathbb{N}$-graded conformal vertex algebra and let $M$ be a $h + \mathbb{N}$-graded $V$-module with $L_0^M$ diagonalizable.
  Then the Li filtration and standard filtration satisfy
  \begin{equation*}
    F_pM_{h + n} = G^{n - p}M_{h + n} \quad \text{for }p, n \in \mathbb{N}.
  \end{equation*}
  An explicit isomorphism $\gr_F(M) \xrightarrow{\sim} \gr^G(M)$ of modules is defined by extending linearly the isomorphisms of vector spaces given by
  \begin{align*}
    \sigma^M_p(F_pM_{h + n}) &\xrightarrow{\sim} \alpha^{n - p}_M(G^{n - p}M_{h + n}) \\
    \sigma^M_p(m) &\mapsto \alpha^{n - p}_M(m),
  \end{align*}
  where $p, n\in \mathbb{N}$ and $m \in F_pM_{h + n}$.
\end{proposition}

\begin{proof}
  We recall three facts:
  \begin{enumerate}
  \item $G^nM \supseteq M_{h + n}$ for $n \in \mathbb{N}$,
  \item $\Delta_{a^M_{(n)}m} = \Delta_a + \Delta_m - n - 1$ for homogeneous $a \in V$, $m \in M$ and $n \in \mathbb{Z}$,
  \item $F_pM_{h + n} = \vspan\{a^{i_1M}_{(-n_1 - 1)}m \mid i_1 \in I, n_1 \in \mathbb{Z}_+\text{ and }m \in F_{p - n_1}M_{\Delta_m}\text{ with }\Delta_{a^{i_1}} + \Delta_m + n_1 = h + n\}$ for any $p \in \mathbb{Z}_+$ and $n \in \mathbb{N}$.
  \end{enumerate}
  First, we prove the inclusion $F_pM_{h + n} \subseteq G^{n - p}M_{h + n}$ for $p, n \in \mathbb{N}$.
  We do this by induction on $p \in \mathbb{N}$.
  The base case $p = 0$ is true by property (a) above.
  Now assume $p \ge 1$ and $F_qM_{h + n} \subseteq G^{n - q}M_{h + n}$ for $q < p$ and all $n \in \mathbb{N}$.
  Pick an element $a^{i_1M}_{(-n_1 - 1)}m$ from the spanning set of $F_pM_{h + n}$ in (c) above, with $i_1 \in I$, $n_1 \in \mathbb{Z}_+$ and $m \in F_{p - n_1}M_{\Delta_m}$.
  We know that $m \in G^{\Delta_m - h - p + n_1}M_{\Delta_m}$ by the induction hypothesis and also $a \in V_{\Delta_a^{i_1}} \subseteq G^{\Delta_{a^{i_1}}}V$.
  Therefore by property (d) of the filtration $(G^pM)_{p \in \mathbb{Z}}$, $a^{i_1M}_{(-n_1 - 1)}m \in G^{\Delta_{a^{i_1}} + \Delta_m - h - p + n_1}M = G^{n - p}M$.

  Now we prove the inclusion $G^{n - p}M_{h + n} \subseteq F_pM_{h + n}$.
  Pick an element $a^{i_1M}_{(-n_1 - 1)}\dots a^{i_sM}_{(-n_s - 1)}m$ from the spanning set of $G^{n - p}M_{h + n}$ where $m \in M_{\Delta_m}, s, n_1, \dots, n_s\in \mathbb{N}, i_1, \dots, i_s \in I\text{ and }\Delta_{a^{i_1}} + \dots + \Delta_{a^{i_s}} + \Delta_m - h \le n - p$.
  We must have $\Delta_{a^{i_1}} + \dots + \Delta_{a^{i_s}} + \Delta_m + n_1 + \dots + n_s = h + n$ by property (b) above.
  Therefore $p \le n_1 + \dots + n_s$, so we get $a^{i_1M}_{(-n_1 - 1)}\dots a^{i_sM}_{(-n_s - 1)}m \in F_pM$ straight from the definition of $F_pM$.

  We verify that we obtain an isomorphism $\gr_F(M) \xrightarrow{\sim} \gr^G(M)$ directly from the definitions (cf.\ \Cref{prp:9}).
\end{proof}

\Cref{prp:12} translates into an identity of refined characters of $\gr_F(M)$ and $\gr^G(M)$.

\begin{proposition}
  \label{prp:13}
  Let $V$ be a $\mathbb{N}$-graded conformal vertex algebra and let $M$ be a $h + \mathbb{N}$-graded $V$-module with $L_0^M$ diagonalizable.
  Then
  \begin{enumerate}
  \item $\ch_M(q) = \ch_{\gr_F(M)}(q) = \ch_{\gr^G(M)}(q) = \ch_{\gr_F(M)}(1, q) = \ch_{\gr^G(M)}(1, q)$;
  \item $\ch_{\gr^G(M)}(t^{-1}, tq) = t^h\ch_{\gr_F(M)}(t, q)$.
  \end{enumerate}
\end{proposition}

\begin{proof}\leavevmode
  \begin{enumerate}
  \item This is clear from the properties and definition of the filtrations.
  \item This follows from \Cref{prp:12} and replacing $p$ by $n - p$ in the following computation
    \begin{align*}
    \ch_{\gr^G(M)}(t^{-1}, tq) &= \sum_{p, n \in \mathbb{N}}\dim(\alpha^p_M(G^pM_{h + n}))t^{-p}(tq)^{h+n} \\
    &= \sum_{p, n \in \mathbb{N}}\dim(\alpha^p_M(G^pM_{h + n}))t^{h + n - p}q^{h + n} \\
    &= \sum_{p, n \in \mathbb{N}}\dim(\alpha^{n - p}_M(G^{n - p}M_{h + n}))t^{h + p}q^{h + n} \\
    &= t^h\ch_{\gr_F(M)}(t, q).\qedhere
    \end{align*}
  \end{enumerate}
\end{proof}

\begin{example}[$\gr^G(M(c, h))$]
  \label{exa:12}
  Pick $c, h \in \mathbb{C}$.
  By \Cref{thr:21}, the Verma module $M(c, h)$ is a module over $\Vir^c$ as a vertex algebra.
  From \Cref{exa:11}, $\gr^G(\Vir^c)$ is isomorphic to $\mathbb{C}[L_{-2}, L_{-3}, \dots]$.
  In a similar way, we can prove that $\gr^G(M(c, h))$ is a free $\gr^G(\Vir^c)$-module:
  \begin{equation*}
    \gr^G(M(c, h)) = \bigoplus_{k \in \mathbb{N}}\gr^G(\Vir^c)L_{-1}^k.
  \end{equation*}
  More precisely, the isomorphism is given by
  \begin{align*}
    \gr^G(M(c, h)) &\xrightarrow{\sim} \bigoplus_{k \in \mathbb{N}}\mathbb{C}[L_{-2}, L_{-3}, \dots]L_{-1}^k \\
    \alpha_M^{2s + k}(L_{-n_1 - 2}^M\dots L_{-n_s - 2}^M(L_{-1}^M)^k|c, h\rangle) &\mapsto L_{-n_1 - 2}\dots L_{-n_s - 2}L_{-1}^k,
  \end{align*}
  where $s, k, n_1, \dots, n_s \in \mathbb{N}$.
\end{example}

\begin{remark}
  \label{rmk:29}
  That $+k$ in the isomorphism above is what makes this filtration different from the PBW filtration where all $L_n$ for $n \le -1$ have the same length.
  On the other hand, with the standard filtration $L_{-1}$ has length equal to $1$ while $L_{-2}$, $L_{-3}, \dots$ have length equal to $2$.
\end{remark}

\section{Modules over the simple Virasoro vertex algebras}
\label{sec:zhu-algebra}

\subsection{The Zhu algebra}
\label{sec:zhu-algebra-1}

Let $V$ be a $\mathbb{Z}$-graded vertex algebra with Hamiltonian $H$ and let $M = \bigoplus_{n \in \mathbb{N}} M(n)$ be an admissible $V$-module (note we are not assuming $M$ graded).
We see that
\begin{equation}
  \label{eq:48}
  \begin{split}
    a^M_{0}M(m) &\subseteq M(m) \quad \text{for } m \in \mathbb{N}, \\
    a^M_{n}M(m) &= 0 \quad \text{for }m \in \mathbb{N}\text{ and }n \in \mathbb{Z}_+.
  \end{split}
\end{equation}

We are thus lead to consider the linear map
\begin{align*}
  \pi_M: V &\to \End(M(0)) \\
  \pi_M(a) &= a^M_{0}|_{M(0)}
\end{align*}

Set $m = 1$ and $n = k = -1$ in the graded Borcherds identity \eqref{eq:28} where $a, b \in V$ and $c \in M_0$.
The RHS becomes
\begin{equation*}
  \sum_{j \in \mathbb{N}}\binom{\Delta_a}{j}(a_{(j - 1)}b)^M_0c.
\end{equation*}
By \eqref{eq:48}, the LHS becomes $a^M_0(b^M_0c)$.
Therefore, we are led to define the following (nonassociative) operation on $V$:
\begin{equation*}
  a*b = \sum_{j \in \mathbb{N}}\binom{\Delta_a}{j}a_{(j - 1)}b.
\end{equation*}
We have just seen that $\pi_M$ is a representation of $(V, *)$, i.e.\
\begin{equation}
  \label{eq:49}
  \pi_M(a)\pi_M(b) = \pi_M(a*b) \quad \text{for }a, b \in V.
\end{equation}

\begin{theorem}[{\cite[\S 2]{de_sole_finite_2006}}]
  \label{thr:37}
  With the notation above, we have:
  \begin{enumerate}
  \item $\vac$ is a unit for $*$;
  \item $J(V) = ((T + H)V)*V$ is a two sided ideal of the algebra $(V, *)$;
  \item $(a*b)*c - a*(b*c) \in J(V)$ for $a, b, c \in V$;
  \item $(T + H)V \subseteq J(V) \subseteq \ker(\pi_M)$;
  \item $a*b - b*a = \sum_{j \in \mathbb{N}}\binom{\Delta_a - 1}{j}a_{(j)}b$ for $a, b \in V$.
  \end{enumerate}
\end{theorem}

By \Cref{thr:37}(i)-(iii), we define the associative unital algebra
\begin{equation*}
  \Zhu(V) = V/J(V)
\end{equation*}
called the Zhu algebra of $V$.
The unit of $\Zhu(V)$ is the $\vac + J(V)$.

By \eqref{eq:49}, for a $\mathbb{Z}$-graded vertex algebra $V$, we have defined a restriction functor:
\begin{align*}
  \Omega: \{\text{admissible, positive energy $V$-modules}\} &\to \{\text{$\Zhu(V)$-modules}\} \\
  M & \mapsto \Omega(M) = M(0)
\end{align*}

\subsection{The inverse of the restriction functor}
\label{sec:inverse-restr-funct}

By \Cref{thr:37}(iv)-(v), we have a natural Lie algebra epimorphism $\Lie_0(V) \twoheadrightarrow [\Zhu(V)]$.
Moreover, we know that $\Lie_0(V)$ is isomorphic to $\Lie(V)$.
Therefore, we have a well-defined epimorphism
\begin{align*}
  \Lie(V)_0 &\twoheadrightarrow [\Zhu(V)] \\
  a_{\{\Delta_a - 1\}} &\mapsto a + J(V)
\end{align*}

Let $U$ be a $\Zhu(V)$-module.
Note that $U$ is a fortiori a $[\Zhu(V)]$-module and by the epimorphism above it is a $\Lie(V)_0$-module.
Let $\Lie(V)_-$ act trivially on $U$ and set
\begin{equation*}
  M(U) = \Ind^{\Lie(V)}_{\Lie(V)_- \oplus \Lie(V)_0}(U) = U(\Lie(V)) \otimes_{U(\Lie(V)_- \oplus \Lie(V)_0)}U.
\end{equation*}
Give $U$ degree $0$ and extend the gradation of $\Lie(V)$ to $M(U)$.
Then $M(U)$ becomes $\mathbb{N}$-graded with $M(U)(0) = U$.
We are tempted to define $Y^{M(U)}(a, z) = \sum_{n \in \mathbb{N}}a_{\{n\}}z^{-n - 1}$ for $a \in V$ but that doesn't necessarily satisfy the Borcherds identity.

\begin{theorem}[{\cite[\S 2]{de_sole_finite_2006}}]
  \label{thr:38}
  The $\mathbb{N}$-graded $\Lie(V)$-module $M(U)$ has a unique maximal graded $\Lie(V)$-submodule $J$ with the property that $J \cap U = 0$.
  Then $L(U) = M(U)/J$ is an admissible $V$-module satisfying $\Omega(L(U)) \cong U$.
  Therefore, we have defined a functor
  \begin{equation*}
  L: \{\text{$\Zhu(V)$-modules}\} \to \{\text{admissible $V$-modules}\}
  \end{equation*}
  such that $\Omega \circ L$ is naturally isomorphic to the identity functor.
\end{theorem}

The functors $\Omega$ and $L$ are not yet equivalences but they are close to be.
We call an admissible $V$-module $M = \bigoplus_{n \in \mathbb{N}}M(n)$ almost irreducible if $M(0)$ generates $M$ and $M$ contains no graded submodules intersecting $M(0)$ trivially.

\begin{theorem}[{\cite[\S 2]{de_sole_finite_2006}}]
  \label{thr:39}
  Let $V$ be a $\mathbb{Z}$-graded vertex algebra.
  The functors $\Omega$ and $L$ establish an equivalence between the category of almost irreducible admissible $V$-modules and the category of $\Zhu(V)$-modules.
  In particular, these functors establish a bijective correspondence between irreducible admissible $V$-modules and irreducible $\Zhu(V)$-modules.
\end{theorem}

\subsection{Modules over the simple Virasoro vertex algebras}
\label{sec:modules-over-simple}

\appendix
\section{Almost commutative algebras}
\label{sec:almost-comm-algebr}

Let $A$ be an associative (not necessarily commutative) algebra with a unit and a filtration $(A^p)_{p \in \mathbb{Z}}$ such that:
\begin{enumerate}
\item $A^p = 0$ for $p < 0$;
\item $A^0 \subseteq A^1 \subseteq \dots$;
\item $A^iA^j \subseteq A^{i + j}$ for $i, j \in \mathbb{Z}$.
\end{enumerate}
Let
\begin{equation*}
  \gr(A) = \bigoplus_{p \in \mathbb{N}}A^p/A^{p - 1}
\end{equation*}
be the associated graded vector space.
The vector space $\gr(A)$ is an associative algebra with multiplication given as follows: for $p, q \in \mathbb{N}$, $a \in A^p$ and $b \in A^p$, set
\begin{equation*}
  \gamma^p(a)\gamma^q(b) = \gamma^{p + q}(ab),
\end{equation*}
where $\gamma^p: A^p \to \gr(A)$  is the principal symbol map which is the composition of the natural maps $A^p \to A^p/A^{p - 1}$ and $A^p/A^{p - 1} \to \gr(A)$.

We say $A$ is almost commutative if the filtration $(A^p)_{p \in \mathbb{Z}}$ satisfies the following condition: for $i, j \in \mathbb{Z}$, if $a \in A^i$ and $b \in A^j$, then $ab - ba \in A^{i + j -1}$.
If $A$ is almost commutative, then $\gr(A)$ is commutative.

Let $\mathfrak{g}$ be a Lie algebra and let $U(\mathfrak{g})$ be its universal enveloping algebra.
The PBW filtration $(U(\mathfrak{g})^p)_{p \in \mathbb{Z}}$ is given by
\begin{equation*}
  U(\mathfrak{g})^p = \vspan\{x_1x_2\dots x_s \mid s\le p, x_1, \dots, x_s \in \mathfrak{g}\} \quad \text{for }p \in \mathbb{Z}.
\end{equation*}
This filtration clearly satisfy axioms (i)-(iii).
Furthermore, $U(\mathfrak{g})^1 = \mathfrak{g}$.
By \cite[2.1.5. Lemma.]{dixmier_enveloping_1996}, $U(\mathfrak{g})$ is almost commutative, so $\gr(U(\mathfrak{g}))$ is commutative.
Let $S(\mathfrak{g})$ denote the symmetric algebra of $\mathfrak{g}$.
We have two natural inclusions of $\mathfrak{g}$: $\mathfrak{g} \to U(\mathfrak{g})$ and $\mathfrak{g} \to S(\mathfrak{g})$.
By the universal property of $S(\mathfrak{g})$, there is a homomorphism of commutative algebra $S(\mathfrak{g}) \to \gr(U(\mathfrak{g}))$ such that $1 \mapsto 1$ and the following diagram commutes
\begin{equation*}
  \begin{tikzcd}
    \mathfrak{g} \arrow[r] \arrow[rd] & S(\mathfrak{g}) \arrow[d] \\
    & \gr(U(\mathfrak{g}))      
  \end{tikzcd}
\end{equation*}
By \cite[2.3.6. Proposition]{dixmier_enveloping_1996}, the homomorphism $S(\mathfrak{g}) \to \gr(U(\mathfrak{g}))$ is in fact an isomorphism.
If $(x_i)_{i \in I}$ is a basis of $\mathfrak{g}$, then $S(\mathfrak{g})$ is isomorphic to the polynomial algebra $\mathbb{C}[(x_i)_{i \in I}]$.
We have described $\gr(U(\mathfrak{g}))$ explicitly.

% \printindex

\bibliographystyle{alpha}
\bibliography{ising-modules.bib}

\end{document}
