\documentclass[a4paper, 12pt, reqno]{amsart}

\usepackage{amssymb}
\usepackage{enumerate}
\usepackage{hyperref}
\usepackage[margin = 0.8in]{geometry}
\usepackage[inline]{enumitem}
\usepackage{tikz-cd}
\usepackage[nameinlink]{cleveref}
\usepackage{stmaryrd}

\newtheorem{theorem}{Theorem}[subsection]
\newtheorem{lemma}[theorem]{Lemma}
\newtheorem{proposition}[theorem]{Proposition}
\newtheorem{corollary}[theorem]{Corollary}

\theoremstyle{remark}
\newtheorem{remark}[theorem]{Remark}
\newtheorem{example}[theorem]{Example}

\numberwithin{equation}{subsection}

\setenumerate[0]{label = \normalfont(\roman*)}

\DeclareMathOperator{\Vir}{Vir}
\DeclareMathOperator{\Id}{Id}
\DeclareMathOperator{\gr}{gr}
\DeclareMathOperator{\End}{End}
\DeclareMathOperator{\ch}{ch}
\DeclareMathOperator{\lm}{lm}
\DeclareMathOperator{\vspan}{span}
\DeclareMathOperator{\Ind}{Ind}
\DeclareMathOperator{\len}{len}
\DeclareMathOperator{\psn}{psn}
\DeclareMathOperator{\Frac}{Frac}
\DeclareMathOperator{\Res}{Res}
\DeclareMathOperator{\vac}{|0\rangle}
\DeclareMathOperator{\vachalf}{|1/2\rangle}
\DeclareMathOperator{\vacsixteen}{|1/16\rangle}
\DeclareMathOperator{\zero}{\overline{0}}
\DeclareMathOperator{\one}{\overline{1}}
\DeclareMathOperator{\Der}{Der}
\DeclareMathOperator{\Lie}{Lie}
\DeclareMathOperator{\ad}{ad}
\DeclareMathOperator{\Cur}{Cur}
\DeclareMathOperator{\Hom}{Hom}
\DeclareMathOperator{\fs}{fs}

\makeindex

\begin{document}

\begin{abstract}
  To every $h + \mathbb{N}$-graded module $M$ over a $\mathbb{N}$-graded conformal vertex algebra $V$ we associate an increasing filtration $(G^pM)_{p \in \mathbb{Z}}$ which is compatible with the filtrations introduced by Haisheng Li.
  The associated graded vector space $\gr^G(M)$ is naturally a module over the Poisson vertex algebra $\gr^G(V)$.
  We study $\gr^G(M)$ for the three irreducible modules of the Ising model $\Vir_{3, 4}$, namely $\Vir_{3,4} = L(1/2, 0)$, $L(1/2, 1/2)$ and $L(1/2, 1/16)$.
  We obtain an explicit monomial basis for each of these modules and a formula for their refined characters which are related to Nahm sums for the matrix $\left(\begin{smallmatrix} 8 & 3 \\ 3 & 2 \end{smallmatrix}\right)$.
\end{abstract}

\title{PBW bases of irreducible Ising modules}
\author{Diego Salazar Gutierrez}
\address{Instituto de Matemática Pura e Aplicada, Rio de Janeiro, RJ, Brazil}
\email{diego.salazar@impa.br}
\date{\today}
\maketitle

\tableofcontents

\section{Introduction}
\label{sec:introduction}

\section{Vertex algebras and their modules}
\label{sec:vert-algebr-their}

\subsection{Formal calculus}
\label{sec:formal-calculus}

All vector spaces and all algebras are over $\mathbb{C}$, the field of complex numbers.
The set of natural numbers $\{0, 1, \dots\}$ is denoted by $\mathbb{N}$, the set of integers is denoted by $\mathbb{Z}$ and the set of positive integers $\{1, 2, \dots\}$ is denoted by $\mathbb{Z}_+$.

The vector space of \index{formal distribution|emph}\emph{formal distributions} in $n \in \mathbb{N}$ variables, denoted by $\mathbb{C}[[X_1^{\pm 1}, \dots, X_n^{\pm 1}]]$, is the set of functions $f: \mathbb{Z}^n \to \mathbb{C}$, written as $f = \sum_{\upsilon \in \mathbb{Z}^n}f_{\upsilon}X_{\upsilon}$ where $X_{\upsilon} = X^{\upsilon_1}_1\dots X^{\upsilon_n}_n$, with the natural operations of addition and multiplication by a scalar.
The field of \index{rational function|emph}\emph{rational functions} in $n$ variables, denoted by $\mathbb{C}(X_1, \dots, X_n)$, is the field of fractions $\Frac(\mathbb{C}[X_1, \dots, X_n])$.
The field of \index{formal Laurent series|emph}\emph{formal Laurent series} is the subalgebra of elements $f \in \mathbb{C}[[Z^{\pm 1}]]$ such that there is $N \in \mathbb{Z}$ with $f_n = 0$ for $n \le N$.
We also have $\mathbb{C}((X)) = \Frac(\mathbb{C}[[X]])$.
The field of \index{joint Laurent series|emph}\emph{joint Laurent series} in $n$ variables, denoted by $\mathbb{C}((X_1, \dots, X_n))$, is $\Frac(\mathbb{C}[[X_1, \dots, X_n]])$.

If $V$ is a vector space, we similarly define $V[[X_1^{\pm 1}, \dots, X_n^{\pm 1}]]$ and $V((X))$ but in this case, $V((X))$ is only a vector space.

Let $V$ be a vector space.
The \index{formal distribution!Fourier expansion of|emph}\emph{Fourier expansion} of a formal distribution $a(z) \in V[[z^{\pm 1}]]$, written as $a = \sum_{n \in \mathbb{Z}} a_nz^n$, is conventionally written in the theory of vertex algebras as
\begin{equation*}
  a(z) = \sum_{n \in \mathbb{Z}}a_{(n)}z^{-n - 1},
\end{equation*}
where
\begin{equation*}
  a_{(n)} = a_{-n-1}.
\end{equation*}
The \index{formal distribution!residue of|emph}\emph{residue} of a formal distribution $a(z) \in V[[z^{\pm 1}]]$ is defined as
\begin{equation*}
  \Res_z(a(z)) = a_{(0)} = a_{-1}.
\end{equation*}

If $P\in R[[z_1^{\pm 1},\dots, z_n^{\pm 1}]]$ and $Q\in R[[w_1^{\pm 1},\dots, w_m^{\pm 1}]]$, then $PQ\in R[[z_1^{\pm 1},\dots z_n^{\pm 1},w_1^{\pm 1},\dots w_m^{\pm 1}]]$ is defined in the natural way.
However, if both $P$ and $Q$ belong to $R[[z_1^{\pm 1},\dots, z_n^{\pm 1}]]$, we may run into trouble because infinite sums may appear.

An important formal distribution in two variables $z, w$ is the \index{formal distribution!delta|emph}\emph{formal delta distribution} which is defined by
\begin{equation*}
  \delta(z, w) = \sum_{n \in \mathbb{Z}}z^nw^{-n - 1} \in \mathbb{C}[[z^{\pm 1}, w^{\pm 1}]].
\end{equation*}

The \emph{expansion in the domain} $|z| > |w|$ is the field homomorphism $i_{z, w}: \mathbb{C}((z, w)) \to \mathbb{C}((z))((w))$ such that the following diagram commutes
\begin{equation*}
  \begin{tikzcd}
    {\mathbb{C}[[z, w]]} \arrow[rd] \arrow[r] & {\mathbb{C}((z, w))} \arrow[d, "{i_{z,w}}"] \\
    & \mathbb{C}((z))((w))                                          
  \end{tikzcd}
\end{equation*}
where the unlabeled homomorphisms are the natural ones.
Similarly, \index{expansion in the domain|emph}expansion in the domain $|w| > |z|$ is the field homomorphism $i_{w, z}: \mathbb{C}((z, w)) \to \mathbb{C}((w))((z))$ such that the following diagram commutes
\begin{equation*}
  \begin{tikzcd}
    {\mathbb{C}[[z, w]]} \arrow[rd] \arrow[r] & {\mathbb{C}((z, w))}\arrow[d, "{i_{w, z}}"] \\
    & \mathbb{C}((w))((z))                                          
  \end{tikzcd}
\end{equation*}

We have natural inclusions $\mathbb{C}((z))((w)) \to \mathbb{C}[[z^{\pm 1}, w^{\pm 1}]]$ and $\mathbb{C}((w))((z)) \to \mathbb{C}[[z^{\pm 1}, w^{\pm 1}]]$.
The diagram
\begin{equation*}
  \begin{tikzcd}
    & {\mathbb{C}((z, w))} \arrow[ld, "{i_{z,w}}"'] \arrow[rd, "{i_{w,z}}"] &                                 \\
    \mathbb{C}((z))((w)) \arrow[rd] &                                                                      & \mathbb{C}((w))((z)) \arrow[ld] \\
    & {\mathbb{C}[[z^{\pm 1}, w^{\pm 1}]]}                                  &                                
  \end{tikzcd}
\end{equation*}
doesn't commute. In fact, the formal delta distribution can be expressed as
\begin{equation*} 
  \delta(z, w) = i_{z, w}\left(\frac{1}{z - w}\right) - i_{w, z}\left(\frac{1}{z - w}\right),
\end{equation*}
where we consider $i_{z, w}(\frac{1}{z - w})$ and $i_{w, z}(\frac{1}{z - w})$ as elements of $\mathbb{C}[[z^{\pm 1}, w^{\pm 1}]]$.
From now on, we will consider $i_{z, w}$ and $i_{w, z}$ as mapped into $\mathbb{C}[[z^{\pm 1}, w^{\pm 1}]]$.

Let $V$ be a vector space.
A formal distribution $a(z, w) \in V[[z^{\pm 1}, w^{\pm 1}]]$ is \index{formal distribution!local|emph}\emph{local} if there is $N \in \mathbb{N}$ such that
\begin{equation*}
  (z - w)^Na(z, w)=0.
\end{equation*}
For example, the \index{formal distribution!delta}formal delta distribution $\delta(z, w)$ is local.

\begin{theorem}[{\cite[Proposition 2.2]{kac_vertex_1998}}]
  \label{thr:1}
  Let $a(z, w) \in V[[z^{\pm 1}, w^{\pm 1}]]$ be a local formal distribution.
  Then $a(z, w)$ can be written uniquely as a sum
  \begin{equation}
    \label{eq:1}
    a(z, w) = \sum_{j \in \mathbb{N}}c^j(w)\frac{\partial_w^j\delta(z, w)}{j!}
  \end{equation}
  where $c^j(w) \in V[[w^{\pm 1}]]$ are formal distributions given by
  \begin{equation}
    \label{eq:2}
    c^j(w) = \Res_z(z - w)^ja(z, w).
  \end{equation}
  In addition, the converse is true.
\end{theorem}

Let $V$ be a vector superspace and  $a(z) \in V[[z^{\pm 1}]]$ be a formal distribution.
Define
\begin{equation*}
  i_{z, w}a(z + w) = \sum_{n \in \mathbb{Z}} a_ni_{z, w}(z + w)^n.
\end{equation*}
We have a formal version of the usual Taylor series expansion.

\begin{theorem}
  \label{thr:2}
  We have
  \begin{equation*}
    i_{z, w}a(z + w) = \sum_{j \in \mathbb{N}}\frac{\partial^ja(z)}{j!}w^j.
  \end{equation*}
\end{theorem}

We now define the notion of Fourier transform in two cases: in one and two variables. Let $V$ be a vector space and let $a(z) \in V[[z^{\pm 1}]]$.
We define the \index{Fourier transform!in one variable|emph}\emph{Fourier transform in one variable} of $a(z)$ by
\begin{equation*}
  F^\lambda_za(z) = \Res_z(e^{\lambda z}a(z)) \in V[[\lambda]].
\end{equation*}

\begin{proposition}[{\cite[Proposition 1.5.2]{nozaradan_introduction_2008}}]
  \label{prp:1}
  $F^\lambda_z$ satisfies the following properties:
  \begin{enumerate}
  \item $F^\lambda_z\partial_za(z) = -\lambda F^\lambda_za(z)$;
  \item $F^\lambda_z(e^{zT}a(z)) = F^{z + T}_z(a(z))$ where $T \in \End(V)$ and $a(z) \in V((z))$;
  \item $F^\lambda_z(a(-z)) = -F^{-\lambda}_za(z)$;
  \item $F^\lambda_z\partial^n_w\delta(z, w) = e^{\lambda w}\lambda^n$.
  \end{enumerate}
\end{proposition}

Now let $a(z, w) \in V[[z^{\pm 1}, w^{\pm 1}]]$.
We define the \index{Fourier transform!in two variables|emph}\emph{Fourier transform in two variables} of $a(z, w)$ by
\begin{equation*}
  F^\lambda_{z, w}a(z, w) = \Res_ze^{\lambda(z - w)}a(z, w) \in V[[w^{\pm 1}]][[\lambda]].
\end{equation*}

Expanding the definition of $F^\lambda_{z, w}$, we obtain another expression
\begin{equation*}
  F^\lambda_{z, w}a(z, w) = \sum_{j \in \mathbb{N}}\frac{\lambda^j}{j!}c^j(w),
\end{equation*}
where
\begin{equation*}
  c^j(w) = \Res_z(z - w)^ja(z, w).
\end{equation*}
\begin{proposition}[{\cite[Proposition 1.5.4]{nozaradan_introduction_2008}}]
  \label{prp:2}
  $F^\lambda_{z, w}$ satifies the following properties
  \begin{enumerate}
  \item If $a(z, w)$ is local then $F^\lambda_{z, w}a(z, w) \in V[[w^{\pm 1}]][\lambda]$;
  \item $F^\lambda_{z, w}\partial_za(z, w) = -\lambda F^\lambda_{z, w}a(z, w) = [\partial_w, F^{\lambda}_{z, w}]a(z, w)$;
  \item If $a(z, w)$ is local, then $F^\lambda_{z, w}a(w, z) = F^{-\lambda - \partial_w}_{z, w}a(z, w)$, where $F^{-\lambda - \partial_w}_{z, w}a(z, w) = F^\mu_{z, w}a(z, w)|_{u = -\lambda - \partial_w}$.
  \end{enumerate}
\end{proposition}

\subsection{Lie conformal superalgebras}
\label{sec:lie-conf-super}

A \index{vector superspace|emph}\emph{vector superspace} is a $\mathbb{Z}_2$-graded vector space $V = V_{\zero} \oplus V_{\one}$ where $\mathbb{Z}_2 = \mathbb{Z}/2\mathbb{Z} = \{\zero, \one\}$, $\zero = 0 + 2\mathbb{Z}$ and $\one = 1 + 2\mathbb{Z}$.
% Elements of $V_{\zero}$ are called \index{vector superspace!even element of |emph}\emph{even}, elements of $V_{\one}$ are called \index{vector superspace!odd element of|emph}\emph{odd}.
We call $V_{\zero}$ the \index{vector superspace!even subspace of|emph}\emph{even subspace} of $V$ and $V_{\one}$ the \index{vector superspace!odd subspace of|emph}\emph{odd subspace} of $V$.
Nonzero elements of $V_{\zero}\cup V_{\one}$ are called \index{vector superspace!homogeneous element of|emph}\emph{homogeneous}.
A \index{superalgebra|emph}\emph{superalgebra} is a $\mathbb{Z}_2$-graded algebra $A = A_{\zero} \oplus A_{\one}$.
This means $A_{\alpha}A_{\beta} \subseteq A_{\alpha + \beta}$ for all $\alpha, \beta \in \mathbb{Z}_2$.

We set $(-1)^{\zero} = 1$, $(-1)^{\one} = -1$.
If $a\in V_\alpha$, $v\neq 0$ is homogeneous, we set $p(a) = \alpha$ and call it the \index{parity!of homogeneous element|emph}\emph{parity} of $a$.
If $a$ and $b$ are homogeneous, we set $p(a, b) = (-1)^{p(a)p(b)}$.
A \index{Lie superalgebra|emph}\emph{Lie superalgebra} is a superalgebra $\mathfrak{g} = \mathfrak{g}_{\zero} \oplus \mathfrak{g}_{\one}$ with the product written $[a, b]$ for $a, b \in \mathfrak{g}$ and called \index{Lie superalgebra!Lie superbracket of|emph}\emph{Lie superbracket} satisfying the following properties:
\begin{enumerate}
\item $[\bullet, \bullet]$ is \index{graded antisymmetric|emph}\emph{graded antisymmetric}
  \begin{equation*}
    [a, b] = -p(a, b)[b, a];
  \end{equation*}
\item $[\bullet, \bullet]$ satisfies the \index{graded Jacobi identity|emph}\emph{graded Jacobi identity}
  \begin{equation*}
    p(a, c)[a, [b, c]] + p(b, a)[b, [c, a]] + p(c, b)[c, [a, b]] = 0.
  \end{equation*}
\end{enumerate}

In an \index{associative superalgebra}associative superalgebra $A$, we can define the superbracket of homogeneous elements by
\begin{equation*}
  [a, b] = ab - p(a, b)ba.
\end{equation*}
It can then be extended by linearity to nonhomogeneous elements.
With this superbracket, $A$ becomes a Lie superalgebra called the \index{associative superalgebra!underlying Lie superalgebra of|emph}\emph{underlying Lie superalgebra} of $A$ and is denoted by $[A]$.

\begin{remark}
  \label{rmk:1}
  The even part $\mathfrak{g}_{\zero}$ of a Lie superalgebra is just a standard Lie algebra.
  However, unlike superspaces and superalgebras, a Lie superalgebra is not always a Lie algebra.
  This is why we prefer the term Lie superalgebra instead of $\mathbb{Z}_2$-graded Lie algebra.
\end{remark}

The most important example of an associative superalgebra is the \index{endomorphism superalgebra}endomorphism superalgebra $\End(V)$ of a superspace $V$ with the $\mathbb{Z}_2$-grading given by:
\index{parity!of endomorphism|emph}
\begin{equation*}
  \End(V)_{\alpha} = \{a \in \End(V) \mid \forall \beta \in \mathbb{Z}_2: aV_\beta \subseteq V_{\alpha + \beta}\},
\end{equation*}
for $\alpha \in \mathbb{Z}_2$.
We denote $\mathfrak{gl}(V) = [\End(V)]$.

Let $A$ be a not necessarily associative superalgebra.
A \index{superalgebra!superderivation of|emph}\emph{superderivation} of $A$ is an homogeneous endomorphism $\partial \in \End(A)$ such that
\begin{equation*}
  \partial(ab) = \partial(a)b + (-1)^{p(\partial)p(a)}b\partial(c),
\end{equation*}
for all $a, b \in A$.
The \index{superalgebra!subspace of superderivations of|emph}\emph{subspace of superderivations} of $A$ is denoted by $\Der(A)$.

Let $\mathfrak{g}$ be a Lie superalgebra.
We first extend the Lie superbracket on $\mathfrak{g}$ to the \index{formal distribution!Lie superbracket of|emph}\emph{Lie superbracket} between two $\mathfrak{g}$-valued formal distributions in one variable. Starting from $a(z) = \sum_{m \in \mathbb{Z}}a_{(m)}z^{-m - 1} \in \mathfrak{g}[[z^{\pm 1}]]$ and $b(w) = \sum_{n \in \mathbb{Z}}b_{(n)}z^{-n - 1} \in \mathfrak{g}[[w^{\pm 1}]]$, we define a new formal distribution in two variables by defining the superbracket
\begin{equation*}
  [a(z), b(w)] = \sum_{m, n}[a_{(m)}, b_{(n)}]z^{-m - 1}w^{-n - 1} \in \mathfrak{g}[[z^{\pm 1}, w^{\pm 1}]]
\end{equation*}

Let $\mathfrak{g}$ be a Lie superalgebra.
A pair $(a(z), b(z))$ of $\mathfrak{g}$-valued formal distributions is said \index{formal distribution!local pair of|emph}\emph{local} if $[a(z), b(w)]$ is local.
By \Cref{thr:1}, this means that
\begin{equation*}
  [a(z), b(w)] = \sum_{j \in \mathbb{N}}c^j(w)\frac{\partial^j_w\delta(z, w)}{j!}.
\end{equation*}
with $c^j(w) = \Res_z(z - w)^j[a(z), b(w)] \in \mathfrak{g}[[w^{\pm 1}]]$.
Equivalently, we can write this equation as
\begin{equation}
  \label{eq:3}
  [a_{(m)}, b_{(n)}] = \sum_{j \in \mathbb{N}}c^j(w)_{(m + n - j)}.
\end{equation}
\begin{remark}
  \label{rmk:2}
  If $(a(z), b(z))$ is a local pair then $(\partial_za(z), b(z))$ is also a local pair.
\end{remark}

Let $\mathfrak{g}$ be a Lie superalgebra.
A subset $\mathfrak{F} \subseteq \mathfrak{g}[[z^{\pm 1}]]$ of formal distributions is called a \index{formal distribution!local family of|emph}\emph{local family} if all pairs of its elements are local.
Let $a(w)$ and $b(w)$ be two $\mathfrak{g}$-valued formal distributions.
For $j \in \mathbb{N}$, their \index{formal distribution!$j$-product of|emph}\emph{$j$-product} is the $\mathbb{C}$-bilinear map defined by
\begin{align}
  \nonumber
  \bullet_{(j)}\bullet: \mathfrak{g}[[w^{\pm 1}]] \times \mathfrak{g}[[w^{\pm 1}]] &\to \mathfrak{g}[[w^{\pm 1}]]\\
  \label{eq:4}
  a(w)_{(j)}b(w) &= \Res_z(z - w)^j[a(z), b(w)].
\end{align}
We define $a(w)_{(j)} \in \End(\mathfrak{g}[[z^{\pm 1}]])$ in the natural way.
If $(a(z), b(z))$ is a local pair, then \eqref{eq:3} becomes
\begin{equation*}
  [a_{(m)}, b_{(n)}] = \sum_{j \in \mathbb{N}}\binom{m}{j}(a(w)_{(j)}b(w))_{(m + n - j)}.
\end{equation*}
By \Cref{thr:1}, we also have
\begin{equation*}
  [a(z),b(w)]=\sum_{j\in \mathbb{N}}(a(w)_{(j)}b(w))\frac{\partial_w^j\delta(z,w)}{j!}.
\end{equation*}
All these identities led us to the define the following new algebraic structure that encodes the relevant information compactly.

Let $\mathfrak{g}$ be a Lie superalgebra.
The \index{formal distribution!$\lambda$-bracket of|emph}\emph{$\lambda$-bracket} of two $\mathfrak{g}$-valued formal distributions is defined by the $\mathbb{C}$-bilinear map
\begin{align*}
  [\bullet_{\lambda}\bullet]: \mathfrak{g}[[w^{\pm 1}]] \times \mathfrak{g}[[w^{\pm 1}]] &\to \mathfrak{g}[[w^{\pm 1}]][[\lambda]] \\
  [a(w)_{\lambda}b(w)] &= F^{\lambda}_{z, w}[a(z), b(w)]
\end{align*}
It can easily be shown that the $\lambda$-bracket is related to the $j$-products by
\begin{equation*}
  [a(w)_{\lambda}b(w)] = \sum_{j \in \mathbb{N}}a(w)_{(j)}b(w)\frac{\lambda^j}{j!}.
\end{equation*}
This suggest to see the $\lambda$-bracket as the generating function of the $j$-products.
It allows us to gather all the $j$-products in one product alone, the price to pay being the additional formal variable $\lambda$.
Note that for a local pair, the sum in the expansion of $[a(w)_{\lambda} b(w)]$ in terms of the $j$-products is finite, i.e.\ $[a(w)_{\lambda}b(w)] \in \mathfrak{g}[[w^{\pm 1}]][\lambda]$.

\begin{theorem}[{\cite[Section 2.3]{nozaradan_introduction_2008}}]
  \label{thr:3}
  The $j$-products and the $\lambda$-bracket satisfy the following properties:
  \begin{enumerate}
  \item $(\partial a(w))_{(j)}b = -ja(w)_{(j - 1)}b(w)$;
  \item $a(w)_{(j)}\partial b(w) = \partial(a(w)_{(j)}b(w)) + ja(w)_{(j - 1)}b(w)$;
  \item $\partial(a(w)_{(j)}b(w))=(\partial a(w))_{(j)}b(w)+a(w)_{(j)}\partial b(w)$;
  \item $[\partial a(w)_{\lambda}b(w)] = -\lambda [a(w)_{\lambda}b(w)]$;
  \item $[a(w)_{\lambda}\partial b(w)] = (\partial + \lambda)[a(w)_{\lambda}b(w)]$;
  \item $\partial[a(w)_\lambda b(w)]=[\partial a(w)_\lambda b(w)]+[a(w)_\lambda \partial b(w)]$.
  \end{enumerate}
\end{theorem}
\begin{remark}
  \label{rmk:3}
  Properties (c) and (f) tell us that $\partial: \mathfrak{g}[[z^{\pm 1}]] \to \mathfrak{g}[[z^{\pm 1}]]$ acts as a derivation on the $j-$products and the $\lambda$-bracket.
\end{remark}

Let $V$ be vector superspace.
From now on, all coefficients of a formal distribution are assumed to have the same parity.
Therefore, we can define the \index{parity!of formal distribution|emph}\emph{parity} of a formal distribution $a(z) \in V[[z^{\pm 1}]]$ as $p(a(z)) = p(a_{(n)})$ for any $n \in \mathbb{Z}$.

\begin{theorem}[{\cite[Section 2.3]{nozaradan_introduction_2008}}]
  \label{thr:4}
  Let $\mathfrak{g}$ be a Lie superalgebra
  The $j$-products and the $\lambda$-brackets between $\mathfrak{g}$-valued formal distributions satisfy the following properties
  \begin{enumerate}
  \item $b(w)_{(j)}a(w) = -p(a(w), b(w))\sum_{l = 0}^{\infty}(-1)^{j + l}\frac{\partial^l(a(w)_{(j + l)}b(w))}{l!}$ if $(a(w), b(w))$ is a local pair;
  \item $[a(w)_{(p)}, b(w)_{(m)}] = \sum_{k = 0}^p\binom{p}{k}(a(w)_{(k)}b(w))_{(p + m - k)}$;
  \item $[b(w)_{\lambda}a(w)] = -p(a(w), b(w))[a(w)_{-\lambda - \partial}b(w)]$ if $(a(w), b(w))$ is a local pair;
  \item $[a(w)_{\lambda}[b(w)_{\mu}c(w)]] = [[a(w)_{\lambda}b(w)]_{\lambda + \mu}c(w)] + p(a(w), b(w))[b(w)_{\mu}[a(w)_{\lambda}c(w)]]$;
  \item $F^{\lambda + \mu}_z[a(z)_{\lambda}b(z)] = [F^{\lambda}_za(z), F^{\mu}_zb(z)]$.
  \end{enumerate}
\end{theorem}

Let $\mathfrak{g}$ be a Lie superalgebra.
A \index{formal distribution Lie superalgebra|emph}\emph{formal distribution Lie superalgebra} is a pair $(\mathfrak{g}, \mathfrak{F})$ such that $\mathfrak{F}$ is a local family of $\mathfrak{g}$-valued formal distributions, denoted by $\{a^j(z) = \sum_{n \in \mathbb{Z}}a^j_{(n)}z^{-n - 1}\}_{j \in J}$, such that the coefficients $\{a^j_{(n)} \mid j \in J, n \in \mathbb{Z}\}$ span the whole $\mathfrak{g}$.
A \index{formal distribution Lie superalgebra!regular|emph}\emph{regular} formal distribution Lie superalgebra is a triple $(\mathfrak{g}, \mathfrak{F}, T)$ such that
\begin{enumerate}
\item $(\mathfrak{g}, \mathfrak{F})$ is a formal distribution Lie superalgebra;
\item $\mathbb{C}[\partial_z]\mathfrak{F}$ is closed under all $n$-th products for $n \in \mathbb{N}$;
\item $T \in \Der(\mathfrak{g})$ satisfies
  \begin{equation*}
    T(a^j(z)) = \partial_za^j(z)
  \end{equation*}
  which is equivalent to
  \begin{equation}
    \label{eq:5}
    T(a^j_{(n)}) = -na^j_{(n - 1)},
  \end{equation}
  for all $j \in J$ and $n \in \mathbb{Z}$. 
\end{enumerate}
\begin{remark}
  \label{rmk:4}
  Note that \eqref{eq:5} and the fact that $\{a^j_{(n)} \mid j \in J, n \in \mathbb{Z}\}$ span $\mathfrak{g}$ imply that if such $T$ exists, it is even and unique so we could remove $T$ from the notation, but we won't.
\end{remark}

Let $(\mathfrak{g}, \mathfrak{F})$ be a formal distribution Lie superalgebra.
The \index{formal distribution Lie superalgebra!annihilation subalgebra of|emph}\emph{annihilation subalgebra} of $(\mathfrak{g}, \mathfrak{F})$ is
\begin{equation*}
  \mathfrak{g}_- = \vspan\{a^j_{(n)} \mid j \in J, n \in \mathbb{N}\},
\end{equation*}
the \index{formal distribution Lie superalgebra!creation subalgebra of|emph}\emph{creation subalgebra} of $(\mathfrak{g}, \mathfrak{F})$ is
\begin{equation*}
  \mathfrak{g}_+ = \vspan\{a^j_{(-n - 1)} \mid j \in J, n \in \mathbb{N}\},
\end{equation*}
and the \index{formal distribution Lie superalgebra!polar decomposition|emph}\emph{polar decomposition} of $(\mathfrak{g}, \mathfrak{F})$ is
\begin{equation*}
  \mathfrak{g} = \mathfrak{g}_- \oplus \mathfrak{g}_+.
\end{equation*}

By \Cref{rmk:2}, if $(\mathfrak{g}, \mathfrak{F})$ is a formal distribution Lie superalgebra then $\mathbb{C}[\partial_z]\mathfrak{F}$ is a local family.

The notions of $j$-products and $\lambda$-bracket were previously defined from $\mathfrak{g}$-valued formal distributions with $\mathfrak{g}$ being a given Lie superalgebra.
Those products were shown to satisfy several properties, coming either from their definition or from the fact that $\mathfrak{g}$ is a Lie superalgebra.
Now we take those properties as axioms of a new algebraic structure, defined intrinsically, without any reference either to $\mathfrak{g}$, nor to formal distributions.
For this reason, we write $\partial$ instead of $\partial_z$ in the following definition.

A $\mathbb{C}[\partial]$-module $\mathcal{R}$ is called a Lie conformal superalgebra if it is endowed with a $\mathbb{C}$-bilinear map called $\lambda$-bracket
\begin{equation*}
  [\bullet_{\lambda}\bullet]: \mathcal{R}\times\mathcal{R} \to \mathcal{R}[\lambda]
\end{equation*}
and satisfying the following properties for all $a, b, c \in \mathcal{R}$:
\begin{enumerate}
\item $[\partial a_{\lambda}b] = -\lambda[a_{\lambda}b]$;
\item $[b_{\lambda}a] = -p(a, b)[a_{-\lambda - \partial}b]$;
\item $[a_{\lambda}[b_{\mu}c]] = [[a_{\lambda}b]_{\lambda +\mu}c] + p(a, b)[b_{\mu}[a_{\lambda}c]]$.
\end{enumerate}

If we write
\begin{equation*}
  [a_{\lambda}b] = \sum_{j \in \mathbb{N}}(a_{(j)}b)\frac{\lambda^j}{j!},
\end{equation*}
where $a_{(j)} \in \End(\mathcal{R})$, these properties translate in terms of $j$-products as follows
\begin{enumerate}[label = (\alph*)]
\item $(\partial a)_{(j)} = -j a_{(j - 1)}$; 
\item $b_{(j)}a = -p(a,b)\sum_{l = 0}^\infty(-1)^{j + l}\frac{\partial^l(a_{(j + l)}b)}{l!}$;
\item $[a_{(p)},b_{(m)}] = \sum_{k = 0}^p\binom{p}{k}(a_{(k)}b)_{(p + m - k)}$.
\end{enumerate}

\begin{proposition}[{\cite[Remark 2.5.3]{nozaradan_introduction_2008}}]
  \label{prp:3}
  Let $\mathcal{R}$ be a Lie conformal superalgebra and $a, b \in \mathcal{R}$.
  Then
  \begin{equation*}
    [a_{\lambda}\partial b] = (\partial + \lambda)[a_{\lambda}b],
  \end{equation*}
  or equivalently
  \begin{equation*}
    a_{(j)}\partial b = \partial(a_{(j)}b) + ja_{(j - 1)}b,
  \end{equation*}
  for all $j \in \mathbb{N}$.
  In particular, $\partial$ is a derivation of the $\lambda$-bracket.
\end{proposition}

We have previously shown that any regular formal distribution Lie superalgebra $(\mathfrak{g}, \mathfrak{F}, T)$ was given the structure of Lie Conformal superalgebra $\mathcal{R}$ with $\mathcal{R} = \mathbb{C}[\partial_z]\mathfrak{F}$, $\partial = \partial_z$ and $[a_{\lambda}b] = F^{\lambda}_{z, w}([a(z), b(w)])$.
It turns out that the process can be reverted: to any conformal superalgebra we can associate a regular formal distribution Lie superalgebra.
According to the definition of a formal distribution Lie superalgebra, we first have to define a Lie superalgebra, denoted by $\Lie(\mathcal{R})$ and then associate to it a conformal family $\mathcal{R}$ of $\Lie(\mathcal{R})$-valued formal distributions, whose coefficients span $\Lie(\mathcal{R})$ so that $(\Lie(\mathcal{R}), \mathcal{R})$ is then the expected formal distribution Lie algebra.
We proceed in two steps.

We first consider the space $\widetilde{\mathcal{R}} = \mathcal{R}[t, t^{-1}] = \mathcal{R}\otimes\mathbb{C}[t, t^{-1}]$ with $\widetilde{\partial} = \partial \otimes I + I \otimes \partial_t$ where $I$ appearing on the left (resp.\ right) of $\otimes$ is the identity operator acting on $\mathcal{R}$ (resp.\ $\mathbb{C}[t, t^{-1}]$).
This space is called the affinization of $\mathcal{R}$.
Its generating elements can be written $a\otimes t^m$, where $a \in \mathcal{R}$ and $m \in \mathbb{Z}$.
For clarity, we will use the notation $at^m$ for its elements and $\widetilde{\partial} = \partial + \partial_t$.
Define the commutation relation on $\widetilde{\mathcal{R}}$ as follows:
\begin{equation*}
  [at^m, bt^n] = \sum_{j \in \mathbb{N}}\binom{m}{j}(a_{(j)}b)t^{m + n - j},
\end{equation*}
which gives $\widetilde{\mathcal{R}}$ the structure of algebra, denoted by $(\widetilde{\mathcal{R}},[\bullet, \bullet])$.

Now the second step. We have to check that the commutator verifies the antisymmetry and Jacobi identities, considering that the terms $a_{(j)}b$ of the definition of $[\bullet, \bullet]$ satisfy the axioms.
The latter ones are not sufficient.
Indeed, as we will see, another constraint has to be imposed on elements of $\widetilde{\mathcal{R}}$, namely $\widetilde{\partial}(at^m) = 0$.
The algebraic formulation of the latter condition is as follows: the space $\widetilde{\mathcal{R}}$ has to be quotiented by the subspace $I$ spanned by the elements of the form $\{(\partial a)t^n + nat^{n - 1} \mid n \in \mathbb{Z}\}$.
Using $\widetilde{\partial}$, we can write $I = \widetilde{\partial}\widetilde{\mathcal{R}}$.
This process has two goals: first transferring on $\widetilde{\mathcal{R}}$ the structure of algebra of $(\widetilde{\mathcal{R}}, [\bullet, \bullet])$ and then endowing $(\widetilde{\mathcal{R}}/\widetilde{\partial}\widetilde{\mathcal{R}}, [\bullet, \bullet])$ with the structure of Lie superalgebra.
The first goal is not direct.
Indeed, $\widetilde{\partial}\widetilde{\mathcal{R}}$ has to be a two-sided ideal of the algebra $(\widetilde{\mathcal{R}}, [\bullet, \bullet])$, which is the case.

\begin{lemma}[{\cite[Proposition 2.6.1]{nozaradan_introduction_2008}}]
  \label{lmm:1}
  $\widetilde{\partial}\widetilde{\mathcal{R}}$ is a two-sided ideal of the algebra $(\widetilde{\mathcal{R}}, [\bullet, \bullet])$.
\end{lemma}

Define the homomorphism $\phi: \widetilde{\mathcal{R}} \to \widetilde{\mathcal{R}}/\widetilde{\partial}\widetilde{\mathcal{R}}$ as the natural quotient map.
The commutator between two elements of $\widetilde{\partial}\widetilde{\mathcal{R}}$ is defined by
\begin{equation}
  [\phi(at^m), \phi(bt^n)] = \sum_{j \in \mathbb{N}}\binom{m}{j}\phi((a_{(j)}b)t^{m + n - j}),
\end{equation}
where $a, b \in \mathcal{R}$.

\begin{proposition}[{\cite[Proposition 2.6.3]{nozaradan_introduction_2008}}]
  \label{prp:4}
  $(\widetilde{\mathcal{R}}/\widetilde{\partial}\widetilde{\mathcal{R}}, [\bullet, \bullet])$ is a Lie superalgebra.
\end{proposition}

We set
\begin{equation*}
  \Lie(\mathcal{R}) = \widetilde{\mathcal{R}}/\widetilde{\partial}\widetilde{\mathcal{R}}.
\end{equation*}
Abusing notation, we define the family $\mathcal{R}$ of $\Lie(\mathcal{R})$-valued formal distributions, whose coefficients span $\Lie(\mathcal{R})$, by
\begin{equation*}
  \mathcal{R} = \left\{\sum_{n \in \mathbb{Z}}\phi(at^n)z^{-n - 1} \mid a \in \mathcal{R}\right\}.
\end{equation*}

\begin{theorem}[{\cite[Proposition 2.6.4]{nozaradan_introduction_2008}}]
  \label{thr:5}
  Let $\mathcal{R}$ be a Lie conformal superalgebra.
  Then $(\Lie(\mathcal{R}), \mathcal{R},-\partial_t)$ is a regular formal distribution Lie superalgebra. 
\end{theorem}

\begin{remark}
  \label{rmk:5}
  We haven't defined the category of regular formal distribution Lie superalgebras nor the category of Lie conformal superalgebras.
  But it's clear how they should be and they are equivalent categories:
  \begin{align*}
    \{\text{regular formal distribution Lie superalgebra}\} &\leftrightarrow \{\text{Lie conformal superalgebra}\} \\
    (\mathfrak{g}, \mathfrak{F}, T) &\mapsto (\mathbb{C}[\partial_z]\mathfrak{F}, F^{\lambda}_{z, w}([\bullet, \bullet])) \\
    (\Lie(\mathcal{R}), \mathcal{R},-\partial_t) &\mapsfrom \mathcal{R}
  \end{align*}
\end{remark}

We now show three examples of regular formal distribution Lie superalgebras and their respective Lie conformal superalgebras.

\begin{example}[\index{Virasoro!Lie conformal algebra|emph}\emph{Virasoro Lie conformal algebra}]
  \label{exa:1}
  The Virasoro Lie algebra, denoted $\Vir$, is the complex Lie algebra given by
  \begin{equation*}
    \Vir = \bigoplus_{n \in \mathbb{Z}}\mathbb{C}L_{n} \oplus \mathbb{C}C.
  \end{equation*}
  These elements satisfy the following commutation relations:
  \begin{equation}
    \label{eq:6}
    \begin{aligned}
      [L_m, L_n] &= (m - n)L_{m + n} + \delta_{m, -n}\frac{m^3 - m}{12}C, \\
      [\Vir, C] &= 0,
    \end{aligned}
  \end{equation}
  for all $m, n \in \mathbb{Z}$.
  We construct a $\Vir$-valued formal distribution by setting
  
  \begin{equation*}
    L(z) = \sum_{n \in \mathbb{Z}}L_{(n)}z^{-n - 1}\text{ with }L_{(n)} = L_{n - 1}.
  \end{equation*}
  Hence $L(z) = \sum_{n \in \mathbb{Z}}L_nz^{-n - 2}$.
  In terms of formal distributions, the commutation relations become:
  \begin{equation}
    \label{eq:7}
    \begin{aligned}
      [L(z), L(w)] &= \partial_wL(w)\delta(z, w)+2L(w)\partial_w\delta(z,w)+\frac{C}{12}\partial^3_w\delta(z,w) \\
      [L(z), C] &= 0,
    \end{aligned}
  \end{equation}
  where $C$ denotes the constant formal distribution equal to $C \in \Vir$.
  In terms of $j$-products, the commutation relations become:
  \begin{equation}
    \label{eq:8}
    \begin{aligned}
      L(z)_{(0)}L(z) &= \partial_zL(z), \\
      L(z)_{(1)}L(z) &= 2L(z), \\
      L(z)_{(3)}L(z) &= \frac{C}{2}, \\
      L(z)_{(j)}L(z) &= 0 \quad (j \neq 0, 1, 3), \\
      L(z)_{(j)}C &= 0 \quad (j \in \mathbb{N}).
    \end{aligned}
\end{equation}
  In terms of the $\lambda$-bracket, the commutation relations become:
  \begin{equation}
    \label{eq:9}
    \begin{aligned}
      [L(z)_{\lambda}L(z)] &= (\partial + 2\lambda)L(z) + \frac{\lambda^3}{12}C \\
      [L(z)_{\lambda}C] &= 0.
    \end{aligned}
\end{equation}
  By \Cref{thr:1}, $\{L(z), C\}$ is a local family.
  Therefore, $(\Vir, \{L(z), C\})$ is a formal distribution Lie algebra.
  Moreover, we can verify directly that $(\Vir, \{L(z), C\}, \ad(L_{-1}))$ is regular.
  We obtain a Lie conformal algebra $\mathcal{R} = \mathbb{C}[\partial]L + \mathbb{C}C$, with $L = L(z)$, $\partial C = 0$ and $\partial = \partial_z$.
  This is actually a direct sum and  we get the Virasoro Lie conformal algebra
  \begin{equation*}
    \Vir = \mathbb{C}[\partial]L \oplus \mathbb{C}C.
  \end{equation*}
\end{example}

\begin{remark}
  \label{rmk:6}
  The notation $L(z) = \sum_{n \in \mathbb{Z}}L_nz^{-n - 2}$ is contradictory with the notation we wrote in \Cref{sec:formal-calculus}.
  However, this notation will acquire a meaning when we treat the notion of weight of an eigendistribution.
  In fact, this notation usually simplifies calculations, as we will see later.
\end{remark}

Let $\mathfrak{g}$ be a Lie superalgebra and $C \in \mathfrak{g}$.
A Virasoro formal distribution $L(z) \in \mathfrak{g}[[z^{\pm 1}]]$ with central charge $C$ is a $\mathfrak{g}$-valued formal distribution satisfying \eqref{eq:6}, or equivalently, \eqref{eq:7}, \eqref{eq:8} or \eqref{eq:9}.

\begin{example}[Current Lie conformal superalgebra]
  \label{exa:2}
  Let $\mathfrak{g} = \mathfrak{g}_{\zero} \oplus \mathfrak{g}_{\one}$ be a Lie superalgebra.
  A supersymmetric bilinear form is a bilinear map $(\bullet| \bullet): \mathfrak{g} \times \mathfrak{g} \to \mathbb{C}$ such that
  \begin{equation*}
    (a| b) = (-1)^{p(a)}(b| a),
  \end{equation*}
  where $a$ and $b$ are homogeneous elements of $\mathfrak{g}$.
  Alternatively, we can define a supersymmetric bilinear form as a bilinear form that vanishes on $\mathfrak{g}_{\zero} \oplus \mathfrak{g}_{\one}$ and $\mathfrak{g}_{\one} \oplus \mathfrak{g}_{\zero}$, symmetric on $\mathfrak{g}_{\zero} \oplus \mathfrak{g}_{\zero}$ and antisymmetric on $\mathfrak{g}_{\one} \oplus \mathfrak{g}_{\one}$.
  A bilinear form $(\bullet| \bullet): \mathfrak{g} \times \mathfrak{g} \to \mathbb{C}$ is said invariant if
  \begin{equation*}
    ([a, b]| c)=(a| [b, c])\quad a, b, c \in \mathfrak{g}.
  \end{equation*}

  Let $\mathfrak{g}$ be a Lie superalgebra endowed with a supersymmetric, invariant bilinear form $(\bullet| \bullet)$.
  The associated loop algebra, denoted by $\tilde{\mathfrak{g}}$, is the algebra $\tilde{\mathfrak{g}} = \mathfrak{g} \otimes \mathbb{C}[t, t^{-1}] = \mathfrak{g}[t, t^{-1}]$, endowed with the superbracket defined by
  \begin{equation*}
    [a\otimes f(t), b\otimes g(t)] = [a, b] \otimes f(t)g(t),
  \end{equation*}
  where $a, b \in \mathfrak{g}$, $f(t), g(t) \in \mathbb{C}[t, t^{-1}]$ and with parity given by $p(a\otimes f(t)) = p(a)$.
  This makes $\tilde{\mathfrak{g}}$ into a Lie superalgebra.
  Replacing $a\otimes t^n$ by $at^n$ for brevity, the commutation relations become
  \begin{equation*}
    [at^m, bt^n]=[a, b]t^{m + n}.
  \end{equation*}
  The central extension of the loop algebra is the algebra $\hat{\mathfrak{g}} = \tilde{\mathfrak{g}} \oplus \mathbb{C}K$ with the superbracket
  \begin{equation*}
    \begin{aligned}
    [at^m, bt^n] &= [a, b]t^{m + n} + m\delta_{m, -n}(a| b)K, \\
    [\hat{\mathfrak{g}}, K] &= 0
    \end{aligned}
  \end{equation*}
  where $a, b \in \mathfrak{g}$, $n, m \in \mathbb{Z}$ and with parity given by $p(K) = \zero$.
  This makes $\hat{\mathfrak{g}}$ into a Lie superalgebra called the affinization of $\mathfrak{g}$.
  If $\mathfrak{g}$ is a finite-dimensional simple Lie superalgebra, then the affinization of $\mathfrak{g}$ leads to a Kac-Moody affinization.
  We now construct $\hat{\mathfrak{g}}$-valued formal distributions by setting
  \begin{equation*}
    a(z) = \sum_{n \in \mathbb{Z}}at^nz^{-n - 1},
  \end{equation*}
  for $a \in \mathfrak{g}$.
  These formal distributions are called currents.
  In terms of currents, the commutation relations become:
  \begin{equation*}
    \begin{aligned}
    [a(z), b(w)] &= [a, b](w)\delta(z, w) + K(a| b)\partial_w\delta(z, w), \\
    [a(z), K] &= 0
    \end{aligned}
  \end{equation*}
  In terms of the $\lambda$-bracket, the commutation relations become:
  \begin{equation*}
    \begin{aligned}
    [a(z)_{\lambda}b(z)] &= [a(z), b(z)] + (a| b)K\lambda, \\
    [a(z)_{\lambda}K] &= 0.
    \end{aligned}
  \end{equation*}
  By \Cref{thr:1}, $\{a(z) \mid a \in \mathfrak{g}\} \cup \{K\}$ is a local family.
  Therefore, $(\hat{\mathfrak{g}}, \{a(z) \mid a \in \mathfrak{g}\} \cup \{K\})$ is a formal distribution Lie superalgebra.
  Moreover, we can verify directly that $(\hat{\mathfrak{g}}, \{a(z) \mid a \in \mathfrak{g}\} \cup \{K\}, -\partial_t)$ is regular.
  In a similar way to the Virasoro Lie conformal algebra, we obtain the current Lie conformal superalgebra
  \begin{equation*}
    \Cur(\mathfrak{g}) = \mathbb{C}[\partial]\mathfrak{g} \oplus \mathbb{C}K.
  \end{equation*}
  Let $\mathfrak{g}$ be an abelian Lie superalgebra.
  In that case, $\Cur(\mathfrak{g})$ is known as the conformal algebra of free bosons associated with the free bosons algebra $\hat{\mathfrak{g}}$, the latter being endowed with the relations $[at^m, bt^n] = m(a| b)\delta_{m, -n}K$. 
\end{example}

\begin{example}[Fermionic Lie conformal superalgebra]
  \label{exa:3}
  Let $V = V_{\zero} \oplus V_{\one}$ be a superspace. A bilinear form $\langle\bullet, \bullet\rangle: V\times V \to \mathbb{C}$ is antisupersymmetric if it satisfies the relation
  \begin{equation*}
    \langle a, b\rangle = -(-1)^{p(a)}\langle b, a\rangle
  \end{equation*}
  for homogeneous elements $a, b\in V$.
  Alternatively, we can define an antisymmetric bilinear form  as a bilinear form that vanishes on $V_{\zero} \oplus V_{\one}$ and $V_{\one}\oplus V_{\zero}$, is antisymmetric on the even part $V_{\zero}\oplus V_{\zero}$ and symmetric on the odd part $V_{\one}\oplus V_{\one}$.

  The Clifford affinization of $V$ is defined by
  \begin{equation*}
    \widehat{V} = V[t, t^{-1}] \oplus \mathbb{C}K
  \end{equation*}
  with the superbracket
  \begin{equation}
    \label{eq:10}
    \begin{aligned}
      [at^m, bt^n] &= \delta_{m, -n - 1}\langle a, b\rangle K, \\
      [at^m, K] &= 0,
    \end{aligned}
  \end{equation}
  where $a, b\in V$ and with parity given by $p(at^n) = p(a)$ and $p(K) = \zero$.
  This makes $\widehat{V}$ into a Lie superalgebra.
  We now construct $\widehat{V}$-valued formal distributions by setting
  \begin{equation*}
    a(z) = \sum_{m \in \mathbb{Z}}at^mz^{-m - 1},
  \end{equation*}
  for $a \in V$.
  In terms of formal distributions, the commutation relations become:
  \begin{equation*}
    \begin{aligned}
      [a(z), b(w)] &= \langle a, b\rangle K\delta(z, w), \\
      [a(z),K] &= 0.
    \end{aligned}
  \end{equation*}
  In terms of the $\lambda$-bracket, the commutation relations become:
  \begin{equation*}
    \begin{aligned}
      [a(z)_{\lambda}b(z)] &= \langle a, b\rangle K, \\
      [a(z)_{\lambda}K] &= 0.
    \end{aligned}
  \end{equation*}
  As before, we obtain a regular formal distribution Lie superalgebra $(\widehat{V}, \{a(z) \mid a \in V\} \cup \{K\}, -\partial_t)$ from which we obtain the Fermionic Lie conformal superalgebra
  \begin{equation*}
    F(V) = \mathbb{C}[\partial]V \oplus \mathbb{C}K.
  \end{equation*}
\end{example}

\subsection{Fields over vector spaces}
\label{sec:fields-over-vector}

In this subsection, we fix a vector superspace $V = V_{\zero} \oplus V_{\one}$ and all formal distributions are $\End(V)$-valued unless otherwise stated.
Let $a(z)$ be a formal distribution.
Set
\begin{align*}
  a(z)_+&=\sum_{n\le -1}a_{(n)}z^{-n-1} \\
  a(z)_-&=\sum_{n\ge 0}a_{(n)}z^{-n-1}.
\end{align*}

Let $a(z)$ and $b(z)$ be two formal distributions.
We define the normal product between $a(z)$ and $b(z)$ as the following formal distribution in two variables
\begin{equation*}
  :a(z)b(w):=a(z)_+b(w)+p(a,b)b(w)a(z)_-.
\end{equation*}

\begin{theorem}[{\cite[Proposition 3.2.3]{nozaradan_introduction_2008}}]
  \label{thr:6}
  Let $(a(z), b(z))$ be a pair of local formal distributions.
  The following identities are known as the operator product expansion of $a(z)$ and $b(w)$:
  \begin{align*}
    a(z)b(w) &= \sum_{j \in \mathbb{N}}a(w)_{(j)}b(w)i_{z, w}\frac{1}{(z - w)^{j + 1}} + :a(z)b(w):, \\
    p(a, b)b(w)a(z) &= \sum_{j \in \mathbb{N}}a(w)_{(j)}b(w)i_{w, z}\frac{1}{(z - w)^{j + 1}} + :a(z)b(w):.
  \end{align*}
\end{theorem}

A formal distribution $a(z)$ is a field if
\begin{equation*}
  a(z)b = \sum_{n \in \mathbb{Z}}a_{(n)}bz^{-n - 1} \in V((z))
\end{equation*}
for all $b \in V$.
The set of fields on $V$ is denoted by $\mathcal{F}(V)$.
Note that
\begin{equation*}
  \mathcal{F}(V) = \Hom(V, V((z))).
\end{equation*}
Therefore, we can define a field $a(z)$ by defining $a(z)b \in V((z))$ for $b \in V$.

\begin{proposition}[{\cite[Proposition 3.3.2]{nozaradan_introduction_2008}}]
  \label{prp:5}
  Let $a(z), b(z)\in \mathcal{F}(V)$ be two fields.
  Then $:a(z)b(z):\in \End(V)[[z,z^{-1}]]$ is again a field where $:a(z)b(z):$ is defined by
  \begin{equation*}
    :a(z)b(z):c = a(z)_+b(z)c + p(a, b)b(z)a(z)_-c,
  \end{equation*}
  for $c \in V$.
\end{proposition}

We thus defined the notion of Normal Ordered Product $:a(z)b(z):$ between fields $a(z), b(z) \in \mathcal{F}(V)$.
In general, the operation of normal ordered product is neither commutative nor associative. We follow the convention that the normal ordered ordering is read from right to left, so that by definition
\begin{equation*}
  :a(z)b(z)c(z): = :a(z)(:b(z)c(z):):.
\end{equation*}
The normal ordered product between $a(z)$ and $b(z)$ is explicitly given in the following proposition.

\begin{proposition}[{\cite[Proposition 3.3.3]{nozaradan_introduction_2008}}]
  \label{prp:6}
  Let $a(z)$ and $b(z)$ be two fields. Their normal ordered product is written explicitly as
  \begin{equation*}
    :a(z)b(z): = \sum_{j \in \mathbb{Z}}:ab:_{(j)}z^{-j - 1},
  \end{equation*}
  with
  \begin{equation*}
    :ab:_{(j)}c = \sum_{n \le -1}a_{(n)}b_{(j - n - 1)}c + p(a, b)\sum_{n \ge 0}b_{(j - n - 1)}a_{(n)}c,
  \end{equation*}
  for $c \in V$.
\end{proposition}

Now let's extend the $j$-products.
For $j \in \mathbb{N}$, set
\begin{equation}
  \label{eq:11}
  a(w)_{(-1 - j)}b(w) = \frac{:(\partial^j_wa(w))b(w):}{j!}.
\end{equation}

\begin{theorem}[{\cite[Proposition 3.4.3]{nozaradan_introduction_2008}}]
  \label{thr:7}
  The $j$-products \eqref{eq:4} and \eqref{eq:1} are special cases of the following generalized $j$-product defined by
  \begin{equation*}
    a(w)_{(j)}b(w)c = \Res_z(i_{z, w}(z - w)^ja(z)b(w)c - p(a,b)i_{w, z}(z - w)^jb(w)a(z)c),
  \end{equation*}
  for $c \in V$.
\end{theorem}

The usual properties of $j$-products for $j \in \mathbb{N}$ carry over to the generalized $j$-products.

\begin{proposition}[{\cite[Proposition 3.4.4]{nozaradan_introduction_2008}}]
  \label{prp:7}
  Let $a(z)$ and $b(z)$ be formal distributions.
  For all $j \in \mathbb{Z}$, we have:
  \begin{enumerate}
  \item $(\partial a(z))_{(j)}b(z) = -ja(z)_{(j - 1)}b(z)$;
  \item $\partial(a(z)_{(j)}b(z)) = (\partial_za(z))_{(j)}b(z) + a(z)_{(j)}\partial_zb(z)$.
  \end{enumerate}
\end{proposition}

\begin{lemma}[Dong's lemma {\cite[LEMMA 3.2]{kac_vertex_1998}}]
  \label{lmm:2}
  If $a(z)$, $b(z)$ and $c(z)$ are pairwise mutually local fields, then $(a(z),b(z)_{(n)}c(z))$ is a local pair of fields as well for $n\in \mathbb{Z}$.
\end{lemma}

Let $L(z)$ be a formal distribution.
An eigendistribution of weight $\Delta_a$ with respect to $L(z)$ is a formal distribution $a(z)$ satisfying
\begin{equation}
  \label{eq:12}
  [L_{\lambda}a] = (\partial + \Delta_a\lambda)a + O(\lambda^2),
\end{equation}
where $O(\lambda^2)$ denotes sums of terms of monomials $s\lambda^j$ with $j \ge 2$.
If $L$ is a Virasoro formal distribution, $\Delta_a$ is called the conformal weight.
Clearly, $a(z)$ is an eigendistribution in the sense that it is an eigenvector of the endomorphism $L_{(1)}$ whose action is defined by $L_{(1)}(b) = L_{(1)}b$.
Indeed, \eqref{eq:12} implies $L_{(1)}a = \Delta_aa$.
Note that, by definition, a Virasoro formal distribution $L(z)$ is an eigendistribution of conformal weight $2$ with respect to itself.

\begin{theorem}[{\cite[Proposition 3.7.4]{nozaradan_introduction_2008}}]
  \label{thr:8}
  If $a(z)$ and $b(z)$ have weights $\Delta_a$ and $\Delta_b$ with an even formal distribution $L$, then $\Delta_{a_{(n)}b} = \Delta_a + \Delta_b - n - 1$ with respect to $L$.
  In particular, $\Delta_{:ab:} = \Delta_a + \Delta_b$ and $\Delta_{\partial a} = \Delta_a + 1$.
\end{theorem}

The expansion of an eigendistribution $a(z)$ of weight $\Delta_a$ is often adapted as follows:
\begin{equation*}
  a(z) = \sum_{n \in \mathbb{Z} - \Delta_a}a_nz^{-n - \Delta_a}.
\end{equation*}
This justifies the way we wrote the Virasoro formal distribution when we defined the Virasoro Lie conformal algebra.
By comparison with the usual way of writing $a(z) = \sum_{m \in \mathbb{Z}}a_{(m)}z^{-m - 1}$, we must have
\begin{equation*}
  a_{(m)} = a_{m - \Delta_a + 1},
\end{equation*}
or the other way
\begin{equation*}
  a_n = a_{(n + \Delta_a - 1)}.
\end{equation*}
One of the interesting features of this change of notation is that it reveals the gradation of the superbracket.

\begin{theorem}[{\cite[Proposition 3.7.6]{nozaradan_introduction_2008}}]
  \label{thr:9}
  In the new notation, we can write
  \begin{equation*}
    [a_m, b_n] = \sum_{j \in \mathbb{N}}\binom{m + \Delta_a - 1}{j}(a_{(j)}b)_{m + n}.
  \end{equation*}
\end{theorem}

An eigendistribution $a(z)$ is called primary of conformal weight $\Delta_a$ if
\begin{equation*}
  [L_\lambda a]=(\partial +\Delta_a\lambda)a,
\end{equation*}
where $L(z)$ is a Virasoro formal distribution.

Let's fix an operator $T\in \End(V)$.
A formal distribution $a(z)$ is covariant with respect to $T$ if
\begin{equation*}
  [T, a(z)] = \partial_za(z).
\end{equation*}

\begin{theorem}[{\cite[Lemma 1]{callegaro_introduction_2017}}]
  \label{thr:10}
  Assume that $\vac \in V$ is such that $T\vac = 0$.
  Then
  \begin{enumerate}
  \item For any translation covariant field $a(z)$ we have $a(z)\vac \in V[[z]]$.
  \item Set $a = a_{(-1)}\vac$.
    Then
    \begin{equation*}
      a(z)\vac = e^{Tz}a = \sum_{n = 0}^\infty\frac{T^n(a)}{n!}z^n.
    \end{equation*}
  \end{enumerate}
\end{theorem}

Let
\begin{equation*}
  \mathcal{F}_{tc} = \{a(z) \in \mathcal{F}(V) \mid [T, a(z)] = \partial_za(z)\}
\end{equation*}
be the subspace of translation covariant fields.

\begin{lemma}[{\cite[Lemma 3]{callegaro_introduction_2017}}]
  \label{lmm:3}
  $\mathcal{F}_{tc}$ contains $\Id_V$, it is $\partial_z$-invariant and is closed under all $n$-products, i.e., $\partial_za(z), a(z)_{(n)}b(z) \in \mathcal{F}_{tc}$ for $a(z), b(z) \in \mathcal{F}_{tc}$ and $n \in \mathbb{Z}$.
\end{lemma}

By \Cref{thr:10}, we can define a linear map
\begin{align*}
  \fs: \mathcal{F}_{tc} &\to V \\
  \fs(a(z)) &= a(z)\vac|_{z = 0},
\end{align*}
called the field-state correspondence.

\begin{lemma}[{\cite[Proposition 4.3.2]{nozaradan_introduction_2008}}]
  \label{lmm:4}
  Let $A$ be a linear operator on a linear space $V$.
  The formal differential equation
  \begin{equation*}
    \frac{df(z)}{dz} = Af(z) \quad (f(z) \in V[[z]])
  \end{equation*}
  admits a unique solution, given an initial condition $f(0)=f_0$.
\end{lemma}

\begin{theorem}
  \label{thr:11}
  Let $a(z), b(z) \in \mathcal{F}_{tc}$, $a = \fs(a(z))$, $b = \fs(b(z))$ and $n \in \mathbb{Z}$.
  Write $a(z) = \sum_{j \in \mathbb{Z}}a_{(j)}z^{-j - 1}$ and $b(z) = \sum_{j \in \mathbb{Z}}b_{(j)}z^{-j - 1}$.
  Then
  \begin{enumerate}
  \item $\fs(\Id_V) = \vac$;
  \item $\fs(\partial_za(z)) = Ta$;
  \item ($n$-product identity) $\fs(a(z)_{(n)}b(z)) = a_{(n)}b$;
  \item $T(a_{(n)}b) = -na_{(n - 1)}b + a_{(n)}Tb$;
  \item $e^{Tw}a(z)e^{-Tw} = i_{z, w}a(z + w)$;
  \item (Borcherds identity). If $a(z)$ and $b(z)$ are local then
    \begin{equation*}
      i_{z,w}(z-w)^na(z)b(w)c-p(a,b)i_{w,z}(z-w)^nb(w)a(z)c=\sum_{j\in \mathbb{N}}a(w)_{(n+j)}b(w)\frac{\partial^j_w\delta(z,w)}{j!}c,
    \end{equation*}
    for $c\in V$;
  \item (Skewsymmetry) If $a(z)$ and $b(z)$ are local then
    \begin{equation*}
      a(z)b=p(a,b)e^{Tz}b(-z)a.
    \end{equation*}
  \end{enumerate}
\end{theorem}

\begin{proof}
  \begin{enumerate}
  \item Clear.
  \item
    \begin{align*}
      \fs(\partial_za(z)) &= [T, a(z)]\vac|_{z = 0} \\
      &= (Ta(z) - p(a(z), T)a(z)T)\vac|_{z = 0} \\
      &= Ta.
    \end{align*}
  \item By definition, we have
    \begin{equation*}
      \fs(a(z)_{(n)}b(z)) = a(z)_{(n)}b(z)\vac|_{z = 0},
    \end{equation*}
    and the right hand side, by \Cref{thr:7}, is equal to
    \begin{equation*}
      \Res_w(a(w)b(z)i_{w, z}(w - z)^n\vac - p(a, b)b(z)a(w)i_{z, w}(w - z)^n\vac)|_{z = 0}.
    \end{equation*}
    Now, since $a(w)\vac \in V[[w]]$ and $i_{z, w}(w-z)^n$ has only nonnegative powers of $w$, we have
    \begin{equation*}
      \Res_w(b(z)a(w)i_{z, w}(w - z)^n\vac) = 0.
    \end{equation*}
    For the first term, since $b(z)\vac \in V[[z]]$, we can let $z = 0$ before we calculate the residue, which gives
    \begin{equation*}
      \Res_w(a(w)b(z)i_{w, z}(w - z)^n\vac)|_{z = 0} = \Res_w(a(w)bw^n) = a_{(n)}b.
    \end{equation*}
  \item This follows from $[T, a_{(n)}] = -na_{(n - 1)}$ which is equivalent to translation covariance of the field $a(z)$.
  \item Set $b_1(z, w) = i_{z, w}a(z + w)$ and $b_2(z, w) = e^{Tw}a(z)e^{-Tw}$.
    By \Cref{thr:2} and translation covariance,
    \begin{align*}
      \frac{\partial b_1(z, w)}{\partial w} &= \sum_{j \in \mathbb{N}}\frac{\partial^{j + 1}_za(z)}{j!}w^j = \sum_{j \in \mathbb{N}}\frac{\partial^j[T, a(z)]}{j!} = [T, b_1(z, w)], \\
      b_1(z, 0) &= a(z), \\
      \frac{\partial b_1(z, w)}{\partial w} &= Te^{Tw}a(z)e^{-Tw} + e^{Tw}a(z)(-T)e^{-Tw} = [T, b_2(z, w)], \\
      b_2(z, 0) &= a(z).
    \end{align*}
    By \Cref{lmm:4}, $b_1(z, w) = b_2(z, w)$.
  \item The left hand side of the Borcherds identity is a local formal distribution in $z$ and $w$ applied to $c$.
    Apply \Cref{thr:1} to it to get that it is equal to
    \begin{equation*}
      \sum_{j \in \mathbb{N}} c^j(w)\frac{\partial^j_w\delta(z,w)}{j!}c,
    \end{equation*}
    where
    \begin{align*}
      c^j(w)c &= (\Res_z(z - w)^j(i_{z, w}(z - w)^na(z)b(w) - p(a, b)i_{w, z}(z - w)^nb(w)a(z)))c \\
      &= \Res_z(i_{z, w}(z - w)^{n + j}a(z)b(w)c - p(a, b)i_{w, z}(z - w)^{n + j}b(w)a(z)c) \\
      &= a(w)_{(n + j)}b(w)c.
    \end{align*}
  \item By locality, there is $N \in \mathbb{Z}$ such that
    \begin{equation*}
      (z - w)^Na(z)b(w) = p(a, b)(z - w)^Nb(w)a(z).
    \end{equation*}
    Apply $\vac$ to both sides; by \Cref{thr:10}, we get
    \begin{equation*}
      (z - w)^Na(z)e^{Tw}b = p(a, b)(z - w)^Nb(w)e^{Tz}a.
    \end{equation*}
    Now use (f) and \Cref{thr:2}:
    \begin{equation*}
      RHS = p(a, b)(z - w)^Ne^{Tz}e^{-Tz}b(w)e^{Tz}a = p(a, b)(z - w)^Ne^{Tz}i_{w, z}b(w - z)a.
    \end{equation*}
    For $N$ big enough, this is a formal power series in $(z - w)$, so we can set $w = 0$ and get
    \begin{equation*}
      LHS = z^Na(z)b = p(a, b)e^{Tz}z^Nb(-z)a = RHS,
    \end{equation*}
    which proves the desired formula. \qedhere
  \end{enumerate}
\end{proof}

\begin{lemma}[{\cite[Lemma 3]{callegaro_introduction_2017}}]
  \label{lmm:5}
  Let $\mathcal{F}' \subseteq \mathcal{F}_{tc}$ and $a(z) \in \mathcal{F}_{tc}$.
  Assume that
  \begin{enumerate}
  \item $\fs(a(z)) = 0$;
  \item $a(z)$ is local with any element in $\mathcal{F}'$;
  \item $\fs(\mathcal{F}') = V$.
  \end{enumerate}
  Then $a(z) = 0$.
\end{lemma}

\begin{proof}
  Let $b(z) \in \mathcal{F}'$.
  By the locality of $a(z)$ and $b(z)$, we have $(z - w)^N[a(z), b(w)] = 0$, for some $N \in \mathbb{N}$.
  Apply $\vac$ to both sides to get
  \begin{equation*}
    (z - w)^Na(z)b(w)\vac = \pm(z - w)^Nb(w)a(z)\vac.
  \end{equation*}
  By the property (i), we have $a_{(-1)}\vac = 0$ and $a(z)$ is translation covariant, hence by \Cref{thr:10}(i), $b(w)\vac \in V[[w]]$, so we can let $w = 0$ and get $z^Na(z)b = 0$, which means $a_{(n)}b = 0$ for any $n \in \mathbb{Z}$.
  This is true for any $b \in V$ by the property (c).
  So in fact, we have $a(z) = 0$.
\end{proof}

\subsection{Vertex algebras}
\label{sec:vertex-algebras}

A vertex algebra is the data consisting of four elements $(V, \vac, T, Y)$ satisfying the following properties:
\begin{enumerate}
\item $V$ is a superspace called the state space;
\item $\vac \in V_{\zero}$ is called the vacuum vector;
\item $T \in \End(V)_{\zero}$ is called the translation operator;
\item $Y: V \to \mathcal{F}(V)$ is a linear and parity preserving map called the state-field correspondence.
\end{enumerate}
By parity preserving map we mean that for $a \in V$ homogeneous, $p(a_{(n)}) = p(a)$ for all $n \in \mathbb{Z}$.
The operator $(a) = Y(a, z) \in \End(V)[[z^{\pm 1}]]$ for $a \in V$ is sometimes called a vertex operator.
The data must satisfy the following axioms for $a \in V$:
\begin{enumerate}
\item (Vacuum axiom)
  \begin{align*}
    Y(\vac,z) &= \Id_V, \\
    Y(a, z)\vac &\in V[[z]], \\
    Y(a, z)\vac|_{z = 0} &= a, \\
    T\vac &= 0;
  \end{align*}
\item (Translation covariance) $[T, Y(a, z)] = \partial_zY(a, z)$;
\item (Locality) $\{Y(a, z) \mid a \in V\}$ is a local family of fields.
\end{enumerate}

\begin{remark}
  \label{rmk:7}
  Writing $Y(a, z) = \sum_{n \in \mathbb{Z}}a_{(n)}z^{-n - 1}$ for $a \in V$, the first two vertex algebra axioms imply that for $n \in \mathbb{Z}$:
  \begin{align*}
    \vac_{(n)}a &= a_{(n)}\vac = \delta_{n, -1}a, \\  
    [T, a_{(n)}] &= -na_{(n - 1)}.
  \end{align*}
\end{remark}

\begin{remark}
  \label{rmk:8}
  The translation covariance axiom together with \Cref{thr:10} permit us to express $T$ by
  \begin{equation}
    \label{eq:13}
    Ta = a_{(-2)}\vac.
  \end{equation}
  As a consequence, the data of the translation operator $T$ is redundant.
  The original definition with $T$ appears to be more natural, though.
\end{remark}

A vertex algebra homomorphism $f: (V, \vac, T, Y) \to (V', \vac', T', Y')$ is a linear map $f: V \to V'$ such that $f(\vac) = \vac'$ and for $a, b \in V$,
\begin{equation*}
  f(Y(a, z)b) = \sum_{n \in \mathbb{Z}}f(a_{(n)}b)z^{-n - 1} = \sum_{n \in \mathbb{Z}}f(a)_{(n)}f(b)z^{-n - 1} = Y'(f(a), z)f(b).
\end{equation*}
We obtain the category of vertex algebras.

\begin{remark}
  \label{rmk:9}
  It is easy to verify that category of vertex algebras is abelian in a natural way.
\end{remark}

\begin{example}[Commutative vertex algebras]
  \label{exa:4}
  A vertex algebra $V$ is called commutative if all vertex operators $Y(a, z)$, $a \in V$ commute with each other.
  
  Suppose we are given a commutative vertex algebra $V$.
  Then for $a, b \in A$,
  \begin{equation*}
    Y(a, z)b = Y(a, z)Y(b, w)\vac|_{w = 0} = Y(b, w)Y(a, z)\vac|_{w = 0}.
  \end{equation*}
  But by the vacuum axiom, the last expression has no negative power of $z$.
  Therefore $Y(a, z)b \in V[[z]]$ for $a, b \in V$ so $Y(a, z) \in \End(V)[[z]]$ for $a \in V$.
  Conversely, suppose that we are given a vertex algebra $V$ in which $Y(a, z) \in \End(V)[[z]]$ for all $a \in V$.
  Observe that if the equality $(z - w)^Nf_1(z, w) = (z - w)^Nf_2(z, w)$ holds for $f_1, f_2 \in \mathbb{C}[[z, w]]$ and $N \in \mathbb{N}$, then necessarily $f_1(z, w) = f_w(z, w)$.
  Therefore we obtain that $[Y(a, z), Y(b, w)] = 0$ for all $a, b \in V$, so $V$ is commutative.
  
  Thus, a commutative vertex algebra may be defined as one in which all $Y(a, z)$ belong to $\End(V)[[z]]$.

  Denote by $Y_a$ the endomorphism of a commutative vertex algebra $V$ which is the constant term of $Y(a, z)$ for $a \in V$, and define a bilinear operation $\circ$ on $V$ by setting $a\circ b=Y_ab$.
  By construction, $Y_aY_b = Y_bY_a$.
  This implies both commutativity and associativity of $\circ$.
  Furthermore, the vacuum vector $\vac$ is a unit, and the operator $T$ is a derivation with respect to this product.
  Thus, we have the structure of a commutative algebra with a derivation on $V$.

  Conversely, let $V$ be an associative and commutative algebra with a unit $1$ and a derivation $T$.
  We call this object a differential algebra $(V, T)$.
  Then $V$ can become a vertex algebra.
  Take $\vac = 1$, and set
  \begin{equation*}
    Y(a, z) = e^{Tz}a.
  \end{equation*}
  It is straightforward to check that all the axioms of a commutative vertex algebra are satisfied.

  Therefore, we obtain an isomorphism between the category of differential algebras and the category of commutative vertex algebras.
\end{example}

The theory done in \Cref{sec:formal-calculus} and \Cref{sec:fields-over-vector} translates into the language of vertex algebras.

\begin{theorem}[Properties of vertex algebras]
  \label{thr:12}
  Let $V$ be a vertex algebra.
  For $a, b, c \in V$ and $n, j \in \mathbb{Z}$:
  \begin{enumerate}
  \item $Y: V \to \mathcal{F}(V)$ is injective;
  \item $Y(a, z)\vac \in V[[z]]$ and $Y(a, z)\vac = e^{Tz}a$ so $T^ma = m!a_{(-m - 1)}\vac$ for $m \in \mathbb{N}$;
  \item $Y(Ta, z) = \partial_zY(a, z)$;
  \item ($n$-product identity) $Y(a, z)_{(n)}Y(b, z) = Y(a_{(n)}b, z)$;
  \item ($T$ as an even derivation) $T(Y(a, z)b) = Y(Ta, z)b + Y(a, z)Tb$;
  \item $e^{Tw}Y(a,z)e^{-Tw}=i_{z,w}Y(a,z+w)$;
  \item (Borcherds identity)
    \begin{equation*}
      i_{z, w}(z - w)^nY(a, z)Y(b, w)c - p(a, b)i_{w, z}(z - w)^nY(b, w)Y(a, z)c = \sum_{j \in \mathbb{N}}Y(a_{(n + j)}b, w)\frac{\partial^j_w\delta(z, w)}{j!}c;
    \end{equation*}
  \item (Skew-symmetry)
    \begin{equation*}
      Y(a, z)b = p(a, b)e^{Tz}Y(b, -z)a;
    \end{equation*}
  \item $(a_{(n)}b)_{(j)}c = \sum_{k \in \mathbb{N}}(-1)^k\binom{n}{k}(a_{(n - k)}b_{(j + k)}c - (-1)^np(a, b)b_{(n + j - k)}a_{(k)}c)$.
  \end{enumerate}
\end{theorem}

\begin{theorem}[Original Borcherds identity]
  \label{thr:13}
  Let $V$ be a vertex algebra.
  For $a, b, c \in V$,
  \begin{equation*}
    \sum_{j \in \mathbb{N}}(-1)^j\binom{n}{j}\left(a_{(m + n - j)}(b_{(k + j)}c) - (-1)^np(a, b)b_{(n + k - j)}(a_{(m + j)}c)\right) = \sum_{j \in \mathbb{N}}\binom{m}{j}(a_{(n + j)}b)_{(m + k - j)}c,
  \end{equation*}
  or, equivalently, for all $F(z, w) = z^mw(z - w)^k$ where $m, n, k \in \mathbb{Z}$,
  \begin{equation*}
    \begin{split}
      &\Res_{z - w}(Y(Y(a, z - w)b, w)ci_{w, z - w}F(z, w)) = \\
      &\Res_z(Y(a, z)Y(b, w)ci_{z, w}F(z, w)) - \Res_z(Y(b, w)Y(a, z)ci_{w, z}F(z, w)).
    \end{split}
  \end{equation*}
\end{theorem}

Let $V$ be a vertex algebra.
A vertex subalgebra of $V$ is a subspace of $W$ of $V$ which contains $\vac$ and such that $Y(a, z)b \in W((z))$ for $a, b\in W$.
Because $Ta = a_{(-2)}\vac$ and $\vac\in W$, this implies that $TW \subseteq W$.
Thus, $(W, \vac, T|_{W}: W \to W, Y|_{W}: W \to \mathcal{F}(W))$ is a vertex algebra in its own right.
Let $S \subseteq V$ be a subset.
The vertex subalgebra generated by $S$ is the smallest vertex subalgebra containing $S$ which is the intersection of all subalgebras containing $S$.
It is denoted $\langle S \rangle$ and we can prove that
\begin{equation*}
  \langle S \rangle = \vspan \{a^1_{(n_1)}\dots a^s_{(n_s)}\vac \mid s \in \mathbb{N}, a^i \in S, n_1, \dots, n_s \in \mathbb{Z}\}.
\end{equation*}
The vertex algebra $V$ is strongly generated by $S \subseteq V$ if
\begin{equation*}
  V = \vspan \{a^1_{(-n_1 - 1)}\dots a^s_{(-n_s - 1)}\vac \mid s \in \mathbb{N}, a^i \in S, n_1, \dots, n_s \in \mathbb{N}\}.
\end{equation*}

An ideal of a vertex algebra $V$ is a subspace $I$ of $V$ such that $Y(a, z)b \in I((z))$ and $Y(b, z)a \in I((z))$ for all $a \in V$ and $b \in I$.
For example, the kernel of vertex algebra homomorphism is an ideal.
Note that a right ideal $I$ is automatically $T$-invariant ($TI \subseteq I$) because of \eqref{eq:13}.
Also, right ideals and $T$-invariant left ideals are automatically two-sided ideals because of skewsymmetry.
However, to prove that a subspace is an ideal it is usally easier to check that it is $T$-invariant and a left ideal.
It follows that for any ideal $I$, $V/I$ inherits a natural quotient vertex algebra structure $(V/I, \vac + I,T_{V/I}: V/I \to V/I, Y_{V/I}: V/I \to \mathcal{F}(V/I))$.
Let $S \subseteq V$ be a subset.
The ideal generated by $S$ is the smallest ideal containing $S$ which is the intersection of all ideals containing $S$.
It is denoted by $(S)$ and we can prove that
\begin{equation*}
  (S) = \vspan\{b_{(n)}T^ma \mid b \in V, n \in \mathbb{Z}, m \in \mathbb{N}, a \in S\}.
\end{equation*}

We don't examples of vertex algebras other than the ones coming from differential algebras.
It turns out is not an easy task to construct nontrivial vertex algebras.
We need a preliminary concept to do that task. 
A pre-vertex algebra is a quadruple $(V, \vac, T, \mathcal{F})$ where $V = V_{\zero} \oplus V_{\one}$ is a vector superspace, $\vac \in V_{\zero}$, $T \in \End(V)_{\zero}$ and $\mathcal{F} = \{a^j(z) = \sum_{n \in \mathbb{Z}}a^j_{(n)}z^{-n - 1}\}_{j \in J}$ is a collection of $\End(V)$-valued fields such that for $j \in J$, all $a^j_{(n)}$ for $n \in \mathbb{Z}$ have the same parity.
The above data satisfies the following axioms
\begin{enumerate}
\item (Vacuum axiom) $T\vac = 0$;
\item (Translation covariance) $[T, a^j(z)] = \partial_za^j(z)$ for all $j \in J$;
\item (Locality) $a^i(z)$ and $a^j(z)$ are mutually local for all $i, j \in J$;
\item (Completeness) $\vspan\{a^{j_1}_{(n_1)}\dots a^{j_s}_{(n_s)}\vac \mid s \in \mathbb{N}, j_i \in J, n_i \in \mathbb{Z}\} = V$.
\end{enumerate}

Let $(V, \vac, T, \mathcal{F})$ be a pre-vertex algebra.
Define the following subspaces of $\mathcal{F}(V)$:
\begin{align*}
  \mathcal{F}_{\min} &= \vspan\{a^{j_1}(z)_{(n_1)}(a^{j_2}(z)_{(n_2)}\dots(a^{j_s}(z)_{(n_s)}\Id_V)\dots) \mid s \in \mathbb{N}, n_i \in \mathbb{Z}, j_i \in J\}, \\
  \mathcal{F}_{\max} &= \{a(z) \in \mathcal{F}(V) \mid [T, a(z)] = \partial_za(z)\text{ and for }j \in J, (a(z),a^j(z))\text{ is a local pair}\}.
\end{align*}
We have inclusions
\begin{equation*}
  \mathcal{F} \subseteq \mathcal{F}_{\min} \subseteq \mathcal{F}_{\max} \subseteq \mathcal{F}_{tc}.
\end{equation*}
The first inclusion is because for $a(z) \in \mathcal{F}$, $a(z)_{(-1)}\Id_V = a(z) \in \mathcal{F}_{\min}$.
The second inclusion is by \Cref{lmm:5} and Dong's Lemma.
The last inclusion is by definition.

Now we come to a very fundamental theorem, which allows us to construct noncommutative vertex algebras and is the backbone of several of our most important examples of vertex algebras.
\begin{theorem}[Extension theorem]
  \label{thr:14}
  Let $(V, \vac, T, \mathcal{F})$ be a pre-vertex algebra and $\mathcal{F}_{\min}, \mathcal{F}_{\max}$ defined as above.
  Then:
  \begin{enumerate}
  \item $\mathcal{F}_{\min} = \mathcal{F}_{\max}$;
  \item The linear map
    \begin{align*}
      \fs: \mathcal{F}_{\max} &\to V \\
      \fs(a(z)) &= a(z)\vac|_{z = 0}
    \end{align*}
    is well-defined and bijective.
    Denote by $Y: V \to \mathcal{F}(V)$ the inverse map;
  \item $(V, \vac, T, Y)$ is a vertex algebra with $Y: V \to \mathcal{F}(V)$ given explicitly by
    \begin{equation}
      \label{eq:14}
      Y(a^{j_1}_{(n_1)}a^{j_2}_{(n_2)}\dots a^{j_s}_{(n_s)}\vac) = a^{j_1}(z)_{(n_1)}(a^{j_2}(z)_{(n_2)}\dots (a^{j_s}(z)_{(n_s)}\Id_V)\dots),
    \end{equation}
    for $s \in \mathbb{N}$, $j_1, \dots, j_s \in J$ and $n_1, \dots, n_s \in \mathbb{Z}$;
  \item The vertex algebra $V$ is generated by $\{a^j_{(-1)}\vac \mid j \in J\}$;
  \item The only vertex algebra structure on $V$ with $Y(a^j_{(-1)}\vac, z) = a^j(z)$ for $j \in J$ is the one given by \eqref{eq:14}.
  \end{enumerate}
\end{theorem}

\begin{proof}
  By \Cref{thr:10}, the map $\fs$ is well defined as was noted already in \Cref{sec:fields-over-vector}.
  By \Cref{thr:11}(i) and \Cref{thr:11}(iii), $\fs|_{\mathcal{F}_{\min}}: \mathcal{F}_{\mathcal{F}_{\min}} \to V$ is given by
  \begin{equation}
    \label{eq:15}
    \fs|_{\mathcal{F}_{\min}}(a^{j_1}(z)_{(n_1)}(a^{j_2}(z)_{(n_2)}\dots (a^{j_s}(z)_{(n_s)}\Id_V)\dots)) = a^{j_1}_{(n_1)}a^{j_2}_{(n_2)}\dots a^{j_s}_{(n_s)}\vac.
  \end{equation}
  By the completeness axiom of pre-vertex algebras, $\fs|_{\mathcal{F}_{\min}}$ is surjective.

  The map $\fs: \mathcal{F}_{\max} \to V$ is injective using \Cref{lmm:5} with $\mathcal{F}' = \mathcal{F}_{\min}$.
  Recall the inclusion $\mathcal{F}_{\min} \subseteq \mathcal{F}_{\max}$.
  We know that $\fs|_{\mathcal{F}_{\min}}$ is surjective and $\fs$ is injective, so we can conclude that is in fact bijective and $\mathcal{F}_{\min} = \mathcal{F}_{\max}$.
  This proves (i) and (ii).

  For (iii), we need to show that $Y(a, z)$ is translation covariant for $a \in V$ and that each pair $(Y(a, z), Y(b, w))$ is local for $a, b \in V$.
  Translation covariance comes from \Cref{lmm:3} and locality comes from Dong's lemma.
  
  Note that we have $Y(a^j_{(-1)}\vac, z) = a^j(z)$ for $j \in J$.
  Therefore, $(a^j_{(-1)}\vac)_{(n)} = a^j_{(n)}$ for $j \in J$ and $n \in \mathbb{Z}$.
  By the completeness axiom, we get (iv).

  Uniqueness of the vertex algebra structure follows from the completeness axiom of pre-vertex algebras, the $n$-product identity and the fact that $\vac \mapsto \Id_V$ in any vertex algebra.
  This finishes (v) and the proof of the theorem.
  
\end{proof}

\begin{corollary}
  \label{crl:1}
  Let $V$ be a vertex algebra, $s \in \mathbb{N}$, $a^1, \dots, a^s \in V$ and $n_1, \dots, n_s \in \mathbb{Z}$.
  Then 
  \begin{equation*}
    Y(a^1_{(n_1)}a^2_{(n_2)}\dots a^s_{(n_s)}\vac, z) = Y(a^j, z)_{(n_1)}(Y(a^2, z)_{(n_2)}\dots (Y(a^s, z)_{(n_s)}\Id_V)\dots).
  \end{equation*}
  In particular, for $s, n_1, \dots, n_s \in \mathbb{N}$ and $a^1, \dots, a^s \in V$,
  \begin{equation}
    \label{eq:16}
    Y(a^1_{(-n_1 - 1)}\dots a^s_{(-n_s - 1)}\vac, z) = \frac{:\partial^{n_1}_zY(a^1,z)\dots \partial^{n_s}_zY(a^s,z):}{n_1!\dots n_s!}.
  \end{equation}
  If $V$ is given by a pre-vertex algebra $(V, \vac, T, \mathcal{F})$ where $\mathcal{F} = \{a^j(z) = \sum_{n \in \mathbb{Z}}a^j_{(n)}z^{-n - 1}\}_{j \in J}$, then for $s, n_1, \dots, n_s \in \mathbb{N}$ and $j_1, \dots, j_s \in J$,
  \begin{equation}
    \label{eq:17}
    Y(a^{j_1}_{(-n_1 - 1)}\dots a^{j_s}_{(-n_s - 1)}\vac, z) = \frac{:\partial^{n_1}_za^{j_1}(z)\dots \partial^{n_s}_za^{j_s}(z):}{n_1!\dots n_s!}.
  \end{equation}
\end{corollary}

Formulas \eqref{eq:16} and \eqref{eq:17} are used to do explicit computations.

\printindex

\bibliographystyle{alpha}
\bibliography{ising-modules.bib}

\end{document}
