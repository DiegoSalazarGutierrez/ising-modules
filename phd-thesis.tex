\documentclass[a4paper, 12pt, reqno]{amsart}

\usepackage{amssymb}
\usepackage{enumerate}
\usepackage{hyperref}
\usepackage[margin = 0.8in]{geometry}
\usepackage{enumitem}
\usepackage{tikz-cd}
\usepackage[nameinlink]{cleveref}

\newtheorem{theorem}{Theorem}[subsection]
\newtheorem{lemma}[theorem]{Lemma}
\newtheorem{proposition}[theorem]{Proposition}
\newtheorem{corollary}[theorem]{Corollary}

\theoremstyle{remark}
\newtheorem{remark}[theorem]{Remark}

\numberwithin{equation}{subsection}

\DeclareMathOperator{\Vir}{Vir}
\DeclareMathOperator{\Id}{Id}
\DeclareMathOperator{\gr}{gr}
\DeclareMathOperator{\End}{End}
\DeclareMathOperator{\ch}{ch}
\DeclareMathOperator{\lm}{lm}
\DeclareMathOperator{\vspan}{span}
\DeclareMathOperator{\Ind}{Ind}
\DeclareMathOperator{\len}{len}
\DeclareMathOperator{\psn}{psn}
\DeclareMathOperator{\Frac}{Frac}
\DeclareMathOperator{\Res}{Res}
\DeclareMathOperator{\vac}{|0\rangle}
\DeclareMathOperator{\vachalf}{|1/2\rangle}
\DeclareMathOperator{\vacsixteen}{|1/16\rangle}
\DeclareMathOperator{\zero}{\overline{0}}
\DeclareMathOperator{\one}{\overline{1}}
\DeclareMathOperator{\Der}{Der}

\makeindex

\begin{document}

\begin{abstract}
  To every $h + \mathbb{N}$-graded module $M$ over a $\mathbb{N}$-graded conformal vertex algebra $V$ we associate an increasing filtration $(G^pM)_{p \in \mathbb{Z}}$ which is compatible with the filtrations introduced by Haisheng Li.
  The associated graded vector space $\gr^G(M)$ is naturally a module over the Poisson vertex algebra $\gr^G(V)$.
  We study $\gr^G(M)$ for the three irreducible modules of the Ising model $\Vir_{3, 4}$, namely $\Vir_{3,4} = L(1/2, 0)$, $L(1/2, 1/2)$ and $L(1/2, 1/16)$.
  We obtain an explicit monomial basis for each of these modules and a formula for their refined characters which are related to Nahm sums for the matrix $\left(\begin{smallmatrix} 8 & 3 \\ 3 & 2 \end{smallmatrix}\right)$.
\end{abstract}

\title{PBW bases of irreducible Ising modules}
\author{Diego Salazar Gutierrez}
\address{Instituto de Matemática Pura e Aplicada, Rio de Janeiro, RJ, Brazil}
\email{diego.salazar@impa.br}
\date{\today}
\maketitle

\tableofcontents

\section{Introduction}
\label{sec:introduction}

\section{Vertex algebras and their modules}
\label{sec:vert-algebr-their}

\subsection{Formal calculus}
\label{sec:formal-calculus}

All vector spaces and all algebras are over $\mathbb{C}$, the field of complex numbers.
The set of natural numbers $\{0, 1, \dots\}$ is denoted by $\mathbb{N}$, the set of integers is denoted by $\mathbb{Z}$ and the set of positive integers $\{1, 2, \dots\}$ is denoted by $\mathbb{Z}_+$.

The vector space of \index{formal distribution|emph}\emph{formal distributions} in $n \in \mathbb{N}$ variables, denoted by $\mathbb{C}[[X_1^{\pm 1}, \dots, X_n^{\pm 1}]]$, is the set of functions $f: \mathbb{Z}^n \to \mathbb{C}$ with the natural operations of addition and multiplication by a scalar.
The field of \index{rational function|emph}\emph{rational functions} in $n$ variables, denoted by $\mathbb{C}(X_1, \dots, X_n)$, is the field of fractions $\Frac(\mathbb{C}[X_1, \dots, X_n])$.
The field of \index{formal Laurent series|emph}\emph{formal Laurent series} is the subalgebra of elements $f \in \mathbb{C}[[Z^{\pm 1}]]$ such that there is $N \in \mathbb{Z}$ with $f_n = 0$ for $n \le N$.
We also have $\mathbb{C}((X)) = \Frac(\mathbb{C}[[X]])$.
The field of \index{joint Laurent series|emph}\emph{joint Laurent series} in $n$ variables, denoted by $\mathbb{C}((X_1, \dots, X_n))$, is $\Frac(\mathbb{C}[[X_1, \dots, X_n]])$.

If $V$ is a vector space, we similarly define $V[[X_1^{\pm 1}, \dots, X_n^{\pm 1}]]$ and $V((X))$ but in this case, $V((X))$ is only a vector space.

Let $V$ be a vector space.
The \index{formal distribution!Fourier expansion of|emph}\emph{Fourier expansion} of a formal distribution $a(z) \in V[[z^{\pm 1}]]$, written as $a = \sum_{n \in \mathbb{Z}} a_nz^n$, is conventionally written in the theory of vertex algebras as
\begin{equation*}
  a(z) = \sum_{n \in \mathbb{Z}}a_{(n)}z^{-n - 1},
\end{equation*}
where
\begin{equation*}
  a_{(n)} = a_{-n-1}.
\end{equation*}
The \index{formal distribution!residue of|emph}\emph{residue} of a formal distribution $a(z) \in V[[z^{\pm 1}]]$ is defined as
\begin{equation*}
  \Res_z(a(z)) = a_{(0)} = a_{-1}.
\end{equation*}

If $P\in R[[z_1^{\pm 1},\dots, z_n^{\pm 1}]]$ and $Q\in R[[w_1^{\pm 1},\dots, w_m^{\pm 1}]]$, then $PQ\in R[[z_1^{\pm 1},\dots z_n^{\pm 1},w_1^{\pm 1},\dots w_m^{\pm 1}]]$ is defined in the natural way.
However, if both $P$ and $Q$ belong to $R[[z_1^{\pm 1},\dots, z_n^{\pm 1}]]$, we may run into trouble because infinite sums may appear.

An important formal distribution in two variables $z, w$ is the \index{formal distribution!delta|emph}\emph{formal delta distribution} which is defined by
\begin{equation*}
  \delta(z, w) = \sum_{n \in \mathbb{Z}}z^nw^{-n - 1} \in \mathbb{C}[[z^{\pm 1}, w^{\pm 1}]].
\end{equation*}

The \emph{expansion in the domain} $|z| > |w|$ is the field homomorphism $i_{z, w}: \mathbb{C}((z, w)) \to \mathbb{C}((z))((w))$ such that the following diagram commutes
\begin{equation*}
  \begin{tikzcd}
    {\mathbb{C}[[z, w]]} \arrow[rd] \arrow[r] & {\mathbb{C}((z, w))} \arrow[d, "{i_{z,w}}"] \\
    & \mathbb{C}((z))((w))                                          
  \end{tikzcd}
\end{equation*}
where the unlabeled homomorphisms are the natural ones.
Similarly, \index{expansion in the domain|emph}expansion in the domain $|w| > |z|$ is the field homomorphism $i_{w, z}: \mathbb{C}((z, w)) \to \mathbb{C}((w))((z))$ such that the following diagram commutes
\begin{equation*}
  \begin{tikzcd}
    {\mathbb{C}[[z, w]]} \arrow[rd] \arrow[r] & {\mathbb{C}((z, w))}\arrow[d, "{i_{w, z}}"] \\
    & \mathbb{C}((w))((z))                                          
  \end{tikzcd}
\end{equation*}

We have natural inclusions $\mathbb{C}((z))((w)) \to \mathbb{C}[[z^{\pm 1}, w^{\pm 1}]]$ and $\mathbb{C}((w))((z)) \to \mathbb{C}[[z^{\pm 1}, w^{\pm 1}]]$.
The diagram
\begin{equation*}
  \begin{tikzcd}
    & {\mathbb{C}((z, w))} \arrow[ld, "{i_{z,w}}"'] \arrow[rd, "{i_{w,z}}"] &                                 \\
    \mathbb{C}((z))((w)) \arrow[rd] &                                                                      & \mathbb{C}((w))((z)) \arrow[ld] \\
    & {\mathbb{C}[[z^{\pm 1}, w^{\pm 1}]]}                                  &                                
  \end{tikzcd}
\end{equation*}
doesn't commute. In fact, the formal delta distribution can be expressed as
\begin{equation*} 
  \delta(z, w) = i_{z, w}(\tfrac{1}{z - w}) - i_{w, z}(\tfrac{1}{z - w}),
\end{equation*}
where we consider $i_{z, w}(\frac{1}{z - w})$ and $i_{w, z}(\frac{1}{z - w})$ as elements of $\mathbb{C}[[z^{\pm 1}, w^{\pm 1}]]$.
From now on, we will consider $i_{z, w}$ and $i_{w, z}$ as mapped into $\mathbb{C}[[z^{\pm 1}, w^{\pm 1}]]$.

Let $V$ be a vector space.
A formal distribution $a(z, w) \in V[[z^{\pm 1}, w^{\pm 1}]]$ is \index{formal distribution!local|emph}\emph{local} if there is $N \in \mathbb{N}$ such that
\begin{equation*}
  (z - w)^Na(z, w)=0.
\end{equation*}
For example, the \index{formal distribution!delta}formal delta distribution $\delta(z, w)$ is local.

\begin{theorem}[{\cite[Proposition 2.2]{kac_vertex_1998}}]
  \label{thr:1}
  Let $a(z, w) \in V[[z^{\pm 1}, w^{\pm 1}]]$ be a local formal distribution.
  Then $a(z, w)$ can be written uniquely as a sum
  \begin{equation}
    \label{eq:1}
    a(z, w) = \sum_{j \in \mathbb{N}}c^j(w)\frac{\partial_w^j\delta(z, w)}{j!}
  \end{equation}
  where $c^j(w) \in V[[w^{\pm 1}]]$ are formal distributions given by
  \begin{equation}
    \label{eq:2}
    c^j(w) = \Res_z(z - w)^ja(z, w).
  \end{equation}
  In addition, the converse is true.
\end{theorem}

We now define the notion of Fourier transform in two cases: in one and two variables. Let $V$ be a vector space and let $a(z) \in V[[z^{\pm 1}]]$.
We define the \index{Fourier transform!in one variable|emph}\emph{Fourier transform in one variable} of $a(z)$ by
\begin{equation*}
  F^\lambda_za(z) = \Res_z(e^{\lambda z}a(z)) \in V[[\lambda]].
\end{equation*}
\begin{proposition}[{\cite[Proposition 1.5.2]{nozaradan_introduction_2008}}]
  \label{prp:1}
  $F^\lambda_z$ satisfies the following properties:
  \begin{enumerate}[label = (\alph*)]
  \item $F^\lambda_z\partial_za(z) = -\lambda F^\lambda_za(z)$;
  \item $F^\lambda_z(e^{zT}a(z)) = F^{z + T}_z(a(z))$ where $T \in \End(V)$ and $a(z) \in V((z))$;
  \item $F^\lambda_z(a(-z)) = -F^{-\lambda}_za(z)$;
  \item $F^\lambda_z\partial^n_w\delta(z, w) = e^{\lambda w}\lambda^n$.
  \end{enumerate}
\end{proposition}

Now let $a(z, w) \in V[[z^{\pm 1}, w^{\pm 1}]]$.
We define the \index{Fourier transform!in two variables|emph}\emph{Fourier transform in two variables} of $a(z, w)$ by
\begin{equation*}
  F^\lambda_{z, w}a(z, w) = \Res_ze^{\lambda(z - w)}a(z, w) \in V[[w^{\pm 1}]][[\lambda]].
\end{equation*}

Expanding the definition of $F^\lambda_{z, w}$, we obtain another expression
\begin{equation*}
  F^\lambda_{z, w}a(z, w) = \sum_{j \in \mathbb{N}}\frac{\lambda^j}{j!}c^j(w),
\end{equation*}
where
\begin{equation*}
  c^j(w) = \Res_z(z - w)^ja(z, w).
\end{equation*}
\begin{proposition}[{\cite[Proposition 1.5.4]{nozaradan_introduction_2008}}]
  \label{prp:2}
  $F^\lambda_{z, w}$ satifies the following properties
  \begin{enumerate}[label = (\alph*)]
  \item If $a(z, w)$ is local then $F^\lambda_{z, w}a(z, w) \in V[[w^{\pm 1}]][\lambda]$;
  \item $F^\lambda_{z, w}\partial_za(z, w) = -\lambda F^\lambda_{z, w}a(z, w) = [\partial_w, F^{\lambda}_{z, w}]a(z, w)$;
  \item If $a(z, w)$ is local, then $F^\lambda_{z, w}a(w, z) = F^{-\lambda - \partial_w}_{z, w}a(z, w)$, where $F^{-\lambda - \partial_w}_{z, w}a(z, w) = F^\mu_{z, w}a(z, w)|_{u = -\lambda - \partial_w}$.
  \end{enumerate}
\end{proposition}

\subsection{Lie conformal superalgebras}
\label{sec:lie-conf-super}

A \index{vector superspace|emph}\emph{vector superspace} is a $\mathbb{Z}_2$-graded vector space $V = V_{\zero} \oplus V_{\one}$ where $\mathbb{Z}_2 = \mathbb{Z}/2\mathbb{Z} = \{\zero, \one\}$, $\zero = 0 + 2\mathbb{Z}$ and $\one = 1 + 2\mathbb{Z}$.
% Elements of $V_{\zero}$ are called \index{vector superspace!even element of |emph}\emph{even}, elements of $V_{\one}$ are called \index{vector superspace!odd element of|emph}\emph{odd}.
We call $V_{\zero}$ the \index{vector superspace!even subspace of|emph}\emph{even subspace} of $V$ and $V_{\one}$ the \index{vector superspace!odd subspace of|emph}\emph{odd subspace} of $V$.
Nonzero elements of $V_{\zero}\cup V_{\one}$ are called \index{vector superspace!homogeneous element of|emph}\emph{homogeneous}.
A \index{superalgebra|emph}\emph{superalgebra} is a $\mathbb{Z}_2$-graded algebra $A = A_{\zero} \oplus A_{\one}$.
This means $A_{\alpha}A_{\beta} \subseteq A_{\alpha + \beta}$ for all $\alpha, \beta \in \mathbb{Z}_2$.

We set $(-1)^{\zero} = 1$, $(-1)^{\one} = -1$.
If $a\in V_\alpha$, $v\neq 0$ is homogeneous, we set $p(a) = \alpha$ and call it the \index{vector superspace!homogeneous element of!parity of|emph}\emph{parity} of $a$.
If $a$ and $b$ are homogeneous, we set $p(a, b) = (-1)^{p(a)p(b)}$.
A \index{Lie superalgebra|emph}\emph{Lie superalgebra} is a superalgebra $\mathfrak{g} = \mathfrak{g}_{\zero} \oplus \mathfrak{g}_{\one}$ with the product written $[a, b]$ for $a, b \in \mathfrak{g}$ and called \index{Lie superalgebra!Lie superbracket of|emph}\emph{Lie superbracket} satisfying the following properties:
\begin{enumerate}[label = (\alph*)]
\item $[\bullet, \bullet]$ is \index{graded antisymmetric|emph}\emph{graded antisymmetric}
  \begin{equation*}
    [a, b] = -p(a, b)[b, a];
  \end{equation*}
\item $[\bullet, \bullet]$ satisfies the \index{graded Jacobi identity|emph}\emph{graded Jacobi identity}
  \begin{equation*}
    p(a, c)[a, [b, c]] + p(b, a)[b, [c, a]] + p(c, b)[c, [a, b]] = 0.
  \end{equation*}
\end{enumerate}

In an \index{associative superalgebra}associative superalgebra $A$, we can define the superbracket of homogeneous elements by
\begin{equation*}
  [a, b] = ab - p(a, b)ba.
\end{equation*}
It can then be extended by linearity to non-homogeneous elements.
With this superbracket, $A$ becomes a Lie superalgebra called the \index{associative superalgebra!underlying Lie superalgebra of|emph}\emph{underlying Lie superalgebra} of $A$ and is denoted by $[A]$.

\begin{remark}
  \label{rmk:1}
  The even part $\mathfrak{g}_{\zero}$ of a Lie superalgebra is just a standard Lie algebra.
  However, unlike superspaces and superalgebras, a Lie superalgebra is not always a Lie algebra.
  This is why we prefer the term Lie superalgebra instead of $\mathbb{Z}_2$-graded Lie algebra.
\end{remark}

The most important example of an associative superalgebra is the endomorphism algebra $\End(V)$ of a superspace $V$ with the $\mathbb{Z}_2$-grading given by:
\begin{equation*}
  \End(V)_{\alpha} = \{a \in \End(V) \mid \forall \beta \in \mathbb{Z}_2: aV_\beta \subseteq V_{\alpha + \beta}\},
\end{equation*}
for $\alpha \in \mathbb{Z}_2$.
We denote $\mathfrak{gl}(V) = [\End(V)]$.

Let $\mathfrak{g}$ be a Lie superalgebra.
We first extend the Lie superbracket on $\mathfrak{g}$ to the \index{formal distribution!Lie superbracket of|emph}\emph{Lie superbracket} between two $\mathfrak{g}$-valued formal distributions in one variable. Starting from $a(z) = \sum_{m \in \mathbb{Z}}a_{(m)}z^{-m - 1} \in \mathfrak{g}[[z^{\pm 1}]]$ and $b(w) = \sum_{n \in \mathbb{Z}}b_{(n)}z^{-n - 1} \in \mathfrak{g}[[w^{\pm 1}]]$, we define a new formal distribution in two variables by defining the superbracket
\begin{equation*}
  [a(z), b(w)] = \sum_{m, n}[a_{(m)}, b_{(n)}]z^{-m - 1}w^{-n - 1} \in \mathfrak{g}[[z^{\pm 1}, w^{\pm 1}]]
\end{equation*}

Let $\mathfrak{g}$ be a Lie superalgebra.
A pair $(a(z), b(z))$ of $\mathfrak{g}$-valued formal distributions is said \index{formal distribution!local pair of|emph}\emph{local} if $[a(z), b(w)]$ is local.
By \Cref{thr:1}, this means that
\begin{equation*}
  [a(z), b(w)] = \sum_{j \in \mathbb{N}}c^j(w)\frac{\partial^j_w\delta(z, w)}{j!}.
\end{equation*}
with $c^j(w) = \Res_z(z - w)^j[a(z), b(w)] \in \mathfrak{g}[[w^{\pm 1}]]$.
Equivalently, we can write this equation as
\begin{equation}
  \label{eq:3}
  [a_{(m)}, b_{(n)}] = \sum_{j \in \mathbb{N}}c^j(w)_{(m + n - j)}.
\end{equation}

Let $\mathfrak{g}$ be a Lie superalgebra.
A subset $\mathfrak{F} \subseteq \mathfrak{g}[[z^{\pm 1}]]$ of formal distributions is called a \index{formal distribution!local family of|emph}\emph{local family} if all pairs of its elements are local.
Let $a(w)$ and $b(w)$ be two $\mathfrak{g}$-valued formal distributions.
For $j \in \mathbb{N}$, their \index{formal distribution!$j$-product of|emph}\emph{$j$-product} is the $\mathbb{C}$-bilinear map defined by
\begin{align*}
  \bullet_{(j)}\bullet: \mathfrak{g}[[w^{\pm 1}]]\times\mathfrak{g}[[w^{\pm 1}]] &\to \mathfrak{g}[[w^{\pm 1}]]\\
  a(w)_{(j)}b(w) &= \Res_z(z - w)^j[a(z), b(w)].
\end{align*}
We define $a(w)_{(j)} \in \End(\mathfrak{g}[[z^{\pm 1}]])$ in the natural way.
If $(a(z), b(z))$ is a local pair, then \eqref{eq:3} becomes
\begin{equation}
  \label{eq:4}
  [a_{(m)}, b_{(n)}] = \sum_{j \in \mathbb{N}}\binom{m}{j}(a(w)_{(j)}b(w))_{(m + n - j)}.
\end{equation}
All these identities led us to the define the following new algebraic structure that encodes the relevant information compactly.

Let $\mathfrak{g}$ be a Lie superalgebra.
The \index{formal distribution!$\lambda$-bracket of|emph}\emph{$\lambda$-bracket} of two $\mathfrak{g}$-valued formal distributions is defined by the $\mathbb{C}$-bilinear map
\begin{align*}
  [\bullet_{\lambda}\bullet]: \mathfrak{g}[[w^{\pm 1}]]\times \mathfrak{g}[[w^{\pm 1}]] &\to \mathfrak{g}[[w^{\pm 1}]][[\lambda]] \\
  [a(w)_{\lambda}b(w)] &= F^{\lambda}_{z, w}[a(z), b(w)]
\end{align*}
It can easily be shown that the $\lambda$-bracket is related to the $j$-products by
\begin{equation*}
  [a(w)_{\lambda}b(w)] = \sum_{j \in \mathbb{N}}a(w)_{(j)}b(w)\frac{\lambda^j}{j!}.
\end{equation*}
This suggest to see the $\lambda$-bracket as the generating function of the $j$-products.
It allows us to gather all the $j$-products in one product alone, the price to pay being the additional formal variable $\lambda$.
Note that for a local pair, the sum in the expansion of $[a(w)_{\lambda} b(w)]$ in terms of the $j$-products is finite, i.e.\ $[a(w)_{\lambda}b(w)] \in \mathfrak{g}[[w^{\pm 1}]][\lambda]$.

\begin{theorem}[{\cite[Section 2.3]{nozaradan_introduction_2008}}]
  \label{thr:2}
  The $j$-products and the $\lambda$-bracket satisfy the following properties:
  \begin{enumerate}[label = (\alph*)]
  \item $(\partial a(w))_{(j)}b = -ja(w)_{(j - 1)}b(w)$;
  \item $a(w)_{(j)}\partial b(w) = \partial(a(w)_{(j)}b(w)) + ja(w)_{(j - 1)}b(w)$;
  \item $\partial(a(w)_{(j)}b(w))=(\partial a(w))_{(j)}b(w)+a(w)_{(j)}\partial b(w)$;
  \item $[\partial a(w)_{\lambda}b(w)] = -\lambda [a(w)_{\lambda}b(w)]$;
  \item $[a(w)_{\lambda}\partial b(w)] = (\partial + \lambda)[a(w)_{\lambda}b(w)]$;
  \item $\partial[a(w)_\lambda b(w)]=[\partial a(w)_\lambda b(w)]+[a(w)_\lambda \partial b(w)]$.
  \end{enumerate}
\end{theorem}
\begin{remark}
  \label{rmk:2}
  Properties (c) and (f) tell us that $\partial: \mathfrak{g}[[z^{\pm 1}]] \to \mathfrak{g}[[z^{\pm 1}]]$ acts as a derivation on the $j-$products and the $\lambda$-bracket.
\end{remark}

Let $V$ be vector superspace.
From now on, all coefficients of a formal distribution are assumed to have the same parity.
Therefore, we can define the \index{formal distribution!parity of|emph}\emph{parity} of a formal distribution $a(z) \in V[[z^{\pm 1}]]$ as $p(a(z)) = p(a_{(n)})$ for any $n \in \mathbb{Z}$.

\begin{theorem}[{\cite[Section 2.3]{nozaradan_introduction_2008}}]
  \label{thr:3}
  Let $\mathfrak{g}$ be a Lie superalgebra
  The $j$-products and the $\lambda$-brackets between $\mathfrak{g}$-valued formal distributions satisfy the following properties
  \begin{enumerate}[label = (\alph*)]
  \item $b(w)_{(j)}a(w) = -p(a(w), b(w))\sum_{l = 0}^{\infty}(-1)^{j + l}\frac{\partial^l(a(w)_{(j + l)}b(w))}{l!}$ if $(a(w), b(w))$ is a local pair;
  \item $[a(w)_{(p)}, b(w)_{(m)}] = \sum_{k = 0}^p\binom{p}{k}(a(w)_{(k)}b(w))_{(p + m - k)}$;
  \item $[b(w)_{\lambda}a(w)] = -p(a(w), b(w))[a(w)_{-\lambda - \partial}b(w)]$ if $(a(w), b(w))$ is a local pair;
  \item $[a(w)_{\lambda}[b(w)_{\mu}c(w)]] = [[a(w)_{\lambda}b(w)]_{\lambda + \mu}c(w)] + p(a(w), b(w))[b(w)_{\mu}[a(w)_{\lambda}c(w)]]$;
  \item $F^{\lambda + \mu}_z[a(z)_{\lambda}b(z)] = [F^{\lambda}_za(z), F^{\mu}_zb(z)]$.
  \end{enumerate}
\end{theorem}

Let $\mathfrak{g}$ be a Lie superalgebra and denote by $\Der(\mathfrak{g})$ the \index{subspace of derivations}subspace of derivations of $\End(\mathfrak{g})$.
A \index{formal distribution Lie superalgebra|emph}\emph{formal distribution Lie superalgebra} is a pair $(\mathfrak{g}, \mathfrak{F})$ such that $\mathfrak{F}$ is a local family of $\mathfrak{g}$-valued formal distributions, denoted by $\{a^j(z) = \sum_{n \in \mathbb{Z}}a^j_{(n)}z^{-n - 1}\}_{j \in J}$, such that the coefficients $\{a^j_{(n)} \mid j \in J, n \in \mathbb{Z}\}$ span the whole $\mathfrak{g}$.
A \index{formal distribution Lie superalgebra!regular|emph}\emph{regular} formal distribution Lie superalgebra is a triple $(\mathfrak{g}, \mathfrak{F}, T)$ such that
\begin{enumerate}[label = (\alph*)]
\item $(\mathfrak{g}, \mathfrak{F})$ is a formal distribution Lie superalgebra;
\item $\mathbb{C}[\partial_z]\mathfrak{F}$ is closed under all $n$-th products for $n \in \mathbb{N}$;
\item $T \in \Der(\mathfrak{g})$ satisfies
  \begin{equation*}
    T(a^j(z)) = \partial_za^j(z)
  \end{equation*}
  which is equivalent to
  \begin{equation}
    \label{eq:5}
    T(a^j_{(n)}) = -na^j_{(n - 1)},
  \end{equation}
  for all $j \in J$ and $n \in \mathbb{Z}$. 
\end{enumerate}
\begin{remark}
  \label{rmk:3}
  Note that \eqref{eq:5} and the fact that $\{a^j_{(n)} \mid j \in J, n \in \mathbb{Z}\}$ span $\mathfrak{g}$ imply that if such $T$ exists, it is unique so we could remove $T$ from the notation, but we won't.
\end{remark}

Let $(\mathfrak{g}, \mathfrak{F}, T)$ be a formal distributions Lie superalgebra.
The annihilation subalgebra of 
\printindex

\bibliographystyle{alpha}
\bibliography{ising-modules.bib}

\end{document}
